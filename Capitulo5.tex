\chapter{Análisis financiero}

El presente capítulo tiene como objetivo evaluar los costos asociados al desarrollo, despliegue y potencial comercialización del prototipo a desarrollar. Para ello, se realiza un análisis detallado mediante dos enfoques: la estimación de costos por recursos necesarios y la estimación basada en actividades y tareas específicas. También, se contempla una proyección exploratoria del costo de venta en caso de escalar el prototipo hacia una fase comercial. Esta evaluación financiera permite identificar los recursos requeridos, estimar el presupuesto total del proyecto y valorar su viabilidad dentro del contexto académico y en escenarios de implementación futura.


\section{Informe de costos de creación del producto}

Este informe presenta el análisis financiero asociado al desarrollo del prototipo de la aplicación móvil de traducción de frases del español a la Lengua de Señas Mexicana (LSM) mediante animaciones 3D. El objetivo es identificar y estructurar los costos necesarios para garantizar la viabilidad y factibilidad del proyecto en su etapa académica de construcción y prueba.

\subsection{Enfoque de análisis}

Para una adecuada planeación financiera, se consideran dos enfoques principales de análisis de costos: 
\begin{itemize}
	\item La presupuestación mediante la cotización de recursos necesarios para el desarrollo y operación del prototipo.
	\item La estimación de costos basada en las actividades específicas requeridas para cada fase del proyecto.
\end{itemize}
Ambos enfoques permiten obtener una visión clara y complementaria de los recursos financieros que serán requeridos.

\subsection{Presupuesto mediante recursos}

Dentro del análisis basado en recursos, el proceso de desarrollo se ha estructurado en tres etapas principales, cada una integrada por actividades específicas que permiten avanzar de manera ordenada, asegurando la calidad y funcionalidad del producto final. Las etapas contempladas son:

\begin{itemize}
	\item \textbf{Creación del producto:} Desarrollo inicial, diseño técnico y elaboración de animaciones 3D.
	\item \textbf{Despliegue del producto:} Publicación en entorno Android 14, pruebas piloto y validación funcional.
	\item \textbf{Evaluación financiera:} Análisis de costos finales y valoración preliminar de viabilidad futura.
\end{itemize}

Cada una de estas etapas será detallada en los apartados siguientes, permitiendo una mejor comprensión de los recursos y esfuerzos asociados.

\subsubsection{Creación del prototipo}

Esta primera etapa, abarca todas las actividades necesarias para el desarrollo inicial del prototipo. Estas actividades incluyen la definición de objetivos, la elaboración de especificaciones técnicas, el establecimiento de requisitos del prototipo, el diseño de las animaciones, el desarrollo del código base y la realización de pruebas funcionales. En esta fase también se implementan técnicas de Procesamiento de Lenguaje Natural (PLN) para la segmentación de frases en español, así como el modelado y animación 3D fluida para representar las señas en Lengua de Señas Mexicana (LSM).

\subsubsection{Despliegue del prototipo}

La segunda etapa, contempla todas las acciones necesarias para poner en funcionamiento la aplicación en dispositivos Android, específicamente optimizada para la versión Android 14. Esta fase incluye la publicación del prototipo, la configuración y adaptación técnica para su correcta operación, la elaboración de documentación técnica y guías de usuario, así como la ejecución de pruebas piloto con usuarios finales. A partir de estas pruebas se obtendrá retroalimentación valiosa que permitirá realizar los ajustes necesarios para optimizar la experiencia de uso y garantizar la comprensión efectiva de las animaciones.

\subsubsection{Evaluación financiera}

La tercera etapa, evaluación financiera, implica el análisis de los costos derivados de las etapas anteriores, así como la valoración de los beneficios esperados en términos de impacto social, accesibilidad y sostenibilidad del prototipo. Aunque el proyecto tiene fines académicos, esta evaluación contempla un análisis básico de los recursos humanos, tecnológicos y operativos involucrados, con el objetivo de valorar su viabilidad en escenarios de uso real o en un posible proceso de escalamiento futuro.

Este enfoque estructurado de desarrollo permite garantizar una planeación clara y eficiente, considerando los elementos técnicos, operativos y financieros necesarios para la ejecución del proyecto.

A partir de la identificación de los recursos requeridos, se realiza también una estimación de costos por actividades, desglosando las tareas correspondientes a cada etapa, asignándoles tiempos estimados en horas y costos unitarios por hora de trabajo. La sumatoria de estos costos constituye la base del cálculo del presupuesto total del proyecto.

Por último, como medida preventiva ante posibles riesgos o imprevistos, se contempla una reserva de contingencia equivalente al 15\% del costo total estimado, conforme a lo definido en el análisis de riesgos del proyecto.

\subsection{Presupuesto mediante actividades}

Complementando el análisis anterior, se presenta el presupuesto detallado por actividades, donde se especifica el tiempo estimado y el costo asociado a cada tarea del proyecto. El desglose completo de actividades se encuentra disponible en el siguiente enlace:

El presente proyecto, aunque es de carácter académico, contempla una visión de posible escalabilidad comercial. Por tal motivo, además de estimar las horas de trabajo requeridas para el desarrollo de cada actividad, se realizó una simulación de costos basada en tarifas de mercado aplicables a proyectos de tecnología y desarrollo de software. 

En el siguiente apartado, se presenta el presupuesto detallado por actividades, donde se especifica el tiempo estimado y el costo asociado a cada tarea del proyecto.
El desglose completo de actividades se encuentra disponible en el siguiente enlace:

\begin{flushleft} \href{ruta_del_anexo_o_enlace}{\textbf{Anexo 1: Desglose de Actividades, Horas Estimadas y Costos Simulados}} \end{flushleft}

Cabe señalar que los costos presentados son únicamente proyecciones hipotéticas, utilizadas con el propósito de ilustrar la posible viabilidad financiera y escalabilidad comercial del proyecto. No representan un presupuesto real para la ejecución académica del presente trabajo de titulación.

\texttt{\url{Insertar link}}

A continuación, se ilustran las fases del proyecto de manera visual para facilitar su comprensión.


\section{Análisis de recursos}

\subsection{Costos de servicios}

El presente apartado detalla los costos asociados a los servicios necesarios para el desarrollo del prototipo, considerando un periodo estimado de ejecución de cuatro meses. Los servicios contemplados incluyen electricidad, internet, almacenamiento en la nube y pruebas en dispositivos Android, así como una reserva para servicios externos o pruebas adicionales que pudieran surgir durante el proceso de desarrollo.

\begin{table}[H]
	\centering
	\renewcommand{\arraystretch}{1.5}
	\setlength{\tabcolsep}{12pt}
	\resizebox{\textwidth}{!}{%
		\begin{tabular}{|l|r|r|r|}
			\hline
			\textbf{Recurso} & \textbf{Costo mensual (MXN \$)} & \textbf{Costo a 4 meses (MXN \$)} & \textbf{Costo a 1 año (MXN \$)} \\ \hline
			Electricidad e Internet (compartido entre integrantes) & \$500.00 & \$2,000.00 & \$6,000.00 \\ \hline
			Almacenamiento en la nube (Google Drive / GitHub) & \$300.00 & \$1,200.00 & \$3,600.00 \\ \hline
			Pruebas en dispositivos Android (emulador físico o virtual) & \$500.00 & \$2,000.00 & \$6,000.00 \\ \hline
			Reserva para servicios externos o pruebas adicionales & \$1,000.00 & \$4,000.00 & \$12,000.00 \\ \hline
			\textbf{Total estimado de servicios} & \textbf{\$2,300.00} & \textbf{\$9,200.00} & \textbf{\$27,600.00} \\ \hline
		\end{tabular}%
	}
	\caption{Costos estimados de servicios durante el desarrollo del prototipo}
	\label{tab:costos_servicios}
\end{table}


\subsection{Compras no recurrentes}

En esta sección se presentan los costos estimados de artículos y adquisiciones necesarias para el desarrollo del prototipo, clasificadas como compras no recurrentes. Estas compras representan inversiones únicas que no implican costos periódicos, pero que son fundamentales para el correcto desarrollo y prueba de la aplicación.

\begin{table}[H]
	\centering
	\renewcommand{\arraystretch}{1.5}
	\setlength{\tabcolsep}{8pt}
	\resizebox{\textwidth}{!}{%
		\begin{tabular}{|l|c|r|r|}
			\hline
			\textbf{Recurso} & \textbf{Unidades} & \textbf{Costo unitario (MXN \$)} & \textbf{Costo total (MXN \$)} \\ \hline
			Equipo de cómputo personal (propio de los integrantes) & 3 & -- & -- \\ \hline
			Dispositivo Android para pruebas físicas & 1 & \$5,000.00 & \$5,000.00 \\ \hline
			Capacitación online (cursos: Blender, MediaPipe, PLN) & 3 cursos & \$800.00 & \$2,400.00 \\ \hline
			Compra de modelos 3D o recursos gráficos (opcional) & 1 paquete & \$2,500.00 & \$2,500.00 \\ \hline
			\textbf{Total compras no recurrentes} & & & \textbf{\$9,900.00} \\ \hline
		\end{tabular}%
	}
	\caption{Costos estimados de artículos y compras no recurrentes}
	\label{tab:compras_no_recurrentes}
\end{table}

\noindent \textbf{Nota aclaratoria:}  
El equipo de cómputo utilizado corresponde a dispositivos personales de los integrantes del proyecto, por lo cual no se ha considerado un costo adicional en esta categoría. La compra de modelos 3D o recursos gráficos es considerada opcional y dependerá de la necesidad de complementar el material gráfico disponible de manera gratuita o de libre acceso.

\newpage
\subsection{Sueldos y asesorías}

El presente apartado presenta el costo estimado de los sueldos y asesorías considerados para el desarrollo del prototipo. El cálculo se basa en el tiempo dedicado por los integrantes del equipo de desarrollo (tres estudiantes) y en la contratación puntual de asesorías externas para la validación de señas en Lengua de Señas Mexicana (LSM) y animación 3D. 

Cabe destacar que los valores presentados no representan sueldos formales, sino una estimación del costo de oportunidad asociado al esfuerzo y tiempo invertido, a efectos de contar con una valoración financiera adecuada para el proyecto académico.

\begin{table}[H]
	\centering
	\renewcommand{\arraystretch}{1.5}
	\setlength{\tabcolsep}{10pt}
	\resizebox{\textwidth}{!}{%
		\begin{tabular}{|l|c|r|r|r|r|}
			\hline
			\textbf{Equipo} & \textbf{Cantidad} & \multicolumn{2}{c|}{\textbf{Desarrollo}} & \multicolumn{2}{c|}{\textbf{Mantenimiento / Ajustes}} \\ \hline
			\textbf{Tipo} & & \textbf{Mensual (MXN)} & \textbf{4 meses (MXN)} & \textbf{Mensual (MXN)} & \textbf{A un año (MXN)} \\ \hline
			Desarrollador (estudiante) & 3 & \$8,000.00 & \$96,000.00 (32,000 c/u) & \$4,000.00 (referencial) & \$48,000.00 (escenario futuro) \\ \hline
			Asesoría en animación 3D (freelance) & 1 parcial & \$3,000.00 & \$12,000.00 & -- & -- \\ \hline
			Asesoría en LSM (validación de señas) & 1 parcial & \$2,500.00 & \$2,500.00 (por sesiones) & -- & -- \\ \hline
			\textbf{Total} & \textbf{5} & \textbf{\$13,500.00} & \textbf{\$110,500.00} & \textbf{\$4,000.00} & \textbf{\$48,000.00} \\ \hline
		\end{tabular}%
	}
	\caption{Costos estimados de sueldos y asesorías durante el desarrollo del prototipo}
	\label{tab:sueldos_asesorias}
\end{table}

\noindent \textbf{Nota aclaratoria:}  
La columna de “Mantenimiento / Ajustes” se incluye únicamente como referencia para una posible fase futura de operación continua, en caso de que el prototipo evolucione hacia un producto comercial o requiera soporte extendido. Para efectos del presente trabajo terminal, los costos de mantenimiento no forman parte del presupuesto activo del proyecto.


\subsection{Resumen de costos estimados}

El presente apartado presenta el resumen de los costos estimados para el desarrollo y posible despliegue comercial del prototipo de la aplicación móvil de traducción de frases del español a la Lengua de Señas Mexicana (LSM) mediante animaciones 3D. El análisis incluye el presupuesto derivado de las actividades de desarrollo, los recursos necesarios para las fases de creación y despliegue, las compras no recurrentes, los sueldos del equipo de trabajo y una reserva contemplada para cubrir posibles riesgos o imprevistos durante la implementación.

\begin{table}[H]
	\centering
	\renewcommand{\arraystretch}{1.5}
	\setlength{\tabcolsep}{12pt}
	\resizebox{\textwidth}{!}{%
		\begin{tabular}{|l|r|}
			\hline
			\textbf{Concepto} & \textbf{Monto estimado (MXN \$)} \\ \hline
			Presupuesto total de actividades & \$243,050.00 \\ \hline
			Presupuesto total de recursos (Desarrollo) & \$89,500.00 \\ \hline
			Presupuesto total de recursos (Despliegue) & \$268,500.00 \\ \hline
			Compras no recurrentes & \$196,000.00 \\ \hline
			Sueldo del equipo (Desarrollo) & \$1,500,000.00 \\ \hline
			Sueldo del equipo (Mantenimiento) & \$408,000.00 \\ \hline
			Reserva para riesgos e imprevistos & \$36,457.50 \\ \hline
			\textbf{Presupuesto total estimado del proyecto} & \textbf{\$2,742,211.50} \\ \hline
		\end{tabular}%
	}
	\caption{Resumen de costos estimados para el desarrollo y despliegue del prototipo}
	\label{tab:costos}
\end{table}

\noindent \textbf{Nota aclaratoria:}  
El presupuesto presentado considera un escenario de escalamiento comercial del prototipo, por lo que las cifras reflejan una estimación basada en tarifas de mercado, infraestructura de operación real y recursos humanos contratados de manera formal. Cabe resaltar que, para efectos académicos, los costos efectivos en la fase de prototipo fueron significativamente menores, basados en costos de oportunidad y recursos propios. La reserva para riesgos e imprevistos contempla un porcentaje adicional sobre el subtotal, como medida preventiva ante ajustes, retrasos o necesidades técnicas no previstas.


\newpage

\section{Informe de costo de despliegue del prototipo}

El despliegue del prototipo constituye una etapa crucial dentro del proyecto, ya que permite verificar la correcta operación de la aplicación, validar la experiencia de usuario y obtener retroalimentación que contribuya a la mejora continua de la solución. Esta fase busca asegurar que el prototipo sea funcional, accesible y que cumpla con los objetivos de accesibilidad e inclusión establecidos para su prueba en un entorno académico o controlado.

A continuación, se detallan los principales aspectos relacionados con esta etapa:

\subsection{Objetivos del despliegue}

Los objetivos principales del despliegue del prototipo son los siguientes:

\begin{itemize}
	\item Verificar la funcionalidad de la aplicación y su compatibilidad en dispositivos con Android 14.
	\item Realizar pruebas piloto para validar la comprensión y fluidez de las animaciones 3D en Lengua de Señas Mexicana (LSM).
	\item Obtener retroalimentación de usuarios potenciales y especialistas para identificar áreas de oportunidad y mejora en el prototipo.
\end{itemize}

\subsection{Cronograma del despliegue}

El cronograma de despliegue se fundamenta en la planificación estratégica de las actividades necesarias para la publicación, prueba y evaluación del prototipo. Estas actividades incluyen la ejecución de pruebas internas, la validación con usuarios, la recopilación sistemática de observaciones y la realización de ajustes finales basados en los resultados obtenidos.

Las actividades específicas y sus tiempos estimados se encuentran detallados en las tablas correspondientes al presupuesto por actividades y en la planificación del proyecto.

\subsection{Costos del despliegue}

La fase de despliegue implica costos asociados a las principales actividades de validación y ajuste del prototipo. A continuación se presenta el resumen de estos costos:

\begin{itemize}
	\item \textbf{Gestión de calidad y pruebas internas:} \$8,500.00 MXN.
	\item \textbf{Validación con usuarios (pruebas piloto y retroalimentación):} \$6,000.00 MXN.
	\item \textbf{Documentación y ajustes finales del prototipo:} \$5,250.00 MXN.
\end{itemize}

El costo total estimado para esta fase es de \textbf{\$19,750.00 MXN}.

\subsection{Métricas para medir el éxito}

Para evaluar la efectividad del despliegue del prototipo, se definirán los siguientes indicadores clave de desempeño (KPIs):

\begin{itemize}
	\item Porcentaje de pruebas funcionales exitosas (fluidez y comprensión de las animaciones 3D): superior al 90\%.
	\item Nivel de satisfacción o retroalimentación positiva de los usuarios evaluadores: superior al 85\%.
\end{itemize}

El cumplimiento de estas métricas permitirá validar que el prototipo alcanza los niveles de funcionalidad, accesibilidad e inclusión planteados como objetivos principales del proyecto.


\newpage
\section{Análisis exploratorio de costo de venta}

\subsection{Escenario de comercialización}

Aunque el presente trabajo terminal se centra en el desarrollo de un prototipo académico para la traducción de frases del español a la Lengua de Señas Mexicana (LSM), se considera pertinente incluir un análisis exploratorio sobre la viabilidad comercial de la solución. Este ejercicio tiene carácter preliminar y no forma parte del alcance operativo actual del proyecto; sin embargo, proporciona un marco de referencia para una eventual implementación en el mercado.

\subsection{Análisis de costos}

El siguiente cuadro presenta el desglose de los costos estimados para una posible fase de comercialización del prototipo, ajustados a valores razonables para un proyecto en etapa inicial:

\begin{table}[H]
	\centering
	\renewcommand{\arraystretch}{1.5}
	\setlength{\tabcolsep}{12pt}
	\resizebox{\textwidth}{!}{%
		\begin{tabular}{|l|r|}
			\hline
			\textbf{Concepto} & \textbf{Monto estimado (MXN)} \\ \hline
			Presupuesto total de actividades & \$243,050.00 \\ \hline
			Presupuesto total de recursos (desarrollo) & \$89,500.00 \\ \hline
			Presupuesto total de recursos (despliegue) & \$268,500.00 \\ \hline
			Compras no recurrentes & \$196,000.00 \\ \hline
			Sueldo del equipo (desarrollo) & Ajustado a horas hombre \\ \hline
			Sueldo del equipo (mantenimiento) & Ajustado a horas hombre \\ \hline
			Reserva para riesgos e imprevistos (15\%) & Recalculada sobre subtotal \\ \hline
			\textbf{Presupuesto total estimado del proyecto} & \textbf{\$705,829.20} \\ \hline
		\end{tabular}%
	}
	\caption{Proyección de costos para una posible fase comercial}
	\label{tab:costos_venta}
\end{table}

\noindent \textbf{Nota:} Los valores presentados consideran tarifas racionales para etapas iniciales, ajustadas al uso de recursos de nivel estudiante y asesorías externas especializadas.

\subsection{Determinación del precio de venta}

Para estimar un precio competitivo de suscripción mensual, se modela un escenario base con los siguientes supuestos:

\begin{itemize}
	\item Número de usuarios: \textbf{200}.
	\item Tiempo promedio de uso por usuario: \textbf{40 horas mensuales}.
	\item Periodo de recuperación de inversión: \textbf{24 meses}.
\end{itemize}

\subsubsection{Justificación del tiempo promedio de uso}

El tiempo promedio estimado de 40 horas mensuales por usuario se fundamenta en tendencias observadas en aplicaciones similares, equilibrando el comportamiento entre usuarios intensivos y ocasionales. Este parámetro permite una proyección razonable de costos e ingresos para la fase inicial de comercialización.

\noindent \textbf{Nota:} Estos valores son preliminares y podrían ajustarse conforme se avance en la validación del producto y del mercado objetivo.

\subsection{Cálculo del precio base mensual}

El costo mensual del proyecto es:

\[
\text{Costo mensual del proyecto} = \frac{\$705,829.20}{24} = \$29,409.55
\]

El costo mensual por usuario es:

\[
\text{Costo mensual por usuario} = \frac{\$29,409.55}{200} = \$147.05
\]

El costo por hora de uso es:

\[
\text{Costo por hora de uso} = \frac{\$147.05}{40} = \$3.68
\]

\subsection{Propuesta de tarifa adicional por exceso de uso}

En caso de que un usuario exceda las 40 horas de uso mensual, se propone una tarifa adicional del \textbf{10\%} sobre el costo base por hora:

\[
\text{Tarifa adicional por hora} = \$3.68 \times 1.10 = \$4.05
\]

\subsection{Propuesta de precio final}

Considerando un margen de ganancia del \textbf{20\%}, el precio base mensual por suscripción sería:

\[
\text{Precio base mensual} = \$147.05 \times 1.20 = \$176.46
\]

Redondeando al entero más cercano:

\[
\text{Precio base mensual redondeado} = \$176.00
\]

En caso de exceso de uso, el monto adicional a pagar se calcularía de la siguiente forma:

\[
\text{Total mensual con exceso de uso} = \$176.00 + (\text{Horas extra} \times \$4.05)
\]

\subsection{Ejemplo de ajuste por exceso de uso}

Si un usuario utiliza \textbf{50 horas} en un mes:

\[
\text{Horas extra} = 50 - 40 = 10
\]
\[
\text{Cargo adicional} = 10 \times \$4.05 = \$40.50
\]
\[
\text{Total mensual} = \$176.00 + \$40.50 = \$216.50
\]

\begin{flushleft}
	\textbf{Nota final:} Este análisis exploratorio tiene un carácter preliminar y busca evaluar la posible viabilidad económica del proyecto en una eventual fase de comercialización. Las cifras presentadas podrían modificarse conforme se avance en el desarrollo del producto y en el estudio de mercado correspondiente.
\end{flushleft}
