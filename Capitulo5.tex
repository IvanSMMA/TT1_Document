\chapter{Análisis financiero}

El presente capítulo tiene como objetivo evaluar los costos asociados al desarrollo, despliegue y potencial comercialización del prototipo de una aplicación móvil de traducción de frases del español a la Lengua de Señas Mexicana (LSM), mediante técnicas de Procesamiento de Lenguaje Natural (PLN) y modelado 3D.

Para ello, se plantea un análisis financiero estructurado en dos niveles: 

\begin{itemize}
	\item \textbf{Enfoque académico:} Incluye estimaciones de costos simbólicos o reducidos, considerando el uso de recursos propios, herramientas gratuitas, trabajo no remunerado por parte de los estudiantes y asesorías puntuales. Este enfoque representa fielmente la ejecución del proyecto dentro de un marco académico.
	
	\item \textbf{Enfoque exploratorio de comercialización:} Presenta una simulación de costos reales basada en tarifas aproximadas de mercado, incluyendo sueldos profesionales, licencias, infraestructura tecnológica y servicios externos. Este análisis permite anticipar los requerimientos financieros de una futura etapa de comercialización.
\end{itemize}

Ambas perspectivas se estructuran mediante dos métodos complementarios de análisis financiero:

\begin{itemize}
	\item \textbf{Estimación por recursos:} Identifica y valora los insumos materiales, humanos y tecnológicos necesarios para el desarrollo y operación del prototipo.
	\item \textbf{Estimación por actividades:} Desglosa las tareas específicas del proyecto, asignando tiempos estimados y costos unitarios para cada una.
\end{itemize}

Esta estructura metodológica permite identificar con precisión los recursos involucrados, calcular el presupuesto total estimado y analizar la factibilidad financiera del proyecto, tanto en su ejecución académica como en un escenario de implementación futura.




\section{Informe de costos de creación del prototipo}
Este informe presenta el análisis financiero asociado al desarrollo y creación del prototipo de la aplicación móvil de apoyo para la traducción de frases del español a la Lengua de Señas Mexicana (LSM) mediante Procesamiento de Lenguaje Natural (PLN) y modelado 3D. El objetivo es identificar y estructurar los costos necesarios para garantizar la viabilidad y factibilidad del proyecto en su etapa de creación.

\subsection{Presupuesto mediante recursos}

Dentro del análisis basado en recursos, el proceso de desarrollo se ha estructurado en tres etapas principales, cada una integrada por actividades específicas que permiten avanzar de manera ordenada, asegurando la calidad y funcionalidad del producto final. Las etapas contempladas son:

\begin{itemize}
	\item \textbf{Creación del prototipo}. 
	\item \textbf{Despliegue del prototipo}.
	\item \textbf{Evaluación financiera}. 
\end{itemize}

Cada una de estas etapas será detallada en los apartados siguientes, permitiendo una mejor comprensión del contenido.

\subsubsection{Creación del prototipo}

Esta primera etapa, abarca todas las actividades necesarias para el desarrollo inicial del prototipo. Estas actividades incluyen la definición de objetivos, la elaboración de especificaciones técnicas, el establecimiento de requisitos del prototipo, el diseño de las animaciones, el desarrollo del código base y la realización de pruebas funcionales. En esta fase también se implementan técnicas de Procesamiento de Lenguaje Natural (PLN) para la segmentación de frases en español, así como el modelado y animación 3D fluida para representar las señas en Lengua de Señas Mexicana (LSM).

\subsubsection{Despliegue del prototipo}

La segunda etapa, contempla todas las acciones necesarias para poner en funcionamiento la aplicación en dispositivos Android, específicamente optimizada para la versión Android 14. Esta fase incluye la publicación del prototipo, la configuración y adaptación técnica para su correcta operación, la elaboración de documentación técnica y guías de usuario, así como la ejecución de pruebas piloto con usuarios finales. A partir de estas pruebas se obtendrá retroalimentación valiosa que permitirá realizar los ajustes necesarios para optimizar la experiencia de uso y garantizar la comprensión efectiva de las animaciones.

\subsubsection{Evaluación financiera}

La tercera etapa, evaluación financiera, implica el análisis de los costos derivados de las etapas anteriores, así como la valoración de los beneficios esperados en términos de impacto social, accesibilidad y sostenibilidad del prototipo. Aunque el proyecto tiene fines académicos, esta evaluación contempla un análisis básico de los recursos humanos, tecnológicos y operativos involucrados, con el objetivo de valorar su viabilidad en escenarios de uso real o en un posible proceso de escalamiento futuro.

Este enfoque estructurado de desarrollo permite garantizar una planeación clara y eficiente, considerando los elementos técnicos, operativos y financieros necesarios para la ejecución del proyecto.

Por último, como medida preventiva ante posibles riesgos o imprevistos, se contempla una reserva de contingencia equivalente al 15\% del costo total estimado, conforme a lo definido en el análisis de riesgos del proyecto.


\subsection{Análisis de recursos}

\subsubsection{Costos de servicios}

El presente apartado detalla los costos asociados a los servicios necesarios para el desarrollo del prototipo, considerando un periodo estimado de ejecución de cuatro meses. Se incluyen los servicios con ambos enfoques, el académico y el exploratorio de comercialización.

\begin{table}[H]
	\centering
	\renewcommand{\arraystretch}{1.5}
	\setlength{\tabcolsep}{12pt}
	\resizebox{\textwidth}{!}{%
		\begin{tabular}{|l|r|r|r|}
			\hline
			\textbf{Recurso} & \textbf{Costo mensual (MXN \$)} & \textbf{Costo a 4 meses (MXN \$)} & \textbf{Costo a 1 año (MXN \$)} \\ \hline
			Electricidad e Internet (compartido entre integrantes) & \$500.00 & \$2,000.00 & \$6,000.00 \\ \hline
			Almacenamiento en la nube (Google Drive / GitHub) & \$300.00 & \$1,200.00 & \$3,600.00 \\ \hline
			Pruebas en dispositivos Android (emulador físico o virtual) & \$500.00 & \$2,000.00 & \$6,000.00 \\ \hline
			Reserva para servicios externos o pruebas adicionales & \$1,000.00 & \$4,000.00 & \$12,000.00 \\ \hline
			\textbf{Total estimado de servicios} & \textbf{\$2,300.00} & \textbf{\$9,200.00} & \textbf{\$27,600.00} \\ \hline
		\end{tabular}%
	}
	\caption{Costos estimados de servicios en el escenario académico durante la creación del prototipo.}
	\label{tab:costos_servicios}
\end{table}

\begin{table}[H]
	\centering
	\renewcommand{\arraystretch}{1.5}
	\setlength{\tabcolsep}{10pt}
	\resizebox{\textwidth}{!}{%
		\begin{tabular}{|l|r|r|r|}
			\hline
			\textbf{Recurso o servicio} & \textbf{Costo mensual (MXN \$)} & \textbf{Costo a 4 meses (MXN \$)} & \textbf{Costo a 1 año (MXN \$)} \\ \hline
			Electricidad e internet (oficina dedicada) & \$2,500.00 & \$10,000.00 & \$30,000.00 \\ \hline
			Renta de espacio de coworking (para 3 personas) & \$9,000.00 & \$36,000.00 & \$108,000.00 \\ \hline
			Suscripción a plataformas de desarrollo (GitHub Copilot, Blender Studio, etc.) & \$800.00 & \$3,200.00 & \$9,600.00 \\ \hline
			API de lenguaje natural (OpenAI GPT, DialogFlow) & \$1,000.00 & \$4,000.00 & \$12,000.00 \\ \hline
			Servicios de almacenamiento en la nube (Google Cloud, Firebase, etc.) & \$750.00 & \$3,000.00 & \$9,000.00 \\ \hline
			Servidor para aplicación móvil (Firebase/AWS) & \$1,200.00 & \$4,800.00 & \$14,400.00 \\ \hline
			Licencia de software para animación (Unity, MediaPipe, etc.) & \$1,500.00 & \$6,000.00 & \$18,000.00 \\ \hline
			Validación profesional de señas LSM (freelancer mensual) & \$5,000.00 & \$20,000.00 & \$60,000.00 \\ \hline
			Servicios de soporte técnico y mantenimiento & \$3,000.00 & \$12,000.00 & \$36,000.00 \\ \hline
			Marketing digital (redes sociales, web, SEO) & \$2,000.00 & \$8,000.00 & \$24,000.00 \\ \hline
			Traducción y adaptación de contenido LSM (consultoría externa) & \$4,000.00 & \$16,000.00 & \$48,000.00 \\ \hline
			\textbf{TOTAL ESTIMADO} & \textbf{\$30,750.00} & \textbf{\$123,000.00} & \textbf{\$369,000.00} \\ \hline
		\end{tabular}%
	}
	\caption{Costos estimados de servicios en un escenario comercial para la creación del prototipo.}
	\label{tab:costos_comercial}
\end{table}

\subsubsection{Compras no recurrentes}

En esta sección se presentan los costos estimados de artículos y adquisiciones necesarias para la creación del prototipo, clasificadas como compras no recurrentes. Estas compras representan inversiones únicas que no implican costos periódicos, pero que son fundamentales para el correcto desarrollo y prueba de la aplicación.

\begin{table}[H]
	\centering
	\renewcommand{\arraystretch}{1.5}
	\setlength{\tabcolsep}{8pt}
	\resizebox{\textwidth}{!}{%
		\begin{tabular}{|l|c|r|r|}
			\hline
			\textbf{Recurso} & \textbf{Unidades} & \textbf{Costo unitario (MXN \$)} & \textbf{Costo total (MXN \$)} \\ \hline
			Equipo de cómputo personal (propio de los integrantes) & 3 & -- & -- \\ \hline
			Dispositivo Android para pruebas físicas & 1 & -- & -- \\ \hline
			Capacitación online (cursos: Blender, MediaPipe, PLN) & 3 cursos & \$800.00 & \$2,400.00 \\ \hline
			Compra de modelos 3D o recursos gráficos (opcional) & 1 paquete & \$2,500.00 & \$2,500.00 \\ \hline
			\textbf{Total compras no recurrentes} & & & \textbf{\$4,900.00} \\ \hline
		\end{tabular}%
	}
	\caption{Costos estimados de artículos y compras no recurrentes en el escenario académico.}
	\label{tab:compras_no_recurrentes}
\end{table}

\noindent \textbf{Nota aclaratoria:}  
El equipo de cómputo utilizado corresponde a dispositivos personales de los integrantes del proyecto, por lo cual no se ha considerado un costo adicional en esta categoría. La compra de modelos 3D o recursos gráficos es considerada opcional y dependerá de la necesidad de complementar el material gráfico disponible de manera gratuita o de libre acceso.

\begin{table}[H]
	\centering
	\renewcommand{\arraystretch}{1.5}
	\setlength{\tabcolsep}{8pt}
	\resizebox{\textwidth}{!}{%
		\begin{tabular}{|l|c|r|r|}
			\hline
			\textbf{Recurso} & \textbf{Unidades} & \textbf{Costo unitario (MXN \$)} & \textbf{Costo total (MXN \$)} \\ \hline
			Equipo de cómputo profesional (para desarrollo y edición 3D) & 3 & \$25,000.00 & \$75,000.00 \\ \hline
			Dispositivos móviles de prueba (Android gama media/alta) & 2 & \$7,000.00 & \$14,000.00 \\ \hline
			Cámara y sensor de movimiento (para captura LSM y pruebas) & 1 & \$12,000.00 & \$12,000.00 \\ \hline
			Paquete profesional de modelos 3D con licencia comercial & 1 & \$10,000.00 & \$10,000.00 \\ \hline
			Tablet para testing y revisión de interfaz (Android 14) & 1 & \$6,000.00 & \$6,000.00 \\ \hline
			Cursos y certificaciones profesionales (PLN, UX, IA, Unity) & 4 & \$3,000.00 & \$12,000.00 \\ \hline
			Equipo de audio y grabación (para interfaz voz/signos) & 1 & \$4,500.00 & \$4,500.00 \\ \hline
			\textbf{Total compras no recurrentes} & & & \textbf{\$133,500.00} \\ \hline
		\end{tabular}%
	}
	\caption{Costos estimados de compras no recurrentes en un escenario de comercialización}
	\label{tab:compras_no_recurrentes_comercial}
\end{table}


\newpage
\subsubsection{Sueldos y asesorías}

El presente apartado presenta el costo estimado de los sueldos y asesorías considerados para la creación del prototipo en cuestión. Se adopta un enfoque dual que contempla tanto el escenario académico de ejecución como una proyección orientada a una futura etapa de comercialización del producto.

\begin{table}[H]
	\centering
	\renewcommand{\arraystretch}{1.5}
	\setlength{\tabcolsep}{10pt}
	\resizebox{\textwidth}{!}{%
		\begin{tabular}{|l|c|r|r|r|r|}
			\hline
			\textbf{Equipo} & \textbf{Cantidad} & \multicolumn{2}{c|}{\textbf{Desarrollo}} & \multicolumn{2}{c|}{\textbf{Mantenimiento / Ajustes}} \\ \hline
			\textbf{Tipo} & & \textbf{Mensual (MXN \$)} & \textbf{4 meses (MXN \$)} & \textbf{Mensual (MXN \$)} & \textbf{A un año (MXN)} \\ \hline
			Desarrollador (estudiante) & 3 & \$100.00 & \$1,200.00 (400 c/u) & -- & -- \\ \hline
			Asesoría en animación 3D (freelance) & 1 parcial & \$300.00 (por sesión)& \$1,200.00 (cuatro sesiones) & -- & -- \\ \hline
			Asesoría en LSM (validación de señas) & 1 parcial & \$200.00 (por sesión)& \$800.00 (cuatro sesiones) & -- & -- \\ \hline
			\textbf{Total} & \textbf{5} & \textbf{\$600.00} & \textbf{\$3,200.00} & -- & --\\ \hline
		\end{tabular}%
	}
	\caption{Costos estimados n el escenario académico de sueldos y asesorías durante el creación del prototipo.}
	\label{tab:sueldos_asesorias}
\end{table}

\noindent \textbf{Nota aclaratoria:}  
El sueldo ene ste caso, se considera un pago simbólico.  

\begin{table}[H]
	\centering
	\renewcommand{\arraystretch}{1.5}
	\setlength{\tabcolsep}{10pt}
	\resizebox{\textwidth}{!}{%
		\begin{tabular}{|l|c|r|r|r|r|}
			\hline
			\textbf{Rol / Servicio} & \textbf{Cantidad} & \multicolumn{2}{c|}{\textbf{Desarrollo}} & \multicolumn{2}{c|}{\textbf{Mantenimiento / Ajustes}} \\ \hline
			\textbf{Tipo} & & \textbf{Mensual (MXN \$)} & \textbf{4 meses (MXN \$)} & \textbf{Mensual (MXN \$)} & \textbf{A un año (MXN \$)} \\ \hline
			Desarrollador backend / frontend & 2 & \$28,000.00 & \$224,000.00 & \$16,000.00 & \$192,000.00 \\ \hline
			Especialista en PLN & 1 & \$30,000.00 & \$120,000.00 & \$15,000.00 & \$180,000.00 \\ \hline
			Diseñador 3D / animador (Unity / Blender) & 1 & \$25,000.00 & \$100,000.00 & \$12,000.00 & \$144,000.00 \\ \hline
			Asesoría profesional en LSM & 1 parcial & \$10,000.00 & \$40,000.00 & \$5,000.00 & \$60,000.00 \\ \hline
			Soporte técnico / DevOps & 1 & \$15,000.00 & \$60,000.00 & \$10,000.00 & \$120,000.00 \\ \hline
			\textbf{Total estimado} & \textbf{6} & \textbf{\$108,000.00} & \textbf{\$544,000.00} & \textbf{\$58,000.00} & \textbf{\$696,000.00} \\ \hline
		\end{tabular}%
	}
	\caption{Costos estimados de sueldos y asesorías en un entorno comercial durante el creación del prototipo.}
	\label{tab:sueldos_comercial}
\end{table}


\noindent \textbf{Nota aclaratoria:}  
La columna de “Mantenimiento / Ajustes” se incluye únicamente como referencia para una posible fase futura de operación continua, en caso de que el prototipo evolucione hacia un producto comercial o requiera soporte extendido. 


\subsection{Presupuesto mediante actividades}

En el siguiente apartado, se presenta el presupuesto detallado por actividades, donde se especifica el tiempo estimado y el costo asociado a cada tarea del proyecto.
El desglose completo de actividades se encuentra disponible en el siguiente enlace:

\begin{flushleft} \href{ruta_del_anexo_o_enlace}{\textbf{Anexo 1: Desglose de Actividades, Horas Estimadas y Costos Simulados}} \end{flushleft}

Cabe señalar que los costos presentados son únicamente proyecciones hipotéticas, utilizadas con el propósito de ilustrar la posible viabilidad financiera y escalabilidad comercial del proyecto. No representan un presupuesto real para la ejecución académica del presente trabajo de titulación.



\subsection{Resumen de costos estimados}

El presente apartado presenta el resumen de los costos estimados para el desarrollo y posible despliegue comercial del prototipo de la aplicación móvil de traducción de frases del español a la Lengua de Señas Mexicana (LSM) mediante animaciones 3D. El análisis incluye el presupuesto derivado de las actividades de desarrollo, los recursos necesarios para las fases de creación y despliegue, las compras no recurrentes, los sueldos del equipo de trabajo y una reserva contemplada para cubrir posibles riesgos o imprevistos durante la implementación.

\begin{table}[H]
	\centering
	\renewcommand{\arraystretch}{1.5}
	\setlength{\tabcolsep}{12pt}
	\resizebox{\textwidth}{!}{%
		\begin{tabular}{|l|r|}
			\hline
			\textbf{Concepto} & \textbf{Monto estimado (MXN \$)} \\ \hline
			Presupuesto total de actividades & \$243,050.00 \\ \hline
			Presupuesto total de recursos (Desarrollo) & \$89,500.00 \\ \hline
			Presupuesto total de recursos (Despliegue) & \$268,500.00 \\ \hline
			Compras no recurrentes & \$196,000.00 \\ \hline
			Sueldo del equipo (Desarrollo) & \$1,500,000.00 \\ \hline
			Sueldo del equipo (Mantenimiento) & \$408,000.00 \\ \hline
			Reserva para riesgos e imprevistos & \$36,457.50 \\ \hline
			\textbf{Presupuesto total estimado del proyecto} & \textbf{\$2,742,211.50} \\ \hline
		\end{tabular}%
	}
	\caption{Resumen de costos estimados para el desarrollo y despliegue del prototipo}
	\label{tab:costos}
\end{table}

\noindent \textbf{Nota aclaratoria:}  
El presupuesto presentado considera un escenario de escalamiento comercial del prototipo, por lo que las cifras reflejan una estimación basada en tarifas de mercado, infraestructura de operación real y recursos humanos contratados de manera formal. Cabe resaltar que, para efectos académicos, los costos efectivos en la fase de prototipo fueron significativamente menores, basados en costos de oportunidad y recursos propios. La reserva para riesgos e imprevistos contempla un porcentaje adicional sobre el subtotal, como medida preventiva ante ajustes, retrasos o necesidades técnicas no previstas.


\newpage

\section{Informe de costo de despliegue del prototipo}

El despliegue del prototipo constituye una etapa crucial dentro del proyecto, ya que permite verificar la correcta operación de la aplicación, validar la experiencia de usuario y obtener retroalimentación que contribuya a la mejora continua de la solución. Esta fase busca asegurar que el prototipo sea funcional, accesible y que cumpla con los objetivos de accesibilidad e inclusión establecidos para su prueba en un entorno académico o controlado.

A continuación, se detallan los principales aspectos relacionados con esta etapa:

\subsection{Objetivos del despliegue}

Los objetivos principales del despliegue del prototipo son los siguientes:

\begin{itemize}
	\item Verificar la funcionalidad de la aplicación y su compatibilidad en dispositivos con Android 14.
	\item Realizar pruebas piloto para validar la comprensión y fluidez de las animaciones 3D en Lengua de Señas Mexicana (LSM).
	\item Obtener retroalimentación de usuarios potenciales y especialistas para identificar áreas de oportunidad y mejora en el prototipo.
\end{itemize}

\subsection{Cronograma del despliegue}

El cronograma de despliegue se fundamenta en la planificación estratégica de las actividades necesarias para la publicación, prueba y evaluación del prototipo. Estas actividades incluyen la ejecución de pruebas internas, la validación con usuarios, la recopilación sistemática de observaciones y la realización de ajustes finales basados en los resultados obtenidos.

Las actividades específicas y sus tiempos estimados se encuentran detallados en loa anexo [INSERTAR ANEXOS].

\subsection{Costos del despliegue}

La fase de despliegue representa una etapa crítica del proyecto, ya que permite validar el funcionamiento del prototipo, ajustar detalles técnicos y garantizar que cumpla con los criterios de calidad y usabilidad esperados.

Para ofrecer una visión integral, esta sección presenta dos enfoques diferenciados en la estimación de costos:

\begin{itemize}
	\item \textbf{Enfoque académico:} Se consideran costos simbólicos asociados a la ejecución del proyecto en un entorno universitario, utilizando recursos propios y trabajo colaborativo de los estudiantes. Este enfoque refleja la realidad operativa del desarrollo del prototipo en su contexto actual.
	
	\item \textbf{Enfoque comercial (exploratorio):} Se presenta una estimación realista basada en tarifas de mercado, que contempla los costos necesarios para validar, documentar y lanzar el prototipo como un producto funcional en un entorno profesional, incluyendo personal especializado, infraestructura y herramientas de control de calidad.
\end{itemize}

\vspace{1em}
\noindent\textbf{Resumen de costos (despliegue) – Enfoque académico:}
\begin{itemize}
	\item \textbf{Gestión de calidad y pruebas internas:} \$2,400.00 MXN (simbólico).
	\item \textbf{Validación con usuarios (pruebas piloto y retroalimentación):} \$5,000.00 MXN (simbólico).
	\item \textbf{Documentación y ajustes finales del prototipo:} \$5,000.00 MXN (simbólico).
\end{itemize}

\noindent\textbf{Costo total estimado (académico):} \textbf{\$12,400.00 MXN}

\vspace{1em}
\noindent\textbf{Resumen de costos (despliegue) – Enfoque comercial:} (según tarifas profesionales conservadoras)

\begin{itemize}
	\item \textbf{Gestión de calidad (definición de estándares, métricas y control):} \$6,520.00 MXN.
	\item \textbf{Pruebas técnicas e integración (unitarias, con usuarios, corrección de errores):} \$15,000.00 MXN.
	\item \textbf{Preparación de entorno de producción y despliegue técnico:} \$16,000.00 MXN.
\end{itemize}

\noindent\textbf{Costo total estimado (comercial):} \textbf{\$37,520.00 MXN}

\vspace{1em}
Esta doble perspectiva permite comparar la ejecución del despliegue del prototipo en un entorno académico y en un escenario comercial futuro, facilitando la toma de decisiones sobre su escalabilidad y viabilidad económica.


\subsection{Métricas para medir el éxito}

Para evaluar la efectividad del despliegue del prototipo, se definirán los siguientes indicadores clave de desempeño (KPIs):

\begin{itemize}
	\item Porcentaje de pruebas funcionales exitosas (fluidez y comprensión de las animaciones 3D): superior al 90\%.
	\item Nivel de satisfacción o retroalimentación positiva de los usuarios evaluadores: superior al 85\%.
\end{itemize}

El cumplimiento de estas métricas permitirá validar que el prototipo alcanza los niveles de funcionalidad, accesibilidad e inclusión planteados como objetivos principales del proyecto.


\newpage
\section{Análisis exploratorio de costo de venta}

\subsection{Escenario de comercialización}

Este apartado presenta una proyección financiera elaborada bajo un enfoque comercial, con el objetivo de estimar los costos y condiciones necesarias para escalar el prototipo de aplicación a un producto funcional en el mercado. Aunque el presente trabajo terminal tiene un carácter académico, esta simulación permite anticipar los requerimientos económicos y técnicos para una eventual implementación comercial.

\subsection{Análisis de costos}

A continuación, se presenta el desglose de costos estimados para una fase inicial de comercialización del prototipo, considerando sueldos profesionales, infraestructura tecnológica, servicios especializados, mantenimiento y una reserva para riesgos:

\begin{table}[H]
	\centering
	\renewcommand{\arraystretch}{1.5}
	\setlength{\tabcolsep}{12pt}
	\resizebox{\textwidth}{!}{%
		\begin{tabular}{|l|r|}
			\hline
			\textbf{Concepto} & \textbf{Monto estimado (MXN \$)} \\ \hline
			Presupuesto total de actividades & \$179,280.00 \\ \hline
			Presupuesto total de recursos (desarrollo) & \$123,000.00 \\ \hline
			Presupuesto total de recursos (despliegue) & \$369,000.00 \\ \hline
			Compras no recurrentes & \$133,500.00 \\ \hline
			Sueldo del equipo (desarrollo) & \$544,000.00 \\ \hline
			Sueldo del equipo (mantenimiento) & \$696,000.00 \\ \hline
			Reserva para riesgos e imprevistos (15\%) & \$26,892.00 \\ \hline
			\textbf{Presupuesto total estimado del proyecto} & \textbf{\$2,072,346.00} \\ \hline
		\end{tabular}%
	}
	\caption{Proyección de costos para una posible fase comercial}
	\label{tab:costos_venta}
\end{table}

\noindent \textbf{Nota:} Esta estimación considera valores realistas para una implementación comercial, incluyendo tarifas conservadoras por perfil profesional, recursos tecnológicos y una reserva destinada a cubrir contingencias operativas o técnicas.

\subsection{Determinación del precio de venta}

Para calcular un precio base de suscripción mensual, se considera el siguiente escenario:

\begin{itemize}
	\item Número de usuarios esperados: \textbf{200}.
	\item Tiempo promedio de uso por usuario: \textbf{40 horas mensuales}.
	\item Periodo de recuperación de inversión: \textbf{24 meses}.
\end{itemize}

\subsection{Cálculo del precio base mensual}

\[
\text{Costo mensual del proyecto} = \frac{\$2,072,346.00}{24} = \$86,347.75
\]

\[
\text{Costo mensual por usuario} = \frac{\$86,347.75}{200} = \$431.74
\]

\[
\text{Costo por hora de uso} = \frac{\$431.74}{40} = \$10.79
\]

\subsection{Propuesta de tarifa adicional por exceso de uso}

Si un usuario supera el límite mensual de 40 horas, se aplicará una tarifa adicional del \textbf{10\%} sobre el costo base por hora:

\[
\text{Tarifa adicional por hora} = \$10.79 \times 1.10 = \$11.87
\]

\subsection{Propuesta de precio final}

Considerando un margen de ganancia del \textbf{20\%}, el precio sugerido de suscripción mensual sería:

\[
\text{Precio base mensual} = \$431.74 \times 1.20 = \$518.09
\]

Redondeando al entero más cercano:

\[
\text{Precio base mensual redondeado} = \$518.00
\]

\noindent En caso de exceso de uso:

\[
\text{Total mensual con exceso de uso} = \$518.00 + (\text{Horas extra} \times \$11.87)
\]

\subsection{Ejemplo de ajuste por exceso de uso}

Si un usuario utiliza \textbf{50 horas} en un mes:

\[
\text{Horas extra} = 50 - 40 = 10
\]
\[
\text{Cargo adicional} = 10 \times \$11.87 = \$118.70
\]
\[
\text{Total mensual} = \$518.00 + \$118.70 = \$636.70
\]

\begin{flushleft}
	\textbf{Nota final:} Esta simulación financiera tiene carácter exploratorio y busca establecer un marco inicial para la evaluación económica del producto. Las cifras deberán ser validadas y refinadas mediante estudios de mercado y pruebas piloto antes de su implementación real.
\end{flushleft}

