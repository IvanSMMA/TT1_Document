\chapter{Análisis financiero}
\section{Introducción}
En la actualidad, la inclusión y la accesibilidad representan pilares fundamentales para el desarrollo de tecnologías que permitan garantizar la comunicación efectiva entre todas las personas, sin importar sus capacidades físicas o sensoriales. Dentro de este contexto, la Inteligencia Artificial (IA) y el Procesamiento de Lenguaje Natural (PLN) ofrecen herramientas innovadoras para enfrentar desafíos relacionados con la accesibilidad lingüística, especialmente en comunidades que utilizan lenguas de señas.

El presente proyecto propone el diseño, desarrollo e implementación de un prototipo de aplicación móvil que permite la traducción de frases del español a la Lengua de Señas Mexicana (LSM), utilizando animaciones 3D fluidas y naturales. El objetivo es facilitar la interacción entre personas oyentes y personas con discapacidad auditiva, promoviendo la inclusión y el acceso a la comunicación en situaciones cotidianas, especialmente en saludos y escenarios de emergencia.

Mediante el uso de técnicas de PLN para el procesamiento de las frases en español, así como el modelado 3D apoyado en la biblioteca MediaPipe, el prototipo busca ofrecer una solución accesible, eficiente y compatible con la última versión de Android, considerando el amplio uso de este sistema operativo en México. Esta herramienta está pensada como un recurso complementario para apoyar la inclusión, contribuyendo a reducir las barreras de comunicación que enfrenta la comunidad sorda.

\section{Objetivos del producto}
\begin{itemize}
	\item Desarrollar un prototipo de aplicación móvil capaz de traducir frases del español a LSM mediante animaciones 3D continuas y comprensibles.
	\item Facilitar la comunicación en situaciones de emergencia y en interacciones cotidianas mediante un conjunto predefinido de frases y dactilología para casos no contemplados.
	\item Implementar técnicas de Procesamiento de Lenguaje Natural para mejorar el reconocimiento y adaptación de las frases al contexto de la traducción.
	\item Garantizar la compatibilidad y el rendimiento del prototipo en dispositivos con Android 14, optimizando la experiencia del usuario final.
\end{itemize}

\section{Descripción del producto}
El producto consiste en una aplicación móvil orientada a traducir frases escritas en español hacia representaciones animadas en Lengua de Señas Mexicana (LSM), utilizando modelos 3D que permitan una visualización fluida, continua y natural de las señas. En caso de no contar con una traducción directa para alguna frase, la aplicación ofrecerá la opción de dactilología para asegurar siempre una respuesta adecuada.

\subsection{Características principales}
\begin{itemize}
	\item \textbf{Animaciones 3D fluidas y naturales:} Modelado de las señas de LSM con especial atención a la continuidad de los movimientos, evitando transiciones abruptas o cortes entre gestos.
	\item \textbf{Procesamiento de Lenguaje Natural (PLN):} Análisis y segmentación de las frases en español para adaptar las traducciones al contexto, facilitando la correcta interpretación de las oraciones.
	\item \textbf{Compatibilidad con Android 14:} Desarrollo y optimización del prototipo para dispositivos con la versión más reciente del sistema operativo Android, asegurando estabilidad y rendimiento.
	\item \textbf{Dactilología como recurso complementario:} Deletreo mediante el alfabeto manual en casos donde la frase no esté predefinida, garantizando siempre una alternativa de comunicación.
	\item \textbf{Interfaz simple e intuitiva:} Diseño centrado en la experiencia del usuario, especialmente pensado para la facilidad de uso por parte de personas oyentes que desean comunicarse con personas sordas.
\end{itemize}

\newpage

\section*{Informe de costo de creación del producto}

Este informe presenta el análisis financiero para el desarrollo del prototipo de la aplicación móvil de traducción de frases del español a Lengua de Señas Mexicana (LSM) mediante animaciones 3D. Para estructurar adecuadamente los costos del proyecto, se contemplan dos enfoques principales: la presupuestación mediante la cotización de recursos y la estimación de costos por actividades.

\subsection*{Presupuesto mediante recursos}

Con el objetivo de garantizar una planificación adecuada para la creación del prototipo, el proceso se ha dividido en tres etapas principales. Cada etapa abarca actividades específicas que permiten avanzar de manera ordenada en el desarrollo, asegurando la calidad y funcionalidad del producto final. Las etapas consideradas son:

\begin{itemize}
	\item Creación del producto.
	\item Despliegue del producto.
	\item Evaluación financiera.
\end{itemize}

\subsubsection*{Creación del producto}

La primera etapa, \textbf{Creación del producto}, comprende todas las actividades necesarias para el desarrollo del prototipo, desde la definición de los objetivos, especificaciones técnicas y requisitos del sistema, hasta el diseño de las animaciones, el desarrollo del código y las pruebas funcionales. En esta fase se considera la implementación del Procesamiento de Lenguaje Natural (PLN) para la segmentación de las frases en español, así como el modelado y la animación 3D fluida para representar las señas en LSM.

\subsubsection*{Despliegue del producto}

En la etapa de \textbf{Despliegue del producto}, se contemplan las acciones necesarias para poner en funcionamiento la aplicación en dispositivos Android. Esto incluye la publicación del prototipo, la configuración para su funcionamiento en Android 14, así como la elaboración de documentación técnica y guías de usuario. También se considera la realización de pruebas piloto con usuarios para validar la comprensión de las animaciones y obtener retroalimentación que permita realizar mejoras.

\subsubsection*{Evaluación financiera}

La etapa de \textbf{Evaluación financiera} implica el análisis de los costos involucrados en las etapas anteriores y la valoración de los beneficios esperados en términos de impacto social y accesibilidad. Aunque el proyecto tiene fines académicos, se contempla un análisis básico de costos, incluyendo recursos humanos, tecnológicos y operativos, para evaluar la viabilidad de la solución en escenarios de uso real o potencial escalamiento.

Este enfoque estructurado permite asegurar una planificación clara y eficiente, contemplando los elementos técnicos, operativos y financieros que intervienen en el desarrollo del prototipo.

A partir del análisis de presupuesto mediante la cotización de recursos, se realiza además la estimación de costos por actividades. Cada etapa del proyecto se desglosa en actividades y tareas, a las cuales se les asigna un tiempo estimado en horas y un costo por hora. La suma total de estas tareas representa el costo de desarrollo del prototipo.

Finalmente, se considera una reserva de contingencia del 15\% sobre el costo total estimado, como previsión ante posibles riesgos o imprevistos identificados en el análisis de riesgos.

% Análisis de Presupuesto
\newpage

\section{Análisis de recursos}

\subsection{Costos de servicios}

El cálculo de los servicios se realizó considerando un periodo estimado de desarrollo de cuatro meses.

\begin{table}[h!]
	\centering
	\renewcommand{\arraystretch}{1.5}
	\setlength{\tabcolsep}{10pt}
	\begin{tabular}{|l|r|r|r|}
		\hline
		\textbf{Recurso}                                    & \textbf{Costo mensual} & \textbf{A 4 meses} & \textbf{A un año} \\ \hline
		Electricidad e Internet (compartido entre integrantes)  & \$500.00              & \$2,000.00         & \$6,000.00        \\ \hline
		Almacenamiento en la nube (Google Drive / GitHub)       & \$300.00              & \$1,200.00         & \$3,600.00        \\ \hline
		Pruebas en dispositivos Android (emulador físico o virtual) & \$500.00          & \$2,000.00         & \$6,000.00        \\ \hline
		Reserva para servicios externos o pruebas adicionales    & \$1,000.00            & \$4,000.00         & \$12,000.00       \\ \hline
		\textbf{TOTAL SERVICIOS}                               & \textbf{\$2,300.00}   & \textbf{\$9,200.00} & \textbf{\$27,600.00} \\ \hline
	\end{tabular}
	\caption{Costos estimados de servicios ajustados al proyecto académico}
\end{table}



\subsection{Compras no recurrentes}

A continuación, se presenta la tabla con los costos estimados de artículos o compras no recurrentes necesarios para el desarrollo del proyecto.

\begin{table}[h!]
	\centering
	\renewcommand{\arraystretch}{1.5}
	\setlength{\tabcolsep}{8pt}
	\begin{tabular}{|l|c|r|r|}
		\hline
		\textbf{Recurso}                                    & \textbf{Unidades} & \textbf{Costo unitario} & \textbf{Costo total} \\ \hline
		Equipo de cómputo personal (propio de los integrantes) & 3                & -                      & -                   \\ \hline
		Dispositivo Android para pruebas físicas             & 1                & \$5,000.00             & \$5,000.00          \\ \hline
		Capacitación online (cursos: Blender, MediaPipe, PLN) & 3 cursos        & \$800.00               & \$2,400.00          \\ \hline
		Compra de modelos 3D o recursos gráficos (opcional)   & 1 paquete        & \$2,500.00             & \$2,500.00          \\ \hline
		\textbf{TOTAL COMPRAS NO RECURRENTES}                &                  &                        & \textbf{\$9,900.00} \\ \hline
	\end{tabular}
	\caption{Artículos y compras no recurrentes}
\end{table}

\newpage
\subsection{Sueldos}

El costo estimado de sueldos se calculó tomando como referencia el tiempo dedicado por los integrantes del equipo de desarrollo (3 estudiantes) y las posibles asesorías externas requeridas (validación de señas LSM y animación 3D). Este cálculo no representa un salario formal, sino una estimación del costo de oportunidad y la valoración del tiempo invertido, para efectos de análisis financiero del proyecto.

\begin{table}[h!]
	\centering
	\renewcommand{\arraystretch}{1.5}
	\setlength{\tabcolsep}{10pt}
	\resizebox{\textwidth}{!}{%
		\begin{tabular}{|l|c|r|r|r|r|}
			\hline
			\textbf{Equipo}                     & \textbf{Cantidad} & \multicolumn{2}{c|}{\textbf{Desarrollo}} & \multicolumn{2}{c|}{\textbf{Mantenimiento / Ajustes}} \\ \hline
			\textbf{Tipo}                       &                   & \textbf{Mensual} & \textbf{4 meses}     & \textbf{Mensual}        & \textbf{A un año}         \\ \hline
			Desarrollador (estudiante)          & 3                 & \$8,000.00        & \$96,000.00 (32,000 c/u) & \$4,000.00 (referencial)     & \$48,000.00 (escenario futuro)   \\ \hline
			Asesoría en animación 3D (freelance) & 1 parcial         & \$3,000.00        & \$12,000.00           & -                     & -                       \\ \hline
			Asesoría en LSM (validación señas)   & 1 parcial         & \$2,500.00        & \$2,500.00 (por sesiones) & -                  & -                       \\ \hline
			\textbf{TOTAL}                      & \textbf{5}         & \textbf{\$13,500.00} & \textbf{\$110,500.00} & \textbf{\$4,000.00}         & \textbf{\$48,000.00}               \\ \hline
		\end{tabular}%
	}
	\caption{Costos estimados de sueldos y asesorías para el desarrollo del prototipo}
\end{table}

\noindent \textbf{Nota aclaratoria:}  
La columna de “Mantenimiento / Ajustes” se incluye únicamente como referencia para una posible fase futura de operación continua, en caso de que el prototipo evolucione hacia un producto comercial o se requiera soporte extendido. Para efectos de este trabajo terminal, dicha fase no será implementada, por lo que los costos de mantenimiento no forman parte del presupuesto activo del proyecto.

\subsection{Resumen de costos estimados del proyecto}

El siguiente cuadro presenta el resumen de los costos estimados para el desarrollo del prototipo de la aplicación móvil de traducción de frases del español a Lengua de Señas Mexicana (LSM) mediante animaciones 3D. Este análisis incluye los rubros de sueldos y asesorías, compras no recurrentes, servicios asociados y una reserva de contingencia para cubrir posibles riesgos o imprevistos durante el proceso de desarrollo.

\begin{table}[h!]
	\centering
	\renewcommand{\arraystretch}{1.5}
	\setlength{\tabcolsep}{12pt}
	\begin{tabular}{|l|r|}
		\hline
		\textbf{Concepto}                        & \textbf{Costo total (MXN)} \\ \hline
		Sueldos y asesorías                      & \$110,500.00              \\ \hline
		Compras no recurrentes                   & \$9,900.00                \\ \hline
		Servicios                                & \$9,200.00                \\ \hline
		\textbf{Subtotal}                        & \textbf{\$129,600.00}     \\ \hline
		Reserva de contingencia (15\% del subtotal) & \$19,440.00            \\ \hline
		\textbf{Total estimado del proyecto}     & \textbf{\$149,040.00}     \\ \hline
	\end{tabular}
	\caption{Resumen de costos estimados para el desarrollo del prototipo}
\end{table}

Este resumen permite visualizar de manera clara y estructurada la distribución de los recursos necesarios para la concreción del prototipo. Cabe resaltar que, debido a la naturaleza académica del proyecto, las cifras aquí presentadas corresponden a estimaciones de costos de oportunidad y esfuerzo invertido por los integrantes, sin representar sueldos formales ni un esquema de operación comercial. La reserva de contingencia se incluye como una medida preventiva ante posibles ajustes, retrasos o necesidades técnicas adicionales que puedan surgir durante el desarrollo.

\section{Presupuesto mediante actividades}

El desglose de presupuesto por actividades se encuentra en el siguiente enlace: \\
\texttt{\url{Insertar link}}.

A continuación, se presentan las fases del proyecto representadas visualmente:

\newpage

\section{Informe de costo de despliegue del prototipo}

\subsection{Informe de despliegue del prototipo}

El despliegue del prototipo representa una etapa crucial dentro del proyecto, ya que permite verificar la correcta operación de la aplicación, validar la experiencia del usuario y obtener retroalimentación que contribuya a la mejora continua de la solución. Esta fase asegura que el prototipo esté funcional, accesible y que cumpla con los objetivos establecidos para la prueba en el entorno académico o controlado.

A continuación, se describe el detalle de esta etapa:

\subsection{Objetivos del despliegue}

\begin{itemize}
	\item Verificar la funcionalidad del prototipo y su compatibilidad en dispositivos con Android 14.
	\item Facilitar la realización de pruebas piloto para validar la comprensión y fluidez de las animaciones 3D de la Lengua de Señas Mexicana (LSM).
	\item Obtener retroalimentación por parte de usuarios potenciales y especialistas, con el fin de identificar oportunidades de mejora.
\end{itemize}

\subsection{Cronograma del despliegue}

El cronograma del despliegue se fundamenta en la distribución estratégica de las actividades necesarias para la publicación, prueba y evaluación del prototipo. Estas actividades consideran la ejecución de pruebas internas, la validación con usuarios, la recopilación de observaciones y los ajustes finales con base en los resultados obtenidos.

Las actividades específicas y la estimación de tiempos para cada una de ellas se encuentran reflejadas en las tablas de presupuesto por actividades y su planificación respectiva.

\subsection{Costos del despliegue}

A continuación, se presenta un resumen de los costos asociados a las principales actividades de la fase de despliegue del prototipo:

\begin{itemize}
	\item \textbf{Gestión de calidad y pruebas internas:} \$8,500.00.
	\item \textbf{Validación con usuarios (pruebas piloto y retroalimentación):} \$6,000.00.
	\item \textbf{Documentación y ajustes finales del prototipo:} \$5,250.00.
\end{itemize}

El costo total estimado para la fase de despliegue es de \textbf{\$19,750.00 MXN}.

\subsection{Métricas para medir el éxito}

Para evaluar la efectividad del despliegue del prototipo, se utilizarán los siguientes indicadores clave de desempeño (KPIs):

\begin{itemize}
	\item Porcentaje de pruebas funcionales exitosas (fluidez y comprensión de las animaciones 3D): \textgreater 90\%.
	\item Nivel de satisfacción o retroalimentación positiva por parte de usuarios evaluadores: \textgreater 85\%.
\end{itemize}

Este informe de despliegue asegura una planificación organizada y enfocada en la calidad y funcionalidad del prototipo, maximizando las posibilidades de cumplir con los objetivos de accesibilidad e inclusión planteados en el proyecto.

\newpage

\section{Análisis exploratorio de costo de venta (proyección futura)}

\subsection{Escenario de comercialización y costo de venta}

Si bien el presente trabajo terminal se enfoca en el desarrollo de un prototipo académico para la traducción de frases del español a Lengua de Señas Mexicana (LSM), se considera pertinente incluir un análisis exploratorio sobre la posible viabilidad comercial de la solución. Este ejercicio tiene carácter preliminar y no forma parte del alcance operativo actual del proyecto; sin embargo, permite proyectar un modelo de costos y precios que podría utilizarse en caso de que el prototipo evolucione hacia una implementación real en el mercado.

\subsection{Análisis de costos}

El desglose de los costos estimados para una eventual fase de comercialización, ajustado a valores racionales para un proyecto en su etapa inicial, es el siguiente:

\begin{table}[h!]
	\centering
	\renewcommand{\arraystretch}{1.5}
	\setlength{\tabcolsep}{12pt}
	\begin{tabular}{|l|r|}
		\hline
		\textbf{Concepto}                                  & \textbf{Monto estimado (MXN)} \\ \hline
		Presupuesto total de actividades                  & \$243,050.00                  \\ \hline
		Presupuesto total de recursos (desarrollo)        & \$89,500.00                   \\ \hline
		Presupuesto total de recursos (despliegue)        & \$268,500.00                  \\ \hline
		Compras no recurrentes                            & \$196,000.00                  \\ \hline
		Sueldo equipo (desarrollo)                        & Ajustado a horas hombre       \\ \hline
		Sueldo equipo (mantenimiento)                     & Ajustado a horas hombre       \\ \hline
		Reserva para riesgos e imprevistos (15\%)         & Recalculada sobre subtotal    \\ \hline
		\textbf{PRESUPUESTO TOTAL ESTIMADO DEL PROYECTO} & \textbf{\$705,829.20}         \\ \hline
	\end{tabular}
	\caption{Proyección de costos para una posible fase comercial (ajustada a prototipo)}
	\label{tab:costos_venta}
\end{table}

\textbf{Nota:} Los valores presentados se basan en los ajustes de costos por hora y horas estimadas según la versión optimizada del presupuesto, con tarifas racionales para etapas iniciales de comercialización y considerando el uso de recursos a nivel estudiante y asesorías externas puntuales.

\subsection{Determinación del precio de venta}

Para estimar un precio de suscripción mensual competitivo, se plantea un escenario hipotético con los siguientes supuestos:
\begin{itemize}
	\item Número de usuarios: \textbf{200}.
	\item Tiempo promedio de uso por usuario: \textbf{40 horas al mes}.
	\item Periodo de recuperación de la inversión: \textbf{24 meses}.
\end{itemize}

\subsubsection{Justificación del tiempo promedio de uso}

El tiempo promedio de 40 horas mensuales por usuario se determinó a partir del análisis de tendencias en aplicaciones similares, donde este nivel de uso representa un equilibrio razonable entre usuarios intensivos y ocasionales. Este valor permite mantener la sostenibilidad financiera y ajustar las tarifas en función del consumo real, sin penalizar a los usuarios regulares.

\textbf{Nota:} Este promedio es una referencia preliminar y podría modificarse si se identifican patrones de uso distintos en fases futuras de implementación.

\subsection{Cálculo del precio base mensual}

El costo mensual del proyecto se calcula de la siguiente manera:

\[
\text{Costo mensual del proyecto} = \frac{\$705,829.20}{24} = \$29,409.55
\]

El costo mensual por usuario es:

\[
\text{Costo mensual por usuario} = \frac{\$29,409.55}{200} = \$147.05
\]

El costo por hora de uso es:

\[
\text{Costo por hora de uso} = \frac{\$147.05}{40} = \$3.68
\]

\subsection{Propuesta de tarifa adicional por Exceso de uso}

En caso de que un usuario exceda las 40 horas mensuales, se propone aplicar una tarifa adicional del \textbf{10\%} sobre el costo por hora para cada hora extra utilizada:

\[
\text{Tarifa adicional por hora} = \$3.68 \times 1.10 = \$4.05
\]

\subsection{Propuesta de precio final}

Incorporando un margen de ganancia del \textbf{20\%}, el precio base mensual por suscripción es:

\[
\text{Precio base mensual} = \$147.05 \times 1.20 = \$176.46
\]

Redondeando al entero más cercano:

\[
\text{Precio base mensual redondeado} = \$176.00
\]

Si el usuario excede las 40 horas de uso mensual, se aplicaría la tarifa adicional correspondiente:

\[
\text{Total mensual con exceso de uso} = \$176.00 + (\text{Horas extra} \times \$4.05)
\]

\subsection{Ejemplo de ajuste por exceso de uso}

Si un usuario utiliza \textbf{50 horas} en un mes:

\[
\text{Horas extra} = 50 - 40 = 10
\]
\[
\text{Cargo adicional} = 10 \times \$4.05 = \$40.50
\]
\[
\text{Total mensual} = \$176.00 + \$40.50 = \$216.50
\]

\begin{flushleft}
	\textbf{Nota final:} Este análisis es de carácter exploratorio y busca evaluar la posible viabilidad económica del proyecto en caso de escalarse hacia una fase de comercialización. Las cifras aquí presentadas son estimaciones preliminares y podrían ajustarse conforme se avance en la validación del producto y el análisis de mercado.
\end{flushleft}

