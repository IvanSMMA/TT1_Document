\begin{thebibliography}{99}
    \bibitem{ref1}
    MacMillan Education, “Aspectos generales para entender la comunicación. Unidad 1”, MacMillan, 2018. [En línea]. Disponible en: \url{https://www.macmillaneducation.es/wp-content/uploads/2018/10/comunicacion_cliente_libroalumno_unidad1muestra.pdf.} [Recuperado: 26-feb-2025].

    \bibitem{ref2}
    S. L. Hernández Mendoza y D. Aduana Avila, “Barreras de comunicación”, Revista ICEA, vol. 9, no. 18, pp. 47-48, 5 de mayo de 2021. [En línea]. Disponible en: \url{https://repository.uaeh.edu.mx/revistas/index.php/icea/article/view/7125/8008}. [Recuperado: 26-feb-2025].

    \bibitem{ref3}
    Secretaría de Salud, “Con discapacidad auditiva, 2.3 millones de personas: Instituto Nacional de Rehabilitación”, 2021. [En línea]. Disponible en: \url{https://www.gob.mx/salud/prensa/530-con-discapacidad-auditiva-2-3-millones-de-personas-instituto-nacional-de-rehabilitacion?idiom=es}. [Recuperado: 26-feb-2025].

    \bibitem{ref4}
    J. Morales Novas y A. K. Sánchez Zepeda, “Lengua de Señas Mexicana (LSM): Su importancia”, s.f. [En línea]. Disponible en: \url{https://trabajosocial.unam.mx/copred/doc/infografia_2_lengua%20de%20senas_mexicana.pdf}. [Recuperado: 26-feb-2025].

    \bibitem{ref5}
    M. Florencia Melo, “El mapa mundial de Android e iOS”, 2024. [En línea]. Disponible en: \url{https://es.statista.com/grafico/29620/sistema-operativo-movil-con-la-mayor-cuota-de-mercado-por-pais/}. [Recuperado: 26-feb-2025].

    \bibitem{ref6}
    Instituto Federal de Telecomunicaciones (IFT), “ANDROID ES EL SISTEMA OPERATIVO MÁS UTILIZADO EN MÉXICO; POR SU PARTE, GOOGLE/CHROME PREDOMINA PARA REALIZAR BÚSQUEDAS EN INTERNET”, IFT, 2022. [En línea]. Disponible en: \url{https://www.ift.org.mx/sites/default/files/contenidogeneral/usuarios-y-audiencias/encuestassobresistemasoperativosynavegadores2022.pdf}. [Recuperado: 20-abr-2025].

    \bibitem{ref7}
    C. Lugaresi, J. Tang, H. Nash, C. McClanahan, E. Uboweja, M. Hays, F. Zhang, C.-L. Chang, M. G. Yong, J. Lee et al., “Mediapipe: A framework for building perception pipelines”, arXiv preprint, arXiv:1906.08172, 2019.

    \bibitem{ref8}
    J. C. Hernández-Cruz, C. E. Rose-Gómez y S. González-López, “Translation of Spanish Text to Mexican Sign Language Glossed Text Using Rules and Deep Learning”, Resilience and Future of Smart Learning, J. Yang et al., Eds. Springer, 2022. doi: 10.1007/978-981-19-5967-7\_25.

    \bibitem{ref9}
    O. Pichardo-Lagunas y B. Martínez-Seis, “Resource Creation for Automatic Translation System from Texts in Spanish into Mexican Sign Language”, Research in Computing Science, vol. 100, pp. 129-137, 2015.

    \bibitem{ref10}
    Instituto de Pedagogía Aplicada, “Voz y Señas, traductor LSM”, Voz\&Señas, 2018. [En línea]. Disponible en: \url{https://www.vozysenas.com/}. [Recuperado: 03-mar-2025].

    \bibitem{ref11}
    HeTaH, “Avatar Lengua de Señas,” HeTaH, 2017. [En línea]. Disponible en: \url{https://hetah.net/avatar}. [Recuperado: 03-mar-2025].

    \bibitem{ref12}
    Hand Talk, “Meet the Hand Talk Sign Language Translator App”, Hand Talk, 2025. [En línea]. Disponible en: \url{https://www.handtalk.me/en/blog/meet-the-hand-talk-sign-language-translator-app/.}. [Recuperado: 03-mar-2025].

    \bibitem{ref13}
    Sign4ALL, “Hablemos de Inclusión”, Sign4ALL, 2018. [En línea]. Disponible en: \url{https://www.sign4all.net/}. [Recuperado: 03-mar-2025].

    \bibitem{ref14}
    UW News, “UW undergraduate team wins \$10,000 Lemelson-MIT Student Prize for gloves that translate sign language”, University of Washington News, 2016. [En línea]. Disponible en: \url{https://www.washington.edu/news/2016/04/12/uw-undergraduate-team-wins-10000-lemelson-mit-student-prize-for-gloves-that-translate-sign-language/}. [Recuperado: 03-mar-2025].

    \bibitem{ref15}
    J. A. Lara, “Sistema traductor de la Lengua de Señas Mexicana a español mediante dactilología y de español a español signado,” Tesis de maestría, CIC, IPN, Ciudad de México, México, 2024.

    \bibitem{ref16}
    H. Takeuchi y I. Nonaka, “The new product development game”, Harvard Business Review, ene.-feb., pp. 137-146, 1986.

    \bibitem{ref17}
    A. Navarro Cadavid, J. D. Fernández Martínez, y J. Morales Vélez, "Revisión de metodologías ágiles para el desarrollo de software", Prospectiva, vol. 11, no. 2, pp. 30-39, jul.-dic. 2013. [En línea]. Disponible en: \url{https://www.redalyc.org/pdf/4962/496250736004.pdf}. [Recuperado: 03-feb-2025].

    \bibitem{ref18}
    K. Schwaber y J. Sutherland, The Scrum guide, 2011. [En línea]. Disponible en: \url{http://www.scrumguides.org/}. [Recuperado: 03-feb-2025].

\end{thebibliography}