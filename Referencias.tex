\cleardoublepage
\addcontentsline{toc}{chapter}{Bibliografía}
\renewcommand{\bibname}{Bibliografía} % o \refname en article

\begingroup
\let\clearpage\relax % Evita el salto de página

\begin{thebibliography}{200}
    \bibitem{ref1}
    MacMillan Education, “Aspectos generales para entender la comunicación. Unidad 1”, MacMillan, 2018. [En línea]. Disponible en: \url{https://www.macmillaneducation.es/wp-content/uploads/2018/10/comunicacion_cliente_libroalumno_unidad1muestra.pdf.} [Recuperado: 26-feb-2025].

    \bibitem{ref2}
    S. L. Hernández Mendoza y D. Aduana Avila, “Barreras de comunicación”, Revista ICEA, vol. 9, no. 18, pp. 47-48, 5 de mayo de 2021. [En línea]. Disponible en: \url{https://repository.uaeh.edu.mx/revistas/index.php/icea/article/view/7125/8008}. [Recuperado: 26-feb-2025].

    \bibitem{ref3}
    Secretaría de Salud, “Con discapacidad auditiva, 2.3 millones de personas: Instituto Nacional de Rehabilitación”, 2021. [En línea]. Disponible en: \url{https://goo.su/fZ8L4}. [Recuperado: 26-feb-2025].

    \bibitem{ref4}
    J. Morales Novas y A. K. Sánchez Zepeda, “Lengua de Señas Mexicana (LSM): Su importancia”, s.f. [En línea]. Disponible en: \url{https://trabajosocial.unam.mx/copred/doc/infografia_2_lengua%20de%20senas_mexicana.pdf}. [Recuperado: 26-feb-2025].

    \bibitem{ref5}
    M. Florencia Melo, “El mapa mundial de Android e iOS”, 2024. [En línea]. Disponible en: \url{https://es.statista.com/grafico/29620/sistema-operativo-movil-con-la-mayor-cuota-de-mercado-por-pais/}. [Recuperado: 26-feb-2025].

    \bibitem{ref6}
    Instituto Federal de Telecomunicaciones (IFT), “ANDROID ES EL SISTEMA OPERATIVO MÁS UTILIZADO EN MÉXICO; POR SU PARTE, GOOGLE/CHROME PREDOMINA PARA REALIZAR BÚSQUEDAS EN INTERNET”, IFT, 2022. [En línea]. Disponible en: \url{https://www.ift.org.mx/sites/default/files/contenidogeneral/usuarios-y-audiencias/encuestassobresistemasoperativosynavegadores2022.pdf}. [Recuperado: 20-abr-2025].

    \bibitem{ref7}
    C. Lugaresi, J. Tang, H. Nash, C. McClanahan, E. Uboweja, M. Hays, F. Zhang, C.-L. Chang, M. G. Yong, J. Lee et al., “Mediapipe: A framework for building perception pipelines”, arXiv preprint, arXiv:1906.08172, 2019.

    \bibitem{ref8}
    J. C. Hernández-Cruz, C. E. Rose-Gómez y S. González-López, “Translation of Spanish Text to Mexican Sign Language Glossed Text Using Rules and Deep Learning”, Resilience and Future of Smart Learning, J. Yang et al., Eds. Springer, 2022. doi: 10.1007/978-981-19-5967-7\_25.

    \bibitem{ref9}
    O. Pichardo-Lagunas y B. Martínez-Seis, “Resource Creation for Automatic Translation System from Texts in Spanish into Mexican Sign Language”, Research in Computing Science, vol. 100, pp. 129-137, 2015.

    \bibitem{ref10}
    Instituto de Pedagogía Aplicada, “Voz y Señas, traductor LSM”, Voz\&Señas, 2018. [En línea]. Disponible en: \url{https://www.vozysenas.com/}. [Recuperado: 03-mar-2025].

    \bibitem{ref11}
    HeTaH, “Avatar Lengua de Señas,” HeTaH, 2017. [En línea]. Disponible en: \url{https://hetah.net/avatar}. [Recuperado: 03-mar-2025].

    \bibitem{ref12}
    Hand Talk, “Meet the Hand Talk Sign Language Translator App”, Hand Talk, 2025. [En línea]. Disponible en: \url{https://www.handtalk.me/en/blog/meet-the-hand-talk-sign-language-translator-app/.}. [Recuperado: 03-mar-2025].

    \bibitem{ref13}
    Sign4ALL, “Hablemos de Inclusión”, Sign4ALL, 2018. [En línea]. Disponible en: \url{https://www.sign4all.net/}. [Recuperado: 03-mar-2025].

    \bibitem{ref14}
    UW News, “UW undergraduate team wins \$10,000 Lemelson-MIT Student Prize for gloves that translate sign language”, University of Washington News, 2016. [En línea]. Disponible en: \url{https://goo.su/T9jf}. [Recuperado: 03-mar-2025].

    \bibitem{ref15}
    J. A. Lara, “Sistema traductor de la Lengua de Señas Mexicana a español mediante dactilología y de español a español signado,” Tesis de maestría, CIC, IPN, Ciudad de México, México, 2024.

    \bibitem{ref16}
    H. Takeuchi y I. Nonaka, “The new product development game”, Harvard Business Review, ene.-feb., pp. 137-146, 1986.

    \bibitem{ref17}
    A. Navarro Cadavid, J. D. Fernández Martínez, y J. Morales Vélez, "Revisión de metodologías ágiles para el desarrollo de software", Prospectiva, vol. 11, no. 2, pp. 30-39, jul.-dic. 2013. [En línea]. Disponible en: \url{https://www.redalyc.org/pdf/4962/496250736004.pdf}. [Recuperado: 03-feb-2025].

    \bibitem{ref18}
    K. Schwaber y J. Sutherland, The Scrum guide, 2011. [En línea]. Disponible en: \url{http://www.scrumguides.org/}. [Recuperado: 03-feb-2025].

    \bibitem{ref19}
    L. Lomelí, “Metodología Scrum: Roles, Procesos y Artefactos”, Innevo, 26-may-2023. [En línea]. Disponible en: \url{https://innevo.com/blog/metodologia-scrum}. [Accedido: 04-abr-2025].

    \bibitem{ref20}
    D. V. Santos Garcia, “Fundamentos de la comunicación”. Red Tercer Milenio S.C., 2012. [En línea]. Disponible en: \url{https://dspace.itsjapon.edu.ec/jspui/bitstream/123456789/673/1/Fundamentos_de_comunicacion.pdf}. [Recuperado: 01-mar-2025].

    \bibitem{ref21}
    K. Huaylla Gonzales, “LA COMUNICACIÓN EFECTIVA”, Univ. Priv. San Juan Bautista, Fac. Comunicación y Ciencias Administrativas, 2021. [En línea]. Disponible en: \url{https://www.researchgate.net/profile/Katherine-Huaylla-Gonzales/publication/356814831_LA_COMUNICACION_EFECTIVA/links/61ae5473c11c10383694545c/LA-COMUNICACION-EFECTIVA.pdf}. [Recuperado: 01-mar-2025].

    \bibitem{ref22}
    UNAM, “Módulo I. Identidad, Unidad 3”, Comunicación. UNAM. [En línea]. Disponible en: \url{https://www.campus-virtual.mineria.unam.mx/Mineria/Diplomados/2DHabilidades/Documentos/--M1U3_PDF.pdf}. [Recuperado: 01-mar-2025].

    \bibitem{ref23}
    McGrawHill, “La comunicación”. [En línea]. Disponible en: \url{https://www.mheducation.es/bcv/guide/capitulo/8448180445.pdf}. [Recuperado: 01-mar-2025].

    \bibitem{ref24}
    Pressbooks, “La Comunicación y sus Componentes”. [En línea]. Disponible en: \url{https://saalck.pressbooks.pub/spanish-composition-and-grammar-wku/chapter/a-1-la-comunicacion-y-sus-componentes/}. [Recuperado: 01-mar-2025].

    \bibitem{ref25}
    R. Gasperin, “Barreras de la comunicación y en las relaciones humanas”, Univ. Veracruzana, pp. 95-135, 2005. [En línea]. Disponible en: \url{https://www.uv.mx/personal/rdegasperin/files/2011/07/Antologia.Comunicacion-Unidad3.pdf}. [Recuperado: 20-mar-2021].

    \bibitem{ref26}
    K. Ruíz García, “Diferencias culturales entre sordos y oyentes (LSM)”, genially, 25-ene-2023. [En línea]. Disponible en: \url{https://view.genially.com/63d1b7cdc9caf40011513154/presentation-diferencias-culturales-entre-sordos-y-oyentes-lsm}. [Recuperado: 01-mar-2025].

    \bibitem{ref27}
    A. Galván Jordán, “El Privilegio de Oír: Conocimiento, Percepción y Comunicación de la Población Oyente sobre las Personas con Sordera”, Univ. de La Laguna, Fac. Psicología y Logopedia, 2022. [En línea]. Disponible en: \url{https://riull.ull.es/xmlui/bitstream/handle/915/29274/El%20privilegio%20de%20oir%20Conocimiento%2C%20percepcion%20y%20comunicacion%20de%20la%20poblacion%20oyente%20sobre%20las%20personas%20con%20sordera.pdf?sequence=1&isAllowed=y}. [Recuperado: 01-mar-2025].

    \bibitem{ref28}
    Naciones Unidas, “Día Internacional de las Lenguas de Señas, 23 de septiembre”. [En línea]. Disponible en: \url{https://www.un.org/es/observances/sign-languages-day}. [Recuperado: 01-mar-2025].

    \bibitem{ref29}
    M. Restrepo Montes y L. C. Clavijo, “La construcción de la identidad del adolescente sordo”, Univ. de Manizales, Fac. de Educación, 2004. [En línea]. Disponible en: \url{https://ridum.umanizales.edu.co/xmlui/handle/20.500.12746/276}. [Recuperado: 03-mar-2025].

    \bibitem{ref30}
    J. Carrascosa García, “La discapacidad auditiva. Principales modelos y ayudas técnicas para la intervención”, Rev. Int. de Apoyo a la Inclusión, Logopedia, Sociedad y Multiculturalidad, vol. 1, no. 1, pp. 24-36, ene. 2015. [En línea]. Disponible en: \url{https://revistaselectronicas.ujaen.es/index.php/riai/article/view/4141/3367}. [Recuperado: 03-mar-2025].

    \bibitem{ref31}
    National Institute on Deafness and Other Communication Disorders, “Partes del oído”, NIH/NIHD, 14-jun-2022. [En línea]. Disponible en: \url{https://www.nidcd.nih.gov/es/multimedia/partes-del-oido}. [Accedido: 04-abr-2025]. 

    \bibitem{ref32}
    V. C. Abello Gómez, “Interacción comunicativa entre comunidad sorda y oyente, y la incidencia de aspectos sociales y culturales en las prácticas comunicativas”, Univ. Distrital Francisco José de Caldas, Fac. de Educación, 2017. [En línea]. Disponible en: \url{https://repository.udistrital.edu.co/server/api/core/bitstreams/fcbd2ab4-a027-4823-b914-f3229247925e/content}. [Recuperado: 03-mar-2025].

    \bibitem{ref33}
    A. Ruiz Villa, “La lengua de señas en un mundo globalizado”, IDJ: Blog Digital Universitario, vol. 1, pp. 1-12, ago. 2021. [En línea]. Disponible en: \url{https://edu.ijd.org.mx/data/files/La-lengua-de-se-as-en-un-mundo-globalizado_Alejandra-Ruiz-Villa_VBLOG_vf_3.pdf}. [Recuperado: 04-mar-2025].

    \bibitem{ref34}
    Cámara de Diputados del H. Congreso de la Unión, “Ley General para la Inclusión de las Personas con Discapacidad, última reforma publicada DOF 14-06-2024”, Diario Oficial de la Federación, México, 30 de mayo de 2011. [En línea]. Disponible en: \url{https://www.diputados.gob.mx/LeyesBiblio/pdf/LGIPD.pdf}. [Recuperado: 05-mar-2025].

    \bibitem{ref35}
    A. Valdez, "¿Qué tan accesible es aprender Lengua de Señas?", Forbes México, 16 sep. 2021. [En línea]. Disponible en: \url{https://forbes.com.mx/que-tan-acceible-es-aprender-lengua-de-senas/}. [Recuperado: 04-mar-2025].

    \bibitem{ref36}
    M. E. Serafín de Fleischmann y R. González Pérez, “Manos con Voz: Diccionario de Lengua de Señas Mexicana”, Consejo Nacional para Prevenir la Discriminación, 2011. [En línea]. Disponible en: \url{https://educacionespecial.sep.gob.mx/storage/recursos/2023/05/xzrfl019nV-4Diccionario_lengua_%20Senas.pdf}. [Recuperado: 20-abr-2025].

    \bibitem{ref37}
    Instituto de las Personas con Discapacidad (INDEPEDI), “Diccionario de Lengua de Señas Mexicana LSM”, Gobierno de la Ciudad de México, 2017. [En línea]. Disponible en: \url{https://pdh.cdmx.gob.mx/storage/app/media/banner/Dic_LSM%202.pdf}. [Recuperado: 20-abr-2025].

    \bibitem{ref38}
    M. González Moraga, “El proceso de construcción del rol de los educadores Sordos chilenos”, Universidade Federal de Sergipe, 2017. [En línea]. Disponible en: \url{https://www.researchgate.net/publication/324279135_El_proceso_de_construccion_del_rol_de_los_educadores_Sordos_chilenos}. [Recuperado: 20-abr-2025].

    \bibitem{ref39}
    G, Acevedo, R. Flores, S. Lima y B. Alducin., “Lenguaje de Señas por Celular”, Instituto Tecnológico de Milpa Alta e Instituto Tecnológico de Cuautla, 2009. [En línea]. Disponible en: \url{https://www.iiis.org/CDs2009/CD2009CSC/CISCI2009/PapersPdf/C828UG.pdf}. [Recuperado: 20-abr-2025].

    \bibitem{ref40}
    J. Vilches Vilela, “La dactilología, ¿qué, cómo, cuándo?”, Escuela Universitaria de Magisterio “Sagrado Corazón” de Córdoba, 2005. [En línea]. Disponible en: \url{https://www.uco.es/%7Efe1vivim/alfabeto_dactilologico.pdf}. [Recuperado: 07-mar-2025].

    \bibitem{ref41}
    L. P. Rouhiainen, “Inteligencia Artificial, 101 cosas que debes saber hoy sobre nuestro futuro”, Editorial Planeta, 2018. [En línea]. Disponible en: \url{https://planetadelibrosar0.cdnstatics.com/libros_contenido_extra/40/39307_Inteligencia_artificial.pdf}. [Recuperado: 05-mar-2025].

    \bibitem{ref42}
    F. Morandín-Ahuerma, “¿Qué es la inteligencia artificial?”. [En línea]. Disponible en: \url{https://philarchive.org/archive/MORQEI-2v2}. [Recuperado: 05-mar-2025].

    \bibitem{ref43}
    UNAM, “Capítulo 4. Inteligencia artificial”. [En línea]. Disponible en: \url{http://www.ptolomeo.unam.mx:8080/xmlui/bitstream/handle/132.248.52.100/219/A7.pdf}. [Recuperado: 05-mar-2025].

    \bibitem{ref44}
    F. Bravo-Marquez y J. Dunstan, “Procesamiento de lenguaje natural: dónde estamos y qué estamos haciendo”, Rev. Bits de Ciencia, Univ. de Chile, núm. 21, 2021. [En línea]. Disponible en: \url{https://revistasdex.uchile.cl/index.php/bits/article/view/2772}. [Recuperado: 07-mar-2025].
    
    \bibitem{ref45}
    N. C. Beltrán y E. C. Rodríguez, “Procesamiento del lenguaje natural (PLN) -GPT-3, y su aplicación en la Ingeniería de Software”, Tecnol. Investig. AcademiaTIA, vol. 8, no. 1, pp. 18-37, 2021.

    \bibitem{ref46}
    A. C. Vásquez, J. P. Quispe y A. M. Huayna, “Procesamiento de lenguaje natural”, Revista de Investigación de Sistemas e Informática, vol. 6, no. 2, pp. 45-54, 2009.

    \bibitem{ref47}
    F. M. Ramos y J. I. Velez, “Integración de Técnicas de Procesamiento de Lenguaje Natural a través de Servicios Web”, Universidad Nacional del Centro de la Provincia de Buenos Aires, Facultad de Ciencias Exactas, mayo 2016. [En línea]. Disponible en: \url{http://alejandrorago.com.ar/files/advising/2016-thesis-velez&ramos.pdf?i=1}. [Recuperado: 06-mar-2025].

    \bibitem{ref48}
    M. Hernández y J. Gómez, “Aplicaciones de Procesamiento de Lenguaje Natural”, Revista Politécnica, vol. 32, pp. 87-96, 2013. [En línea]. Disponible en: \url{https://www.redalyc.org/articulo.oa?id=688773657016}. [Recuperado: 02-abr-2025].

    \bibitem{ref49}
    Nanobaly, “MediaPipe: la caja de herramientas esencial para la Computer Vision”, Innovatiana, 21-ago-2024. [En línea]. Disponible en: \url{https://es.innovatiana.com/post/mediapipe-101}. [Recuperado: 07-mar-2025].

    \bibitem{ref50}
    N, Buhl, “Google’s MediaPipe Framework: Deploy Computer Vision Pipelines with Ease”, Encord, 21-jun-2024. [En línea]. Disponible en: \url{https://encord.com/blog/google-mediapipe/}. [Accedido: 04-abr-2025].

    \bibitem{ref51}
    MediaPipe, “MediaPipe Hands”, MediaPipe, 2025. [En línea]. Disponible en: \url{https://mediapipe.readthedocs.io/en/latest/solutions/hands.html}. [Recuperado: 20-abr-2025].

    \bibitem{ref52}
    Google AI for Developers, “Hand landmarks detection guide”, Google, 2025. [En línea]. Disponible en: \url{https://ai.google.dev/edge/mediapipe/solutions/vision/hand_landmarker?hl=es-419}. [Recuperado: 20-abr-2025].

    \bibitem{ref53}
    Google AI for Developers, “Guía de detección de puntos de referencia de posiciones”, Google, 2025. [En línea]. Disponible en: \url{https://ai.google.dev/edge/mediapipe/solutions/vision/pose_landmarker?hl=es-419}. [Recuperado: 07-mar-2025].

    \bibitem{ref54}
    J. S. Viñolo Locubiche, “El modelo de producción industrial de animación 3D estadounidense”, Univ. de Barcelona, 20-jun-2017. [En línea]. Disponible en: \url{https://diposit.ub.edu/dspace/handle/2445/115650}. [Recuperado: 09-mar-2025].

    \bibitem{ref55}
    J. Nava, “Qué es Unity y para qué sirve este motor gráfico”, Empower Talent, 2024. [En línea]. Disponible en: \url{https://empowertalent.com/que-es-unity/}. [Recuperado: 20-abr-2025].

    \bibitem{ref56}
    Cool Efect, “Unity 3D”. [En línea]. Disponible en: \url{https://www.cooleffect.org/unity-3d}. [Accedido: 04-abr-2025].

    \bibitem{ref57}
    D. M. Campos Gavilanez, “Análisis Comparativo entre Sistemas Operativos de Dispositivos Móviles Android, iPhone OS y MIUI”, Universidad Técnica de Babahoyo, Facultad de Administración, Finanzas e Informática, 2023. [En línea]. Disponible en: \url{https://dspace.utb.edu.ec/bitstream/handle/49000/13969/E-UTB-FAFI-SIST-000414.pdf?sequence=1&isAllowed=y}. [Recuperado: 07-mar-2025].

    \bibitem{ref58}
    M. García, “La historia del logo de Android”, brandemia, 29-ene-2024. [En línea]. Disponible en: \url{https://brandemia.org/la-historia-del-logo-de-android}. [Accedido: 04-abr-2025].

    \bibitem{ref59}
    M. Budziński, “What is React Native? Complex Guide for 2024”, Net Guru, 2025. [En línea]. Disponible en: \url{https://www-netguru-com.translate.goog/glossary/react-native?_x_tr_sl=en&_x_tr_tl=es&_x_tr_hl=es&_x_tr_pto=tc}. [Recuperado: 08-may-2025].

    \bibitem{ref60}
    React Native, “Get Started with React Native”, React Native, 2025. [En línea]. Disponible en: \url{https://reactnative.dev/docs/environment-setup}. [Recuperado: 08-may-2025].

    \bibitem{ref61}
    J. Nielsen, “10 Usability Heuristics for User Interface Design”, Nielsen Norman Group, 1994. [En línea]. Disponible en: \url{https://www-nngroup-com.translate.goog/articles/ten-usability-heuristics/?_x_tr_sl=en&_x_tr_tl=es&_x_tr_hl=es&_x_tr_pto=tc}. [Recuperado: 15-may-2025].

    \bibitem{ref62}
    International Organization for Standardization, “ISO 9241-210:2019”, ISO, 2019. [En línea]. Disponible en: \url{https://www-iso-org.translate.goog/standard/77520.html?_x_tr_sl=en&_x_tr_tl=es&_x_tr_hl=es&_x_tr_pto=tc}. [Recuperado: 15-may-2025].
\end{thebibliography}