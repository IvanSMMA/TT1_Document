\cleardoublepage
\addcontentsline{toc}{chapter}{Bibliografía}
\renewcommand{\bibname}{Bibliografía} % o \refname en article

\begingroup

\begin{thebibliography}{200}
    \bibitem{ref1}
    MacMillan Education, “Aspectos generales para entender la comunicación. Unidad 1”, MacMillan, 2018. [En línea]. Disponible en: \url{https://www.macmillaneducation.es/wp-content/uploads/2018/10/comunicacion_cliente_libroalumno_unidad1muestra.pdf.} [Recuperado: 26-feb-2025].

    \bibitem{ref2}
    S. L. Hernández Mendoza y D. Aduana Avila, “Barreras de comunicación”, Revista ICEA, vol. 9, no. 18, pp. 47-48, 5 de mayo de 2021. [En línea]. Disponible en: \url{https://repository.uaeh.edu.mx/revistas/index.php/icea/article/view/7125/8008}. [Recuperado: 26-feb-2025].

    \bibitem{ref3}
    Secretaría de Salud, “Con discapacidad auditiva, 2.3 millones de personas: Instituto Nacional de Rehabilitación”, 2021. [En línea]. Disponible en: \url{https://goo.su/fZ8L4}. [Recuperado: 26-feb-2025].

    \bibitem{ref4}
    J. Morales Novas y A. K. Sánchez Zepeda, “Lengua de Señas Mexicana (LSM): Su importancia”, s.f. [En línea]. Disponible en: \url{https://trabajosocial.unam.mx/copred/doc/infografia_2_lengua%20de%20senas_mexicana.pdf}. [Recuperado: 26-feb-2025].

    \bibitem{ref5}
    M. Florencia Melo, “El mapa mundial de Android e iOS”, 2024. [En línea]. Disponible en: \url{https://es.statista.com/grafico/29620/sistema-operativo-movil-con-la-mayor-cuota-de-mercado-por-pais/}. [Recuperado: 26-feb-2025].

    \bibitem{ref6}
    Instituto Federal de Telecomunicaciones (IFT), “ANDROID ES EL SISTEMA OPERATIVO MÁS UTILIZADO EN MÉXICO; POR SU PARTE, GOOGLE/CHROME PREDOMINA PARA REALIZAR BÚSQUEDAS EN INTERNET”, IFT, 2022. [En línea]. Disponible en: \url{https://www.ift.org.mx/sites/default/files/contenidogeneral/usuarios-y-audiencias/encuestassobresistemasoperativosynavegadores2022.pdf}. [Recuperado: 20-abr-2025].

    \bibitem{ref7}
    C. Lugaresi, J. Tang, H. Nash, C. McClanahan, E. Uboweja, M. Hays, F. Zhang, C.-L. Chang, M. G. Yong, J. Lee et al., “Mediapipe: A framework for building perception pipelines”, arXiv preprint, arXiv:1906.08172, 2019.

    \bibitem{ref8}
    J. C. Hernández-Cruz, C. E. Rose-Gómez y S. González-López, “Translation of Spanish Text to Mexican Sign Language Glossed Text Using Rules and Deep Learning”, Resilience and Future of Smart Learning, J. Yang et al., Eds. Springer, 2022. doi: 10.1007/978-981-19-5967-7\_25.

    \bibitem{ref9}
    O. Pichardo-Lagunas y B. Martínez-Seis, “Resource Creation for Automatic Translation System from Texts in Spanish into Mexican Sign Language”, Research in Computing Science, vol. 100, pp. 129-137, 2015.

    \bibitem{ref10}
    Instituto de Pedagogía Aplicada, “Voz y Señas, traductor LSM”, Voz\&Señas, 2018. [En línea]. Disponible en: \url{https://www.vozysenas.com/}. [Recuperado: 03-mar-2025].

    \bibitem{ref11}
    HeTaH, “Avatar Lengua de Señas,” HeTaH, 2017. [En línea]. Disponible en: \url{https://hetah.net/avatar}. [Recuperado: 03-mar-2025].

    \bibitem{ref12}
    Hand Talk, “Meet the Hand Talk Sign Language Translator App”, Hand Talk, 2025. [En línea]. Disponible en: \url{https://www.handtalk.me/en/blog/meet-the-hand-talk-sign-language-translator-app/.}. [Recuperado: 03-mar-2025].

    \bibitem{ref13}
    Sign4ALL, “Hablemos de Inclusión”, Sign4ALL, 2018. [En línea]. Disponible en: \url{https://www.sign4all.net/}. [Recuperado: 03-mar-2025].

    \bibitem{ref14}
    UW News, “UW undergraduate team wins \$10,000 Lemelson-MIT Student Prize for gloves that translate sign language”, University of Washington News, 2016. [En línea]. Disponible en: \url{https://goo.su/T9jf}. [Recuperado: 03-mar-2025].

    \bibitem{ref15}
    J. A. Lara, “Sistema traductor de la Lengua de Señas Mexicana a español mediante dactilología y de español a español signado,” Tesis de maestría, CIC, IPN, Ciudad de México, México, 2024.

    \bibitem{ref16}
    H. Takeuchi y I. Nonaka, “The new product development game”, Harvard Business Review, ene.-feb., pp. 137-146, 1986.

    \bibitem{ref17}
    A. Navarro Cadavid, J. D. Fernández Martínez, y J. Morales Vélez, "Revisión de metodologías ágiles para el desarrollo de software", Prospectiva, vol. 11, no. 2, pp. 30-39, jul.-dic. 2013. [En línea]. Disponible en: \url{https://www.redalyc.org/pdf/4962/496250736004.pdf}. [Recuperado: 03-feb-2025].

    \bibitem{ref18}
    K. Schwaber y J. Sutherland, The Scrum guide, 2011. [En línea]. Disponible en: \url{http://www.scrumguides.org/}. [Recuperado: 03-feb-2025].

    \bibitem{ref19}
    L. Lomelí, “Metodología Scrum: Roles, Procesos y Artefactos”, Innevo, 26-may-2023. [En línea]. Disponible en: \url{https://innevo.com/blog/metodologia-scrum}. [Accedido: 04-abr-2025].

    \bibitem{ref20}
    D. V. Santos Garcia, “Fundamentos de la comunicación”. Red Tercer Milenio S.C., 2012. [En línea]. Disponible en: \url{https://dspace.itsjapon.edu.ec/jspui/bitstream/123456789/673/1/Fundamentos_de_comunicacion.pdf}. [Recuperado: 01-mar-2025].

    \bibitem{ref21}
    K. Huaylla Gonzales, “LA COMUNICACIÓN EFECTIVA”, Univ. Priv. San Juan Bautista, Fac. Comunicación y Ciencias Administrativas, 2021. [En línea]. Disponible en: \url{https://www.researchgate.net/profile/Katherine-Huaylla-Gonzales/publication/356814831_LA_COMUNICACION_EFECTIVA/links/61ae5473c11c10383694545c/LA-COMUNICACION-EFECTIVA.pdf}. [Recuperado: 01-mar-2025].

    \bibitem{ref22}
    UNAM, “Módulo I. Identidad, Unidad 3”, Comunicación. UNAM. [En línea]. Disponible en: \url{https://www.campus-virtual.mineria.unam.mx/Mineria/Diplomados/2DHabilidades/Documentos/--M1U3_PDF.pdf}. [Recuperado: 01-mar-2025].

    \bibitem{ref23}
    McGrawHill, “La comunicación”. [En línea]. Disponible en: \url{https://www.mheducation.es/bcv/guide/capitulo/8448180445.pdf}. [Recuperado: 01-mar-2025].

    \bibitem{ref24}
    Pressbooks, “La Comunicación y sus Componentes”. [En línea]. Disponible en: \url{https://saalck.pressbooks.pub/spanish-composition-and-grammar-wku/chapter/a-1-la-comunicacion-y-sus-componentes/}. [Recuperado: 01-mar-2025].

    \bibitem{ref25}
    R. Gasperin, “Barreras de la comunicación y en las relaciones humanas”, Univ. Veracruzana, pp. 95-135, 2005. [En línea]. Disponible en: \url{https://www.uv.mx/personal/rdegasperin/files/2011/07/Antologia.Comunicacion-Unidad3.pdf}. [Recuperado: 20-mar-2021].

    \bibitem{ref26}
    K. Ruíz García, “Diferencias culturales entre sordos y oyentes (LSM)”, genially, 25-ene-2023. [En línea]. Disponible en: \url{https://view.genially.com/63d1b7cdc9caf40011513154/presentation-diferencias-culturales-entre-sordos-y-oyentes-lsm}. [Recuperado: 01-mar-2025].

    \bibitem{ref27}
    A. Galván Jordán, “El Privilegio de Oír: Conocimiento, Percepción y Comunicación de la Población Oyente sobre las Personas con Sordera”, Univ. de La Laguna, Fac. Psicología y Logopedia, 2022. [En línea]. Disponible en: \url{https://riull.ull.es/xmlui/bitstream/handle/915/29274/El%20privilegio%20de%20oir%20Conocimiento%2C%20percepcion%20y%20comunicacion%20de%20la%20poblacion%20oyente%20sobre%20las%20personas%20con%20sordera.pdf?sequence=1&isAllowed=y}. [Recuperado: 01-mar-2025].

    \bibitem{refsordos}
    SENADIS, “Conociendo la terminología apropiada para referirse a las Personas en Situación de Discapacidad (PeSD)”, Gobierno de Chile. [En línea]. Disponible en: \url{https://www.senadis.gob.cl/resources/upload/documento/b36c2bb728ef85fa68b257ccfe0aff3a.pdf}. [Recuperado: 09-may-2025].

    \bibitem{ref28}
    Naciones Unidas, “Día Internacional de las Lenguas de Señas, 23 de septiembre”. [En línea]. Disponible en: \url{https://www.un.org/es/observances/sign-languages-day}. [Recuperado: 01-mar-2025].

    \bibitem{ref29}
    M. Restrepo Montes y L. C. Clavijo, “La construcción de la identidad del adolescente sordo”, Univ. de Manizales, Fac. de Educación, 2004. [En línea]. Disponible en: \url{https://ridum.umanizales.edu.co/xmlui/handle/20.500.12746/276}. [Recuperado: 03-mar-2025].

    \bibitem{ref30}
    J. Carrascosa García, “La discapacidad auditiva. Principales modelos y ayudas técnicas para la intervención”, Rev. Int. de Apoyo a la Inclusión, Logopedia, Sociedad y Multiculturalidad, vol. 1, no. 1, pp. 24-36, ene. 2015. [En línea]. Disponible en: \url{https://revistaselectronicas.ujaen.es/index.php/riai/article/view/4141/3367}. [Recuperado: 03-mar-2025].

    \bibitem{ref31}
    National Institute on Deafness and Other Communication Disorders, “Partes del oído”, NIH/NIHD, 14-jun-2022. [En línea]. Disponible en: \url{https://www.nidcd.nih.gov/es/multimedia/partes-del-oido}. [Accedido: 04-abr-2025]. 

    \bibitem{ref32}
    V. C. Abello Gómez, “Interacción comunicativa entre comunidad sorda y oyente, y la incidencia de aspectos sociales y culturales en las prácticas comunicativas”, Univ. Distrital Francisco José de Caldas, Fac. de Educación, 2017. [En línea]. Disponible en: \url{https://repository.udistrital.edu.co/server/api/core/bitstreams/fcbd2ab4-a027-4823-b914-f3229247925e/content}. [Recuperado: 03-mar-2025].

    \bibitem{ref33}
    A. Ruiz Villa, “La lengua de señas en un mundo globalizado”, IDJ: Blog Digital Universitario, vol. 1, pp. 1-12, ago. 2021. [En línea]. Disponible en: \url{https://edu.ijd.org.mx/data/files/La-lengua-de-se-as-en-un-mundo-globalizado_Alejandra-Ruiz-Villa_VBLOG_vf_3.pdf}. [Recuperado: 04-mar-2025].

    \bibitem{ref34}
    Cámara de Diputados del H. Congreso de la Unión, “Ley General para la Inclusión de las Personas con Discapacidad, última reforma publicada DOF 14-06-2024”, Diario Oficial de la Federación, México, 30 de mayo de 2011. [En línea]. Disponible en: \url{https://www.diputados.gob.mx/LeyesBiblio/pdf/LGIPD.pdf}. [Recuperado: 05-mar-2025].

    \bibitem{ref35}
    A. Valdez, "¿Qué tan accesible es aprender Lengua de Señas?", Forbes México, 16 sep. 2021. [En línea]. Disponible en: \url{https://forbes.com.mx/que-tan-acceible-es-aprender-lengua-de-senas/}. [Recuperado: 04-mar-2025].

    \bibitem{ref36}
    M. E. Serafín de Fleischmann y R. González Pérez, “Manos con Voz: Diccionario de Lengua de Señas Mexicana”, Consejo Nacional para Prevenir la Discriminación, 2011. [En línea]. Disponible en: \url{https://educacionespecial.sep.gob.mx/storage/recursos/2023/05/xzrfl019nV-4Diccionario_lengua_%20Senas.pdf}. [Recuperado: 20-abr-2025].

    \bibitem{ref37}
    Instituto de las Personas con Discapacidad (INDEPEDI), “Diccionario de Lengua de Señas Mexicana LSM”, Gobierno de la Ciudad de México, 2017. [En línea]. Disponible en: \url{https://pdh.cdmx.gob.mx/storage/app/media/banner/Dic_LSM%202.pdf}. [Recuperado: 20-abr-2025].

    \bibitem{ref38}
    M. González Moraga, “El proceso de construcción del rol de los educadores Sordos chilenos”, Universidade Federal de Sergipe, 2017. [En línea]. Disponible en: \url{https://www.researchgate.net/publication/324279135_El_proceso_de_construccion_del_rol_de_los_educadores_Sordos_chilenos}. [Recuperado: 20-abr-2025].

    \bibitem{ref39}
    G, Acevedo, R. Flores, S. Lima y B. Alducin., “Lenguaje de Señas por Celular”, Instituto Tecnológico de Milpa Alta e Instituto Tecnológico de Cuautla, 2009. [En línea]. Disponible en: \url{https://www.iiis.org/CDs2009/CD2009CSC/CISCI2009/PapersPdf/C828UG.pdf}. [Recuperado: 20-abr-2025].

    \bibitem{refrae}
    Real Academia Española y Asociación de Academias de la Lengua Española, “Glosario de términos gramaticales”, RAE, 2019. [En línea]. Disponible en: \url{https://www.rae.es/gtg/afijo}. [Recuperado: 10-may-2025].

    \bibitem{ref40}
    J. Vilches Vilela, “La dactilología, ¿qué, cómo, cuándo?”, Escuela Universitaria de Magisterio “Sagrado Corazón” de Córdoba, 2005. [En línea]. Disponible en: \url{https://www.uco.es/%7Efe1vivim/alfabeto_dactilologico.pdf}. [Recuperado: 07-mar-2025].

    \bibitem{reftradint}
    M. Hernández, “Diferencias entre traducir e interpretar”, Angloeducativo, 2025. [En línea]. Disponible en: \url{https://angloeducativo.com/blog/traducir-e-interpretar/}. [Recuperado: 16-jul-2025].

    \bibitem{reftradint2}
    Lionbridge, “Cinco diferencias básicas entre traducción e interpretación”, Lionbridge, 30-jun-2022. [En línea]. Disponible en: \url{https://www.lionbridge.com/es/blog/translation-localization/5-major-differences-interpretation-translation/}. [Recuperado: 05-nov-2025].

    \bibitem{ref41}
    L. P. Rouhiainen, “Inteligencia Artificial, 101 cosas que debes saber hoy sobre nuestro futuro”, Editorial Planeta, 2018. [En línea]. Disponible en: \url{https://planetadelibrosar0.cdnstatics.com/libros_contenido_extra/40/39307_Inteligencia_artificial.pdf}. [Recuperado: 05-mar-2025].

    \bibitem{ref42}
    F. Morandín-Ahuerma, “¿Qué es la inteligencia artificial?”. [En línea]. Disponible en: \url{https://philarchive.org/archive/MORQEI-2v2}. [Recuperado: 05-mar-2025].

    \bibitem{ref43}
    UNAM, “Capítulo 4. Inteligencia artificial”. [En línea]. Disponible en: \url{http://www.ptolomeo.unam.mx:8080/xmlui/bitstream/handle/132.248.52.100/219/A7.pdf}. [Recuperado: 05-mar-2025].

    \bibitem{ref44}
    F. Bravo-Marquez y J. Dunstan, “Procesamiento de lenguaje natural: dónde estamos y qué estamos haciendo”, Rev. Bits de Ciencia, Univ. de Chile, núm. 21, 2021. [En línea]. Disponible en: \url{https://revistasdex.uchile.cl/index.php/bits/article/view/2772}. [Recuperado: 07-mar-2025].
    
    \bibitem{ref45}
    N. C. Beltrán y E. C. Rodríguez, “Procesamiento del lenguaje natural (PLN) -GPT-3, y su aplicación en la Ingeniería de Software”, Tecnol. Investig. AcademiaTIA, vol. 8, no. 1, pp. 18-37, 2021.

    \bibitem{ref46}
    A. C. Vásquez, J. P. Quispe y A. M. Huayna, “Procesamiento de lenguaje natural”, Revista de Investigación de Sistemas e Informática, vol. 6, no. 2, pp. 45-54, 2009.

    \bibitem{ref47}
    F. M. Ramos y J. I. Velez, “Integración de Técnicas de Procesamiento de Lenguaje Natural a través de Servicios Web”, Universidad Nacional del Centro de la Provincia de Buenos Aires, Facultad de Ciencias Exactas, mayo 2016. [En línea]. Disponible en: \url{http://alejandrorago.com.ar/files/advising/2016-thesis-velez&ramos.pdf?i=1}. [Recuperado: 06-mar-2025].

    \bibitem{refebd1}
    I. Huerta, “Domina los Embeddings : La clave para el análisis semántico de contenidos, textos, documentaciones y keywords.”, Ikaue, s.f. [En línea]. Disponible en: \url{https://ikaue.com/blog-data/embeddings-de-texto-la-clave-para-el-analisis-semantico-y-de-significados-seo}. [Recuperado: 20-nov-2025].

    \bibitem{refebd2}
    G. Espíndola, “¿Qué son los embeddings y cómo se utilizan en la inteligencia artificial con python?”, Medium, 02-mar-2023. [En línea]. Disponible en: \url{https://gustavo-espindola.medium.com/qu%C3%A9-son-los-embeddings-y-c%C3%B3mo-se-utilizan-en-la-inteligencia-artificial-con-python-45b751ed86a5}. [Recuperado: 20-nov-2025].

    \bibitem{refebd3}
    P. Huet, “Embeddings: Qué son y cómo transforman datos en información”, OpenWebinars, 12-jul-2024. [En línea]. Disponible en: \url{https://openwebinars.net/blog/embeddings/}. [Recuperado: 20-nov-2025].

    \bibitem{refebd4}
    J. Barnard, “¿Qué es embedding?”, IBM, s.f. [En línea]. Disponible en: \url{https://www.ibm.com/mx-es/think/topics/embedding}. [Recuperado: 20-nov-2025].

    \bibitem{refebd5}
    T. Mikolov, K. Chen, G. Corrado, y J. Dean, "Efficient Estimation of Word Representations in Vector Space," arXiv:1301.3781, 2013. [En línea]. Disponible en: \url{https://arxiv.org/abs/1301.3781}. [Recuperado: 21-nov-2025].

    \bibitem{refebd6}
    J. Pennington, R. Socher, y C. Manning, "GloVe: Global Vectors for Word Representation," in Proc. EMNLP, 2014, pp. 1532–1543. [En línea]. Disponible en: \url{https://aclanthology.org/D14-1162/}. [Recuperado: 21-nov-2025].

    \bibitem{refebd7}
    P. Bojanowski, E. Grave, A. Joulin, y T. Mikolov, "Enriching Word Vectors with Subword Information," arXiv:1607.04606, 2017. [En línea]. Disponible en: \url{https://arxiv.org/abs/1607.04606}. [Recuperado: 21-nov-2025].

    \bibitem{refebd8}
    M. E. Peters et al., "Deep contextualized word representations," arXiv:1802.05365, 2018. [En línea]. Disponible en: \url{https://arxiv.org/abs/1802.05365}. [Recuperado: 21-nov-2025]. 

    \bibitem{refebd9}
    J. Devlin, M.-W. Chang, K. Lee, y K. Toutanova, "BERT: Pre-training of Deep Bidirectional Transformers for Language Understanding," in Proc. NAACL-HLT, 2019, pp. 4171–4186. [En línea]. Disponible en: \url{https://aclanthology.org/N19-1423/}. [Recuperado: 21-nov-2025].

    \bibitem{refebd10}
    N. Reimers y I. Gurevych, "Sentence-BERT: Sentence Embeddings using Siamese BERT-Networks," arXiv:1908.10084, 2019. [En línea]. Disponible en: \url{https://arxiv.org/abs/1908.10084}. [Recuperado: 21-nov-2025].

    \bibitem{refcos1}
    T. Krantz y A. Jonker, “¿Qué es la similitud de coseno?”, IBM, s.f. [En línea]. Disponible en: \url{https://www.ibm.com/mx-es/think/topics/cosine-similarity}. [Recuperado: 22-nov-2025].

    \bibitem{refcos2}
    J. De La Rosa Suncar, “NLP 101 : Similitud Por Coseno”, LinkedIn, 26-mar-2024. Disponible en: \url{https://www.linkedin.com/pulse/nlp-101-similitud-por-coseno-juli%C3%A1n-de-la-rosa-suncar-xvlie}. [Recuperado: 22-nov-2025].

    \bibitem{refcos3}
    V. Chugani, “What is Cosine Distance?”, Datacamp, 28-jul-2024. [En línea]. Disponible en: \url{https://www.datacamp.com/tutorial/cosine-distance}. [Recuperado: 22-nov-2025].

    \bibitem{reftrans1}
    M. López Fernández, “Qué es un modelo Transformer y cómo funciona”, AutomatizaPro, 2025. [En línea]. Disponible en: \url{https://www.automatizapro.com.ar/blog/que-es-modelo-transformer/}. [Recuperado: 22-nov-2025].

    \bibitem{reftrans2}
    A. Vaswani, N. Shazeer, N. Parmar, J. Uszkoreit, L. Jones, A. N. Gomez, Ł. Kaiser y I. Polosukhin, “Attention Is All You Need”, arXiv:1706.03762, 2023. [En línea]. Disponible en: \url{https://arxiv.org/abs/1706.03762}. [Recuperado: 22-nov-2025].

    \bibitem{ref48}
    M. Hernández y J. Gómez, “Aplicaciones de Procesamiento de Lenguaje Natural”, Revista Politécnica, vol. 32, pp. 87-96, 2013. [En línea]. Disponible en: \url{https://www.redalyc.org/articulo.oa?id=688773657016}. [Recuperado: 02-abr-2025].

    \bibitem{ref57}
    D. M. Campos Gavilanez, “Análisis Comparativo entre Sistemas Operativos de Dispositivos Móviles Android, iPhone OS y MIUI”, Universidad Técnica de Babahoyo, Facultad de Administración, Finanzas e Informática, 2023. [En línea]. Disponible en: \url{https://dspace.utb.edu.ec/bitstream/handle/49000/13969/E-UTB-FAFI-SIST-000414.pdf?sequence=1&isAllowed=y}. [Recuperado: 07-mar-2025].

    \bibitem{ref58}
    M. García, “La historia del logo de Android”, brandemia, 29-ene-2024. [En línea]. Disponible en: \url{https://brandemia.org/la-historia-del-logo-de-android}. [Accedido: 04-abr-2025].

    \bibitem{ref59}
    M. Budziński, “What is React Native? Complex Guide for 2024”, Net Guru, 2025. [En línea]. Disponible en: \url{https://www-netguru-com.translate.goog/glossary/react-native?_x_tr_sl=en&_x_tr_tl=es&_x_tr_hl=es&_x_tr_pto=tc}. [Recuperado: 08-may-2025].

    \bibitem{ref60}
    React Native, “Get Started with React Native”, React Native, 2025. [En línea]. Disponible en: \url{https://reactnative.dev/docs/environment-setup}. [Recuperado: 08-may-2025].

    \bibitem{refexpo1}
    Campus MVP, “React Native y Expo: qué son y cómo se relacionan”, Campus MVP, 2023. [En línea]. Disponible en: \url{https://www.campusmvp.es/recursos/post/react-native-y-expo-que-son-y-como-se-relacionan.aspx?srsltid=AfmBOooedUkFdB0lHugUXv-owa7hpAovIL_Q3HY2t_HsIAfgi_oig7D7}. [Recuperado: 02-nov-2025].

    \bibitem{refexpologo}
    Y. Smirnov, “React Native is Embracing Frameworks”, Notificare, 20-sep-2024. [En línea]. Disponible en: \url{https://notificare.com/blog/2024/09/20/react-native-and-expo/}. [Recuperado: 02-nov-2025].

    \bibitem{refexpo2}
    V. Odukwe, “Getting Started with React Native (EXPO): A Beginner's Guide”, Dev, 10-oct-2024. [En línea]. Disponible en: \url{https://dev.to/vrinch/getting-started-with-react-native-expo-a-beginners-guide-4ae8}. [Recuperado: 02-nov-2025].

    \bibitem{refexpo3}
    Scanbot SDK, “What is the Expo framework and should you use it for React Native development?”, Scanbot SDK, 01-oct-2024. [En línea]. Disponible en: \url{https://scanbot.io/techblog/what-is-expo-for-react-native/}. [Recuperado: 02-nov-2025].

    \bibitem{refexpo4}
    Ideo Software, “React Native – What is Expo and is it worth using?”, Ideo Software, s.f. [En línea]. Disponible en: \url{https://www.ideosoftware.com/blog/react-native-what-is-expo-and-is-it-worth-using,275.html}. [Recuperado: 02-nov-2025].

    \bibitem{refapi1}
    AWS, “¿Qué es una interfaz de programación de aplicaciones (API)?”, Amazon, 2025. [En línea]. Disponible en: \url{https://aws.amazon.com/es/what-is/api/}. [Recuperado: 02-nov-2025].

    \bibitem{refapi2}
    A. Ken, “¿Qué es una API y para qué sirve?”, Gluo, 12-jul-2023. [En línea]. Disponible en: \url{https://www.gluo.mx/blog/que-es-una-api-y-para-que-sirve}. [Recuperado: 02-nov-2025].

    \bibitem{refapi3}
    SAP, “¿Qué es una API (interfaz de programación de aplicaciones)?”, SAP, 2025. [En línea]. Disponible en: \url{https://www.sap.com/latinamerica/products/technology-platform/integration-suite/what-is-api.html}. [Recuperado: 02-nov-2025].

    \bibitem{refapi4}
    Salesforce, “Qué es una API: definición, retos”, Salesforce, s.f. [En línea]. Disponible en: \url{https://www.salesforce.com/es/resources/articles/definition-api/}. [Recuperado: 02-nov-2025].

    \bibitem{refapi5}
    L. Gupta, “What is REST?”, REST API Tutorial, 01-Abr-2025. [En línea]. Disponible en: \url{https://restfulapi.net/}. [Recuperado: 23-Nov-2025].

    \bibitem{refapi6}
    AWS, “¿Qué es una API RESTful?”, AWS, s.f. [En línea]. Disponible en: \url{https://aws.amazon.com/es/what-is/restful-api/}. [Recuperado: 23-Nov-2025].

    \bibitem{refmic1}
    Google Cloud, “¿Qué es la arquitectura de microservicios?”, Google Cloud, s.f. [En línea]. Disponible en: \url{https://cloud.google.com/learn/what-is-microservices-architecture?hl=es}. [Recuperado: 02-nov-2025].

    \bibitem{refmic2}
    Azure, “¿Qué son los microservicios?”, Microsoft, s.f. [En línea]. Disponible en: \url{https://azure.microsoft.com/es-es/solutions/microservice-applications#tabx01590c8f9ffb4e108effe063732de07d}. [Recuperado: 02-nov-2025].

    \bibitem{refmic3}
    Atlassian, “Microservicios: qué son y qué ventajas tienen”, Atlassian, s.f. [En línea]. Disponible en: \url{https://www.atlassian.com/es/microservices}. [Recuperado: 02-nov-2025].

    \bibitem{refmic4}
    Red Hat, “¿Qué son y para qué sirven los microservicios?”, Red Hat, 07-feb-2023. [En línea]. Disponible en: \url{https://www.redhat.com/es/topics/microservices}. [Recuperado: 02-nov-2025].
    
    \bibitem{refmic5}
    YArquitectura, “Arquitectura De Microservicios Java”, YArquitectura, s.f. [En línea]. Disponible en: \url{https://www.yarquitectura.com/arquitectura-de-microservicios-java/}. [Recuperado: 02-nov-2025].

    \bibitem{refnub1}
    Nutanix, “¿Qué es un servicio de nube?”, Nutanix, 19-jul-2025. [En línea]. Disponible en: \url{https://www.nutanix.com/es_mx/info/what-is-a-cloud-service#}. [Recuperado: 03-nov-2025].

    \bibitem{refnub2}
    Cloudflare, “¿Qué es la computación en la nube?”, Cloudflare, s.f. [En línea]. Disponible en: \url{https://www.cloudflare.com/es-es/learning/cloud/what-is-the-cloud/}. [Recuperado: 03-nov-2025].

    \bibitem{refnub3}
    Google Cloud, “¿Qué es un proveedor de servicios en la nube?”, Google, s.f. [En línea]. Disponible en: \url{https://cloud.google.com/learn/what-is-a-cloud-service-provider?hl=es}. [Recuperado: 03-nov-2025].

    \bibitem{refnub4}
    Red Hat, “Servicios de nube gerenciados”, Red Hat, 10-jul-2023. [En línea]. Disponible en: \url{https://www.redhat.com/es/topics/cloud-computing/what-are-cloud-services}. [Recuperado: 03-nov-2025].

    \bibitem{refnub5}
    Stark Cloud, “¿Qué son los Servicios en la Nube? Explicación y Ejemplos”, Stark Cloud, 07-ene-2025. [En línea]. Disponible en: \url{https://starkcloud.com/starkcloud-blog/que-son-los-servicios-en-la-nube-explicacion-y-ejemplos/}. [Recuperado: 03-nov-2025].

    \bibitem{refaws1}
    M. Gimenez, “Amazon Web Services (AWS): ¿qué es y qué ofrece?”, Hiberus Blog, 20-jul-2020. [En línea]. Disponible en: \cite{https://www.hiberus.com/crecemos-contigo/amazon-web-services-aws-que-es-y-que-ofrece/}. [Recuperado: 03-nov-2025].

    \bibitem{refaws2}
    The Information Lab, “Qué es Amazon Web Services y para qué sirve”, The Information Lab, 06-abr-2021. [En línea]. Disponible en: \url{https://www.theinformationlab.es/blog/que-es-amazon-web-services-y-para-que-sirve/}. [Recuperado: 03-nov-2025].

    \bibitem{refaws3}
    V. Singh, “¿Qué es AWS? Introducción a Amazon Web Services”, Datacamp, 26-feb-2025. [En línea]. Disponible en: \url{https://www.datacamp.com/es/blog/what-is-aws}. [Recuperado: 03-nov-2025].

    \bibitem{refaws4}
    J. Lopez Lopez, “Conociendo AWS cloud provider”, Cloud\&Bytes, 2023. [En línea]. Disponible en: \url{https://cloudnbytes.com/tech/conociendo-aws-cloud-provider/}. [Recuperado: 03-nov-2025].

    \bibitem{ref61}
    J. Nielsen, “10 Usability Heuristics for User Interface Design”, Nielsen Norman Group, 1994. [En línea]. Disponible en: \url{https://www-nngroup-com.translate.goog/articles/ten-usability-heuristics/?_x_tr_sl=en&_x_tr_tl=es&_x_tr_hl=es&_x_tr_pto=tc}. [Recuperado: 15-may-2025].

    \bibitem{ref62}
    International Organization for Standardization, “ISO 9241-210:2019”, ISO, 2019. [En línea]. Disponible en: \url{https://www-iso-org.translate.goog/standard/77520.html?_x_tr_sl=en&_x_tr_tl=es&_x_tr_hl=es&_x_tr_pto=tc}. [Recuperado: 15-may-2025].

    \bibitem{ref63}
    Glassdoor, “Sueldos para desarrollador en México”, Glassdoor LLC, 2025. [En línea]. Disponible en: \url{https://www.glassdoor.com.mx/Sueldos/desarrollador-full-stack-junior-sueldo-SRCH_KO0,31.htm}. [Recuperado: 15-may-2025].

    \bibitem{ref64}
    Glassdoor, “Sueldos para Desarrollador De Backend en México”, Glassdoor LLC, 2025. [En línea]. Disponible en: \url{https://www.glassdoor.com.mx/Sueldos/desarrollador-de-backend-sueldo-SRCH_KO0,24.htm}. [Recuperado: 16-may-2025].

    \bibitem{ref65}
    Talent.com, “Salario medio para Desarrollador Back End en México 2025”, Talent.com, 2025. [En línea]. Disponible en: \url{https://mx.talent.com/salary?job=Desarrollador+Back+End}. [Recuperado: 15-may-2025].

    \bibitem{ref66}
    Talent.com, “Salario medio para Desarrollador Front End en México 2025”, Talent.com, 2025. [En línea]. Disponible en: \url{https://mx.talent.com/salary?job=Desarrollador+Front+End}. [Recuperado: 15-may-2025].

    \bibitem{ref67}
    OCC, “Bolsa de trabajo de AI en México”, OCC, 2025. [En línea]. Disponible en: \url{https://www.occ.com.mx/empleos/de-ai/en-mexico/?jobid=20351511}. [Recuperado: 15-may-2025].

    \bibitem{ref68}
    Jooble, “Salario de Animador 3D”, Jooble, 2025. [En línea]. Disponible en: \url{https://mx.jooble.org/salary/animador-3d}. [Recuperado: 15-may-2025]. 

    \bibitem{ref69}
    Redacción 24 horas, “¿Cuánto cobra un traductor en lengua de señas en México?”, 24 horas, 2023. [En línea]. Disponible en: \url{https://24-horas.mx/vida/cuanto-cobra-un-traductor-en-lenguaje-de-senas-en-mexico/}. [Recuperado: 15-may-2025].

    \bibitem{ref70}
    Indeed, “Sueldo de Devops en México”, Indeed, 2025. [En línea]. Disponible en: \url{https://mx.indeed.com/career/devops/salaries}. [Recuperado: 15-may-2025].

    \bibitem{ref71}
    Jobted, “Sueldo de un Project Manager en México”, Jobted, 2025. [En línea]. Disponible en: \url{https://www.joted.com.mx/salario/project-manager}. [Recuperado: 15-may-2025].

    \bibitem{ref72}
    Computrabajo, “Salarios de Analista funcional en México”, Computrabajo, 2025. [En línea]. Disponible en: \url{https://mx.computrabajo.com/salarios/analista-funcional}. [Recuperado: 15-may-2025].

    \bibitem{ref73}
    Hireline, “ ¿Qué es un Diseñador UX?”, Hireline, 2025. [En línea]. Disponible en: \url{https://hireline.io/mx/enciclopedia-de-perfiles-de-tecnologia/disenador-ux}. [Recuperado: 15-may-2025].

    \bibitem{ref74}
    Jobted, “Sueldo de un Full Stack Desarrollador en México”, Jobted, 2025. [En línea]. Disponible en: \url{https://www.jobted.com.mx/salario/full-stack-developer}. [Recuperado: 15-may-2025].

    \bibitem{ref75}
    Computrabajo, “Salarios de Finanzas en México”, Computrabajo, 2025. [En línea]. Disponible en: \url{https://mx.computrabajo.com/salarios/finanzas}. [Recuperado: 15-may-2025].

    \bibitem{ref76}
    Indeed, “Sueldo de Devops en México”, Indeed, 2025. [En línea]. Disponible en: \url{https://mx.indeed.com/career/devops/salaries}. [Recuperado: 15-may-2025].

    \bibitem{ref77}
    Computrabajo, “Salarios de Redactor/a en México”, Computrabajo, 2025. [En línea]. Disponible en: \url{https://mx.computrabajo.com/salarios/redactora}. [Recuperado: 15-may-2025].

    \bibitem{refcocomo}
    Y. Miranda Silva, “MODELO COCOMO (INGENIERA DE SOFTWARE)”, SlideShare, s.f. [En línea]. Disponible en: \url{https://es.slideshare.net/slideshow/modelo-cocomo-ingeniera-de-software/55338602}. [Recuperado: 15-may-2025].

    \bibitem{ref78}
    I. Sommerville, “Ingeniería del software”, 7.ª ed. Madrid, España: Pearson Educación, 2005, ISBN 84-7829-074-5, 712 págs.

    \bibitem{refimpl1}
    Pydantic, “Pydantic Validation, Pydantic, s.f. [En línea]. Disponible en: \url{https://docs.pydantic.dev/latest/}. [Recuperado: 24-nov-2025].

    \bibitem{refimpl2}
    S. Ramírez Montaño, “FastAPI”, FastAPI, 2018. [En línea]. Disponible en: \url{https://fastapi.tiangolo.com/#additional-optional-dependencies}. [Recuperado: 24-nov-2025].

    \bibitem{refimpl3}
    Uvicorn,  “Uvicorn, an ASGI web server, for Python.”, Uvicorn, s.f. [En línea]. Disponible en: \url{https://uvicorn.dev/}. [Recuperado: 24-nov-2025].

    \bibitem{refimpl4}
    M. A. Magaña Fuentes, “RapidFuzz: Potencia tu Análisis de Datos con Comparación de Texto Fácil y Eficiente - Parte I”, LinkedIn, 27-nov-2024. [En línea]. Disponible en: \url{https://es.linkedin.com/pulse/rapidfuzz-potencia-tu-an%C3%A1lisis-de-datos-con-texto-y-maga%C3%B1a-fuentes-0j1xc}. [Recuperado: 24-nov-2025]

    \bibitem{refimpl5}
    Md5decrypt.net, “Leet Speak Translator”, Md5decrypt.net, 2024. [En línea]. Disponible en: \url{https://md5decrypt.net/en/Leet-translator/}. [Recuperado: 25-nov-2025].

    \bibitem{refimpl6}
    Refactoring.Guru, “El catálogo de patrones de diseño”, Refactoring.Guru, 2025. [En línea]. Disponible en: \url{https://refactoring.guru/es/design-patterns/catalog}. [Recuperado: 25-nov-2025].

    \bibitem{refimpl7}
    D. Sabalete Rodríguez, “Los principios SOLID de programación orientada a objetos explicados en Español sencillo”, freeCodeCamp, 28-nov-2022. [En línea]. Disponible en: \url{https://www.freecodecamp.org/espanol/news/los-principios-solid-explicados-en-espanol/}. [Recuperado: 24-nov-2025].

    \bibitem{refimpl8}
    Python, “Welcome to Python”, Python, s.f. [En línea]. Disponible en: \url{https://www.python.org/}. [Recuperado: 24-nov-2025].

    \bibitem{refimpl9}
    NumPy Team, “Numpy: the fundamental package for scientific computing with Python”, Numpy, 2025. [En línea]. Disponible en: \url{https://numpy.org/}. [Recuperado: 24-nov-2025].

    \bibitem{refimpl10}
    PyTorch, “Get Started with PyTorch”, PyTorch, 2025. [En línea]. Disponible en: \url{https://pytorch.org/}. [Recuperado: 24-nov-2025].

    \bibitem{refimpl11}
    Swagger, “The Future of AI Relies on API Quality”, Swagger, 2025. [En línea]. Disponible en: \url{https://swagger.io/}. [Recuperado: 24-nov-2025].

    \bibitem{refimpl12}
    Swagger, “OpenAPI Specification”, Swagger, 2025. [En línea]. Disponible en: \url{http://swagger.io/specification/}. [Recuperado: 24-nov-2025].

    \bibitem{refimpl13}
    Sbert.net, “SentenceTransformers Documentation”, Sbert.net., s.f. [En línea]. Disponible en: \url{https://sbert.net/}. [Recuperado: 24-nov-2025].  

    \bibitem{refimpl14}
    Scikit-learn, “scikit-learn, Machine Learning in Python”, scikit-learn, s.f. [En línea]. Disponible en: \url{https://scikit-learn.org/stable/}. [Recuperado: 24-nov-2025]. 

    \bibitem{refimpl15}
    Pytest, “pytest: helps you write better programs”, Pytest, 2015. [En línea]. Disponible en: \url{https://docs.pytest.org/en/stable/}. [Recuperado: 24-nov-2025].
    
    \bibitem{refimpl16}
    Git, “Git, versión control system”, Git, s.f. [En línea]. Disponible en: \url{https://git-scm.com/}. [Recuperado: 24-nov-2025]. 

    \bibitem{refimpl17}
    J. A. Alonso, “La distancia Levenshtein (con programación dinámica)”, Exercitium, 04-oct-2023. [En línea]. Disponible en: \url{https://www.glc.us.es/~jalonso/exercitium/04-oct-23/}. [Recuperado: 25-nov-2025].

    \bibitem{ref49}
    Nanobaly, “MediaPipe: la caja de herramientas esencial para la Computer Vision”, Innovatiana, 21-ago-2024. [En línea]. Disponible en: \url{https://es.innovatiana.com/post/mediapipe-101}. [Recuperado: 07-mar-2025].

    \bibitem{ref50}
    N, Buhl, “Google’s MediaPipe Framework: Deploy Computer Vision Pipelines with Ease”, Encord, 21-jun-2024. [En línea]. Disponible en: \url{https://encord.com/blog/google-mediapipe/}. [Accedido: 04-abr-2025].

    \bibitem{ref51}
    MediaPipe, “MediaPipe Hands”, MediaPipe, 2025. [En línea]. Disponible en: \url{https://mediapipe.readthedocs.io/en/latest/solutions/hands.html}. [Recuperado: 20-abr-2025].

    \bibitem{ref52}
    Google AI for Developers, “Hand landmarks detection guide”, Google, 2025. [En línea]. Disponible en: \url{https://ai.google.dev/edge/mediapipe/solutions/vision/hand_landmarker?hl=es-419}. [Recuperado: 20-abr-2025].

    \bibitem{ref53}
    Google AI for Developers, “Guía de detección de puntos de referencia de posiciones”, Google, 2025. [En línea]. Disponible en: \url{https://ai.google.dev/edge/mediapipe/solutions/vision/pose_landmarker?hl=es-419}. [Recuperado: 07-mar-2025].

    \bibitem{refhol1}
    Google AI For Developers, “Holistic landmarks detection task guide”, Google, 24-abr-2024. [En línea]. Disponible en: \url{https://ai.google.dev/edge/mediapipe/solutions/vision/holistic_landmarker}. [Recuperado: 04-nov-2025].

    \bibitem{refhol2}
    S. Harisudhan, “Media Pipe- Exploring Holistic Model”, Medium, 06-mar-2024. [En línea]. Disponible en: \url{https://medium.com/@speaktoharisudhan/media-pipe-exploring-holistic-model-32b851901f8a}. [Recuperado: 04-nov-2025].

    \bibitem{refhol3}
    Geeks for Geeks, “Python - Facial and hand recognition using MediaPipe Holistic”, Geeks for Geeks, 04-ene-2023. [En línea]. Disponible en: \url{https://www.geeksforgeeks.org/machine-learning/python-facial-and-hand-recognition-using-mediapipe-holistic/}. [Recuperado: 04-nov-2025].

    \bibitem{ref54}
    J. S. Viñolo Locubiche, “El modelo de producción industrial de animación 3D estadounidense”, Univ. de Barcelona, 20-jun-2017. [En línea]. Disponible en: \url{https://diposit.ub.edu/dspace/handle/2445/115650}. [Recuperado: 09-mar-2025].

    \bibitem{ref55}
    J. Nava, “Qué es Unity y para qué sirve este motor gráfico”, Empower Talent, 2024. [En línea]. Disponible en: \url{https://empowertalent.com/que-es-unity/}. [Recuperado: 20-abr-2025].

    \bibitem{ref56}
    Cool Efect, “Unity 3D”. [En línea]. Disponible en: \url{https://www.cooleffect.org/unity-3d}. [Accedido: 04-abr-2025].

    \bibitem{refbl1}
    The Core, “¿Qué es Blender y cómo puedes sacarle partido para la animación 3D?”, The Core, s.f. [En línea]. Disponible en: \url{https://www.thecoreschool.com/blog/que-es-blender-y-como-puedes-sacarle-partido-para-la-animacion-3d/}. [Recuperado: 04-nov-2025].

    \bibitem{refbl2}
    J. Nava, “Qué es blender? sus usos y beneficios”, Empower Talent, 22-jul-2024. [En línea]. Disponible en: \url{https://empowertalent.com/que-es-blender/}. [Recuperado: 04-nov-2025].

    \bibitem{refbl3}
    Universidad Europea, “Blender 3D: qué es, usos y diferencias con Cinema 4D”, Universidad Europea, 04-sep-2024. [En línea]. Disponible en: \url{https://creativecampus.universidadeuropea.com/blog/blender/}. [Recuperado: 04-nov-2025].

    \bibitem{refbl4}
    A. Chirivella González, “Blender, qué es y para qué se utiliza”, Profesional Review, 20-feb-2022. [En línea]. Disponible en: \url{https://www.profesionalreview.com/2022/02/20/blender-que-es-y-para-que-se-utiliza/}. [Recuperado: 04-nov-2025].

    \bibitem{refbl5}
    1000 Marcas, “BLENDER LOGO”, 1000 Marcas, 01-jun-2022. [En línea]. Disponible en: \url{https://1000marcas.net/blender-logo/}. [Recuperado: 04-nov-2025].
\end{thebibliography}