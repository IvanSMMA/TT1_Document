\documentclass[12pt, letterpaper,oneside]{book}
\usepackage{booktabs}
% \usepackage{float}  % Para usar el modificador H
\usepackage{graphicx}         %gráficas eps
\usepackage{subfigure}        %subfiguras con títulos a) b) ...
\usepackage[utf8]{inputenc}   %letras acentuadas
\usepackage[spanish]{babel}   %nombre de capítulos, secciones, 
\usepackage{amsmath}
\usepackage[ruled,vlined]{algorithm2e}
\usepackage{array} % para usar m{}
\usepackage{makecell} % para centrar texto en celdas
\usepackage{rotating}
\usepackage{caption}
\usepackage{amssymb}
\usepackage{longtable}
\usepackage{afterpage}
\usepackage[spanish]{nomencl}
\usepackage{textcomp}
\usepackage{cite}
\usepackage{setspace}
\usepackage{float}
\usepackage{longtable}
\captionsetup[longtable]{position=bottom}
\usepackage{pdfpages}
\usepackage[hidelinks]{hyperref}
\usepackage{titlesec}
\usepackage{tikz}
\usepackage{tabularx}
\usepackage{adjustbox}
\usepackage{multirow}
\usepackage{siunitx}
\usepackage{indentfirst}
\usepackage{url}        % permite cortar URLs
         % para ajustar entorno thebibliography
\usepackage{hyperref}
\usepackage{nameref}
\usepackage{pifont}
\usepackage{inconsolata} % o cualquier fuente monoespaciada
\usepackage[utf8]{inputenc}
\usepackage[T1]{fontenc}
\usepackage{listings}
\usepackage{xcolor}


\lstset{
    language=Python,
    basicstyle=\ttfamily\small,
    keywordstyle=\color{blue},
    stringstyle=\color{red},
    commentstyle=\color{gray},
    showstringspaces=false,
    breaklines=true,
    breakatwhitespace=true,
    frame=single,
    inputencoding=utf8,
    extendedchars=true,
    literate=%
      {á}{{\'a}}1
      {é}{{\'e}}1
      {í}{{\'i}}1
      {ó}{{\'o}}1
      {ú}{{\'u}}1
      {Á}{{\'A}}1
      {É}{{\'E}}1
      {Í}{{\'I}}1
      {Ó}{{\'O}}1
      {Ú}{{\'U}}1
      {ñ}{{\~n}}1
      {Ñ}{{\~N}}1
      {¿}{{\textquestiondown}}1
      {¡}{{\textexclamdown}}1
      {→}{{$\rightarrow$}}1
      {↓}{{$\downarrow$}}1
      {✓}{{$\checkmark$}}1
      {✗}{{$\crossmark$}}1
      {•}{{$\point$}}1
      {₂}{{$\doschiquito$}}1
}



% Redefinir la palabra "Apéndice" por "Anexo"
\addto\captionsspanish{%
  \renewcommand{\appendixname}{Anexo}%
}


\usetikzlibrary{shapes.geometric, arrows}

\tikzstyle{startstop} = [rectangle, rounded corners, minimum width=3cm, minimum height=1cm,text centered, draw=black, fill=gray!30]
\tikzstyle{process} = [rectangle, minimum width=3cm, minimum height=1cm, text centered, draw=black, fill=blue!20]
\tikzstyle{decision} = [diamond, minimum width=3.5cm, minimum height=1cm, text centered, draw=black, fill=yellow!30]
\tikzstyle{arrow} = [thick,->,>=stealth]

\titleformat{\section}
  {\normalfont\Large\bfseries}{\thesection}{1em}{}

\titleformat{\subsection}
  {\normalfont\large\bfseries}{\thesubsection}{1em}{}

\titleformat{\subsubsection}
  {\normalfont\normalsize\bfseries\itshape}{\thesubsubsection}{1em}{}


\titlespacing*{\section}{1pt}{*1}{*1}

\usepackage{enumitem}  % Para personalizar listas enumeradas
\usepackage{xcolor}    % Para dar color a los textos

\usepackage[top=3cm,bottom=3cm,left=4cm,right=3cm]{geometry}

\newcommand{\UCactor}{\includegraphics[height=1em]{Images/actor.png}}  % Ícono de usuario
\newcommand{\UCsystem}{\includegraphics[height=1em]{Images/usecase.png}} % Ícono de sistema

\usepackage{etoolbox}
\captionsetup{font=footnotesize}

\makeatletter
\patchcmd{\LT@makecaption}
  {\hrule}
  {}
  {}{}
\makeatother


% Title Page
\title{Tema de tesis}
\author{Estudiante}


\begin{document}
	%Modificaciones para los nombres de algunos títulos
	\renewcommand{\listtablename}{{Índice de tablas}}
	\renewcommand{\tablename}{Tabla}
%	\renewcommand{\algorithmcfname}{Algoritmo}
	
\includepdf[pages={1}]{portada}
	
\clearpage
\thispagestyle{empty}

%================= ENCABEZADO CON LOGOS =================%
\begin{minipage}[c]{0.2\textwidth}
    \centering
    \includegraphics[height=2.5cm]{Images/logo_ipn.jpg}
\end{minipage}
\begin{minipage}[c]{0.6\textwidth}
    \centering
    {\small
    \textbf{INSTITUTO POLITÉCNICO NACIONAL}\\
    \textbf{ESCUELA SUPERIOR DE CÓMPUTO}\\
    \textbf{SUBDIRECCIÓN ACADÉMICA}
    }
\end{minipage}
\begin{minipage}[c]{0.2\textwidth}
    \centering
    \includegraphics[height=2.5cm]{Images/escudoESCOM.png}
\end{minipage}

\vspace{0.6cm}

%================= DATOS =================%
\begin{minipage}{0.45\textwidth}
    \small No. de TT: 2025-B042
\end{minipage}
\hfill
\begin{minipage}{0.45\textwidth}
    \small
    \raggedleft
    10 de diciembre de 2025
\end{minipage}

\vspace{0.8cm}

\begin{center}
    {\small Documento Técnico}\\[0.3cm]
    {\large \textbf{``Prototipo de aplicación móvil de apoyo para la traducción de español a Lengua de Señas Mexicana (LSM), empleando técnicas de Procesamiento de Lenguaje Natural (PLN) y Modelado 3D''}}
\end{center}

\vspace{0.3cm}

%================= AUTORES =================%
\begin{center}
    \textit{Presentan}\\[0.2cm]
    Sánchez Mancilla Ivan Emiliano\footnote{isanchezm1802@alumno.ipn.mx}\\
    Juárez Solano Juan Martin\footnote{jjuarezs2101@alumno.ipn.mx}\\
    Rojas Alarcón Sergio Ulises\footnote{srojasa1800@alumno.ipn.mx}
\end{center}

\vspace{0.3cm}

%================= DIRECTORES =================%
\begin{center}
    \textit{Directores}
\end{center}

\vspace{0.3cm}

\begin{minipage}{0.45\textwidth}
    \centering \textbf{M. en C. Jaime Arturo Lara Cázares}
\end{minipage}
\hfill
\begin{minipage}{0.45\textwidth}
    \centering \textbf{M. en C. Elizabeth Moreno Galván}
\end{minipage}

\vspace{0.4cm}

%================= RESUMEN =================%
\begin{center}
    \textbf{RESUMEN}
\end{center}

\noindent
Este proyecto aborda las limitaciones de las tecnologías actuales para traducir la Lengua de Señas Mexicana (LSM), las cuales suelen carecer de fluidez en las animaciones y de una usabilidad adecuada. Este prototipo de aplicación móvil empleará técnicas de Inteligencia Artificial (IA) y animaciones para traducir frases esenciales relacionadas con situaciones de emergencia, saludos y expresiones de mínima comunicación. El objetivo principal es promover la inclusión social de la comunidad con discapacidad auditiva en México, mediante una interfaz accesible que cumpla con la norma ISO 9241-210 y las heurísticas de usabilidad de Nielsen, garantizando una experiencia de traducción fluida y natural del lenguaje de señas.

\vspace{0.3cm}

\noindent
\textbf{Palabras clave:} Accesibilidad tecnológica, animaciones, procesamiento de lenguaje natural, aplicación móvil.


\newpage
\thispagestyle{empty} % Sin encabezados ni número de página
\begin{center}
    \vspace*{\fill} % Empuja hacia el centro vertical
    \includegraphics[width=1\textwidth]{Images/advertencia.png} % tu imagen
    \vspace*{\fill} % Empuja desde abajo hacia el centro
\end{center}

\newpage

\addtocontents{toc}{\hfill \textbf{Página} \par}
\tableofcontents
\listoffigures     % Índice de figuras
\listoftables      % Índice de tablas

% \let\cleardoublepage\clearpage
\chapter{Introducción}
\section{Motivación}
En México la comunicación es un derecho fundamental para todas las personas, ya que mediante ella se pueden establecer vínculos, e intercambiar información y pensamientos. Sin embargo, las personas de la comunidad sorda enfrentan barreras para poder comunicarse de manera efectiva. \\

Si bien la Lengua de Señas Mexicana (LSM) es el principal medio de comunicación para las personas con discapacidad auditiva en México en interacciones cotidianas o en situaciones de emergencia, las personas dependen de soluciones tecnológicas para poder comunicarse. Existen pocas herramientas digitales que permiten la traducción de español a LSM, pero en su mayoría no son precisas, y en el caso de las que emplean animaciones 3D, las mismas se muestran fragmentadas causando que se dificulte la comprensión de los mensajes.\\

La motivación de este Trabajo Terminal es mejorar la accesibilidad de las personas sordas, a la par que se fomenta la inclusión social y tecnológica para esta comunidad, empleando técnicas de Inteligencia Artificial, Procesamiento de Lenguaje Natural y Modelado 3D para obtener una traducción más fluida y natural.\\

\section{Problemática}
La comunicación es una habilidad fundamental que permite a los seres humanos poder interactuar con su entorno y compartir ideas a través de un sistema de símbolos [1]. No obstante, a menudo existen barreras en la comunicación que limitan el acceso a la información y dificultan la capacidad de las personas para poder expresarse y comunicarse [2]. \\

A pesar de las barreras, las personas sordas han desarrollado su propio medio de comunicación que les permite superar las limitaciones fisiológicas y establecer una conexión efectiva con quienes comparten este lenguaje. En el caso de México, la LSM es el principal medio de comunicación para las personas sordas en el país, con un estimado de 2.3 millones personas que padecen de esta discapacidad [3], lo que evidencia la magnitud de la población que depende de este lenguaje para poder llevar a cabo el proceso de comunicación. Esta lengua posee su propia sintaxis, gramática, léxico y está compuesta por una combinación de señas, expresiones faciales y movimientos corporales que permiten transmitir ideas, mensajes, emociones y sentimientos [4].\\

La Ley General para la Inclusión de las Personas con Discapacidad establece que los medios de comunicación deben implementar tecnologías o intérpretes de Lengua de Señas Mexicana para facilitar el acceso a contenido para la comunidad sorda [4]. En este contexto, las ciencias de la computación enfrentan un reto importante en cuanto a la creación de recursos digitales que respondan a estas necesidades específicas de accesibilidad e inclusión, concretamente en tareas de procesamiento de lenguaje y traducción automática.\\

Actualmente existen diversos proyectos relacionados con la traducción de Lengua de Señas, sin embargo, la mayoría presentan limitaciones significativas. Una de las principales dificultades es la falta de fluidez en las animaciones de la traducción del español a LSM, ya que los traductores actuales no logran representar el lenguaje de señas de manera continua, Como resultado, las animaciones suelen presentar cortes entre palabras o frases, lo que resulta en una traducción fragmentada y poco natural.\\

Estas deficiencias afectan la comprensión de los mensajes por parte de los usuarios, ya que la secuencia de señas no fluye con la rapidez y precisión necesarias para facilitar una comunicación efectiva. Además, algunos de estos proyectos están diseñados para versiones anteriores de Android y no funcionan en las versiones más recientes del sistema operativo.\\


\section{Objetivos}
\subsection{Objetivo General}
Desarrollar un prototipo de aplicación móvil en Android que, utilizando técnicas de Procesamiento de Lenguaje Natural y Modelado 3D, traduzca oraciones específicas empleadas en situaciones de emergencia, así como expresiones cotidianas como saludos y agradecimientos, del español a la Lengua de Señas Mexicana (LSM).

\subsection{Objetivo Especifícos}
\begin{itemize}
 \item Desarrollar el módulo de procesamiento de texto mediante técnicas de procesamiento de lenguaje natural, para interpretar oraciones específicas en español y transformarlas en sentencias manipulables para la traducción a la Lengua de Señas Mexicana (LSM).
    \item Construir un módulo de animación 3D que represente visualmente las señas en Lenguaje de Señas Mexicana (LSM), a partir de las oraciones procesadas.
    \item Implementar el módulo de animación 3D utilizando Inteligencia Artificial, con el fin de optimizar significativamente la fluidez entre las señas, buscando obtener una representación fluida del lenguaje de señas en los avatares.
    \item Crear una aplicación móvil en Android que integre los módulos de procesamiento de lenguaje natural, módulo de generación de animaciones 3D y transiciones fluidas entre los componentes del prototipo.
    \item Validar la funcionalidad y usabilidad de la aplicación móvil mediante pruebas con personas con discapacidad auditiva y personas oyentes, en escenarios simulados de emergencia, evaluando la precisión de la traducción y la fluidez de las animaciones.
\end{itemize}

\section{Alcance}
\subsection{Alcance Genral}
El Trabajo Terminal consiste en el desarrollo de un prototipo de aplicación móvil que traduzca frases de español a LSM mediante animaciones 3D. La aplicación tendrá un conjunto predefinido de frases comunes para saludos y situaciones de emergencia, y en caso de no contar con una frase específica, se empleará la dactilología (representación de las letras de una palabra empleando las manos) para poder garantizar la comunicación.

\subsection{Alcance Especifíco}
Se hará uso de una interfaz de usuario simple e intuitiva, que sea de utilidad para la población objetivo. Mediante técnicas de Procesamiento de Lenguaje Natural (PLN) se pretende hacer un procesamiento del dataset que contiene las frases en español, para posteriormente realizar el modelado de las animaciones 3D empleando Mediapipe; dichas animaciones serán fluidas para mostrar una comunicación natural y fácil de entender.

\subsection{Delimitación}
Es importante aclarar que la traducción será únicamente de un canal de comunicación: de español a LSM. Lo anterior por la razón de que la traducción de LSM a español es más compleja por las diferencias estructurales y gramaticales entre ambos lenguajes.\\

\section{Justificación}
En México la comunicación inclusiva es un reto para las personas con discapacidad auditiva, debido a que la mayoría de la población no conoce la Lengua de Señas Mexicana (LSM), lo que condiciona a la comunidad sorda para poder expresar sus ideas y pensamientos, acceder a servicios esenciales o recibir apoyo en situaciones de emergencia. Pese a que ha habido avances tecnológicos a lo largo de los últimos años, las herramientas de traducción de español a LSM existentes presentan limitaciones, siendo las más importantes la falta de fluidez en las animaciones y la incompatibilidad entre versiones del sistema operativo de Android. \\

Este Trabajo Terminal busca desarrollar un prototipo de aplicación móvil que traduzca frases en español a LSM con animaciones en 3D fluidas, empleando técnicas de PLN y modelado 3D. Al mejorar la calidad de la comunicación entre personas oyentes y personas con discapacidad auditiva, se contribuye a la accesibilidad, a eliminar barreras de comunicación y fomentar la comunicación inclusiva.\\

Se espera que este desarrollo sirva de referencia para sentar las bases de futuras investigaciones relacionadas a la traducción e interpretación de lenguas de señas empleando técnicas de Inteligencia Artificial (IA).\\

\section{Propuesta de solución}
Este proyecto consiste en desarrollar un prototipo que aborde las cuestiones planteadas en el apartado de la problemática, con un enfoque en mejorar la fluidez y precisión de las animaciones de LSM mediante el uso de la Inteligencia Artificial. La aplicación busca facilitar la interacción entre personas oyentes y personas con discapacidad auditiva, integrando técnicas de procesamiento de lenguaje natural y modelado 3D.\\

El traductor estará enfocado en un conjunto limitado de frases predefinidas, como expresiones de emergencia, saludos y agradecimientos. En caso de no encontrar la frase deseada, el sistema utilizará el alfabeto dactilológico de LSM para deletrear la palabra o frase, asegurando siempre una respuesta en la comunicación.\\

El uso de aplicaciones de Procesamiento de Lenguaje Natural (PLN) es fundamental para el procesamiento del español a LSM, ya que permitirá que el sistema no solo reconozca palabras aisladas, sino también frases y oraciones completas, adaptando la traducción al contexto y mejorando la comunicación en situaciones más complejas.\\

El prototipo estará diseñado para la última versión de Android, debido a que este es el sistema operativo más utilizado en México [5], por lo que la solución propuesta tendrá un mayor alcance y beneficiará a una mayor cantidad de usuarios. De acuerdo con el Instituto Federal de Telecomunicaciones (IFT), el sistema operativo más utilizado por las personas en México es Android con 84.6\%, a comparación de iOS que es utilizado por el 6.8\% [s82].\\

Para el desarrollo del prototipo se utilizará Media Pipe [6], una biblioteca eficiente para procesamiento de gestos y movimientos corporales, ideal para capturar de manera precisa los gestos de la Lengua de Señas Mexicana. Esto permitirá obtener datos precisos de las señas, garantizando una traducción más confiable y fluida, mientras se mantiene un rendimiento óptimo incluso en dispositivos de gama media.\\

Finalmente, se busca crear representaciones visuales en 3D para poder facilitar la visualización de señas. Las personas con discapacidad auditiva podrán visualizar una animación asociada a una de las frases que están incluidas dentro del prototipo.
\newline

\textbf{Productos Esperados}
\begin{itemize}
    \item Dataset normalizado de LSM.
    \item Set de animaciones con avatares 3D fluidos e interactivos.
    \item Aplicación móvil en Android.
    \item Documentación del sistema.
\end{itemize}
\begin{center}
    \includegraphics[width=0.8\textwidth]{Images/diacajanegra.jpg}
    \captionof{figure}{Diagrama del funcionamiento de la App, elaboración propia.}  % Pie de foto manual
\end{center}


\section{Estado del arte}
Se han desarrollado diversas investigaciones y desarrollos de sistemas de software que abordan la problemática de la comunicación entre personas oyentes y no oyentes. A continuación se describen los trabajos relacionados: 
\begin{itemize}
    \item Translation of Spanish Text to Mexican Sign Language Glossed Text Using Rules and Deep Learning. En este trabajo se presenta una arquitectura para traducir de español a Lengua de Señas Mexicana (LSM), cuyos resultados fueron evaluados con las métricas BLEU y WER. En este artículo se encontró que la traducción con técnicas tradicionales tiene un mejor desempeño que el aprendizaje profundo [7].
\item Resource Creation for Automatic Translation System from Texts in Spanish into Mexican Sign Language. Este artículo presenta la creación de recursos lingüísticos para la traducción automática del español escrito a LSM, así como el sistema que la implementa. En colaboración con la Casa de la Cultura de los Sordos (CDMX), se tradujeron 150 oraciones pertenecientes a 13 estructuras gramaticales reconocidas por el sistema, identificando 100 signos que pueden representar una oración completa. Se observó que, ante la falta de un signo específico, las personas sordas tienden a deletrear la palabra. El sistema traduce de forma literal al buscar y reproducir palabras en su base de datos, además de contar con signos que representan expresiones completas [8].
\item Aplicación “Voz y Señas”. Esta aplicación, desarrollada por el Instituto de Pedagogia en conjunto con TecnoPrótesis y Bienestar Incluyente A.C.,  permite traducir la LSM por medio del habla o escribiendo texto. Sus objetivos son favorecer la comunicación entre una persona sorda y una persona oyente, y ser una herramienta auxiliar en los procesos de alfabetización, redacción de textos y comprensión lectora [9]. 
\item Hetah. Servicio en línea que funciona como traductor de Lengua de Señas Colombiana (LSC), que permite la comunicación entre personas oyentes y no oyentes mediante un Avatar 3D. Se debe ingresar la frase que se desea traducir y posteriormente, el avatar realizará la traducción mediante gestos [10]. 
\item SignAloud. Sistema de Inteligencia Artificial que emplea guantes capaces de reconocer gestos de las manos correspondientes a palabras y frases en Lenguaje de Señas Americana. Dichos guantes contienen sensores que registran la posición y el movimiento de las manos, para generar datos que son enviados de forma inalámbrica a una computadora central. La computadora analiza los datos de los gestos empleando redes neuronales, y si los datos coinciden con un gesto, la palabra o frase asociada se pronuncia en un altavoz [11].
\item Sistema traductor de la Lengua de Señas Mexicana mediante dactilología y de español a español signado. Sistema de Inteligencia Artificial capaz de ayudar a entablar un diálogo entre una persona sorda y otra oyente, mediante la traducción de las señas de dactilología a  texto plano en español y de forma análoga, la traducción del texto plano a español a español signado [12].

	

\end{itemize}

\section{Metodología}
El proyecto se desarrollará bajo la metodología Scrum, un marco de trabajo ágil que organiza la colaboración del equipo a través de roles, artefactos y reglas que garantizan su correcta implementación [13][14]. Uno de sus principios clave es la configuración de equipos autogestionados y multifuncionales, lo que permite tomar decisiones autónomas sin depender de directrices externas [14].\\

La estructura de Scrum se basa en ciclos iterativos llamados Sprints, en los cuales se genera un incremento funcional del producto. Cada Sprint se gestiona como un proyecto independiente con objetivos específicos [14]. Además, estos ciclos incluyen cinco elementos clave: reunión de planeación, Daily Scrum, trabajo de desarrollo, revisión del Sprint y retrospectiva del Sprint, asegurando así una mejora continua en cada iteración. Para este desarrollo, los Sprints tendrán una duración de entre dos y cuatro semanas, lo que facilitará una transición progresiva hacia esta metodología y garantizará un avance constante.\\

Otro punto a destacar es que Scrum define tres roles esenciales. En primer lugar, el Scrum Master guía la implementación de la metodología y facilita la resolución de impedimentos sin gestionar directamente el desarrollo [15]. En segundo lugar, el Product Owner representa a los interesados y gestiona el Product Backlog, priorizando las funcionalidades para maximizar el valor del producto [15]. Por último, el equipo de desarrollo transforma estos requerimientos en incrementos funcionales, operando sin jerarquías y con un tamaño ideal de tres a nueve integrantes [15].\\

En este proyecto, el equipo asumirá exclusivamente el rol de desarrolladores, encargándose de transformar los requerimientos en incrementos funcionales del producto. La estructura del equipo se distribuirá en tres áreas especializadas. El desarrollador de animaciones 3D y MediaPipe diseñará avatares, implementará gestos y optimizará las transiciones. El desarrollador de Android estructurará la aplicación, desarrollará interfaces y conectará los módulos. Por su parte, el desarrollador de PLN implementará el procesamiento de lenguaje natural, adaptará las frases al contexto de LSM y optimizará la comunicación con los módulos de animación.\\

Esta distribución permitirá aplicar Scrum de manera eficiente, asegurando un desarrollo iterativo y coordinado, en el que cada integrante contribuirá activamente al avance del proyecto mediante la integración de sus respectivas áreas.


\begin{center}
    \includegraphics[width=0.8\textwidth]{Images/metoscrum.png}
    \captionof{figure}{Metodología Scrum, obtenido de [].}  % Pie de foto manual
\end{center}

\chapter{Marco Teórico: conceptos teóricos}
\section{Comunicación}
La comunicación es un proceso dinámico, en el que participa una fuente o emisor que envía un mensaje a través de un canal o medio a un potencial receptor que, a su vez, puede convertirse también en emisor \cite{ref20}. Cuando se transmite el mensaje de una forma clara y efectiva para el receptor sin generar dudas ni confusiones, se logra un comunicación efectiva \cite{ref21}.\\

Comunicar es el acto que permite establecer relaciones efectivas, compartir experiencias, experimentar emociones y sentimientos, así como hacer que los demás lo experimenten \cite{ref22}. A continuación, se describen los tipos de comunicación que existen.

\subsection{Tipos de Comunicación}
Uno de los tipos de comunicación está basado en si se usan palabras o no, es decir, comunicación verbal o no verbal \cite{ref23}:

\begin{itemize}
    \item \textbf{Comunicación verbal}: se emplean palabras y se lleva a cabo a través del habla o de manera escrita.
    \item \textbf{Comunicación no verbal}: se emplea el lenguaje corporal, gestos, signos no lingüísticos y sonidos que no forman palabras.
\end{itemize}

Otro de los tipos de comunicación son la formal y la informal, las cuales se describen a continuación \cite{ref23}:

\begin{itemize}
    \item \textbf{Formal}: se utiliza un lenguaje especializado y estandarizado, sin errores ni coloquialismos, además de que se toman en cuenta las jerarquías sociales.
    \item \textbf{Informal}: no se emplea lenguaje estandarizado, no se siguen protocolos jerárquicos y se emplean coloquialismos.
\end{itemize}

Un tercer tipo de clasificación es aquella que está basada en el tipo de acto comunicativo, la cual contiene los siguientes elementos \cite{ref23}:
\begin{itemize}
    \item \textbf{Comunicación intrapersonal}: conversaciones que un ser humano entabla consigo mismo.
    \item \textbf{Comunicación interpersonal}: intercambio de ideas y pensamientos entre dos personas, la cuál debe ser directa e interactiva.
    \item \textbf{Comunicación grupal}: intercambio de ideas y pensamientos entre un grupo de más de dos personas, las cuales se comunican con un propósito.
    \item \textbf{Comunicación masiva}: dirigida a un gran número de personas, mediante un medio masivo de comunicación como lo puede ser las redes sociales, radio, televisión, entre otros.
\end{itemize}

\subsection{Elementos de la comunicación}
Dentro del proceso de comunicación hay una serie de elementos que hacen posible la transmisión de un mensaje. A continuación, se enlistan cada uno de ellos: 
\begin{itemize}
    
\item \textbf{Emisor}: es el individuo que inicia el intercambio de información al transmitir el mensaje \cite{ref22}. Dicho mensaje debe ser codificado en un sistema de símbolos que deberá ser entendible para el receptor. 

\item \textbf{Receptor}: individuo que recibe el mensaje enviado, el cual es interpretado con base en las experiencias, opiniones, contexto y situación del receptor \cite{ref20}. El receptor también puede ser el emisor.

\item \textbf{Código}: es el sistema de signos que es empleado tanto por el emisor como por el receptor para llevar a cabo el proceso de comunicación. Ese sistema debe ser conocido por ambos para facilitar la codificación y descodificación \cite{ref23}.

\item \textbf{Mensaje}: es la información que el emisor transmite al receptor por medio del código \cite{ref24}.

\item \textbf{Canal}: medio en el que los mensajes del emisor se transmiten hacia el receptor \cite{ref20}.

\item \textbf{Contexto}: se refiere a la situación en la que se lleva a cabo el proceso de comunicación, la cual tiene influencia directa en el entendimiento e interpretación del mensaje \cite{ref24}.

\item \textbf{Retroalimentación}: es la respuesta que el receptor emite tras haber recibido e interpretado un mensaje, convirtiéndose momentáneamente en emisor. Este elemento permite cerrar el ciclo comunicativo al brindar al emisor una señal clara sobre si su mensaje fue comprendido, aceptado o necesita ser aclarado o reformulado \cite{ref23}.

\item \textbf{Ruido o interferencia}: dentro del proceso de comunicación puede haber factores externos que dificultan o impiden el entendimiento de los mensajes \cite{ref23}.
\end{itemize}

% \begin{center}
%     \includegraphics[width=0.9\textwidth]{Images/Cap 2/ProcesoComunicación.png}
%     \captionof{figure}[Proceso de comunicación]{Proceso de comunicación, elaboración propia.} 
% \end{center}

% TOMAR EN CUENTA
% \subsubsection{Diagrama de actividades} 
% \begin{center}
% 	\makebox[\textwidth]{%
% 		\includegraphics[width=1\textwidth]{Images/Cap 3/Actividades.png}
% 	}
%     \captionof{figure}{Diagrama de actividades del sistema}
% \end{center}

\begin{center}
	\makebox[\textwidth]{%
		\includegraphics[width=1\textwidth]{Images/Cap 2/ProcesoComunicación.png}
	}
    \captionof{figure}[Proceso de comunicación]{Proceso de comunicación, elaboración propia.}
\end{center}

La comunicación es un proceso indispensable para la interacción humana ya que por medio de ella las personas pueden intercambiar ideas, pensamientos y emociones. No obstante, como se menciona en el concepto de ruido, en ocasiones hay elementos que impiden que la comunicación se lleve a cabo, como lo pueden ser las barreras de la comunicación.

\newpage
\subsection{Barreras de la comunicación}
Las barreras de la comunicación son elementos que limitan o dificultan que las personas puedan comunicarse, a la par que se dificulta su proceso de comunicación \cite{ref2}. Son todas las perturbaciones que sufre un mensaje, en cualquiera de los elementos que forman parte del proceso de comunicación.\\

Los principales tipos de barreras son:
\begin{enumerate}
    \item \textbf{Barreras físicas}: son interferencias causadas por elementos del entorno o en el medio donde se lleva a cabo la comunicación \cite{ref25}.
    \item \textbf{Barreras psicológicas}: son aquellas que surgen por emociones, prejuicios o estados mentales que afectan la interpretación del mensaje \cite{ref25}.
    \item \textbf{Barreras semánticas}: surgen cuando hay confusión en el significado de las palabras, debido a una interpretación incorrecta del lenguaje. Generalmente ocurren cuando se habla en un idioma que el emisor o el receptor no entienden, o se emplean conceptos técnicos desconocidos \cite{ref25}.
    \item \textbf{Barreras administrativas}: generalmente se presentan en entornos laborales y son causadas por falta de planeación, malentendidos, falta de claridad en los procesos de comunicación y distorsiones semánticas \cite{ref25}.
    \item \textbf{Barreras culturales}: este tipo de barreras se presentan cuando hay diferencias en costumbres, valores, normas o expresiones entre culturas, que imposibilitan la comunicación \cite{ref25}.
    \item \textbf{Barreras interpersonales}: hace referencia a las barreras en las que hay suposiciones incorrectas y diferentes percepciones \cite{ref25}.
    \item \textbf{Barreras tecnológicas}: fallas y limitaciones que se presentan en medios tecnológicos empleados para la comunicación \cite{ref25}.
    \item \textbf{Barreras fisiológicas}: impedimentos físicos o biológicos causados por deficiencias en los sentidos, enfermedades o condiciones médicas que afectan cualquiera de los sentidos de manera parcial o total, afectando la transmisión de información \cite{ref25}. Por ejemplo, voz débil, pronunciación defectuosa, sordera, problemas del habla, problemas visuales, etc.
\end{enumerate}
Para efectos de este Trabajo Terminal se analizarán las barreras fisiológicas, concretamente las que son causadas por problemas de sordera. En el siguiente apartado se describen los términos correctos para referirse a las personas con capacidad de escucha y a las personas con discapacidad auditiva.\\

\section{Personas con discapacidad auditiva}
\subsection{Personas Oyentes}
Un oyente se define como aquella persona con la capacidad de escuchar sonidos que le permiten interpretar mensajes. El término procede del verbo oír, que refiere a la capacidad que posee un individuo para poder percibir sonidos \cite{ref26}.

\subsection{Personas con discapacidad auditiva (sordas)}
Por otro lado, una persona que padece de discapacidad auditiva es aquella que ha sufrido la pérdida de la función del sistema auditivo, teniendo como consecuencia una discapacidad para poder oír, lo que dificulta el acceso al lenguaje oral \cite{ref27}.\\ 

Los términos adecuados para referirse a las personas que padecen de esta condición son personas sordas, personas con discapacidad auditiva o personas de la comunidad sorda \cite{refsordos}. \\

De acuerdo con la Federación Mundial de Sordos, existen aproximadamente 70 millones de personas sordas en todo el mundo, las cuales emplean más de 300 diferentes lenguas de señas \cite{ref28}. Las lenguas de señas varían entre países, presentando cambios principalmente en la estructura gramatical, sintaxis, vocabulario, signos, alfabeto y expresiones corporales \cite{ref26}.\\

Por otro lado, la Secretaría de Salud menciona que en México hay aproximadamente 2.3 millones de personas con discapacidad auditiva, de las cuales más del 50\% son mayores de 60 años, 34\% tienen entre 30 y 59 años, y el 2\% son niñas y niños \cite{ref3}.\\

Las principales causas de problemas de audición son antecedentes familiares de sordera heredados, edad avanzada, enfermedades infecciosas, exposición continua a sonidos intensos, entre otras \cite{ref3}.\\

Las personas sordas enfrentan consecuencias en ámbitos académicos, laborales, sociales y emocionales, debido a que las situaciones de aislamiento, deficiencia en la comunicación y dificultades del día a día repercuten negativamente para integrarse en grupos y para socializar \cite{ref29}. \\

\newpage
\subsection{Tipos de Discapacidad Auditiva}
La discapacidad auditiva se clasifica en tres tipos según distintos criterios: según la parte del oído afectada, según el grado de pérdida auditiva y según el momento en que se adquiere \cite{ref30}:\\
\newline\textbf{Según la parte del oído afectada}
\begin{itemize}
    \item \textbf{Hipoacusia conductiva}: es producida por un impedimento en el trayecto de las ondas sonoras del oído externo y medio al oído interno, causado por tumores, perforación del tímpano, traumatismos o disfunciones del oído.  
    \item \textbf{Hipoacusia neurosensorial}: se produce cuando el nervio auditivo o las células ciliadas son dañadas, ya sea por herencia, anormalidades al momento del nacimiento, exposición a ruidos fuertes, traumatismos, entre otras causas.  
    \item \textbf{Hipoacusia mixta}: combinación de hipoacusia conductiva e hipoacusia neurosensorial, causadas por anormalidades al nacer, infecciones, tumores y lesiones en la cabeza.  
\end{itemize}

\begin{center}
    \includegraphics[width=0.9\textwidth]{Images/Cap 2/PartesOido.jpg}
    \captionof{figure}[Partes del oído humano]{Partes del oído humano, obtenido de \cite{ref31}.} 
\end{center}

\newpage
\textbf{Según el grado de pérdida}\\
El rango normal de audición oscila entre 0 y 20 decíbeles (dB). Tomando en consideración ese rango, se establece la siguiente clasificación de acuerdo con los dB que se hayan perdido:

\begin{itemize}
    \item \textbf{Leve:} 20-40 dB.  
    \item \textbf{Moderada:} 40-70 dB.  
    \item \textbf{Severa:} 70-90 dB.  
    \item \textbf{Profunda:} más de 90 dB.  
\end{itemize}

\textbf{Según el momento de la adquisición}\\
En esta clasificación, la discapacidad auditiva puede ser:

\begin{itemize}
    \item \textbf{Hereditaria}: la discapacidad está contenida en algunos de los genes de uno o ambos progenitores.  
    \item \textbf{Adquirida}: la discapacidad puede ser prenatal (antes del nacimiento) o postnatal (después del nacimiento), y en este último caso se deben tomar en cuenta otros criterios:
        \begin{itemize}
        \item \textbf{Prelocutiva:} antes del desarrollo del lenguaje.  
        \item \textbf{Postlocutiva:} después del desarrollo del lenguaje.  
        \end{itemize}
    \end{itemize}

Las personas sordas enfrentan consecuencias en ámbitos académicos, laborales, sociales y emocionales, debido a que las situaciones de aislamiento, deficiencia en la comunicación y dificultades del día a día repercuten negativamente para integrarse en grupos y para socializar \cite{ref29}. En la siguiente sección, se abordan las brechas entre las personas oyentes y las personas con discapacidad.

\subsection{Brechas entre personas oyentes y personas con discapacidad auditiva}
En el plano sociocultural el lenguaje es esencial en las formas de comunicación en una comunidad, pero cuando no todos los individuos pueden responder a esa lógica comunicativa se crean brechas en los discursos que giran en torno a las formas de relacionarse con los demás, puesto que aquellos que tienen códigos y configuraciones diferentes pasan a estar en un plano de invisibilidad \cite{ref32}.\\

La comunidad sorda, a pesar de ser un grupo portador de un lenguaje cultural particular, debe responder a la lengua “natural” de las personas oyentes, y de no poder hacerlo ocasiona que sean excluidos en diferentes escenarios de la vida cotidiana. Esta comunidad ha sido estereotipada como personas incapaces o con limitaciones para insertarse en la sociedad, por lo que, si no pueden entrar en la “lógica natural” para comunicarse con las personas, se ven forzados a interactuar solamente con las personas que comparten su misma condición \cite{ref32}.\\

A lo largo de los últimos años, se han realizado múltiples esfuerzos a nivel gubernamental y se han puesto en marcha discursos que giran alrededor del reconocimiento e inclusión de todas las personas por igual, como lo es la Ley General para la Inclusión de las Personas con Discapacidad \cite{ref37}, para garantizar una mayor participación de las personas con discapacidad auditiva en escenarios sociales. No obstante, lo expresado en la legalidad dista mucho de las realidades particulares de las personas sordas en el marco sociocultural. La comunidad sorda ha sido reconocida como minoría lingüística y, por sus mismas condiciones, ha sido ubicada socialmente en el plano de la exclusión y la invisibilidad \cite{ref32}. \\ 

La presencia de barreras de comunicación generan aislamiento e impiden el desarrollo de una existencia satisfactoria, lo que puede generar graves problemas psicológicos como la depresión, ansiedad, insomnio, estrés, ideas paranoides y sensibilidad interpersonal \cite{ref27}.\\

Además, la comunidad sorda presenta dificultad para acceder a la información proveniente de la televisión, radio, llamadas telefónicas, megafonías en estaciones de metro y salidas de aeropuertos, etc., debido a que esta es principalmente transmitida hacia la población oyente.\\

A pesar de que las personas sordas presentan muchas dificultades en su vida diaria, hoy en día disponen de numerosas herramientas de apoyo (ver \textbf{\autoref{sec:edoArte}}) para impulsar su inclusión en entornos sociales y favorecer su crecimiento personal, como lo son las prótesis auditivas, señales acústicas y su propia Lengua de Señas. \\

En este Trabajo Terminal, únicamente se centrará el estudio en las Lenguas de Señas, concretamente en la Lengua de Señas Mexicana (LSM), revisando toda la documentación existente hasta el 2024, año de la elaboración de este trabajo.\\

\newpage
\section{Lengua de Señas Mexicana}
\subsection{Definición de Lengua de Señas}
La Lengua de Señas es definida como la lengua natural de expresión y configuración gesto-espacial y percepción visual gracias a la cual los sordos pueden comunicarse con su entorno social, la cual está basada en movimientos y expresiones a través de manos, ojos, rostro, boca y cuerpo \cite{ref33}.\\

En el mundo existen cerca de 300 lenguas de señas distintas, siendo así que cada país posee su propia lengua de señas. Por ejemplo, la Lengua de Señas Mexicana (LSM) es diferente a la Lengua de Señas Española (LSE), que a pesar de estar articulados en el mismo idioma (español), no comparten muchas señas en común debido a que ambas lenguas presentan señas que pueden ser regionalismos de cada país \cite{ref33}.\\

Por su parte, la Lengua de Señas Mexicana (LSM) es la lengua de señas que se emplea en México, que cuenta con su propio vocabulario y gramática. A la LSM se le considera como una lengua, debido a que es completamente capaz de expresar una amplia gama de pensamientos y emociones como cualquier otra lengua \cite{ref33}.

\subsection{Lengua de Señas Mexicana (LSM)}
La Ley General para la Inclusión de las Personas con Discapacidad \cite{ref34} define a la LSM, en el Artículo 2, como la lengua de una comunidad de sordos que consiste en una serie de signos gestuales articulados con las manos y acompañados de expresiones faciales, mirada intencional y movimiento corporal, dotados de función lingüística, que forma parte del patrimonio lingüistico de dicha comunidad y es tan rica y compleja en gramática y vocabulario como cualquier lengua oral \cite{ref34}.\\

Por su parte, el Artículo 20 de dicha ley establece que los medios de comunicación deben implementar la tecnología, más concretamente, de intérpretes de LSM que permitan a la comunidad de sordos las facilidades de comunicación \cite{ref34}.\\

En México hay entre 87,000 y 100,000 personas hablantes de LSM que la dominan y la emplean como vía de comunicación, siendo incluso una población mucho más grande que algunas comunidades hablantes de lenguas indígenas del país \cite{ref35}.\\

\newpage
\subsection{Abecedario de la LSM}
La siguiente tabla explica detalladamente cómo se conforma cada una de las letras del abecedario de LSM:

\begin{longtable}{|m{2cm}|m{5cm}|m{5cm}|}
    \hline
    \textbf{Letra} & \textbf{Descripción} & \textbf{Seña} \\
    \hline
    \endfirsthead
    
    \hline
    \textbf{Letra} & \textbf{Descripción} & \textbf{Seña} \\
    \hline
    \endhead
    
    \hline
    \endfoot
    
    \endlastfoot
    
    A & Con la mano cerrada, se muestran las uñas y se estira el dedo pulgar hacia un lado. La palma mira al frente.
    & \makecell{\colorbox{white}{\includegraphics[width=4cm]{Images/Cap 2/Alfabeto LSM/A.png}}} \\
    \hline
    
    B & Los dedos índice, medio, anular y meñique se estiran unidos y el pulgar se dobla hacia la palma, la cual mira al frente.
    & \makecell{\colorbox{white}{\includegraphics[width=4cm]{Images/Cap 2/Alfabeto LSM/B.png}}} \\
    \hline
    
    C & Los dedos índice, medio, anular y meñique se mantienen unidos y en posición cóncava; el pulgar también se coloca en esa posición. La palma mira a un lado.
    & \makecell{\colorbox{white}{\includegraphics[width=4cm]{Images/Cap 2/Alfabeto LSM/C.png}}} \\
    \hline

    D & Los dedos medio, anular, meñique y pulgar se unen por las puntas y el dedo índice se estira. La palma mira al frente.
    & \makecell{\colorbox{white}{\includegraphics[width=4cm]{Images/Cap 2/Alfabeto LSM/D.png}}} \\
    \hline

    E & Se doblan los dedos completamente y se muestran las uñas. La palma mira al frente.
    & \makecell{\colorbox{white}{\includegraphics[width=4cm]{Images/Cap 2/Alfabeto LSM/E.png}}} \\
    \hline

    F & Con la mano abierta y los dedos unidos, se dobla el índice hasta que su parte lateral toque la yema del pulgar. La palma mira a un lado.
    & \makecell{\colorbox{white}{\includegraphics[width=4cm]{Images/Cap 2/Alfabeto LSM/F.png}}} \\
    \hline

    G & Se cierra la mano y los dedos índice y pulgar se estiran. La palma mira hacia la persona que se comunica.
    & \makecell{\colorbox{white}{\includegraphics[width=4cm]{Images/Cap 2/Alfabeto LSM/G.png}}} \\
    \hline

    H & Se cierra la mano y los dedos índice y medio se unen y se estiran, se extiende el dedo pulgar señalando hacia arriba. La palma mira hacia la persona que se comunica.
    & \makecell{\colorbox{white}{\includegraphics[width=4cm]{Images/Cap 2/Alfabeto LSM/H.png}}} \\
    \hline

    I & Con la mano cerrada, el dedo meñique se estira señalando hacia arriba. La palma se coloca de lado.
    & \makecell{\colorbox{white}{\includegraphics[width=4cm]{Images/Cap 2/Alfabeto LSM/I.png}}} \\
    \hline

    J & Con la mano cerrada, el dedo meñique estirado señala hacia arriba y la palma señala a un lado. La mano dibuja una “j” en el aire.
    & \makecell{\colorbox{white}{\includegraphics[width=4cm]{Images/Cap 2/Alfabeto LSM/J.png}}} \\
    \hline

    K & Se cierra la mano con los dedos índice, medio y pulgar estirados. La yema del pulgar se coloca entre el índice y el medio, moviendo la muñeca hacia arriba.
    & \makecell{\colorbox{white}{\includegraphics[width=4cm]{Images/Cap 2/Alfabeto LSM/K.png}}} \\
    \hline

    L & Con la mano cerrada y los dedos índice y pulgar estiados, se forma una “L”. La palma mira al frente.
    & \makecell{\colorbox{white}{\includegraphics[width=4cm]{Images/Cap 2/Alfabeto LSM/L.png}}} \\
    \hline

    M & Con la mano cerrada, se ponen los dedos índice, medio y anular sobre el pulgar.
    & \makecell{\colorbox{white}{\includegraphics[width=4cm]{Images/Cap 2/Alfabeto LSM/M.png}}} \\
    \hline
 
    N & Con la mano cerrada, se ponen los dedos índice y medio sobre el pulgar. 
    & \makecell{\colorbox{white}{\includegraphics[width=4cm]{Images/Cap 2/Alfabeto LSM/N.png}}} \\
    \hline

    Ñ & Con la mano cerrada, se ponen los dedos índice y medio sobre el pulgar. Se mueve la muñeca a los lados. 
    & \makecell{\colorbox{white}{\includegraphics[width=4cm]{Images/Cap 2/Alfabeto LSM/Ñ.png}}} \\
    \hline

    O & Con la mano se forma una letra “o”. Todos los dedos se tocan por las puntas. 
    & \makecell{\colorbox{white}{\includegraphics[width=4cm]{Images/Cap 2/Alfabeto LSM/O.png}}} \\
    \hline

    P & Con la mano cerrada y los dedos índice, medio y pulgar estirados, se coloca la yema del pulgar entre el índice y el medio.
    & \makecell{\colorbox{white}{\includegraphics[width=4cm]{Images/Cap 2/Alfabeto LSM/P.png}}} \\
    \hline
    
    Q & Con la mano cerrada, se colocan los dedos índice y pulgar en posición de garra. La palma mira hacia abajo, y se mueve hacia los lados.
    & \makecell{\colorbox{white}{\includegraphics[width=4cm]{Images/Cap 2/Alfabeto LSM/Q.png}}} \\
    \hline
    
    R & Con la mano cerrada, se estiran y entrelazan los dedos índice y medio. La palma mira al frente.
    & \makecell{\colorbox{white}{\includegraphics[width=4cm]{Images/Cap 2/Alfabeto LSM/R.png}}} \\
    \hline

    S & Con la mano cerrada, se pone el pulgar sobre los otros dedos. La palma mira al frente.
    & \makecell{\colorbox{white}{\includegraphics[width=4cm]{Images/Cap 2/Alfabeto LSM/S.png}}} \\
    \hline

    T & Con la mano cerrada, el pulgar se pone entre el índice y el medio. La palma mira al frente. 
    & \makecell{\colorbox{white}{\includegraphics[width=4cm]{Images/Cap 2/Alfabeto LSM/T.png}}} \\
    \hline

    U & Con la mano cerrada, se estiran los dedos índice y medio unidos. La palma mira al frente. 
    & \makecell{\colorbox{white}{\includegraphics[width=4cm]{Images/Cap 2/Alfabeto LSM/U.png}}} \\
    \hline

    V & Con la mano cerrada, se estiran los dedos índice y medio separados. La palma mira al frente. 
    & \makecell{\colorbox{white}{\includegraphics[width=4cm]{Images/Cap 2/Alfabeto LSM/V.png}}} \\
    \hline

    W & Con la mano cerrada, se estiran los dedos índice, medio y anular separados. La palma mira al frente. 
    & \makecell{\colorbox{white}{\includegraphics[width=4cm]{Images/Cap 2/Alfabeto LSM/W.png}}} \\
    \hline

    X & Con la mano cerrada, el índice y el pulgar en posición de garra y la palma dirigida a un lado, se realiza un movimiento al frente y de regreso. 
    & \makecell{\colorbox{white}{\includegraphics[width=4cm]{Images/Cap 2/Alfabeto LSM/X.png}}} \\
    \hline
    
    Y & Con la mano cerrada, se estira el meñique y el pulgar. La palma mira hacia la persona que se comunica. 
    & \makecell{\colorbox{white}{\includegraphics[width=4cm]{Images/Cap 2/Alfabeto LSM/Y.png}}} \\
    \hline

    Z & Con la mano cerrada, el dedo índice estirado y la palma al frente, se dibuja una letra z en el aire. 
    & \makecell{\colorbox{white}{\includegraphics[width=4cm]{Images/Cap 2/Alfabeto LSM/Z.png}}} \\
    \hline
    
    \caption[Abecedario de la LSM]{Abecedario de la LSM, obtenido de \cite{ref36}.} \label{tabla:LSM}
\end{longtable}

\clearpage

\subsection{Grámatica de la Lengua de Señas Mexicana}
La gramática estudia cómo se conectan los elementos de una lengua para crear oraciones con sentido. En la Lengua de Señas Mexicana (LSM), esa estructura no depende de sonidos ni palabras habladas, sino del uso visual del cuerpo, el espacio y los movimientos \cite{ref37}.\\

La LSM se desarrolla en un espacio frente al cuerpo, dividido en tres zonas principales \cite{ref37}:

\begin{itemize}
    \item Una línea vertical que va desde la cintura hasta la parte superior de la cabeza.
    \item Un límite horizontal, que se extiende hasta los codos con los brazos en ángulo.
    \item Un tercer límite que marca qué tan lejos están las manos del cuerpo.
\end{itemize}

Si la seña se realiza fuera de estos límites, suele entenderse como un énfasis o exageración del mensaje.\\

A diferencia del español, la LSM no se basa en sonidos, sino en aspectos visuales y espaciales. Las señas suelen representar ideas complejas, como si fueran palabras individuales con sentido propio. Estas señas se consideran morfemas libres, ya que no necesitan agregarse a otras ni modificarse con terminaciones \cite{ref38}.\\

En esta lengua, no se usan frecuentemente sufijos y prefijos para cambiar el significado. En su lugar, expresiones faciales, movimientos de cabeza o del cuerpo ayudan a matizar lo que se dice. Estos gestos pueden aportar información como si la acción se repite, si está terminada, si es deseada, obligatoria, o si es posible \cite{ref38}.\\

Las señas no tienen una categoría gramatical fija. Una misma seña puede funcionar como verbo, adjetivo o sustantivo, dependiendo del contexto en que se use. Esta flexibilidad se debe a un fenómeno llamado prototipicidad, que permite a ciertas formas adaptarse a distintas funciones según la necesidad \cite{ref38}.\\

Existen señas que siguen patrones más estables, como los verbos direccionales, que cambian su movimiento para indicar quién realiza una acción y hacia quién va dirigida. Sin embargo, estas señas también pueden usarse en otros contextos sin perder su significado \cite{ref38}.\\

Las oraciones en LSM se forman con señas que expresan acciones, participantes, tiempo, condiciones o características. En oraciones simples, las señas que representan sujetos u objetos se comportan como nombres, aunque no siempre sean sustantivos. Las acciones o ideas principales se representan con señas que funcionan como predicados \cite{ref38}.\\

También es común ver señas que expresan pronombres, ubicaciones o tiempos. A veces se usa el deletreo dactilológico o nombres propios, los cuales pueden repetirse al final de la oración como una forma de remarcar la información, lo que algunos estudios llaman “etiquetado” o “tags” \cite{ref38}.\\

\begin{center}
    \includegraphics[width=0.9\textwidth]{Images/Cap 2/Estructura_gramatica_LSM.png}
    \captionof{figure}[Estructura gramatical de una oración de LSM]{Estructura gramatical de una oración de LSM, obtenido de \cite{ref38}.}  % Pie de foto manual
\end{center}

\textbf{Fonología}\\
En las lenguas de señas, los fonemas, las unidades mínimas con significado, pueden descomponerse en siete componentes esenciales \cite{ref39}:

\begin{enumerate}
    \item \textbf{Configuración manual}: es la forma específica que adopta la mano al ejecutar un signo determinado.
    \item \textbf{Orientación de la mano}: hace referencia a la dirección en la que se posiciona la palma, ya sea orientada hacia arriba, abajo o frente al emisor.
    \item \textbf{Zona de ejecución}: indica la parte del cuerpo en la que se realiza el signo, como por ejemplo la frente, la boca, el pecho o los hombros.
    \item \textbf{Desplazamiento}: describe el tipo de movimiento que se lleva a cabo con las manos al hacer un signo; este puede ser giratorio, lineal, en vaivén o segmentado.
    \item \textbf{Área de contacto}: se refiere a la parte de la mano dominante (la derecha para personas diestras o la izquierda para personas zurdas) que entra en contacto con el cuerpo. Puede involucrar la palma, las yemas o el dorso de los dedos.
    \item \textbf{Plano de producción}: se trata de la distancia entre el cuerpo y el lugar donde se articula el signo. El Plano 1 está en contacto directo con el cuerpo, mientras que el Plano 4 se encuentra más alejado, con los brazos completamente extendidos.
    \item \textbf{Elementos corporales no manuales}: son señales complementarias que refuerzan el mensaje, como expresiones faciales, movimientos del torso, gesticulaciones orales o el uso del cuello y los hombros. Por ejemplo, para comunicar una acción futura se inclina el cuerpo hacia delante, y para indicar el pasado, hacia atrás.\\
\end{enumerate}

\textbf{La Configuración Manual (CM) en la LSM}\\
\label{sec:config_manual}
En las lenguas de señas, las manos son las principales herramientas para comunicar, aunque no son las únicas. Además de considerar la dirección del movimiento y el lugar en el espacio donde se hace una seña, también es esencial observar la forma que toman las manos, conocida como configuración manual (CM). Esta configuración puede variar tanto en la mano dominante como en la no dominante \cite{ref37}.\\

La configuración manual, entonces, representa la forma específica que adoptan las manos al momento de hacer una seña. Esto incluye aspectos como \cite{ref37}:

\begin{itemize}
    \item \textbf{La posición de los dedos}: si están juntos o separados, doblados o rectos.
    \item \textbf{La forma general de la mano}: abierta, en puño, en forma de garra, etc.
    \item \textbf{La ubicación del pulgar y del índice}: suelen tener movimientos propios.
\end{itemize}

Desde un punto de vista técnico, la configuración manual forma parte de lo que se llama la matriz articulatoria. Dentro de ella, se distinguen dos grupos importantes \cite{ref37}:

\begin{itemize}
    \item Los dedos (índice, medio, anular y meñique), que suelen moverse como bloque.
    \item El pulgar, que, por su movilidad más independiente, se analiza aparte.
\end{itemize}

Debido a todas las combinaciones posibles entre estos elementos, las configuraciones manuales no pueden reducirse a formas simples, sino que son estructuras complejas que generan significado cuando se combinan con otros componentes de la seña.

\newpage

\textbf{Orientación de la Palma de la Mano}\\
Se refiere a la dirección en la que se encuentra la palma de la mano en relación con el cuerpo de la persona que está haciendo la seña, justo en el momento en que adopta la configuración manual \cite{ref37}.\\

En la Lengua de Señas Mexicana (LSM), se han identificado nueve posibles formas de orientar la palma durante la articulación de una seña \cite{ref37}. Estas son:

\begin{enumerate}
    \item Palma hacia arriba, con los dedos apuntando a la izquierda.
    \item Palma hacia arriba, con los dedos apuntando hacia el frente.
    \item Palma hacia abajo, con los dedos dirigidos a la izquierda.
    \item Palma hacia abajo, con los dedos apuntando hacia adelante.
    \item Palma hacia la izquierda, con los dedos hacia arriba.
    \item Palma hacia la izquierda, con los dedos hacia el frente.
    \item Palma hacia el frente, con los dedos señalando hacia arriba.
    \item Palma frente al cuerpo, dedos hacia arriba.
    \item Palma frente al cuerpo, dedos hacia la izquierda.
\end{enumerate}

\begin{center}
    \includegraphics[width=0.7\textwidth]{Images/Cap 2/Orientacion_Palma_Mano.png}
    \captionof{figure}[Orientaciones de la palma de la mano en LSM]{Orientaciones de la palma de la mano en LSM, obtenido de \cite{ref37}.}  % Pie de foto manual
\end{center}

Cada una de estas orientaciones forma parte de la estructura visual y espacial de una seña, y su correcta ejecución es clave para transmitir el significado deseado.\\

\textbf{Ubicación en la Lengua de Señas Mexicana (LSM)}\\
Hace referencia al lugar específico en el espacio donde se realiza una seña. Este espacio, conocido como espacio señante, puede estar frente al cuerpo o sobre él y es clave para transmitir el significado correcto \cite{ref37}.\\

Para describir las señas en este diccionario, se dividió el espacio señante principalmente en niveles de altura y direcciones laterales. En algunos casos, también se toma en cuenta una tercera dimensión, que implica mayor cercanía o profundidad \cite{ref37}.\\

Cuando las señas se hacen sobre el cuerpo, se habla de “alturas” específicas, como por ejemplo:
\begin{itemize}
    \item A la altura del cuello.
    \item A la altura del hombro.
    \item A la altura del pecho.
    \item A la altura del plexo.
    \item A la altura de la cintura.
    \item A la altura de la cadera.
\end{itemize}

Si la seña se mueve entre dos puntos, se describe como un desplazamiento, por ejemplo:

\begin{itemize}
    \item Del pecho a la cintura.
    \item Del hombro a la cadera.
    \item Del cuello a la cadera.
\end{itemize}

Cuando el movimiento es horizontal o de un lado a otro, también se aclara, por ejemplo:
\begin{itemize}
    \item A la altura del pecho, de izquierda a derecha.
\end{itemize}

En el caso de las señas realizadas en la cara, se puede ser más preciso indicando zonas como:
\begin{itemize}
    \item A la altura de los ojos.
    \item A la altura de las cejas.    
\end{itemize}

Finalmente, si la seña se hace sobre el tronco del cuerpo, se especifica si es:
\begin{itemize}
    \item Del lado izquierdo.
    \item Del lado derecho.
    \item Al centro.
\end{itemize}

Aunque algunas ubicaciones pueden detallarse aún más, se busca usar una descripción unificada para facilitar la comprensión \cite{ref37}.\\

\textbf{Dirección del Movimiento de la mano en la LSM}\\
Es la trayectoria que la mano sigue al realizar una seña \cite{ref37}. En la \autoref{direcciones_Mano} se muestran las nueve direcciones que las configuraciones manuales pueden seguir durante la articulación de las señas respecto al cuerpo.

\begin{center}
    \includegraphics[width=0.8\textwidth]{Images/Cap 2/Direccion_Manos_LSM.png}
    \captionof{figure}[Direcciones posibles que sigue la mano en la LSM]{Direcciones posibles que sigue la mano en la LSM, obtenido de \cite{ref37}.} 
    \label{direcciones_Mano}
\end{center}

\newpage
\textbf{Explicación de los movimientos y sus símbolos}\\
La \autoref{tabla:Movimiento_LSM} muestra todos los movimientos que las manos pueden realizar en la LSM. En la primera columna se menciona el nombre del movimiento; en la segunda, la descripción del mismo y en la tercera aparecen flechas o imágenes de la mano para indicar la dirección o la manera en que las configuraciones manuales se mueven.\\

\begin{longtable}{|m{5cm}|m{5cm}|m{5cm}|}
    \hline
    \textbf{Movimiento} & \textbf{Descripción del movimiento} & \textbf{Imagen} \\
    \hline
    \endfirsthead
    
    \hline
    \textbf{Movimiento} & \textbf{Descripción del movimiento} & \textbf{Imagen} \\
    \hline
    \endhead
    
    \hline
    \endfoot
    
    \endlastfoot
    
     & Se emplea una numeración progresiva para señalar cómo cambian las formas de las manos y los desplazamientos en las señas que combinan varios movimientos.
    & \makecell{\colorbox{white}{\includegraphics[width=4cm]{Images/Cap 2/Movimientos LSM/1.png}}} \\
    \hline

    Lineal (lin) & Movimiento rectilíneo.
    & \makecell{\colorbox{white}{\includegraphics[width=4cm]{Images/Cap 2/Movimientos LSM/2.png}}} \\
    \hline

    Arco (ar) & El desplazamiento del brazo, la muñeca o la mano dibuja una curva en forma de arco.
    & \makecell{\colorbox{white}{\includegraphics[width=4cm]{Images/Cap 2/Movimientos LSM/3.png}}} \\
    \hline

    Extensión de dedos (E) & Los dedos se extienden.
    & \makecell{\colorbox{white}{\includegraphics[width=4cm]{Images/Cap 2/Movimientos LSM/4.png}}} \\
    \hline

    Vaivén (va) & Se realiza un movimiento intercalado entre ambas manos o brazos.
    & \makecell{\colorbox{white}{\includegraphics[width=4cm]{Images/Cap 2/Movimientos LSM/5.png}}} \\
    \hline
    
    Circular (circ) & La trayectoria de la mano, muñeca o brazo describe movimientos circulares o semicirculares.
    & \makecell{\colorbox{white}{\includegraphics[width=4cm]{Images/Cap 2/Movimientos LSM/6.png}}} \\
    \hline

    Espiral (es) & La mano o el brazo giran siguiendo un patrón redondeado.
    & \makecell{\colorbox{white}{\includegraphics[width=4cm]{Images/Cap 2/Movimientos LSM/7.png}}} \\
    \hline

    Flexión de dedos (f) & Los dedos se retraen.
    & \makecell{\colorbox{white}{\includegraphics[width=4cm]{Images/Cap 2/Movimientos LSM/8.png}}} \\
    \hline

    Ondular (ond) & El movimiento de la mano o el brazo imita una forma de onda.
    & \makecell{\colorbox{white}{\includegraphics[width=4cm]{Images/Cap 2/Movimientos LSM/9.png}}} \\
    \hline
    
    Salto & La mano o los dedos simulan uno o varios saltos.
    & \makecell{\colorbox{white}{\includegraphics[width=4cm]{Images/Cap 2/Movimientos LSM/10.png}}} \\
    \hline

    Movimiento vibratorio local (vib) & La mano tiembla.
    & \makecell{\colorbox{white}{\includegraphics[width=4cm]{Images/Cap 2/Movimientos LSM/11.png}}} \\
    \hline
    
    Cabeceo de muñeca (cab) & El movimiento parte desde la parte posterior y avanza al frente, usando solo el giro de la muñeca.
    & \makecell{\colorbox{white}{\includegraphics[width=4cm]{Images/Cap 2/Movimientos LSM/12.png}}} \\
    \hline
    
    Aplanado (apl) & Se realiza un contacto breve entre el índice y medio, o el índice y el pulgar, seguido de una separación.
    & \makecell{\colorbox{white}{\includegraphics[width=4cm]{Images/Cap 2/Movimientos LSM/13.png}}} \\
    \hline

    Apulgarado (p) & El índice o medio se libera con impulso del pulgar, pasando de estar doblado a completamente recto.
    & \makecell{\colorbox{white}{\includegraphics[width=4cm]{Images/Cap 2/Movimientos LSM/14.png}}} \\
    \hline

    Cambios progresivos en los dedos (prog) & Los dedos se mueven uno a uno de forma intercalada.
    & \makecell{\colorbox{white}{\includegraphics[width=4cm]{Images/Cap 2/Movimientos LSM/15.png}}} \\
    \hline

    Deslizamiento (desl) & Se realiza un movimiento deslizante de los dedos sobre la superficie del pulgar.
    & \makecell{\colorbox{white}{\includegraphics[width=4cm]{Images/Cap 2/Movimientos LSM/16.png}}} \\
    \hline

    Zig-zag (zig) & El índice traza en el aire la forma de la letra Z.
    & \makecell{\colorbox{white}{\includegraphics[width=4cm]{Images/Cap 2/Movimientos LSM/17.png}}} \\
    \hline
    
    Siete (7) & Se realiza un movimiento que dibuja el número siete en el espacio.
    & \makecell{\colorbox{white}{\includegraphics[width=4cm]{Images/Cap 2/Movimientos LSM/18.png}}} \\
    \hline

    Rotación de muñeca (rot) & La rotación del antebrazo o la muñeca provoca que la mano cambie su dirección.
    & \makecell{\colorbox{white}{\includegraphics[width=4cm]{Images/Cap 2/Movimientos LSM/19.png}}} \\
    \hline
    
    Choque (ch) & Las manos se encuentran y se tocan.
    & \makecell{\colorbox{white}{\includegraphics[width=4cm]{Images/Cap 2/Movimientos LSM/20.png}}} \\
    \hline
    
    Doblar (dob) & Mientras el pulgar permanece quieto, los demás dedos se doblan hacia el centro de la mano.
    & \makecell{\colorbox{white}{\includegraphics[width=4cm]{Images/Cap 2/Movimientos LSM/21.png}}} \\
    \hline
    
    Cruzado (crz) & Los brazos se mueven hacia el centro cruzándose, y las manos se aproximan en un punto medio.
    & \makecell{\colorbox{white}{\includegraphics[width=4cm]{Images/Cap 2/Movimientos LSM/22.png}}} \\
    \hline
    
    Simétrico (sim) & Desde una posición inicial común, las manos se separan hacia diferentes direcciones: superior, inferior o lateral.
    & \makecell{\colorbox{white}{\includegraphics[width=4cm]{Images/Cap 2/Movimientos LSM/23.png}}} \\
    \hline

    Prensar & El índice y el pulgar realizan un gesto de pinza al agarrar otra mano o una zona del cuerpo.
    & \makecell{\colorbox{white}{\includegraphics[width=4cm]{Images/Cap 2/Movimientos LSM/24.png}}} \\
    \hline

    \caption[Movimientos de las manos]{Movimientos de las manos, obtenido de \cite{ref37}.} \label{tabla:Movimiento_LSM} \\
\end{longtable}

\textbf{Rasgos no manuales: expresión facial y gestos}\\
Los rasgos no manuales (RNM) en la Lengua de Señas Mexicana incluyen la expresión facial, los gestos y los movimientos del cuerpo \cite{ref37}. Estos elementos se realizan al mismo tiempo que las señas y tienen una función gramatical clave, ya que aportan significado al mensaje, similar a cómo el tono de voz o la velocidad lo hacen en el español hablado \cite{ref37}.\\

También existen gestos universales que no dependen del idioma o la cultura, como los que expresan felicidad, tristeza, dolor o alegría y que se entienden en cualquier parte del mundo \cite{ref37}.\\

\textbf{Tipos de Señas en la LSM}\\
En la Lengua de Señas Mexicana (LSM), las señas se pueden clasificar de diferentes maneras, principalmente según cuántas manos se usan y cómo se mueven \cite{ref37}:

\begin{itemize}
    \item \textbf{Seña manual (SM)}: se realiza con una sola mano.
    \item \textbf{Seña bimanual (SB)}: usa ambas manos, pero no necesariamente hacen lo mismo; puede haber movimientos distintos entre una y otra.
    \item \textbf{Seña simétrica (SS)}: ambas manos se mueven al mismo tiempo con movimientos similares o en espejo (como si se reflejaran una a la otra).
    \item \textbf{Seña compuesta (SC)}: se forma a partir de dos o más señas simples o al menos tres formas diferentes de las manos.
\end{itemize}

\newpage
Además, según la relación entre la seña y su significado, también se pueden clasificar así \cite{ref37}:
\begin{itemize}
    \item \textbf{Icónicas}: estas señas imitan la forma o alguna característica del objeto al que se refieren. Por ejemplo, la seña de “árbol” muestra cómo es un árbol.
    \item \textbf{Referenciales en el cuerpo}: Algunas señas se hacen en partes del cuerpo relacionadas con el objeto, como la seña de “manzana”, que se realiza en la mejilla.
    \item \textbf{Arbitrarias}: no guardan ninguna relación visual con el objeto o concepto. Ejemplos de estas son las señas de “gracias” u “oportunidad”.
    \item \textbf{Inicializadas o alfabéticas}: se utilizan las letras del alfabeto manual, generalmente la inicial de la palabra en español, como en “mamá” o “alumno”.
    \item \textbf{Indéxicas}: estas señas señalan un lugar, persona u objeto en el espacio, como los pronombres: yo, tú, él, ella, allá, aquí, etc.
    \item \textbf{Numéricas}: son aquellas donde la forma de la mano representa un número, y se usan para nombrar cosas como países o acciones. Ejemplos: “Dinamarca” (con el número 8), “mujer” (número 1), “abanico” (número 4) y “atención” (número 6).\\
\end{itemize}

\textbf{Clasificadores en la LSM}\\
En la LSM, los clasificadores son señas especiales que permiten describir mejor las características de un objeto, las cuales combinan dos elementos: uno que indica a qué tipo de objeto se hace referencia (como persona, animal, cosa) y otro que muestra sus propiedades, como forma, tamaño o cómo se mueve \cite{ref37}. Son muy útiles cuando no existe una seña exacta para algo y ayudan a transmitir el mensaje de manera visual y clara\\

Están basados en la configuración manual (CM), es decir, en la forma en la que se colocan y usan las manos. A través de estas configuraciones, se puede representar si un objeto es redondo, plano, grande, pequeño, blando, rígido, entre otros rasgos y, de igual manera, pueden indicar su ubicación, cantidad u orden \cite{ref37}.\\

Además, existen clasificadores de predicado, que no solo describen el objeto, sino que también aportan información sobre su movimiento, posición o estado. Estos se combinan con “raíces de movimiento” para señalar si algo se mueve, está quieto o entra en contacto con algo \cite{ref37}. Las raíces de movimiento pueden ser:
\begin{itemize}
    \item Movimiento o proceso.
    \item Descripción estática.
    \item Contacto con otro objeto.
\end{itemize}

Los clasificadores también se organizan en diferentes tipos, dependiendo de lo que representan \cite{ref37}:
\begin{itemize}
    \item Clasificadores de entidad (personas, animales, cosas).
    \item De superficie (planos, mesas, pisos).
    \item De profundidad y anchura (algo profundo o ancho).
    \item De extensión o límites (como bordes o extremos).
    \item De perímetro (formas cerradas).
    \item De instrumento (objetos usados para realizar acciones).
\end{itemize}

\textbf{Afijos}\\
De acuerdo con la definición de la Real Academia Española \cite{refrae}, los afijos son morfemas que se agregan a la raíz de una palabra para alterar su sentido o función gramatical. No obstante, en la Lengua de Señas, el uso de afijos no es común como lo es en las lenguas orales \cite{ref37}.\\

En las lenguas habladas, los afijos como prefijos y sufijos son parte esencial de la estructura de muchas palabras. En cambio, en la Lengua de Señas, su presencia es limitada y aparece sobre todo en situaciones en las que interviene el español escrito \cite{ref37}. Generalmente, son personas oyentes que conocen la LSM quienes tienden a emplear estos elementos con mayor frecuencia, adaptando la estructura del español al sistema visual y gestual de la lengua de señas.\\

\textbf{Prefijos}\\
En la Lengua de Señas Mexicana (LSM), los prefijos son elementos que se colocan al inicio de una seña para agregar información como el tiempo, el género o el número \cite{ref37}. Aunque en las lenguas orales son comunes, en LSM su uso es más limitado y, en muchos casos, surge por influencia del español. Actualmente, se conservan pocos prefijos, como por ejemplo “in-” o “im-”, que se representan con una “I” hecha por la mano dominante al tocar la mano base \cite{ref37}.\\

Para expresar el género femenino, se utiliza una seña específica que aparece después de indicar el masculino. El número (singular o plural) suele marcarse antes del sustantivo, y si se quiere enfatizar, puede colocarse también después \cite{ref37}.\\

El tiempo del mensaje se indica al inicio, ya sea mediante una seña específica o con movimientos del cuerpo: inclinarse hacia adelante para el futuro, o hacia atrás para el pasado \cite{ref37}.\\

En cuanto a la negación, existen señas conocidas como dobletes, que tienen una forma afirmativa y otra negativa completamente distinta, sin necesidad de añadir gestos como mover la cabeza \cite{ref37}. Algunos ejemplos son:
\begin{itemize}
    \item Gustar / No gustar.
    \item Poder / No poder.
    \item Saber / No saber.
    \item Todavía / Todavía no.
    \item Querer / No querer.
    \item Haber / No haber.
    \item Sirve / No sirve.
\end{itemize}

\textbf{Sufijos}\\
Son pequeñas unidades de significado que se colocan después de una seña para dar más precisión o detalle al mensaje. Su uso en la Lengua de Señas Mexicana (LSM) se ha visto influenciado principalmente por el idioma español, ya que en las lenguas orales es común agregar estas terminaciones a las palabras \cite{ref37}.\\

En el pasado, los sufijos más frecuentes en LSM eran “-ción” y “-mente”, ya que ayudan a formar sustantivos abstractos o adverbios \cite{ref37}. Sin embargo, con el tiempo han dejado de usarse tanto y actualmente los sufijos que se siguen utilizando con mayor frecuencia son:

\begin{itemize}
    \item -ito / -ita, que indican diminutivo o cercanía con afecto.
    \item -al, usado para formar adjetivos relacionados a un lugar o cosa.
    \item -or / -ora, que hacen referencia a profesiones o a quien realiza una acción.
    \item -dad, que se usa para formar conceptos abstractos, como en “amistad” o “bondad”.
\end{itemize}

\newpage
\subsection{Dactilología}
La dactilología es un sistema de comunicación que transmite información mediante el deletreo manual, y en ocasiones es usado en conjunto con la lengua de señas. Se emplea la mano de diferente manera para pronunciar cada una de las letras \cite{ref30}.\\

Otra definición de la dactilología es que es la representación manual de cada una de las letras que componen el alfabeto para poder transmitir a las personas sordas cualquier palabra que se desee comunicar. Todas las lenguas de señas poseen mecanismos internos que les permiten generar mensajes \cite{ref40}.\\

Para comunicarse por medio de dactilología se emplea la mano dominante a la altura de la barbilla, en conjunto con la articulación oral, siendo necesario que la cara y la boca sean visibles \cite{ref40}. Principalmente se usa para sustantivos, nombres propios, direcciones y palabras para los cuales no existe un signo creado.\\

Si bien la discapacidad auditiva representa una barrera de la comunicación, las personas sordas en los últimos años han buscado superar esa barrera con ayuda de dispositivos tecnológicos que puedan fungir como intérpretes. El desarrollo de la Inteligencia Artificial (IA), más concretamente las técnicas de Procesamiento de Lenguaje Natural (PLN), Visión Artificial, \textit{Deep Learning} y modelado de animaciones 3D (ver \textbf{\autoref{sec:edoArte}}), han ayudado a crear nuevos sistemas que faciliten la interacción entre personas oyentes y personas de la comunidad sorda, derribando las barreras de la comunicación.\\

\subsection{Diferencia entre Traducción e Interpretación}
La traducción es el proceso de convertir textos escritos de un idioma a otro, conservando el sentido, la intención y el tono del mensaje original. La misma se realiza sobre materiales tangibles: documentos, contratos, libros, artículos, guiones, entre otros. El objetivo es transcribir por escrito un mensaje de una lengua origen a una lengua meta, conservando su significado, estilo y contexto \cite{reftradint}. \\

La traducción implica un proceso cognitivo complejo que requiere no sólo la sustitución lingüística, sino también la comprensión del contexto cultural y semántico en el que se enmarca el mensaje original. En el ámbito computacional, este proceso es llevado a cabo por sistemas de traducción automática que analizan la estructura lingüística, la semántica y las relaciones gramaticales del texto fuente, empleando modelos estadísticos o de aprendizaje profundo para generar un texto equivalente en el idioma destino \cite{reftradint2}.\\

Por otro lado, la interpretación es la conversión del mensaje oral de un orador de una lengua hablada a otra, la cual trabaja con el discurso oral, y el proceso se realiza en tiempo real. Por lo tanto, el intérprete debe reaccionar rápidamente, sin margen de corrección ni consulta \cite{reftradint}.\\

La interpretación requiere habilidades cognitivas diferentes a las de la traducción, ya que el intérprete debe procesar simultáneamente la comprensión y la reformulación del mensaje, implicando una alta capacidad de concentración y dominio de ambas lenguas en contextos culturales diversos. A diferencia de la traducción donde existe la posibilidad de revisar el resultado en materiales de consulta, la interpretación se caracteriza por su inmediatez y espontaneidad, lo que la hace esencial en entornos como conferencias, juicios o emergencias médicas \cite{reftradint2}.\\ 

En el caso de la Lengua de Señas Mexicana (LSM), los intérpretes desempeñan un papel fundamental al facilitar la comunicación en espacios presenciales, mientras que un sistema automatizado como el propuesto en este trabajo realiza un proceso de traducción escrita a representación visual animada.\\

La LSM, al ser una lengua visogestual, no puede traducirse de manera literal palabra por palabra, sino que requiere un proceso de adaptación sintáctica y semántica. Por ello, los sistemas automáticos deben incorporar etapas de análisis lingüístico y modelado computacional que permitan representar la estructura y el contexto de la oración en español dentro del marco gramatical propio de la LSM.\\

El desarrollo de este tipo de sistemas contribuye al campo de la accesibilidad tecnológica al proporcionar alternativas comunicativas que promueven la inclusión social. El uso de técnicas de Procesamiento de Lenguaje Natural (PLN) y animaciones permite automatizar la traducción escrita a visual, garantizando resultados comprensibles y gramaticalmente coherentes con las reglas lingüísticas de la LSM. Esta distinción resulta fundamental para entender el alcance del proyecto, centrado en la traducción automatizada y no en la interpretación en tiempo real mediante medios humanos.\\

Este prototipo no pretende reemplazar el trabajo de los intérpretes humanos, sino complementar sus funciones mediante herramientas digitales que permitan traducir texto escrito en español a representaciones visuales animadas en LSM. El proyecto se enmarca dentro del campo de la traducción asistida por computadora, en la cual el procesamiento previo del texto constituye una fase esencial para transformar el lenguaje natural en representaciones visuales comprensibles. Además, si en el futuro el sistema logra procesar voz o señas en tiempo real, podría evolucionar hacia un sistema de interpretación automática, ampliando su alcance y abriendo una nueva línea de investigación en accesibilidad e inteligencia artificial aplicada a las lenguas de señas.\\

En los siguientes apartados se analizarán un par de herramientas que serán necesarias para el desarrollo del prototipo planteado en el capítulo 1 (ver \textbf{\autoref{sec:Intro}}), como lo es la Inteligencia Artificial (IA) y el Procesamiento de Lenguaje Natural (PLN).\\

\section{Inteligencia Artificial}
La Inteligencia Artificial (IA) es la capacidad que poseen las máquinas para usar algoritmos y aprender de los datos para tomar decisiones tal como lo haría un ser humano \cite{ref41}. A diferencia del ser humano, la IA no necesita descansar y es capaz de analizar grandes cantidades de información, reduciendo el margen de error.\\

La IA se basa en el uso de algoritmos y tecnologías de aprendizaje automático para dar a las máquinas la capacidad de aplicar ciertas habilidades cognitivas y realizar tareas por sí mismas de manera autónoma o semiautónoma. A medida que la IA mejora, muchos procesos son más eficientes y algunas tareas que parecían complicadas se realizan con mayor rapidez y precisión \cite{ref42}.\\

\subsection{Clasificación de la Inteligencia Artificial}
La IA puede ser clasificada de varias maneras, ya sea a partir de su grado de capacidad cognitiva o a partir de su grado de autonomía \cite{ref42}.\\

\textbf{Clasificación a partir de su grado de capacidad cognitiva:}
\begin{itemize}
    \item \textbf{Inteligencia Artificial débil o limitada}: está diseñada para realizar tareas específicas de manera eficiente, pero no tiene la capacidad de razonar ni aprender de nuevas situaciones \cite{ref42}.\\
    \item \textbf{Inteligencia Artificial general o fuerte}: este tipo de IA tiene la capacidad de realizar varias tareas cognitivas como el razonamiento, el aprendizaje y la resolución de problemas. A diferencia de la IA débil, la IA fuerte es capaz de adaptarse a nuevas situaciones y entornos \cite{ref42}.\\
    \item \textbf{Super Inteligencia Artificial}: tiene la capacidad de realizar cualquier tarea compleja que requiere Inteligencia Humana, ya que es muy poderosa, y puede superar a los seres humanos en términos de capacidad cognitiva y de aprendizaje \cite{ref42}.\\
\end{itemize}

\textbf{Clasificación de acuerdo con su grado de autonomía:}

\begin{itemize}
    \item \textbf{Inteligencia Artificial Reactiva}: este tipo de IA realiza tareas específicas de manera autónoma, pero no tiene la capacidad de recordar eventos pasados ni de anticipar situaciones futuras. Es útil en situaciones en las que se requieren respuestas rápidas y precisas a situaciones específicas \cite{ref42}.\\
    \item \textbf{Inteligencia Artificial Deliberativa}: tiene la capacidad de planificar y tomar decisiones basándose en información del entorno y en objetivos predeterminados. Es decir, puede analizar situaciones y elegir opciones que le permitan cumplir con objetivos, o adaptarse a entornos empleando información del pasado y del futuro \cite{ref42}.\\
    \item \textbf{Inteligencia Artificial Cognitiva}: se caracteriza por su capacidad de imitar las funciones cognitivas humanas como lo son el razonamiento, la percepción y el aprendizaje, y tienen la capacidad de adaptarse a nuevas situaciones y entornos \cite{ref42}.\\
    \item \textbf{Inteligencia Artificial Autónoma}: es capaz de interactuar de manera autónoma con su entorno, tomar decisiones y aprender de nuevas situaciones, y cambiar sus objetivos y estrategias en función de las estrategias sin la necesidad de la intervención humana \cite{ref42}.
\end{itemize}

De igual manera, la IA emplea diferentes técnicas, las cuales se enlistan a continuación \cite{ref43}:

\begin{itemize}
    \item \textbf{Búsqueda de soluciones}: esta técnica tiene por objetivo encontrar mecanismos de deducción y búsqueda de soluciones para la resolución de problemas cuando no se cuenta con un método directo \cite{ref43}.\\
    \item \textbf{Representación del conocimiento}: elaboración de métodos y técnicas eficientes que sean capaces de organizar conocimientos en un sistema, para posteriormente ser usados en la búsqueda de soluciones para diferentes problemáticas \cite{ref43}.\\
    \item \textbf{Reconocimiento de patrones}: son técnicas de clasificación para identificar subgrupos midiendo el parecido o similitud entre formas, con el objetivo de obtener conclusiones \cite{ref43}.\\
    \item \textbf{Robótica}: esta técnica tiene por objetivo la construcción de robots inteligentes capaces de funcionar con autonomía, que cuenten con la habilidad de realizar procesos mecánicos y manuales con el fin de obtener mayor productividad, suplir mano de obra y proporcionalidad flexibilidad en procesos industriales \cite{ref43}.\\
    \item \textbf{Redes Neuronales}: son sistemas compuestos por estructuras de red, con un gran número de conexiones entre diferentes capas de procesadores, que a su vez tienen asignadas diferentes funciones. Las redes neuronales efectúan una labor de aprendizaje por la reproducción de las salidas de un conjunto de entrenamiento \cite{ref43}.\\
    \item \textbf{Algoritmos genéticos}: son los tipos de algoritmos que tratan de emular el proceso de selección natural a un problema dado, en el que se aplican operadores genéticos para evaluar cada una de las soluciones propuestas. Se emplean procedimientos de búsqueda y optimización para mejorar las soluciones existentes y generar nuevas \cite{ref43}.\\
    \item \textbf{Sistemas expertos}: sistemas que almacenan conocimientos de expertos sobre un área o campo especializado, para obtener una solución mediante una deducción lógica \cite{ref43}. \\
    \item \textbf{Procesamiento de Lenguaje Natural (PLN)}: se centra en el diseño de métodos y algoritmos que toman como entrada o producen como salida datos en la forma del lenguaje humano, ya sea en forma de texto, audio o animación \cite{ref44}.
\end{itemize}

En este Trabajo Terminal nos centraremos en la técnica de Procesamiento de Lenguaje Natural (PLN). En el siguiente apartado se profundizará más en el concepto, características y usos del PLN.\\

\section{Procesamiento de Lenguaje Natural (PLN)}
El Procesamiento de Lenguaje Natural (PLN, o NLP por sus siglas en inglés) es el campo de estudio que busca entender cómo funciona el lenguaje, su construcción, la generación de nuevo lenguaje, así como todas las tareas que tienen relación con el tratamiento del lenguaje como lo es la generación de texto, traductores, generadores de resúmenes, chatbots, entre otros \cite{ref45}.\\

El PLN emplea el lenguaje natural para establecer comunicación entre un ser humano y una computadora. Esta última deberá entender las oraciones que le sean proporcionadas mediante modelos que le ayuden a entender los mecanismos humanos relacionados con el lenguaje \cite{ref46}. 

\subsection{Arquitectura de un sistema de PLN}

La arquitectura de un sistema de PLN está dividida en los siguientes niveles \cite{ref47}:\

\begin{enumerate}[label=\alph*.]
    \item \textbf{Nivel Fonético}: en este nivel se interpretan los sonidos dentro de las palabras.
    \item \textbf{Nivel Fonémico}: se trabajan con los fonemas, los cuales son unidades teóricas básicas para estudiar el nivel fonológico de la lengua humana, ya que analizan la varianza en la pronunciación cuando las palabras están conectadas.
    \item \textbf{Nivel Morfológico}: indica cómo es que las palabras se construyen a partir de unidades de significado más pequeñas, llamadas morfemas.
    \item \textbf{Nivel Léxico}: se encarga del significado individual de cada palabra, analizando cada una de las palabras para conocer su significado y función dentro de una oración, tomando en cuenta el contexto en el que se encuentre.
    \item \textbf{Nivel Sintáctico}: se analiza cómo es que las palabras se unen para formar oraciones, entendiendo la función estructural que cada palabra posee.
    \item \textbf{Nivel Semántico}: se refiere al significado de las palabras, y cómo los mismos se unen para darle sentido a una oración, considerando también el contexto de la oración.
    \item \textbf{Nivel de Discurso}: se encarga de trabajar con unidades de texto grandes, haciendo conexiones entre las oraciones. Se identifica la función que cumple cada oración en el texto, sumando información al significado del texto completo.
    \item \textbf{Nivel Pragmático}: trata de cómo las oraciones son empleadas en diferentes situaciones y cómo es que el uso afecta el significado de las mismas.
\end{enumerate}

\begin{center}
    \includegraphics[width=0.8\textwidth]{Images/Cap 2/Niveles_Arquitectura_PLN.png}
    \captionof{figure}[Niveles de la arquitectura de un Sistema de Procesamiento de Lenguaje Natural]{Niveles de la arquitectura de un Sistema de Procesamiento de Lenguaje Natural, obtenido de \cite{ref46}.}  % Pie de foto manual
\end{center}

\textbf{Los pasos que sigue la arquitectura del sistema de PLN son los siguientes \cite{ref46}:}
\begin{enumerate}
    \item El usuario le expresa a la computadora lo que desea hacer.\\
    \item La computadora analiza las oraciones que el usuario le proporciona, en el sentido morfológico y sintáctico. En otras palabras, se verifican los componentes léxicos definidos y se verifica si se cumple un orden gramatical entre los elementos identificados.\\
    \item Se realiza un análisis sintáctico de las oraciones, para saber cuál es el significado de cada oración.\\
    \item Después de realizar el paso anterior, se lleva a cabo un análisis pragmático de todas las oraciones juntas. Al final de este paso, la computadora obtiene la expresión final.\\
    \item Una vez obtenida la expresión final, la misma es ejecutada para obtener un resultado que será proporcionado al usuario.
\end{enumerate}

\subsection{Técnicas de PLN}
El PLN se apoya de un conjunto de técnicas mediante las cuales se extrae información determinada de un texto. A continuación, se describen algunas de las técnicas más comunes utilizadas \cite{ref47}:

\begin{enumerate}
    \item \textbf{Detección de oraciones}: esta técnica se encarga de recortar una secuencia de caracteres entre dos signos de puntuación; el signo debe estar acompañado por un espacio en blanco y se excluye el caso de la primer frase y en posibles ocasiones la última frase. Corresponde el nivel de procesamiento sintáctico dentro de la arquitectura de PLN.\\
    
La detección de oraciones puede presentar algunas dificultades a la hora de procesar títulos, abreviaturas, o algunos elementos que no siguen algún patrón de texto plano. En esos casos se emplean bancos de palabras, que incluyen aquellos símbolos o abreviaturas necesarias para detectar las sentencias, y posteriormente son cargadas en el modelo \cite{ref47}.
\begin{center}
    \includegraphics[width=0.8\textwidth]{Images/Cap 2/Deteccion_Oraciones.png}
    \captionof{figure}[Ejemplo de la delimitación de oraciones dentro de un párrafo]{Ejemplo de la delimitación de oraciones dentro de un párrafo, obtenido de \cite{ref47}.}
    \label{tabla_delimitacion_oraciones}
\end{center}
La \autoref{tabla_delimitacion_oraciones} muestra que, en el párrafo, el modelo en español determina que “Sr.” es una abreviatura de la palabra “Señor” y por consiguiente ignora el signo de puntuación como final de la oración.

\item \textbf{Segmentación por palabras}: después de que se identifican cada una de las oraciones que componen un texto, se procede a la segmentación por palabras, más conocida como analizador léxico o “\textit{Tokenizer}”.\\

Esta técnica, perteneciente al nivel léxico, consiste en la identificación de \textit{tokens}, los cuales son unidades lingüísticas como palabras, puntuación, números, caracteres alfanuméricos, etc. Para identificar \textit{tokens} en idiomas modernos, se delimitan espacios en blanco con límites de palabra, entre comillas, paréntesis y puntuación.\\

El trato con las abreviaciones es similar a la detección de oraciones, ya que se emplea una lista de palabras recortadas reconocidas \cite{ref47}.\begin{center}
    \includegraphics[width=0.8\textwidth]{Images/Cap 2/Separacion_Palabras.png}
    \captionof{figure}[Ejemplo de separación de palabras en un párrafo]{Ejemplo de separación de palabras en un párrafo, obtenido de \cite{ref47}.}
    \label{separacion_palabras}
\end{center}
En la \autoref{separacion_palabras} se obtiene la lista de palabras del texto, con la separación por palabras indicada por los espacios en blanco y los signos de puntuación. 

\item \textbf{Etiquetado gramatical o \textit{Part-of-Speech (POS) - tagging}}: el proceso de etiquetado gramatical consiste en asignar la categoría gramatical a cada una de las palabras de un texto, de acuerdo con la definición de esta o el contexto en el que aparece, como lo pueden ser los sustantivos, adjetivos, adverbios, etc. \\

Para lograr lo anterior, es primordial establecer las relaciones de una palabra con sus adyacentes dentro de una frase o de un párrafo. Un mismo \textit{token} puede tener múltiples etiquetas POS, pero solo una es válida dependiendo del contexto.
\begin{center}
	\makebox[\textwidth]{%
		\includegraphics[width=1\textwidth]{Images/Cap 2/POS-tagging.png}
	}
    \captionof{figure}[POS Tagger]{\textit{POS Tagger}, obtenido de \cite{ref47}.}  % Pie de foto manual
\end{center}

\item \textbf{Segmentación morfológica}: en esta etapa, se realiza la identificación de morfemas, que son un fragmento mínimo capaz de expresar el significado de una palabra, es decir, es la unidad significativa más pequeña de un idioma.\\

La identificación de morfemas permite el análisis en profundidad de una palabra en un texto, ya que de esta forma se obtiene información específica como el género, modo, tiempo, etc., y es posible ubicar de manera precisa cada palabra de una oración.\\

Los morfemas se clasifican en 2 categorías. Los morfemas independientes admiten cierta libertad fonológica del lexema: 

\begin{itemize}
    \item \textbf{Pronombres}: cuíde-se, di-le, él, ella.
    \item \textbf{Preposiciones}: desde, a, con, de.
    \item \textbf{Conjunciones}: y, e, o, pero, aunque.
    \item \textbf{Determinantes}: él, ella, ese, un, una.
\end{itemize}

Por otro lado, los morfemas dependientes van unidos a otra unidad mínima dotada de significado, conocidos como monemas, para completar su significado.  Los tipos de morfemas dependientes son: 

\begin{enumerate}
    \item \textbf{Derivativos}: estos morfemas son facultativos, es decir, añaden matices al significado de los lexemas.
    \begin{itemize}
        \item Prefijos.
        \item Sufijos.
        \item Interfijos.
    \end{itemize}
    \item \textbf{Flexivos}: estos morfemas son constitutivos, es decir, señalan relaciones gramaticales y sus accidentes entre los diferentes agentes de una acción verbal o una expresión nominal.
    \begin{itemize}
        \item Género
\item Número.
\item Persona.
\item Modo y tiempo.
    \end{itemize}
\end{enumerate}

\item \textbf{Eliminación de \textit{Stop words}}. Mediante esta técnica, se excluyen palabras comunes que tienen poco valor para la recuperación de información, con el fin de reducir el tamaño de un texto y seleccionar las palabras clave. La cantidad de ocurrencias de una palabra en un texto determina si es o no una “\textit{stop word}”, siendo que cuanto más ocurrencias existan menos relevancia tiene en el texto; en su mayoría, los artículos, los pronombres, las preposiciones y las conjunciones.\\

A partir de un listado de palabras \textit{Stop words}, se hace una busqueda de aquellas palabras con mayor ocurrencia dentro de un texto, para su posible eliminación. En ocasiones, al listado de palabras de uso común se le agrega un conjunto de palabras propias del documento que se analiza, empleando la técnica TF-IDF (\textit{Term Frequency - Inverse Document Frequency}), que permite determinar qué palabras son importantes para un documento de acuerdo con la frecuencia de aparición dentro de un texto.
\begin{center}
	\makebox[\textwidth]{%
		\includegraphics[width=1\textwidth]{Images/Cap 2/Deteccion_Stopwords.png}
	}
    \captionof{figure}[Ejemplo de detección de Stop words]{Ejemplo de Detección de \textit{Stop words}, obtenido de \cite{ref47}.}  % Pie de foto manual
\end{center}

\item \textbf{Reconocimiento de Entidades Nombradas (NER)}: se realiza una busqueda y clasificación de elementos de texto que pertenecen a categorías predefinidas, como lo son nombres de personas, nombres de entidades, organizaciones, lugares, expresiones temporales, cantidades, porcentajes. etc.
Para poder hacer el reconocimiento de las diferentes entidades, se utilizan una serie de aproximaciones, siendo necesario ademas, tener una noción del contexto en el cual se encuentra cada una de las entidades para determinar su significado. Finalmente, dentro de las posibles entidades se realiza una asociación con los conceptos del contexto dentro de una base de datos de conocimiento.
\begin{center}
	\makebox[\textwidth]{%
		\includegraphics[width=1\textwidth]{Images/Cap 2/NER.png}
	}
    \captionof{figure}[Reconocimiento de Entidades Nombradas (NER)]{Reconocimiento de Entidades Nombradas (NER), obtenido de \cite{ref47}}  % Pie de foto manual
\end{center}


\item \textbf{\textit{Stemming}}: esta técnica busca un concepto de una palabra mediante la eliminación de prefijos y sufijos para obtener la raíz. De esta manera, se reduce la palabra a su mínimo elemento con significado.
\begin{center}
    \includegraphics[width=0.8\textwidth]{Images/Cap 2/Stemming.png}
    \captionof{figure}[Ejemplo de los términos derivados de la raíz “catalog”]{Ejemplo de los términos derivados de la raíz “catalog”, obtenido de \cite{ref47}.}  % Pie de foto manual
\end{center}
No obstante, es importante mencionar que esta técnica no siempre funciona correctamente debido a que hay palabras que poseen raíces compartidas por más de un significado. 
\begin{table}[H]
    \centering
    \begin{tabular}{|p{3cm}|p{2.5cm}|p{6cm}|}
        \hline
        \textbf{Término con prefijo} & \textbf{Raíz/Stem} & \textbf{Término con el que causaría confusión} \\
        \hline
        Prevalencia & valenc & Valencia, valencia, valenciano, ambivalencia, polivalencia \\
        \hline
        Precatalogar & catalog & Descatalogar, catálogo \\
        \hline
    \end{tabular}
    \caption[Ejemplos de términos con raíces compartidas]{Ejemplos de términos con raíces compartidas, obtenido de \cite{ref47}.}
    \label{tabla:confusion}
\end{table}

\end{enumerate}

\subsection{Embeddings Semánticos}
Los embeddings semánticos constituyen una forma de representar palabras, frases o incluso documentos mediante valores numéricos. En esencia, consisten en transformar el lenguaje natural, difícil de procesar directamente por una computadora, en vectores matemáticos que sí pueden ser manipulados mediante operaciones algebraicas. Esto permite que los sistemas automáticos capten relaciones de significado más allá de la simple coincidencia de palabras, logrando una comprensión más profunda del contenido lingüístico \cite{refebd1}.\\

Para obtener estas representaciones, se emplean algoritmos de aprendizaje automático entrenados con grandes volúmenes de texto, como libros, artículos científicos o recopilaciones masivas de contenido web. Durante el entrenamiento, el modelo identifica patrones, asociaciones y regularidades en el uso del lenguaje. De esta manera, aprende a ubicar cada palabra dentro de un espacio vectorial de muchas dimensiones, donde la proximidad entre vectores refleja su similitud semántica. Así, términos relacionados aparecen cercanos entre sí, mientras que aquellos con significados distintos se separan en el espacio \cite{refebd1}.\\

\noindent \textbf{Vectores semánticos o representaciones vectoriales}

Los vectores semánticos, también denominados representaciones vectoriales o simplemente embeddings, son la base del procesamiento moderno del significado en modelos computacionales. Un vector puede entenderse como una secuencia ordenada de números que define un punto dentro de un espacio con múltiples dimensiones; cada una de estas dimensiones expresa alguna característica aprendida por el modelo durante su entrenamiento \cite{refebd1}.\\

Una forma de visualizar este concepto es compararlo con un mapa geográfico. En un mapa tradicional, la posición de un punto se determina mediante dos coordenadas, como latitud y longitud. De manera análoga, en un espacio semántico cada palabra se asocia a un vector que indica su ubicación dentro de cientos o miles de dimensiones. Cuando dos vectores se encuentran cercanos, significa que sus palabras representan ideas o significados similares; si están lejanos, la relación semántica entre ellas es escasa o inexistente \cite{refebd1}.\\

En la práctica, los embeddings suelen ser representaciones densas de alta dimensionalidad, generalmente entre 128 y 1024 dimensiones, lo que permite capturar matices complejos del significado lingüístico con un grado notable de precisión \cite{refebd2}.\\

\begin{center}
    \includegraphics[width=0.8\textwidth]{Images/Cap 2/Embeddings/1_RepresentaciónVectorial.png}
    \captionof{figure}[Representación Vectorial]{Representación vectorial, obtenido de \cite{refebd1}.}  % Pie de foto manual
\end{center}

\noindent \textbf{Dimensiones de un embedding como una analogía con mapas}

Las dimensiones de un vector semántico pueden entenderse del mismo modo que las coordenadas que ubican un punto en un mapa. Sin embargo, en lugar de señalar una posición geográfica, cada dimensión sitúa una palabra o frase dentro de un espacio semántico abstracto. Cada una de estas dimensiones refleja alguna característica latente del significado, como matices emocionales, atributos gramaticales o cualquier otro rasgo aprendido por el modelo durante el entrenamiento \cite{refebd2}.\\

Cuando todas las dimensiones se combinan, forman una especie de “mapa conceptual” en el que las palabras se distribuyen según sus semejanzas. Por esta razón, dos términos con significados próximos aparecerán cercanos en este espacio, mientras que aquellos que no comparten relación se encontrarán a mayor distancia \cite{refebd2}.\\

Por ejemplo, si una de las dimensiones capturara aspectos relacionados con la polaridad emocional, podría esperarse que palabras como “feliz” o “alegre” obtuvieran valores similares, mientras que palabras como “triste” o “deprimido” se localizarían en posiciones opuestas dentro de esa misma dimensión.\\

\noindent \textbf{Cómo los vectores representan el significado en los textos}

Los embeddings permiten que un sistema automático comprenda el significado de un texto al reflejar patrones de uso y coocurrencia entre las palabras. Aquellos términos que suelen aparecer en contextos similares tienden a adquirir vectores parecidos. De esta manera, palabras como rey y reina se ubicarán cerca entre sí porque suelen relacionarse con conceptos comunes, como la monarquía o el poder \cite{refebd3}.\\

Además, al tratarse de valores numéricos, estos vectores pueden ser manipulados matemáticamente \cite{refebd3}. Por ejemplo, si un modelo ha aprendido atributos vinculados al género gramatical, podría aproximarse la relación:

\begin{center}
    \includegraphics[width=0.9\textwidth]{Images/Cap 2/Embeddings/2_RelacionesSemánticas.png}
    \captionof{figure}[Relaciones Semánticas]{Relaciones semánticas entre palabras, obtenido de \cite{refebd1}.}  % Pie de foto manual
\end{center}

Si bien los modelos reales manejan estructuras mucho más complejas, este tipo de ejemplos ilustra cómo los embeddings capturan no solo definiciones individuales, sino también las relaciones semánticas que existen entre los conceptos.\\

\noindent \textbf{Embeddings: representaciones de alta dimensionalidad}

Los embeddings actuales, como los generados por modelos avanzados de Google, OpenAI y otros sistemas basados en redes neuronales, suelen tener cientos o miles de dimensiones. Estas dimensiones no están definidas manualmente; surgen de manera automática durante el entrenamiento y cada una representa un aspecto abstracto del significado aprendido por el modelo \cite{refebd1}.\\

El uso de una gran cantidad de dimensiones permite representar los textos con un nivel de detalle muy elevado, similar a disponer de un mapa extremadamente preciso con numerosas coordenadas. Gracias a esta alta dimensionalidad, los modelos pueden capturar relaciones semánticas complejas, matices contextuales, analogías y diferentes sentidos de una misma palabra \cite{refebd1}.\\

Por ejemplo, el siguiente vector podría representar el embedding de la frase “La materia no se crea ni se destruye, solo se transforma”:\\

\begin{center}
    \includegraphics[width=0.5\textwidth]{Images/Cap 2/Embeddings/3_RepresentaciónEmbedding.png}
    \captionof{figure}[Representación Vectorial de un embedding]{Representación vectorial de un embedding, obtenido de \cite{refebd1}.}  % Pie de foto manual
\end{center}

En dicho vector, cada componente corresponde a una dimensión del espacio semántico. La combinación de todos estos valores numéricos produce una representación única del significado global de la frase. Aunque estos números no son interpretables por una persona en términos directos, sí permiten que un sistema automático manipule esa representación de forma matemática \cite{refebd1}.\\

De este modo, cuando se genera el embedding de otra frase con un significado parecido. Por ejemplo: “La energía no desaparece, únicamente cambia de forma”, ambos vectores tienden a aparecer muy próximos dentro del espacio semántico, ya que comparten patrones conceptuales y contextuales. Esta cercanía numérica refleja la similitud entre las ideas expresadas en ambas oraciones.\\

\noindent \textbf{Entrenamiento de los modelos de embedding (usualmente con 768 dimensiones)}

Los modelos encargados de producir embeddings se entrenan utilizando cantidades masivas de texto, que pueden incluir libros, artículos científicos, publicaciones web, código y diversos tipos de documentos. En los últimos años, este proceso ha escalado de forma significativa, con modelos que se entrenan sobre volúmenes de datos cada vez mayores \cite{refebd4}.\\

Durante este entrenamiento, el sistema aprende a identificar relaciones entre palabras y a predecir el contexto en el que podrían aparecer. Este aprendizaje es lo que permite que, posteriormente, pueda transformar cualquier frase en un vector numérico dentro del espacio semántico.\\

\begin{center}
    \includegraphics[width=0.9\textwidth]{Images/Cap 2/Embeddings/4_DimensionesEmbedding.png}
    \captionof{figure}[Dimensiones de un embedding]{Dimensiones de un embedding, obtenido de \cite{refebd1}.}  % Pie de foto manual
\end{center}

El número de dimensiones que tendrá cada embedding se establece como un parámetro del modelo. Un valor muy común en modelos modernos es 768 dimensiones, aunque pueden existir variantes con más o menos dimensiones según los objetivos del sistema y la arquitectura utilizada. Este proceso de entrenamiento exige una enorme capacidad computacional y puede requerir varios días o incluso semanas para completarse, especialmente cuando se trabaja con grandes volúmenes de datos o modelos de última generación \cite{refebd4}.\\

Además de comprender el concepto general de los embeddings, es importante reconocer que existen diversas técnicas para generarlos. Estas técnicas han evolucionado con el tiempo y pueden clasificarse en dos grandes categorías: los métodos clásicos, que producen representaciones estáticas, y los métodos modernos, que generan representaciones contextuales capaces de capturar matices más complejos del lenguaje. A continuación, se describen brevemente los modelos más representativos de cada categoría.\\

\noindent \textbf{Técnicas clásicas de generación de embeddings}

Los métodos clásicos fueron pioneros en la representación vectorial del lenguaje y permitieron reemplazar las palabras por vectores que conservaban relaciones semánticas básicas. Entre los modelos más influyentes se encuentran:

\begin{itemize}
    \item \textbf{Word2Vec}, introducido por Mikolov y colaboradores \cite{refebd5}.
    \item \textbf{GloVe}, basado en la factorización de matrices de co-ocurrencias de palabras \cite{refebd6}.
    \item \textbf{FastText}, que mejora los modelos anteriores al incorporar información de subpalabras y prefijos, permitiendo representar adecuadamente palabras nuevas o poco frecuentes \cite{refebd7}.
\end{itemize}

Estos modelos producen embeddings estáticos: cada palabra posee un único vector sin importar el contexto en el que aparece.\\

\noindent \textbf{Técnicas modernas (embeddings contextuales)}

Con el desarrollo de arquitecturas más avanzadas, como redes bidireccionales y modelos transformers, surgieron métodos capaces de generar embeddings que dependen del contexto de cada oración. Esto permite capturar significados más finos y relaciones más complejas entre las palabras. Entre los modelos destacados se encuentran:

\begin{itemize}
    \item \textbf{ELMo}, basado en redes LSTM bidireccionales que generan representaciones sensibles al contexto \cite{refebd8}.
    \item \textbf{BERT}, modelo basado en transformers y entrenado mediante Masked Language Modeling, que produce representaciones contextuales profundas \cite{refebd9}.
    \item \textbf{Sentence-BERT}, una adaptación de BERT diseñada específicamente para generar embeddings de oraciones altamente útiles en tareas de similitud semántica \cite{refebd10}.
\end{itemize}

Sentence-BERT será el modelo utilizado en este Trabajo Terminal debido a su capacidad para producir representaciones compactas y comparables de oraciones completas.\\

\noindent \textbf{Ventajas de Sentence-BERT}

Este modelo presenta características que lo convierten en una opción especialmente adecuada para aplicaciones que requieren medir similitud semántica entre textos \cite{refebd10}:

\begin{itemize}
    \item Captura el significado global de oraciones completas, no solo de palabras individuales.
    \item Ha sido preentrenado utilizando millones de pares de frases comparadas semánticamente.
    \item Está optimizado para tareas de similitud, búsqueda semántica y recuperación de información.
    \item Permite inferencia rápida sin necesidad de entrenamiento adicional (\textit{fine-tuning}) en la mayoría de los casos.
\end{itemize}

\subsection{Similitud coseno}
La similitud del coseno es una medida ampliamente utilizada para evaluar el grado de semejanza entre dos vectores dentro de un espacio multidimensional. A diferencia de otras métricas basadas en distancia, este método no considera la magnitud de los vectores, sino la dirección hacia la cual apuntan. Esta característica la vuelve especialmente útil en espacios de alta dimensionalidad, donde las distancias euclidiana o Manhattan pueden volverse menos representativas.\\

El cálculo consiste en obtener el coseno del ángulo formado por dos vectores distintos de cero. El resultado es un valor numérico comprendido entre –1 y 1, donde:
\begin{itemize}
    \item \textbf{1} indica que ambos vectores apuntan exactamente en la misma dirección, es decir, tienen una similitud completa.
    \item \textbf{0} señala que los vectores son ortogonales, lo que implica que no existe relación direccional entre ellos.
    \item \textbf{-1} representa direcciones totalmente opuestas, lo cual se interpreta como máxima disimilitud.
\end{itemize}

Esta medida puede visualizarse como la comparación entre dos flechas: si ambas apuntan hacia el mismo lugar, la similitud es alta; si forman un ángulo recto, no guardan relación; y si se orientan en direcciones contrarias, son totalmente diferentes.\\

En el ámbito del Machine Learning, Procesamiento de Lenguaje Natural (PLN) e Inteligencia Artificial, la similitud del coseno resulta fundamental debido a que estos sistemas operan con datos representados como vectores (por ejemplo, embeddings de texto). Al convertir palabras, frases o documentos en vectores numéricos, se pueden comparar semánticamente mediante esta métrica.\\

Por ejemplo, un chatbot puede transformar una consulta en un embedding, compararlo con los vectores almacenados en una base de datos y emplear la similitud del coseno para identificar cuál respuesta es la más cercana en significado. Este proceso permite medir cuánta relación existe entre dos representaciones vectoriales y seleccionar la opción más adecuada.\\

En resumen, la similitud del coseno es la métrica estándar para comparar embeddings, ya que ofrece una forma robusta de evaluar el parecido semántico entre textos mediante la orientación de sus vectores en el espacio.\\

\noindent \textbf{Definición matemática:}

La similitud del coseno se define a partir de la relación angular entre dos vectores. Matemáticamente, se expresa como:

\[
\text{cosine\_similarity}(A, B)
= \frac{A \cdot B}{\|A\| \, \|B\|}
\]

donde:
\begin{itemize}
    \item $A \cdot B$ = producto punto (suma de productos elemento a elemento).
    \item $\|A\|$ = norma euclidiana del vector $A$.
    \item Rango: $[-1, 1]$, típicamente $[0, 1]$ cuando se usan embeddings normalizados.
\end{itemize}

Esta formulación permite medir qué tan alineados están dos vectores dentro de un espacio multidimensional, lo cual es fundamental para comparar representaciones semánticas en tareas de PLN.\\

\noindent Propiedades:

La similitud coseno presenta características que la hacen especialmente útil en modelos basados en embeddings:

\begin{itemize}
    \item Invariante a la magnitud de los vectores
    \item Cómputo eficiente: $O(d)$ donde $d$ es la dimensionalidad
    \item Interpretabilidad: 1.0 = idéntico, 0.0 = ortogonal
\end{itemize}

En el proceso de selección de una métrica adecuada para comparar embeddings, se evaluaron diversas alternativas empleadas comúnmente en tareas de análisis vectorial. Sin embargo, varias de ellas presentan limitaciones importantes cuando se trabajan datos de alta dimensionalidad o representaciones semánticas continuas, razón por la cual fueron descartadas:

\noindent \textbf{Alternativas consideradas y rechazadas:}
\begin{itemize}
    \item \textbf{Distancia Euclidiana}: Su desempeño se ve afectado por la magnitud de los vectores, lo que genera resultados inconsistentes cuando los embeddings tienen distintas escalas o longitudes.
    \item \textbf{Distancia de Manhattan}: Aunque útil en ciertos contextos, no captura la direccionalidad entre los vectores, un aspecto clave para evaluar similitud semántica.
    \item \textbf{Índice de Jaccard}: Diseñado para conjuntos discretos; no es adecuado para vectores densos y continuos, como los generados por modelos modernos de PLN.
\end{itemize}

La similitud del coseno se seleccionó como la métrica óptima debido a sus propiedades, que la han consolidado como un estándar en sistemas basados en embeddings. Entre sus principales ventajas destacan:
\begin{itemize}
    \item \textbf{Robustez en espacios de alta dimensionalidad}: A diferencia de las distancias tradicionales, la similitud coseno mantiene un comportamiento estable incluso cuando los vectores poseen cientos o miles de dimensiones.
    \item \textbf{Independencia de la magnitud}: Al centrarse únicamente en el ángulo entre vectores, evita que diferencias en longitud distorsionen la percepción de similitud.
    \item \textbf{Eficiencia computacional}: Su cálculo es ligero y puede implementarse con gran rapidez utilizando herramientas ampliamente adoptadas como NumPy o SciPy.
    \item \textbf{Versatilidad en múltiples dominios}: Es aplicable en tareas como minería de textos, recuperación de información, sistemas de recomendación y búsquedas semánticas en tiempo real, lo que la convierte en una métrica generalista y ampliamente probada.
\end{itemize}

En conjunto, estas características la hacen especialmente adecuada para comparar representaciones vectoriales en el contexto del Procesamiento de Lenguaje Natural empleado en este proyecto.

\subsection{Transformers y Modelos Pre-Entrenados}
Desde su presentación en 2017, la arquitectura Transformer supuso un cambio definitivo en el desarrollo de modelos de aprendizaje profundo, especialmente en el ámbito del Procesamiento de Lenguaje Natural (PLN). Antes de su aparición, el PLN dependía en gran medida de arquitecturas secuenciales como las redes neuronales recurrentes (RNN) y sus variantes (LSTM y GRU). Aunque efectivas para ciertas tareas, estas redes tenían dificultades para capturar dependencias de largo alcance y requerían procesar la información paso a paso, lo que limitaba su escalabilidad.\\

El trabajo \textit{“Attention Is All You Need”} introducido por Vaswani et al. en 2017 [e6] propuso un enfoque completamente distinto: eliminar la recurrencia y reemplazarla por un mecanismo de atención capaz de modelar relaciones globales entre tokens en una sola operación. Esta innovación permitió acelerar drásticamente el entrenamiento y mejorar la calidad contextual de los modelos, convirtiéndose en la base de sistemas modernos como BERT, GPT, T5 y otras variantes avanzadas.\\

A continuación, se describen los elementos esenciales que conforman la arquitectura Transformer, los cuales permiten comprender por qué se ha convertido en el estándar en modelos de PLN:
\begin{enumerate}
    \item \textbf{Embeddings}. Cada token del texto se representa como un vector denso de alta dimensionalidad. Esto permite trabajar con significado numérico en lugar de cadenas de texto.
    \item \textbf{Codificación posicional}: Como el modelo no es secuencial, se introduce información sobre el orden de los tokens usando vectores posicionales, normalmente basados en funciones senoidales.
    \item \textbf{Mecanismo de atención (Self-Attention)}: Este mecanismo permite que cada token “preste atención” a otros tokens de la secuencia para capturar relaciones contextuales. Se utilizan tres vectores: query, key y value.
    \item \textbf{Multi-Head Attention}: Permite al modelo enfocarse en diferentes partes del texto desde distintas perspectivas. Cada “cabeza” aprende un aspecto contextual distinto.
    \item \textbf{Capas Feed-Forward}: Redes neuronales tradicionales que procesan los datos post-atención token por token.
    \item \textbf{Normalización y conexiones residuales}: Estas técnicas estabilizan el entrenamiento y ayudan a preservar el flujo de información.
\end{enumerate}

\noindent \textbf{Arquitectura general: encoder y decoder}

El Transformer original está compuesto por dos bloques principales:
\begin{itemize}
    \item \textbf{Encoder}: procesa la entrada analizando relaciones entre los tokens.
    \item \textbf{Decoder}: genera la salida, utilizando atención tanto a la entrada codificada como al contexto generado anteriormente.
\end{itemize}

Dependiendo del tipo de modelo, puede utilizarse solo una de estas partes:
\begin{itemize}
    \item \textbf{Modelos encoder-only}: Por ejemplo, BERT, empleados para clasificación, extracción de características o embeddings.
    \item \textbf{Modelos decoder-only}: Como GPT, orientados a generación de texto.
    \item \textbf{Modelos encoder–decoder}: Como T5, útiles para tareas de traducción o reformulación.
\end{itemize}

\noindent \textbf{Procesamiento general de un Transformer}
\begin{enumerate}
    \item Tokenización del texto de entrada.
    \item Conversión de tokens a embeddings.
    \item Suma de las codificaciones posicionales.
    \item Procesamiento mediante múltiples capas de self-attention y feed-forward.
    \item (En modelos de generación) aplicación de atención enmascarada para controlar el orden autoregresivo.
    \item Obtención del siguiente token mediante una capa final softmax.
\end{enumerate}

En términos generales, la arquitectura Transformer integra los componentes que se describen a continuación:
\begin{itemize}
    \item \textbf{Encoder}: procesa y contextualiza el texto de entrada.
    \item \textbf{Decoder}: produce la salida token por token (no presente en BERT).
    \item \textbf{Self-Attention}: cada palabra puede relacionarse con todas las demás en la secuencia.
    \item \textbf{Multi-Head Attention}: permite analizar el texto desde diversas perspectivas simultáneamente.
    \item \textbf{Feed-Forward Networks}: capas densas que refinan la representación.
    \item \textbf{Positional Encoding}: incorpora información relativa al orden.
\end{itemize}

A continuación, se enlistan algunos modelos pre-entrenados basados en Transformers.\\

\noindent \textbf{BERT (Bidirectional Encoder Representations from Transformers)}

\begin{itemize}
    \item Entrenado con grandes corpus como Wikipedia y BookCorpus (aprox. 3.3B palabras).
    \item Utiliza dos tareas principales durante el pre-entrenamiento:
    \begin{itemize}
        \item \textbf{Masked Language Model}: predecir palabras ocultas dentro de una oración.
        \item \textbf{Next Sentence Prediction}: aprender relaciones entre oraciones consecutivas.
    \end{itemize}
    \item Su naturaleza bidireccional lo hace ideal para tareas de comprensión del lenguaje.
\end{itemize}

\noindent \textbf{SENTENCE-TRANSFORMERS (BERT)}

Extensión de BERT adaptada para generar embeddings semánticos de oraciones con alta calidad. Se caracteriza por:
\begin{itemize}
    \item Uso de redes siamesas o triplet loss para aprender similitud textual.
    \item Optimización específica para tareas de búsqueda semántica, clustering y clasificación basada en similitud.
    \item Disponibilidad de modelos especializados para distintos idiomas y aplicaciones.
\end{itemize}

El modelo utilizado en este Trabajo Terminal es paraphrase-multilingual-MiniLM-L12-v2. Sus características principales son:
\begin{itemize}
    \item \textbf{Familia}: MiniLM (una variante compacta y eficiente de BERT).
    \item \textbf{Número de capas}: 12 (menor profundidad que BERT-large).
    \item \textbf{Dimensionalidad de los embeddings}: 384.
    \item \textbf{Compatibilidad multilingüe}: soporta más de 50 idiomas, incluido el español.
    \item \textbf{Optimización}: especializado en detección de paráfrasis y similitud semántica.
    \item \textbf{Tamaño reducido (aprox. 120 MB)}: facilita su uso en proyectos con recursos limitados.
    \item \textbf{Velocidad}: capaz de procesar consultas en aproximadamente 40 ms sobre CPU.
\end{itemize}

Este modelo fue elegido debido a su equilibrio entre rendimiento, precisión semántica y eficiencia computacional, lo que lo hace adecuado para el sistema de reconocimiento y correspondencia lingüística requerido en el proyecto.

\subsection{Aplicaciones del PLN}
A continuación, se enlistan las principales aplicaciones del PLN:
\begin{itemize}
    \item \textbf{Recuperación y extracción de información}: la recuperación de información es el proceso de encontrar datos en un repositorio grande, para satisfacer una necesidad de información \cite{ref48}. Por su parte, la extracción de información consiste en la obtención de ciertos elementos dentro de un texto que son de interés, para posteriormente ser pasadas a un formato de base de datos \cite{ref48}.    
    \item \textbf{Minería de datos}: permite descubrir patrones ocultos y relaciones en datos estructurados, empleando técnicas de reconocimiento de información, extracción de información y corpus procesados con técnicas de lingüística computacional \cite{ref48}.
    \item \textbf{Sistemas de búsqueda de respuestas}: son sistemas diseñados para interpretar una pregunta en lenguaje natural y proporcionar una respuesta, para evitar que los usuarios naveguen y lean varias páginas de resultados de búsqueda. Estos sistemas son alimentados con contenido fuente para entender las preguntas del usuario y encontrar las respuestas \cite{ref48}.
    \item \textbf{Generación de resúmenes automáticos}: consiste en emplear herramientas de PLN, para tomar una colección de términos, frases o párrafos significativos que definen el significado del texto original para generar un resumen. También se pueden emplear técnicas de PLN para parafrasear un texto y producir una síntesis \cite{ref48}.
    \item \textbf{Análisis de sentimientos}: identificación y extracción de información subjetiva, empleando herramientas de PLN y software de análisis de textos para automatizar el proceso. El análisis de sentimientos emplea una clasificación polarizada de sentimientos que consiste en un rango de -10 a 10 que se basa en el aprendizaje para evaluar emociones tanto negativas como positivas en corpus etiquetados de entrenamiento \cite{ref48}.
    \item \textbf{Traducción automática}: consiste en tomar el texto escrito en un lenguaje y traducirlo a otro, manteniendo el mismo significado. El proceso de traducción automática sigue tres pasos: en primer lugar, el texto en el lenguaje original se transforma a una representación intermedia, luego se realizan modificaciones a esta representación intermedia basándose en la morfología del lenguaje, y finalmente se transforma al lenguaje destino \cite{ref48}.
\end{itemize}

\section{Sistema Operativo Android}
Un sistema operativo móvil es un conjunto de programas que habilitan características específicas de un teléfono móvil y brindan servicios a las aplicaciones móviles que se ejecutan en él \cite{ref57}.\\

El sistema operativo Android es un sistema operativo móvil desarrollado por la empresa estadounidense Google y que está basado en el sistema operativo Linux. Es un sistema operativo abierto, gratuito, versátil, seguro y altamente personalizable que está desarrollado para dispositivos móviles como smartphones y tablets \cite{ref57}.

\begin{center}
    \includegraphics[width=0.6\textwidth]{Images/Cap 2/Android_Logo.png}
    \captionof{figure}[Logo de Android]{Logo de Android, obtenido de \cite{ref58}.} 
\end{center}

Las principales características de Android son las siguientes \cite{ref57}:

\begin{itemize}

    \item \textbf{Interfaz de Usuario (UI) personalizable}: los usuarios son capaces de cambiar el aspecto de sus dispositivos para adaptarlos a sus necesidades.
    \item \textbf{Compatibilidad con múltiples fabricantes}: este sistema operativo es ejecutado en una gran cantidad de dispositivos de múltiples fabricantes.
    \item \textbf{Google Play Store}: cuenta con una tienda de aplicaciones que permite descargar diferentes aplicaciones de diversa índole, basadas en las necesidades de cada usuario.
    \item \textbf{Asistente Virtual}: los usuarios tienen acceso a un asistente virtual llamado Google Assistant, que ayuda en la realización de tareas.
    \item \textbf{Integración con servicios de Google}: Android está integrado con servicios de Google como Gmail, Google Drive, Google Photos, Maps, entre otros.
    \item \textbf{Compatibilidad con tecnologías emergentes}: es compatible con tecnologías como la Realidad Virtual (VR), la Realidad Aumentada (AR) y los asistentes virtuales. 
\end{itemize}

La versión actual del Sistema Operativo Android, al momento, es la versión Android 14, la cual fue anunciada el 04 de Octubre de 2023.\\

\section{React Native}
React Native es un framework de aplicaciones móviles desarrollado por Meta (Facebook) que permite crear aplicaciones para iOS y Android empleando JavaScript y React \cite{ref59}.\\

Permite crear aplicaciones con un rendimiento similar a las nativas, empleando las mismas API nativas que emplean otras aplicaciones. Sus principales características son \cite{ref59}:

\begin{itemize}
    \item \textbf{Desarrollo multiplataforma}: puede ser desplegado en Android, iOS, Web, etc.
    \item \textbf{Uso de JavaScript}: la base de React Native es JavaScript.
    \item \textbf{Componentes reutilizables}: se emplean componentes de React, que a su vez son bloques de código reutilizables.
    \item \textbf{Acceso a las API nativas}: se puede acceder a características como la cámara, mícrofono, GPS, etc.
    \item \textbf{Rendimiento nativo}: las apps de React Native tienen un rendimiento similar a las apps nativas.
\end{itemize}

La última versión de React Native, al momento de la elaboración de este proyecto, es la 0.79 \cite{ref60}.\\

\begin{center}
    \includegraphics[width=0.6\textwidth]{Images/Cap 2/ReactNative-Logo.png}
    \captionof{figure}[Logo de React Native]{Logo de React Native, obtenido de \cite{ref60}.} 
\end{center}

\subsection{Expo}
Expo es un framework diseñado para facilitar el desarrollo de aplicaciones en React, el cual integra un conjunto de herramientas y servicios que funcionan sobre React Native y las principales plataformas móviles. Su propósito es simplificar las tareas de creación, compilación y despliegue de aplicaciones en iOS, Android e incluso en la Web, utilizando siempre una misma base de código escrita en JavaScript o TypeScript con React \cite{refexpo1}.\\

\begin{center}
    \includegraphics[width=0.8\textwidth]{Images/Cap 2/expo.jpg}
    \captionof{figure}[Logo de Expo]{React Expo, obtenido de \cite{refexpologo}.} 
\end{center}

Además, Expo ofrece acceso a una gran variedad de funciones nativas desde JavaScript, permitiendo trabajar con elementos como la cámara, notificaciones, síntesis de voz, sensores del dispositivo, detección facial y manejo de video, entre otros \cite{refexpo1}.\\

Las características principales de Expo son las siguientes \cite{refexpo2}:
\begin{itemize}
    \item \textbf{Configuración automatizada}: la mayor parte de la preparación del entorno es gestionada directamente por Expo.
    \item \textbf{Código multiplataforma}: con un solo desarrollo se puede ejecutar la aplicación tanto en Android como en iOS.
    \item \textbf{Sin necesidad de configuración nativa}: es posible probar la aplicación mediante Expo Go sin instalar herramientas como Android Studio o Xcode.
    \item \textbf{Bibliotecas integradas}: incluye múltiples API preconstruidas que permiten acceder fácilmente a funcionalidades nativas del dispositivo.
\end{itemize}

Las herramientas destacadas del ecosistema de Expo incluyen \cite{refexpo3}:
\begin{itemize}
    \item \textbf{Expo Go App}: aplicación móvil que permite visualizar y probar el proyecto en tiempo real, sin necesidad de compilarlo por completo.
    \item \textbf{Expo SDK}: conjunto de API listas para usar, que proporcionan acceso rápido a servicios como la cámara, notificaciones push o geolocalización.
    \item \textbf{Expo CLI}: herramienta de línea de comandos que gestiona el ciclo de vida completo del proyecto, desde su creación hasta su publicación.
\end{itemize}

A continuación, se enlistan las ventajas de utilizar Expo para el desarrollo de aplicaciones móviles \cite{refexpo4}:
\begin{itemize}
    \item \textbf{Desarrollo ágil}: crear una app con React Native y Expo resulta más rápido que mantener proyectos nativos separados. Además, permite integrar fácilmente numerosas bibliotecas externas, reduciendo tiempos de implementación.
    \item \textbf{Pruebas simplificadas}: las aplicaciones pueden evaluarse en distintos dispositivos de forma sencilla, lo que mejora la calidad y la transparencia del proceso de desarrollo.
    \item \textbf{Actualizaciones rápidas}: gracias a EAS Update, los cambios o parches pueden implementarse de manera inmediata, garantizando un rendimiento estable y eficiente en todas las plataformas.
\end{itemize}

Expo para React Native constituye una solución potente y optimizada para acelerar el desarrollo de aplicaciones móviles sin perder flexibilidad. Ya sea para la creación de prototipos o proyectos multiplataforma, Expo proporciona un ecosistema integral de herramientas y servicios que simplifican el flujo de trabajo y mejoran la eficiencia en todo el proceso \cite{refexpo3}.\\

\section{API}
Las API (del inglés Application Programming Interfaces) son mecanismos que permiten la comunicación entre dos componentes de software mediante definiciones, protocolos y reglas de intercambio. Actúan como un contrato de servicio que especifica cómo deben comunicarse las aplicaciones a través de solicitudes y respuestas estructuradas \cite{refapi1}. En este modelo, el cliente envía la solicitud y el servidor la procesa y devuelve una respuesta, permitiendo la integración entre distintos sistemas \cite{refapi1}.\\
\newline \textbf{Tipos de API según su funcionamiento}

\begin{center}
    \includegraphics[width=0.8\textwidth]{Images/Cap 2/API.png}
    \captionof{figure}[APIS]{Funcionamiento de las API, obtenido de \cite{refapi2}.} 
\end{center}

Las API pueden operar de distintas formas \cite{refapi1}:
\begin{itemize}
    \item \textbf{SOAP}: usa XML para intercambiar información entre cliente y servidor; actualmente se considera menos flexible.
    \item \textbf{RPC}: permite ejecutar funciones remotas en el servidor y recibir los resultados.
    \item \textbf{WebSocket}: admite comunicación bidireccional en tiempo real mediante objetos JSON.
    \item \textbf{REST}: es el tipo más utilizado, basado en HTTP y en el intercambio de datos entre cliente y servidor.
\end{itemize}

\noindent\textbf{Clasificación por ámbito y uso}

Según su alcance y aplicación, las API se agrupan en \cite{refapi1}\cite{refapi2}:
\begin{itemize}
    \item \textbf{Privadas o internas}: se usan dentro de una empresa para conectar sistemas propios.
    \item \textbf{Públicas o abiertas}: accesibles a cualquier usuario o desarrollador.
    \item \textbf{De socios o aliados}: restringidas a colaboradores autorizados.
    \item \textbf{Compuestas}: combinan varias API para atender procesos más complejos.
\end{itemize}

\vspace{1em} % ← agrega espacio aquí

También pueden clasificarse por su propósito \cite{refapi2}:
\begin{itemize}
    \item \textbf{De datos}: gestionan operaciones CRUD sobre bases de datos.
    \item \textbf{De sistemas operativos}: permiten a las aplicaciones interactuar con los recursos del sistema.
    \item \textbf{Remotas}: conectan sistemas distribuidos a través de una red.
    \item \textbf{Web}: posibilitan la comunicación entre servicios en la web.
\end{itemize}

Las API exponen un conjunto limitado de acciones y puntos de acceso (endpoints) para que otros sistemas interactúen con ellas. Las solicitudes especifican la acción y los datos requeridos, y las respuestas devuelven la información procesada \cite{refapi3}. También pueden activarse por eventos de negocio, como pagos, registros o actualizaciones de inventario, facilitando procesos automatizados entre múltiples sistemas \cite{refapi3}.\\

Entre las principales ventajas de las API destacan \cite{refapi4}:
\begin{itemize}
    \item \textbf{Ahorro de tiempo y costos}: simplifican el desarrollo al reutilizar funcionalidades existentes.
    \item \textbf{Colaboración e integración}: permiten ampliar arquitecturas con nuevos servicios.
    \item \textbf{Seguridad y control}: limitan el acceso mediante pasarelas y autenticación.
    \item \textbf{Innovación}: fomentan la creación de nuevas aplicaciones y oportunidades de negocio.
\end{itemize}

\subsection{API REST}
La Transferencia de Estado Representacional (REST, por sus siglas en inglés) constituye un estilo arquitectónico que define un conjunto de principios para el diseño de interfaces entre sistemas. Su propósito es establecer reglas que permitan crear APIs capaces de comunicarse de forma eficiente, escalable y fiable, incluso en entornos distribuidos como Internet. Una arquitectura basada en REST se caracteriza por ser flexible, fácil de mantener y por ofrecer portabilidad entre distintas plataformas, lo que la vuelve una opción común para el desarrollo de servicios web modernos.\\

Cuando una API se implementa siguiendo estos principios, se le denomina API REST o API RESTful. Del mismo modo, los servicios web construidos bajo esta arquitectura reciben el nombre de servicios web RESTful. En la práctica, los términos REST API y RESTful API suelen emplearse como equivalentes.\\

A continuación, se describen los principales lineamientos que definen el estilo REST:
\begin{enumerate}
    \item \textbf{Interfaz Uniforme}: La uniformidad de la interfaz es el eje central de REST, ya que garantiza que la comunicación entre cliente y servidor siga un formato consistente. Los datos que se transmiten se denominan representaciones y pueden diferir de la forma en la que el servidor almacena internamente sus recursos. Por ejemplo, la información podría gestionarse como texto plano y enviarse al cliente en formato JSON, XML o HTML.\\
    
    Este principio incluye cuatro restricciones importantes:

    \begin{itemize}
        \item Cada solicitud debe identificar el recurso al que desea acceder, normalmente mediante una URL.
        \item La representación enviada por el servidor proporciona suficiente información para que el cliente pueda modificar o eliminar el recurso si es necesario.
        \item Los mensajes deben ser autodescriptivos, es decir, contener los metadatos necesarios para que el cliente sepa cómo procesarlos.
        \item Las representaciones deben incluir enlaces a otros recursos relevantes, de modo que el cliente pueda descubrirlos sin requerir conocimiento previo del sistema.
    \end{itemize}

    \item \textbf{Tecnología sin estado}: REST establece que cada solicitud enviada al servidor debe contener toda la información necesaria para ser atendida, sin depender de solicitudes anteriores. Esto implica que el servidor no mantiene sesiones ni estados entre interacciones. Gracias a esta independencia, el sistema se vuelve más escalable y sencillo de distribuir, ya que cualquier servidor puede atender cualquier petición sin requerir contexto adicional.
    \item \textbf{Sistema por capas}: En REST, la comunicación entre cliente y servidor puede pasar por uno o más intermediarios, como balanceadores de carga, servidores de seguridad o cachés, sin que el cliente sea consciente de ello. Las capas pueden encargarse de tareas como autenticación, lógica de negocio o almacenamiento, y funcionan de manera transparente para quien consume la API. Esta estructura permite mejorar la organización del sistema y distribuir responsabilidades sin afectar la interacción con el cliente.
    \item \textbf{Almacenamiento en caché}: Los servicios RESTful permiten aprovechar mecanismos de almacenamiento en caché para mejorar el rendimiento. Si ciertos datos no cambian con frecuencia, pueden ser guardados temporalmente por el cliente o por intermediarios para evitar solicitudes repetitivas al servidor. La API debe indicar, mediante cabeceras específicas, si una respuesta puede almacenarse y por cuánto tiempo, permitiendo así reducir la carga del servidor y disminuir los tiempos de respuesta.
    \item \textbf{Código bajo demanda}: Opcionalmente, REST permite que el servidor envíe código ejecutable al cliente para ampliar sus capacidades de forma temporal. Un ejemplo habitual ocurre en páginas web que envían fragmentos de JavaScript al navegador para validar formularios o añadir interactividad. Aunque no es obligatorio, este principio proporciona flexibilidad adicional al aumentar las funciones disponibles del lado del cliente sin requerir instalaciones externas.
\end{enumerate}

\subsection{API RESTful}
Una API RESTful es una interfaz que permite que dos sistemas informáticos intercambien información de forma segura a través de Internet. En entornos empresariales, es habitual que diferentes aplicaciones, internas o de terceros, necesiten comunicarse entre sí para completar diversas tareas. Por ejemplo, un sistema de nóminas puede requerir información de un servicio bancario externo para gestionar pagos, o datos de una aplicación interna para validar horas trabajadas.\\

Las API diseñadas bajo el estilo REST facilitan este tipo de integración, ya que emplean estándares abiertos y bien definidos que permiten una comunicación eficiente, confiable y sencilla de mantener.\\

\noindent \textbf{Principios fundamentales de REST}

Los servicios RESTful se basan en una serie de principios arquitectónicos que garantizan interoperabilidad, escalabilidad y simplicidad:

\begin{enumerate}
    \item \textbf{Cliente–Servidor}: Se mantiene una separación clara entre el cliente (quien consume la información) y el servidor (quien la provee), lo que facilita el mantenimiento y la evolución independiente de cada parte.
    \item \textbf{Sin estado (Stateless)}: Cada petición enviada al servidor debe contener toda la información necesaria para procesarla. El servidor no almacena el estado de las interacciones previas, lo que mejora la escalabilidad.
    \item \textbf{Cacheable}: Las respuestas pueden indicar si es posible almacenarlas en caché, reduciendo la carga sobre el servidor y acelerando la comunicación.
    \item \textbf{Interfaz uniforme}: Todos los recursos deben identificarse mediante URIs y utilizar representaciones estandarizadas, proporcionando una interacción consistente entre cliente y servidor.
    \item \textbf{Sistema en capas}: La arquitectura puede incluir múltiples capas como seguridad, balanceo de carga o intermediarios sin que el cliente necesite conocer su existencia.
\end{enumerate}

\noindent \textbf{Métodos HTTP utilizados en REST}

Los servicios RESTful emplean los métodos del protocolo HTTP para definir la operación que se desea realizar sobre un recurso:

\begin{itemize}
    \item \textbf{GET}: Recupera un recurso o una colección de recursos. Es un método seguro e idempotente.
    \item \textbf{POST}: Crea un nuevo recurso o ejecuta una acción específica en el servidor.
    \item \textbf{PUT}: Reemplaza por completo un recurso existente con una nueva representación.
    \item \textbf{PATCH}: Realiza una actualización parcial del recurso.
    \item \textbf{DELETE}: Elimina el recurso identificado por la URI.
\end{itemize}

\noindent \textbf{Códigos de estado HTTP}

Las respuestas de una API RESTful utilizan códigos de estado estándar para indicar el resultado de la operación:

\begin{itemize}
    \item \textbf{2xx – Éxito}: Ejemplos comunes incluyen 200 OK para peticiones correctas y 201 Created para la creación de un nuevo recurso.
    \item \textbf{4xx – Errores del cliente}: Incluyen situaciones como 400 Bad Request cuando la petición está mal formada o 404 Not Found si el recurso solicitado no existe.
    \item \textbf{5xx – Errores del servidor}: Indican fallos internos, como 500 Internal Server Error.
\end{itemize}

El estilo REST ofrece múltiples beneficios que explican su amplia adopción en la industria del software:
\begin{itemize}
    \item \textbf{Alta escalabilidad}: La ausencia de estado facilita el uso de balanceadores de carga y distribuciones horizontales.
    \item \textbf{Desacoplamiento}: Cliente y servidor pueden evolucionar de manera independiente, siempre que mantengan los contratos de la API.
    \item \textbf{Amplio soporte y estandarización}: REST es uno de los enfoques más utilizados, por lo que existe abundante documentación, librerías y compatibilidad multiplataforma.
    \item \textbf{Herramientas maduras}: Ecosistemas como Postman, Swagger/OpenAPI, entre otros, permiten documentar, probar y mantener APIs de manera eficiente.
\end{itemize}

\section{Arquitectura de Microservicios}
La arquitectura de microservicios, abreviada comúnmente como microservicios, es un estilo de diseño de software orientado al desarrollo de aplicaciones como un conjunto de servicios pequeños, independientes y desplegables de forma separada. Cada microservicio posee un dominio de responsabilidad propio y puede implementarse utilizando distintos lenguajes o tecnologías, siempre que mantenga una comunicación coherente mediante contratos de API bien definidos \cite{refmic1,refmic2,refmic3}.\\

A diferencia de las arquitecturas monolíticas, en las que todos los componentes de una aplicación se integran en un único bloque de código, los microservicios promueven un enfoque distribuido y débilmente acoplado, lo que implica que un fallo en un componente no afecta el funcionamiento del resto del sistema \cite{refmic2}\cite{refmic3}. Esta independencia facilita el mantenimiento, la escalabilidad y la actualización continua sin interrumpir el servicio general \cite{refmic3}\cite{refmic4}.\\

Gracias a su flexibilidad, la arquitectura de microservicios permite desarrollar e implementar funcionalidades individuales de manera autónoma, lo que acelera el ciclo de desarrollo y mejora la capacidad de respuesta ante las necesidades del usuario o del negocio \cite{refmic3}\cite{refmic4}. Además, los equipos de desarrollo pueden trabajar de forma paralela en distintos servicios, reduciendo el tiempo de entrega de nuevas características y evitando dependencias entre módulos \cite{refmic2}\cite{refmic4}.\\

Entre los ejemplos más comunes del uso de microservicios se encuentran \cite{refmic1}:
\begin{itemize}
    \item \textbf{Migración de sitios web}: permite dividir una plataforma monolítica en varios microservicios desplegados en la nube, favoreciendo la escalabilidad y el mantenimiento.
    \item \textbf{Gestión de contenido multimedia}: posibilita el almacenamiento y distribución de videos o imágenes mediante sistemas de almacenamiento de objetos escalables.
    \item \textbf{Procesamiento de transacciones y facturación}: separa los módulos de pago y facturación, garantizando que el sistema continúe funcionando incluso si un servicio falla.
    \item \textbf{Procesamiento de datos}: amplía la compatibilidad con la nube de los servicios de análisis y gestión de datos.
\end{itemize}

Las aplicaciones basadas en microservicios presentan varios atributos distintivos \cite{refmic2}\cite{refmic3}, como lo son:
\begin{itemize}
    \item \textbf{Independencia de desarrollo y despliegue}: cada servicio puede diseñarse, probarse y desplegarse sin afectar a los demás.
    \item \textbf{Comunicación basada en API}: los microservicios se comunican entre sí mediante interfaces estandarizadas, lo que permite la integración sin modificar los datos de origen.
    \item \textbf{Escalabilidad selectiva}: los servicios pueden ampliarse individualmente según la demanda, optimizando los recursos y reduciendo costos.
    \item \textbf{Diversidad tecnológica}: cada servicio puede desarrollarse con el lenguaje o framework más adecuado a su función.
\end{itemize}

La adopción de microservicios ofrece múltiples beneficios frente a las arquitecturas tradicionales \cite{refmic3}\cite{refmic4}. A continuación se enlistan algunos:
\begin{itemize}
    \item \textbf{Resiliencia y tolerancia a fallos}: al ser componentes independientes, los errores en un servicio no afectan al resto de la aplicación.
    \item \textbf{Agilidad y velocidad de desarrollo}: la división en módulos pequeños acelera la creación, prueba e implementación del software.
    \item \textbf{Escalabilidad y mantenimiento mejorado}: los equipos pueden escalar servicios específicos y depurar errores sin interrumpir toda la aplicación.
    \item \textbf{Eficiencia operativa}: algunos equipos pueden gestionar sus propios microservicios, reduciendo la necesidad de departamentos de operaciones separados.
    \item \textbf{Dinamismo tecnológico}: se facilita la adopción de nuevas herramientas o tecnologías sin reescribir toda la aplicación.
\end{itemize}

Los microservicios representan un modelo arquitectónico moderno y modular, ideal para entornos nativos de la nube y para organizaciones que buscan agilidad, escalabilidad y resiliencia en sus aplicaciones \cite{refmic1,refmic2,refmic3,refmic4}.\\

\begin{center}
    \includegraphics[width=0.95\textwidth]{Images/Cap 2/microservicios.png}
    \captionof{figure}[Arquitectura de Microservicios]{Arquitectura de Microservicios, obtenido de \cite{refmic5}.} 
\end{center}


\section{Servicios en la Nube}
Los servicios en la nube son soluciones tecnológicas que permiten a usuarios y organizaciones acceder bajo demanda a recursos informáticos, aplicaciones y datos a través de internet, sin necesidad de poseer infraestructura o hardware especializado. Estos servicios son administrados por un proveedor de nube, lo que permite a los usuarios obtener acceso flexible y económico a las herramientas que necesitan, sin asumir los costos ni la complejidad de su gestión \cite{refnub1}.\\

El funcionamiento de la nube se basa en la virtualización de recursos que residen en servidores físicos ubicados en centros de datos gestionados por los proveedores. Los usuarios acceden a estos recursos mediante una conexión a internet, almacenando y procesando la información en servidores remotos en lugar de en equipos locales. Generalmente, el uso de la nube se cobra por suscripción o consumo, lo que representa una alternativa más económica frente al mantenimiento de infraestructura propia \cite{refnub1}.\\

\begin{center}
    \includegraphics[width=0.8\textwidth]{Images/Cap 2/Nube.png}
    \captionof{figure}[Servicios en la Nube]{Funcionamiento de los Servicios en la Nube, obtenido de \cite{refnub2}.} 
\end{center}

Entre las principales ventajas de los servicios en la nube se destacan \cite{refnub1}:
\begin{itemize}
    \item Reducción de hardware físico y costos operativos.
    \item Flexibilidad de pago según el consumo real.
    \item Menor carga del área de TI, ya que el mantenimiento recae en el proveedor.
    \item Acceso remoto desde cualquier dispositivo y ubicación.
    \item Colaboración en tiempo real entre equipos mediante plataformas compartidas.
\end{itemize}

\noindent \textbf{Modelos de servicios en la nube}

Los servicios de nube se clasifican comúnmente en tres modelos principales, con una categoría adicional emergente \cite{refnub1}:
\begin{itemize}
    \item \textbf{Infraestructura como servicio (IaaS)}: ofrece recursos virtualizados como servidores, almacenamiento y redes. Ejemplos: Amazon Web Services (AWS) y Microsoft Azure.
    \item \textbf{Plataforma como servicio (PaaS)}: proporciona entornos completos para el desarrollo, prueba y despliegue de aplicaciones, sin gestionar la infraestructura subyacente.
    \item \textbf{Software como servicio (SaaS)}: ofrece aplicaciones completas alojadas en la nube, accesibles desde internet, como Google Workspace o Microsoft 365.
    \item \textbf{Todo como servicio (XaaS)}: extiende el modelo a cualquier recurso tecnológico disponible como servicio bajo demanda.
\end{itemize}

\vspace{1em} % ← agrega espacio aquí

Contar con un proveedor especializado en la nube aporta beneficios adicionales \cite{refnub3}, como lo son:
\begin{itemize}
    \item Agilidad empresarial y mantenimiento simplificado.
    \item Costos reducidos mediante esquemas de pago por uso.
    \item Escalabilidad inmediata según la demanda.  
    \item Alta fiabilidad e infraestructura redundante.  
    \item Acceso centralizado y movilidad global.
    \item Recuperación ante fallos gracias a planes de continuidad del negocio.  
    \item Componentes de la infraestructura de nube  
\end{itemize}

\vspace{1em} % ← agrega espacio aquí

La infraestructura de nube se compone de varios elementos esenciales \cite{refnub4}:
\begin{itemize}
    \item \textbf{Servidores}: equipos de alto rendimiento ubicados en centros de datos distribuidos.
    \item \textbf{Redes}: conectividad entre aplicaciones, servicios y datos mediante equipos de conmutación y balanceadores de carga.
    \item \textbf{Almacenamiento}: espacios escalables para guardar datos de manera segura y accesible.
    \item \textbf{Software}: herramientas de administración, máquinas virtuales y entornos de análisis que facilitan el uso de los recursos virtualizados.
\end{itemize}

Los servicios en la nube ofrecen una amplia gama de beneficios, que van desde bajos costos iniciales hasta una mayor libertad de acceso al software. Su naturaleza adaptable y escalable permite que se ajusten a diferentes proyectos y empresas de cualquier tamaño o sector. La diversidad de opciones disponibles brinda la posibilidad de seleccionar el tipo de servicio que mejor se alinee con las necesidades de cada proyecto, lo que convierte a la nube en una solución versátil y aplicable en prácticamente cualquier entorno \cite{refnub5}.

\subsection{AWS}
Amazon Web Services (AWS) es uno de los principales proveedores de servicios en la nube a nivel mundial. Ofrece una amplia gama de soluciones tecnológicas que incluyen almacenamiento, recursos de computación, bases de datos, desarrollo de aplicaciones móviles, inteligencia artificial y herramientas empresariales, todo bajo el modelo de cloud computing \cite{refaws1}.\\

AWS tiene presencia en más de 190 países y cuenta con una extensa red de centros de datos distribuidos en América, Europa, Asia y Oceanía. Su infraestructura global garantiza alta disponibilidad, seguridad y escalabilidad. Entre sus ventajas más destacadas se encuentran \cite{refaws1}:
\begin{itemize}
    \item \textbf{Seguridad}: cuenta con certificaciones internacionales como PCI DSS, ISO 27001, HIPAA y auditorías SOC 1 y SOC 2.
    \item \textbf{Amplia oferta de bases de datos}: soporta sistemas como MySQL, Oracle, PostgreSQL, SQL Server, MongoDB y Amazon Aurora.
    \item \textbf{Bajo costo}: elimina la necesidad de inversión en infraestructura local mediante un modelo de pago por uso.
    \item \textbf{Accesibilidad y flexibilidad}: los servicios pueden adaptarse rápidamente a las necesidades del mercado, con incorporación constante de nuevas herramientas.
    \item \textbf{Gobernanza y visibilidad}: permite auditar y controlar los datos a pesar de que la infraestructura sea administrada por AWS.
    \item \textbf{Resiliencia}: sus centros de datos están diseñados para operar incluso en situaciones críticas o de contingencia.
\end{itemize}

Las herramientas y servicios de AWS abarcan múltiples áreas de la computación en la nube \cite{refaws2}:
\begin{itemize}
    \item \textbf{Cloud computing}: incluye modelos de servicio como IaaS, PaaS y SaaS, permitiendo escalar aplicaciones según la demanda.
    \item \textbf{Redes privadas virtuales}: con Amazon VPC es posible crear entornos seguros y personalizados dentro de la nube.
    \item \textbf{Gestión de bases de datos}: Amazon RDS facilita la creación y administración de bases de datos accesibles desde cualquier dispositivo o ubicación.
    \item \textbf{Aplicaciones empresariales y móviles}: AWS ofrece herramientas para el desarrollo de apps, respaldo de datos (backup), recuperación ante desastres (disaster recovery), autenticación doble, IoT e inteligencia de negocios.
\end{itemize}

Entre los servicios más utilizados y representativos de la plataforma se encuentran \cite{refaws3}:
\begin{itemize}
    \item \textbf{Servicios informáticos}: Amazon EC2 (Elastic Compute Cloud) proporciona potencia de cómputo escalable, y AWS Lambda permite ejecutar código sin necesidad de administrar servidores.
    \item \textbf{Almacenamiento}: Amazon S3 ofrece almacenamiento de objetos escalable, Amazon Glacier permite archivado de bajo costo, y Amazon EBS proporciona almacenamiento en bloques de alto rendimiento.
    \item \textbf{Bases de datos}: Amazon RDS gestiona bases relacionales, mientras que DynamoDB ofrece bases NoSQL rápidas y flexibles.
    \item \textbf{Red y distribución de contenido}: AWS Direct Connect ofrece conexiones dedicadas y Amazon CloudFront acelera la entrega global de contenido mediante una red CDN.
    \item \textbf{Seguridad y cumplimiento}: AWS IAM gestiona el acceso a los recursos y AWS Shield protege contra ataques DDoS.
    \item \textbf{Servicios avanzados}: la plataforma incluye soluciones para aprendizaje automático, análisis de datos, IoT y herramientas de desarrollo, facilitando la integración de funciones inteligentes y automatización.
\end{itemize}

AWS constituye una plataforma integral y versátil que ha transformado la forma en que las organizaciones gestionan sus operaciones digitales. Sus ventajas en seguridad, rendimiento y escalabilidad la convierten en una de las opciones más sólidas del mercado, y su amplia oferta de servicios, alcance global y constante innovación la posicionan como un referente líder en computación en la nube \cite{refaws3}.\\

\begin{center}
    \includegraphics[width=0.7\textwidth]{Images/Cap 2/aws.png}
    \captionof{figure}[Logo de AWS]{Amazon Web Services (AWS), obtenido de \cite{refaws4}.} 
\end{center}

\section{Las 10 Reglas Heurísticas de Usabilidad de Nielsen}
En 1994, Jakob Nielsen estableció un conjunto de reglas que todos los sistemas deben cumplir, para la detección de fallos sin realizar pruebas de usuario. Dichas reglas son útiles para realizar una evaluación de usabilidad de un sitio web, aplicación o producto digital \cite{ref61}.\\

A continuación, se enlistan las 10 reglas heurísticas de usabilidad de Nielsen \cite{ref61}:
\begin{enumerate}
    \item \textbf{Visibilidad y estado del sistema}: el diseño de cualquier interfaz debe mantener informado a los usuarios sobre lo que sucede, para evitar confusiones.
    \item \textbf{Coincidencia entre el mundo real y el sistema}: se deben emplear palabras y conceptos que sean familiares para el usuario, para asegurar la comprensión de la información.
    \item \textbf{Control y libertad al usuario}: los usuarios deben tener la posibilidad de poder deshacer y rehacer acciones, para poder tener el control y evitar que se queden atascados al momento de realizar acciones erróneas.
    \item \textbf{Estándares y consistencia}: los elementos visuales y comportamientos del sistema deben ser uniformes, permitiendo que el usuario pueda navegar de forma intuitiva.
    \item \textbf{Prevención de errores}:se debe priorizar la eliminación de aquellas condiciones que sean propensas al error, por medio de mensajes o validaciones.
    \item \textbf{Reconocimiento para evitar el recuerdo}: la reducción de la información que un usuario tiene que recordar facilita el uso de un sistema o aplicación. Los elementos necesarios deben ser visibles.
    \item \textbf{Flexibilidad y eficiencia de uso}: se le debe permitir a los usuarios a los usuarios adaptar las acciones frecuentes, para acelerar la interacción de un usuario.
    \item \textbf{Diseño estético y minimalista}: el contenido y el diseño deben estar centrados en los elementos esenciales, evitando que las interfaces estén sobrecargadas de información irrelevante o innecesaria.
    \item \textbf{Ayudar a los usuarios para reconocer, diagnosticar y recuperarse de los errores}: los mensajes de error deben ser expresados en un lenguaje sencillo, para su facil detección y tratamiento.
    \item \textbf{Ayuda y documentación}: es fundamental contar con documentación en la que se enlisten los pasos concretos que debe seguir el usuario para evitar errores y poder completar tareas.
\end{enumerate}

\section{Norma ISO 9241-210}
La norma ISO 9241-210, o también conocida como norma ISO 9241-210:2019, es una norma internacional que se enmarca en la categoría más amplia de la norma ISO 9241, un conjunto de normas relacionadas con la ergonomía de la interacción persona-sistema. Esta norma proporciona directrices y principios para el diseño de sistemas interactivos que priorizan las necesidades, capacidades y preferencias de los usuarios, con el objetivo de mejorar su satisfacción y usabilidad \cite{ref62}.\\

Los puntos y objetivos clave descritos en la norma ISO 9241-210:2019 \cite{ref62}:

\begin{itemize}
    \item \textbf{Diseño Centrado en el Ser Humano (HCD)}: este punto resalta la importancia de involucrar a los usuarios finales durante todo el proceso de diseño y desarrollo, para que los sistemas desarrollados sean más intuitivos, eficientes y eficaces.
    \item \textbf{Proceso iterativo}: promueve un proceso de diseño iterativo, donde los diseñadores recopilan continuamente la opinión de los usuarios, perfeccionan sus diseños y los vuelven a probar, con el objetivo de identificar y abordar problemas de usabilidad en las primeras etapas del proceso de diseño.
    \item \textbf{Enfoque centrado en el usuario}: la norma subraya la necesidad de comprender a fondo las características, los objetivos y las tareas de los usuarios. El diseño debe ajustarse a las necesidades de los usuarios.
    \item \textbf{Aplicabilidad}: es aplicable a varios tipos de sistemas interactivos, incluidas aplicaciones de software, sitios web, aplicaciones móviles e interfaces de hardware.
    \item \textbf{Evaluación de usabilidad}: el estándar fomenta el uso de métodos de evaluación de usabilidad, como pruebas de usuarios y evaluaciones de expertos, para evaluar la eficacia del diseño y realizar mejoras.
    \item \textbf{Accesibilidad}: los diseñadores deben garantizar que sus sistemas interactivos sean accesibles para personas con discapacidad y personas de la tercera edad.
    \item \textbf{Documentación}: es primordial documentar todo el proceso de diseño, incluidos los resultados de la investigación de usuarios, las decisiones de diseño y los resultados de las pruebas de usabilidad.
\end{itemize}
\chapter{Desarrollo}
\section{Análisis de viabilidad y factibilidad}
En esta sección, se presenta el análisis de viabilidad y factibilidad del prototipo de aplicación móvil de apoyo para la traducción de español a Lengua de Señas Mexicana (LSM) empleando técnicas de Procesamiento de Lenguaje Natural (PLN) y modelado 3D, con el objetivo de evaluar si existen las condiciones técnicas, humanas, tecnológicas y financieras necesarias para su desarrollo exitoso del mismo. En primer lugar, se analiza la viabilidad, considerando los conocimientos del equipo, las tecnologías disponibles, las condiciones de ejecución y el potencial de escalabilidad del prototipo. Posteriormente, se estudia la factibilidad, enfocándose en los recursos financieros, humanos y tecnológicos, así como en el tiempo estimado para completar el proyecto.\\

\subsection{Viabilidad}
El análisis de viabilidad considera los conocimientos técnicos del equipo, la madurez de las tecnologías disponibles y las condiciones actuales para el desarrollo. Este análisis permite evaluar si el proyecto puede llevarse a cabo de manera exitosa bajo las condiciones planteadas.

\subsubsection{Conocimientos y experiencia}
El equipo de desarrollo posee conocimientos en Procesamiento de Lenguaje Natural (PLN), así como habilidades básicas en animaciones y recursos visuales. Aunque la experiencia en animación 3D orientada a señas, en programación de aplicaciones móviles y en el manejo de la Lengua de Señas Mexicana (LSM) es limitada, se considera factible adquirir y aplicar los conocimientos necesarios mediante el uso de recursos de investigación, bibliotecas especializadas y la colaboración con expertos en LSM, tomando en cuenta la versión correspondiente al uso y documentación del año 2024. Esta disposición de aprendizaje y fortalecimiento de competencias respalda la viabilidad técnica del proyecto en función de las capacidades del equipo.

De manera complementaria, se cuenta con diversas tecnologías que facilitarán la implementación del prototipo, tal como se describe a continuación.

\subsubsection{Tecnologías disponibles}
Actualmente, existen diversas tecnologías y herramientas que facilitan la creación de sistemas de traducción de texto a señas, tales como conjunto de datos de señas en video, motores de animación 3D y frameworks para el desarrollo de aplicaciones móviles. Asimismo, se dispone de plataformas de código abierto que permiten representar señas mediante modelos animados o videos precargados, optimizando así los recursos disponibles para el desarrollo de prototipos.

Las tecnologías consideradas para el presente proyecto incluyen:
\begin{itemize} 
	\item Librerías y conjunto de datos de señas mexicanas (videos de señas). 
	\item Herramientas de animación 3D como Blender (versión 4.4.3) o Unity (versión 6.1), así como motores ligeros compatibles con aplicaciones móviles. 
	\item Frameworks de desarrollo móvil como Flutter (versión 3.29.3) y React Native (versión 0.79). 
	\item Herramientas de Procesamiento de Lenguaje Natural (PLN) para el análisis y segmentación de frases en español. 
\end{itemize}

\subsubsection{Condiciones para la ejecución}
El proyecto se desarrolla en el marco de un trabajo terminal académico, lo que garantiza el acceso a recursos institucionales, asesoría especializada y bibliografía técnica actualizada. De igual manera, el creciente interés social y académico por fomentar la inclusión de la comunidad sorda en México genera un entorno favorable para la implementación de este tipo de iniciativas, fortaleciendo así las condiciones de ejecución del prototipo.

\subsection{Factibilidad}
La factibilidad del proyecto se analiza considerando los recursos financieros, humanos y tecnológicos disponibles, así como el tiempo estimado para su desarrollo y finalización.

\subsubsection{Recursos financieros}
El presente apartado tiene como objetivo evaluar los costos asociados al desarrollo, despliegue y potencial comercialización del prototipo. 

Para ello, se plantea un análisis financiero estructurado en dos niveles: 

\begin{itemize}
	\item \textbf{Enfoque académico:} incluye estimaciones de costos simbólicos aproximados del mercado laboral en México correspondientes al año 2025 [CITA IEEE], considerando el uso de recursos propios, herramientas gratuitas y asesorías puntuales. Este enfoque representa fielmente la ejecución del proyecto dentro de un marco académico.
	
	\item \textbf{Enfoque exploratorio de comercialización:} presenta una simulación de costos reales basada en tarifas aproximadas de mercado laboral en México correspondientes al año 2025 [CITA IEEE], incluyendo sueldos profesionales, licencias, infraestructura tecnológica y servicios externos. Este análisis permite anticipar los requerimientos financieros de una futura etapa de comercialización.
\end{itemize}

Ambas perspectivas se estructuran mediante dos métodos complementarios de análisis financiero:

\begin{itemize}
	\item \textbf{Estimación por recursos:} identifica y valora los insumos materiales, humanos y tecnológicos necesarios para el desarrollo y operación del prototipo.
	\item \textbf{Estimación por actividades:} desglosa las tareas específicas del proyecto, asignando tiempos estimados y costos unitarios para cada una.
\end{itemize}

Esta estructura metodológica permite identificar con precisión los recursos involucrados, calcular el presupuesto total estimado y analizar la factibilidad financiera del proyecto, tanto en su ejecución académica como en un escenario de comercialización futuro.

\paragraph{\textbf{Informe de costos de creación del prototipo.}} 
Este informe presenta el análisis financiero asociado a la creación del prototipo en cuestión. El objetivo es identificar y estructurar los costos necesarios para garantizar la factibilidad del proyecto en su etapa de creación.

\paragraph{\textbf{Presupuesto mediante recursos.}} 
Dentro del análisis basado en recursos, el proceso de desarrollo se ha estructurado en tres etapas principales, cada una integrada por actividades específicas que permiten avanzar de manera ordenada, asegurando la calidad y funcionalidad del prototipo final. Las etapas contempladas son:

\begin{itemize}
	\item \textbf{Creación del prototipo}. 
	\item \textbf{Despliegue del prototipo}.
	\item \textbf{Evaluación financiera}. 
\end{itemize}

Cada una de estas etapas será detallada en los apartados siguientes.

\paragraph{\textbf{Creación del prototipo.}} 
Esta primera etapa abarca todas las actividades necesarias para el desarrollo inicial del prototipo: definición de objetivos, especificaciones técnicas, requisitos, diseño de animaciones, etc. Aquí se implementan técnicas de PLN para la segmentación de frases y modelado/animación 3D para representar las señas en LSM.

\paragraph{\textbf{Despliegue del prototipo.}} 
Incluye acciones para poner en funcionamiento la app en Android (versión 14), publicación, configuración técnica, documentación técnica y pruebas piloto con usuarios. Esta retroalimentación permitirá ajustar y mejorar la experiencia de uso.

\paragraph{\textbf{Evaluación financiera.}} 
Implica analizar los costos de las etapas anteriores y valorar beneficios esperados como impacto social, accesibilidad y sostenibilidad. Aunque el proyecto tiene fines académicos, se incluye un análisis básico de posibilidad de escalamiento.

Este enfoque estructurado de desarrollo permite garantizar una planeación clara y eficiente, considerando los elementos técnicos, operativos y financieros necesarios para la ejecución del proyecto.

Por último, como medida preventiva ante posibles riesgos o imprevistos, se contempla una reserva de contingencia equivalente al 15\% del costo total estimado.


\paragraph{\textbf{Análisis de recursos.}} 
\paragraph{\textbf{Costos de servicios.}} 
A continuación, se detallan los costos asociados a los servicios necesarios para el desarrollo del prototipo, considerando un periodo estimado de ejecución de cuatro meses. Se incluyen los servicios con ambos enfoques, el académico y el exploratorio de comercialización.


\begin{table}[H]
	\centering
	\renewcommand{\arraystretch}{1.6}
	\setlength{\tabcolsep}{12pt}
	\Huge % tamaño de fuente más grande posible
	\begin{adjustbox}{max width=\textwidth}
		\begin{tabular}{|p{7cm}|r|r|r|}
			\hline
			\textbf{Recurso} & \textbf{Costo mensual (MXN \$)} & \textbf{Costo a 4 meses (MXN \$)} & \textbf{Costo a 1 año (MXN \$)} \\ \hline
			Electricidad e Internet (compartido entre integrantes) & \$500.00 & \$2,000.00 & \$6,000.00 \\ \hline
			Almacenamiento en la nube (Google Drive / GitHub) & \$300.00 & \$1,200.00 & \$3,600.00 \\ \hline
			Pruebas en dispositivos Android (emulador físico o virtual) & \$500.00 & \$2,000.00 & \$6,000.00 \\ \hline
			Reserva para servicios externos o pruebas adicionales & \$1,000.00 & \$4,000.00 & \$12,000.00 \\ \hline
			\textbf{Total estimado de servicios} & \textbf{\$2,300.00} & \textbf{\$9,200.00} & \textbf{\$27,600.00} \\ \hline
		\end{tabular}
	\end{adjustbox}
	\caption[Costos estimados de servicios en el escenario académico durante la creación del prototipo]{Costos estimados de servicios en el escenario académico durante la creación del prototipo, elaboración propia.}
	\label{tab:costos_servicios}
\end{table}



\begin{table}[H]
	\centering
	\renewcommand{\arraystretch}{1.6}
	\setlength{\tabcolsep}{12pt}
	\Huge % Tamaño de fuente más grande disponible en LaTeX estándar
	\begin{adjustbox}{max width=\textwidth}
		\begin{tabular}{|p{7cm}|r|r|r|}
			\hline
			\textbf{Recurso o servicio} & \textbf{Costo mensual (MXN \$)} & \textbf{Costo a 4 meses (MXN \$)} & \textbf{Costo a 1 año (MXN \$)} \\ \hline
			Electricidad e internet (oficina dedicada) & \$2,500.00 & \$10,000.00 & \$30,000.00 \\ \hline
			Renta de espacio de coworking (para 4 personas) & \$9,000.00 & \$36,000.00 & \$108,000.00 \\ \hline
			Suscripción a plataformas de desarrollo (GitHub Copilot, Blender Studio, etc.) & \$800.00 & \$3,200.00 & \$9,600.00 \\ \hline
			API de lenguaje natural (OpenAI GPT, DialogFlow) & \$1,000.00 & \$4,000.00 & \$12,000.00 \\ \hline
			Servicios de almacenamiento en la nube (Google Cloud, Firebase, etc.) & \$750.00 & \$3,000.00 & \$9,000.00 \\ \hline
			Servidor para aplicación móvil (Firebase/AWS) & \$1,200.00 & \$4,800.00 & \$14,400.00 \\ \hline
			Licencia de software para animación (Unity, MediaPipe, etc.) & \$1,500.00 & \$6,000.00 & \$18,000.00 \\ \hline
			Validación profesional de señas LSM (freelancer mensual) & \$5,000.00 & \$20,000.00 & \$60,000.00 \\ \hline
			Servicios de soporte técnico y mantenimiento & \$3,000.00 & \$12,000.00 & \$36,000.00 \\ \hline
			Marketing digital (redes sociales, web, SEO) & \$2,000.00 & \$8,000.00 & \$24,000.00 \\ \hline
			Traducción y adaptación de contenido LSM (consultoría externa) & \$4,000.00 & \$16,000.00 & \$48,000.00 \\ \hline
			\textbf{TOTAL ESTIMADO} & \textbf{\$30,750.00} & \textbf{\$123,000.00} & \textbf{\$369,000.00} \\ \hline
		\end{tabular}
	\end{adjustbox}
	\caption[Costos estimados de servicios en un escenario comercial para la creación del prototipo]{Costos estimados de servicios en un escenario comercial para la creación del prototipo, elaboración propia.}
	\label{tab:costos_comercial}
\end{table}


\paragraph{\textbf{Compras no recurrentes.}} 
En esta sección se presentan los costos estimados de artículos y adquisiciones necesarias para la creación del prototipo, clasificadas como compras no recurrentes. Estas compras representan inversiones únicas que no implican costos periódicos, pero que son fundamentales para el correcto desarrollo y prueba de la aplicación.

\begin{table}[H]
	\centering
	\renewcommand{\arraystretch}{1.6}
	\setlength{\tabcolsep}{10pt}
	\Huge
	\begin{adjustbox}{max width=\textwidth}
		\begin{tabular}{|p{7cm}|c|r|r|}
			\hline
			\textbf{Recurso} & \textbf{Unidades} & \textbf{Costo unitario (MXN \$)} & \textbf{Costo total (MXN \$)} \\ \hline
			Equipo de cómputo personal (propio de los integrantes) & 3 & -- & -- \\ \hline
			Dispositivo Android para pruebas físicas & 1 & -- & -- \\ \hline
			Capacitación online (cursos: Blender, MediaPipe, PLN) & 3 cursos & \$800.00 & \$2,400.00 \\ \hline
			Compra de modelos 3D o recursos gráficos (opcional) & 1 paquete & \$2,500.00 & \$2,500.00 \\ \hline
			\textbf{Total compras no recurrentes} & & & \textbf{\$4,900.00} \\ \hline
		\end{tabular}
	\end{adjustbox}
	\caption[Costos estimados de artículos y compras no recurrentes en el escenario académico]{Costos estimados de artículos y compras no recurrentes en el escenario académico, elaboración propia.}
	\label{tab:compras_no_recurrentes}
\end{table}


\noindent \textbf{Nota aclaratoria:}  
el equipo de cómputo utilizado corresponde a dispositivos personales de los integrantes del proyecto, por lo cual no se ha considerado un costo adicional en esta categoría. La compra de modelos 3D o recursos gráficos es considerada opcional y dependerá de la necesidad de complementar el material gráfico disponible de manera gratuita o de libre acceso.

\begin{table}[H]
	\centering
	\renewcommand{\arraystretch}{1.6}
	\setlength{\tabcolsep}{10pt}
	\Huge
	\begin{adjustbox}{max width=\textwidth}
		\begin{tabular}{|p{7cm}|c|r|r|}
			\hline
			\textbf{Recurso} & \textbf{Unidades} & \textbf{Costo unitario (MXN \$)} & \textbf{Costo total (MXN \$)} \\ \hline
			Equipo de cómputo profesional (para desarrollo y edición 3D) & 4 & \$25,000.00 & \$100,000.00 \\ \hline
			Dispositivos móviles de prueba (Android gama media) & 2 & \$12,000.00 & \$24,000.00 \\ \hline
			Cámara y sensor de movimiento (para captura LSM y pruebas) & 1 & \$12,000.00 & \$12,000.00 \\ \hline
			Paquete profesional de modelos 3D con licencia comercial & 1 & \$10,000.00 & \$10,000.00 \\ \hline
			Cursos y certificaciones profesionales (PLN, UX, IA, Unity) & 4 & \$3,000.00 & \$12,000.00 \\ \hline
			Equipo de audio y grabación (para interfaz voz/signos) & 1 & \$4,500.00 & \$4,500.00 \\ \hline
			\textbf{Total compras no recurrentes} & & & \textbf{\$162,500.00} \\ \hline
		\end{tabular}
	\end{adjustbox}
	\caption[Costos estimados de compras no recurrentes en un escenario de comercialización]{Costos estimados de compras no recurrentes en un escenario de comercialización, elaboración propia.}
	\label{tab:compras_no_recurrentes_comercial}
\end{table}


\paragraph{\textbf{Sueldos y asesorías.}} 
El presente apartado presenta el costo estimado de los sueldos y asesorías considerados para la creación del prototipo en cuestión. Se adopta un enfoque dual que contempla tanto el escenario académico de ejecución como una proyección orientada a una futura etapa de comercialización del producto.

\begin{table}[H]
	\centering
	\renewcommand{\arraystretch}{1.6}
	\setlength{\tabcolsep}{10pt}
	\Huge
	\begin{adjustbox}{max width=\textwidth}
		\begin{tabular}{|p{5.5cm}|c|r|r|r|r|}
			\hline
			\textbf{Equipo} & \textbf{Cantidad} & \multicolumn{2}{c|}{\textbf{Desarrollo}} & \multicolumn{2}{c|}{\textbf{Mantenimiento / Ajustes}} \\ \hline
			\textbf{Tipo} & & \textbf{Mensual (MXN \$)} & \textbf{4 meses (MXN \$)} & \textbf{Mensual (MXN \$)} & \textbf{A un año (MXN \$)} \\ \hline
			Desarrollador Full Stack Junior (estudiante) & 3 & \$72,000.00 (\$24,000.00 c/u) & \$288,000.00 & -- & -- \\ \hline
			Asesoría en animación 3D (freelance) & 1 parcial & \$300.00 (por sesión) & \$1,200.00 (cuatro sesiones) & -- & -- \\ \hline
			Asesoría en LSM (validación de señas) & 1 parcial & \$200.00 (por sesión) & \$800.00 (cuatro sesiones) & -- & -- \\ \hline
			\textbf{Total} & \textbf{5} & \textbf{\$72,500.00} & \textbf{\$290,000.00} & -- & -- \\ \hline
		\end{tabular}
	\end{adjustbox}
	\caption[Costos estimados en el escenario académico de sueldos y asesorías durante la creación del prototipo]{Costos estimados en el escenario académico de sueldos y asesorías durante la creación del prototipo, elaboración propia.}
	\label{tab:sueldos_asesorias}
\end{table}


\noindent \textbf{Nota aclaratoria:}  
el sueldo en este caso, se considera un pago simbólico, sin embargo, el sueldo mensual se consideró como un aproximado para un Desarrollador Full Stack Junior en México con fecha de mayo de 2025 \cite{ref63}.  

\begin{table}[H]
	\centering
	\renewcommand{\arraystretch}{1.6}
	\setlength{\tabcolsep}{10pt}
	\Huge
	\begin{adjustbox}{max width=\textwidth}
		\begin{tabular}{|p{5.8cm}|c|r|r|r|r|}
			\hline
			\textbf{Rol / Servicio} & \textbf{Cantidad} & \multicolumn{2}{c|}{\textbf{Desarrollo}} & \multicolumn{2}{c|}{\textbf{Mantenimiento / Ajustes}} \\ \hline
			\textbf{Tipo} & & \textbf{Mensual (MXN \$)} & \textbf{4 meses (MXN \$)} & \textbf{Mensual (MXN \$)} & \textbf{A un año (MXN \$)} \\ \hline
			Desarrollador backend / frontend & 2 & \$28,000.00 & \$224,000.00 & \$16,000.00 & \$192,000.00 \\ \hline
			Especialista en PLN & 1 & \$30,000.00 & \$120,000.00 & \$15,000.00 & \$180,000.00 \\ \hline
			Diseñador 3D / animador (Unity / Blender) & 1 & \$25,000.00 & \$100,000.00 & \$12,000.00 & \$144,000.00 \\ \hline
			Asesoría profesional en LSM & 1 parcial & \$10,000.00 & \$40,000.00 & \$5,000.00 & \$60,000.00 \\ \hline
			Soporte técnico / DevOps & 1 & \$15,000.00 & \$60,000.00 & \$10,000.00 & \$120,000.00 \\ \hline
			\textbf{Total estimado} & \textbf{6} & \textbf{\$108,000.00} & \textbf{\$544,000.00} & \textbf{\$58,000.00} & \textbf{\$696,000.00} \\ \hline
		\end{tabular}
	\end{adjustbox}
	\caption[Costos estimados de sueldos y asesorías en un entorno comercial durante la creación del prototipo]{Costos estimados de sueldos y asesorías en un entorno comercial durante la creación del prototipo, elaboración propia.}	
	\label{tab:sueldos_comercial}
\end{table}



\noindent \textbf{Nota aclaratoria:}  
La columna de “Mantenimiento / Ajustes” se incluye únicamente como referencia para una posible fase futura de operación continua, en caso de que el prototipo evolucione hacia un producto comercial o requiera soporte extendido. Los sueldos se consideran como un aproximado en México con fecha de mayo de 2025 \cite{ref64, ref65, ref66, ref67, ref68, ref69, ref70}. 


\paragraph{\textbf{Presupuesto mediante actividades.}} 
En el siguiente apartado, se presenta el presupuesto detallado por actividades, donde se especifica el tiempo estimado y el costo asociado a cada tarea del proyecto. Además, los valores por hora se ajustan para una estimación freelance o contratista externo (no asalariado fijo), ideal para la simulación de costos comerciales.  Los sueldos se consideran como un aproximado en México con fecha de mayo de 2025 \cite{ref71, ref72, ref73, ref74, ref75, ref76, ref77}.

Las actividades específicas y sus tiempos estimados se encuentran detallados en los anexos titulados \textbf{\nameref{anexo:actividades_academicas}} y \textbf{\nameref{anexo:actividades_comercial}} .


\paragraph{\textbf{Resumen de costos estimados.}}
El presente apartado presenta el resumen de los costos que incluye ambos escenarios tanto el académico como el de proyección a comercialización en un futuro. El análisis incluye el presupuesto derivado de las actividades de desarrollo, los recursos necesarios para las fases de creación y despliegue, las compras no recurrentes, los sueldos del equipo de trabajo y una reserva contemplada para cubrir posibles riesgos o imprevistos durante la implementación.

\begin{table}[H]
	\centering
	\renewcommand{\arraystretch}{1.5}
	\setlength{\tabcolsep}{12pt}
	\resizebox{\textwidth}{!}{%
		\begin{tabular}{|l|r|}
			\hline
			\textbf{Concepto} & \textbf{Monto estimado (MXN \$)} \\ \hline
			Presupuesto total de actividades & \$71,100.00 \\ \hline
			Presupuesto total de recursos (desarrollo) & \$9,200.00 \\ \hline
			Presupuesto total de recursos (despliegue) & \$27,600.00 \\ \hline
			Compras no recurrentes & \$4,900.00 \\ \hline
			Sueldo del equipo (desarrollo) & \$290,000.00 \\ \hline
			Sueldo del equipo (mantenimiento) & -- \\ \hline
			Reserva para riesgos e imprevistos (15\%) & \$10,665.00 \\ \hline
			\textbf{Presupuesto total estimado del proyecto} & \textbf{\$413,939.00} \\ \hline
		\end{tabular}%
	}
	\caption[Resumen de costos estimados para el desarrollo y despliegue en el escenario académico]{Resumen de costos estimados para el desarrollo y despliegue en el escenario académico, elaboración propia.}	
	\label{tab:costos_academico}
\end{table}

\begin{table}[H]
	\centering
	\renewcommand{\arraystretch}{1.5}
	\setlength{\tabcolsep}{12pt}
	\resizebox{\textwidth}{!}{%
		\begin{tabular}{|l|r|}
			\hline
			\textbf{Concepto} & \textbf{Monto estimado (MXN \$)} \\ \hline
			Presupuesto total de actividades & \$179,280.00 \\ \hline
			Presupuesto total de recursos (desarrollo) & \$123,000.00 \\ \hline
			Presupuesto total de recursos (despliegue) & \$369,000.00 \\ \hline
			Compras no recurrentes & \$162,500.00 \\ \hline
			Sueldo del equipo (desarrollo) & \$544,000.00 \\ \hline
			Sueldo del equipo (mantenimiento) & \$696,000.00 \\ \hline
			Reserva para riesgos e imprevistos (15\%) & \$26,892.00 \\ \hline
			\textbf{Presupuesto total estimado del proyecto} & \textbf{\$2,101,346.00} \\ \hline
		\end{tabular}%
	}
	\caption[Resumen de costos estimados para el desarrollo y despliegue en un entorno comercial del prototipo]{Resumen de costos estimados para el desarrollo y despliegue en un entorno comercial del prototipo, elaboración propia.}	
	\label{tab:costos_comerciall}
\end{table}

\noindent \textbf{Nota aclaratoria:}  
el presupuesto presentado considera un escenario de escalamiento comercial del prototipo, por lo que las cifras reflejan una estimación basada en tarifas de mercado, infraestructura de operación real y recursos humanos contratados de manera formal. Cabe resaltar que, para efectos académicos, los costos efectivos en la fase de prototipo fueron significativamente menores, basados en costos de oportunidad y recursos propios. La reserva para riesgos e imprevistos contempla un porcentaje adicional sobre el subtotal, como medida preventiva ante ajustes, retrasos o necesidades técnicas no previstas.

\paragraph{\textbf{Informe de costo de despliegue del prototipo.}}
El despliegue del prototipo constituye una etapa crucial dentro del proyecto, ya que permite verificar la correcta operación de la aplicación, validar la experiencia de usuario y obtener retroalimentación que contribuya a la mejora continua de la solución. Esta fase busca asegurar que el prototipo sea funcional, accesible y que cumpla con los objetivos de accesibilidad e inclusión establecidos para su prueba en un entorno académico o controlado.

A continuación, se detallan los principales aspectos relacionados con esta etapa:
\paragraph{\textbf{Objetivos del despliegue.}}
Los objetivos principales del despliegue del prototipo son los siguientes:

\begin{itemize}
	\item Verificar la funcionalidad de la aplicación y su compatibilidad en dispositivos con Android 14.
	\item Realizar pruebas piloto para validar la comprensión y fluidez de las animaciones 3D en Lengua de Señas Mexicana (LSM).
	\item Obtener retroalimentación de usuarios potenciales y especialistas para identificar áreas de oportunidad y mejora en el prototipo.
\end{itemize}

\paragraph{\textbf{Cronograma del despliegue.}}
El cronograma de despliegue se fundamenta en la planificación estratégica de las actividades necesarias para la publicación, prueba y evaluación del prototipo. Estas actividades incluyen la ejecución de pruebas internas, la validación con usuarios, la recopilación sistemática de observaciones y la realización de ajustes finales basados en los resultados obtenidos.


Las actividades específicas y sus tiempos estimados se encuentran detallados en el anexo titulado \textbf{\nameref{anexo:actividades_academicas}} y en \textbf{\nameref{anexo:actividades_comercial}}

\paragraph{\textbf{Costos del despliegue.}}
La fase de despliegue representa una etapa crítica del proyecto, ya que permite validar el funcionamiento del prototipo, ajustar detalles técnicos y garantizar que cumpla con los criterios de calidad y usabilidad esperados.

Para ofrecer una visión integral, esta sección presenta dos enfoques diferenciados en la estimación de costos:

\begin{itemize}
	\item \textbf{Enfoque académico:} se consideran costos simbólicos asociados a la ejecución del proyecto en un entorno universitario, utilizando recursos propios y trabajo colaborativo de los estudiantes. Este enfoque refleja la realidad operativa del desarrollo del prototipo en su contexto actual.
	
	\item \textbf{Enfoque exploratorio de comercialización:} se presenta una estimación realista basada en tarifas de mercado, que contempla los costos necesarios para validar, documentar y lanzar el prototipo como un producto funcional en un entorno profesional, incluyendo personal especializado, infraestructura y herramientas de control de calidad.
\end{itemize}

\vspace{1em}
\noindent\textbf{Resumen de costos (despliegue) – Enfoque académico:}
\begin{itemize}
	\item \textbf{Gestión de calidad y pruebas internas:} \$3,600.00 MXN (simbólico).
	\item \textbf{Validación con usuarios (pruebas piloto y retroalimentación):} \$7,500.00 MXN (simbólico).
	\item \textbf{Documentación y ajustes finales del prototipo:} \$1,500.00 MXN (simbólico).
\end{itemize}

\noindent\textbf{Costo total estimado (académico):} \textbf{\$12,600.00 MXN}

\vspace{1em}
\noindent\textbf{Resumen de costos (despliegue) – Enfoque comercial:} (según tarifas profesionales conservadoras)

\begin{itemize}
	\item \textbf{Gestión de calidad (definición de estándares, métricas y control):} \$6,520.00 MXN.
	\item \textbf{Pruebas técnicas e integración (unitarias, con usuarios, corrección de errores):} \$15,000.00 MXN.
	\item \textbf{Preparación de entorno de producción y despliegue técnico:} \$16,000.00 MXN.
\end{itemize}

\noindent\textbf{Costo total estimado (comercial):} \textbf{\$37,520.00 MXN}

\vspace{1em}
Esta doble perspectiva permite comparar la ejecución del despliegue del prototipo en un entorno académico y en un escenario comercial futuro, facilitando la toma de decisiones sobre su escalabilidad y viabilidad económica.

\paragraph{\textbf{Métricas para medir el éxito.}}
Para evaluar la efectividad del despliegue del prototipo, se definirán los siguientes indicadores clave de desempeño (KPIs):

\begin{itemize}
	\item Porcentaje de pruebas funcionales exitosas (fluidez y comprensión de las animaciones 3D): superior al 90\%.
	\item Nivel de satisfacción o retroalimentación positiva de los usuarios evaluadores: superior al 85\%.
\end{itemize}

El cumplimiento de estas métricas permitirá validar que el prototipo alcanza los niveles de funcionalidad, accesibilidad e inclusión planteados como objetivos principales del proyecto.

\paragraph{\textbf{Análisis exploratorio de costo de desarrollo.}}
\paragraph{\textbf{Escenario de posible futura etapa de comercialización.}}
Este apartado presenta una proyección financiera elaborada bajo un enfoque comercial, con el objetivo de estimar los costos y condiciones necesarias para escalar el prototipo de aplicación a un producto funcional en el mercado. Aunque el presente trabajo terminal tiene un carácter académico, esta simulación permite anticipar los requerimientos económicos y técnicos para una eventual implementación comercial.

\paragraph{\textbf{Análisis de costos.}}
A continuación, se presenta el desglose de costos estimados para una fase inicial de comercialización del prototipo, considerando sueldos profesionales, infraestructura tecnológica, servicios especializados, mantenimiento y una reserva para riesgos:

\begin{table}[H]
	\centering
	\renewcommand{\arraystretch}{1.5}
	\setlength{\tabcolsep}{12pt}
	\resizebox{\textwidth}{!}{%
		\begin{tabular}{|l|r|}
			\hline
			\textbf{Concepto} & \textbf{Monto estimado (MXN \$)} \\ \hline
			Presupuesto total de actividades & \$179,280.00 \\ \hline
			Presupuesto total de recursos (desarrollo) & \$123,000.00 \\ \hline
			Presupuesto total de recursos (despliegue) & \$369,000.00 \\ \hline
			Compras no recurrentes & \$162,500.00 \\ \hline
			Sueldo del equipo (desarrollo) & \$544,000.00 \\ \hline
			Sueldo del equipo (mantenimiento) & \$696,000.00 \\ \hline
			Reserva para riesgos e imprevistos (15\%) & \$26,892.00 \\ \hline
			\textbf{Presupuesto total estimado del proyecto} & \textbf{\$2,101,346.00} \\ \hline
		\end{tabular}%
	}
	\caption[Proyección de costos para una posible fase comercial]{Proyección de costos para una posible fase comercial, elaboración propia.}	
	\label{tab:costos_desarrollo}
\end{table}

\noindent \textbf{Nota:} Esta estimación considera valores realistas para una implementación comercial, incluyendo tarifas conservadoras por perfil profesional, recursos tecnológicos y una reserva destinada a cubrir contingencias operativas o técnicas.

\paragraph{\textbf{Determinación del precio de desarrollo.}}

Para calcular un precio base de suscripción mensual, se considera el siguiente escenario:

\begin{itemize}
	\item Número de usuarios esperados: \textbf{200}.
	\item Tiempo promedio de uso por usuario: \textbf{40 horas mensuales}.
	\item Periodo de recuperación de inversión: \textbf{24 meses}.
\end{itemize}

\paragraph{\textbf{Cálculo del precio base mensual.}}

\[
\text{Costo mensual del proyecto} = \frac{\num{2101346.00}}{24} = \num{87556.08}
\]

\[
\text{Costo mensual por usuario} = \frac{\num{87556.08}}{200} = \num{437.78}
\]

\[
\text{Costo por hora de uso} = \frac{\num{437.78}}{40} = \num{10.94}
\]

\paragraph{\textbf{Propuesta de precio final.}}
Considerando un margen de ganancia del \textbf{20\%}, el precio sugerido de suscripción mensual sería:

\[
\text{Precio base mensual} = \num{437.78} \times 1.20 = \num{525.34}
\]

Redondeando al entero más cercano:

\[
\text{Precio base mensual redondeado} = \num{525.00}
\]

\begin{flushleft}
	\textbf{Nota final:} esta simulación financiera tiene carácter exploratorio y busca establecer un marco inicial para la evaluación económica del producto. Las cifras deberán ser validadas y refinadas mediante estudios de mercado y pruebas piloto antes de su implementación real.
\end{flushleft}



\subsubsection{Recursos humanos}
El equipo de desarrollo está conformado por tres estudiantes de la carrera de Ingeniería en Inteligencia Artificial, con experiencia académica en proyectos de Ingeniería de software. Esta preparación garantiza que el equipo cuenta con las competencias necesarias para realizar el análisis, diseño, implementación, prueba y validación del prototipo propuesto.

Adicionalmente, se contempla la posibilidad de realizar capacitaciones específicas en el manejo de conjuntos de datos de señas y técnicas de animación 3D, con el objetivo de reforzar el conocimiento técnico necesario para los módulos especializados del sistema.

La asignación de tres integrantes al proyecto se fundamenta mediante una estimación del esfuerzo utilizando el modelo \textbf{COCOMO II (Constructive Cost Model II)}, específicamente en su modalidad \textit{Early Design Model}, apropiada para proyectos en etapas iniciales [CITA IEEE].

La fórmula del modelo es la siguiente:

\[
\text{Esfuerzo (PM)} = A \cdot (\text{Tamaño})^E \cdot \prod EM_i
\]

Donde:

\begin{itemize}
	\item \textbf{PM} = Persona-meses estimados de esfuerzo.
	\item \textbf{A} = Constante base del modelo. Se usa el valor estándar de $A = 2.94$.
	\item \textbf{Tamaño} = Tamaño estimado del software en miles de líneas de código (KLOC). Se considera un tamaño de $2$ KLOC (2,000 líneas de código).
	\item \textbf{E} = Exponente de escala. Para este proyecto se asume $E = 1.05$, valor apropiado para proyectos con requisitos moderadamente definidos.
	\item $\prod EM_i$ = Producto de los multiplicadores de esfuerzo. Se asume un valor neutral de $1.0$, al tratarse de un entorno académico con condiciones estándar.
\end{itemize}

Sustituyendo los valores en la fórmula:

\[
\text{PM} = 2.94 \cdot (2)^{1.05} \cdot 1.0 \approx 2.94 \cdot 2.07 = \textbf{6.09 persona-meses.}
\]

Considerando un periodo de desarrollo de aproximadamente ocho meses, el esfuerzo mensual requerido se calcula como:

\[
\frac{6.09 \text{ persona-meses}}{8 \text{ meses}} = \num{0.76125} \approx \num{0.76} \text{ personas por mes.}
\]

Aunque el esfuerzo promedio mensual es inferior a una persona de tiempo completo, se justifica la participación de al menos tres integrantes en el proyecto, debido a las siguientes razones:

\begin{itemize}
	\item Permite distribuir responsabilidades de forma especializada entre los tres módulos principales del sistema: procesamiento de lenguaje natural, generación de animaciones 3D y desarrollo móvil.
	\item Favorece el trabajo en paralelo, lo que reduce los tiempos de espera entre etapas y facilita el cumplimiento de plazos intermedios.
	\item Mejora la calidad del prototipo al facilitar revisiones cruzadas, pruebas continuas y retroalimentación iterativa.
	\item Minimiza el riesgo operativo frente a posibles contingencias como cargas académicas variables, imprevistos personales o rotación parcial del equipo.
\end{itemize}


\subsubsection{Recursos tecnológicos}
El equipo de trabajo dispone del equipo de cómputo necesario para el desarrollo, prueba y validación de la aplicación móvil. Además, se cuenta con acceso a las plataformas de desarrollo y a las herramientas de software requeridas, tales como ambientes de programación, motores de animación 3D y bibliotecas especializadas de procesamiento de lenguaje natural (PLN). Esta disponibilidad de recursos tecnológicos garantiza las condiciones adecuadas para la implementación efectiva del prototipo.

\subsubsection{Plazo}
El tiempo estimado para el desarrollo del prototipo es de aproximadamente ocho meses, considerando las fases de análisis, diseño, desarrollo, pruebas y presentación final. Dado que el alcance del proyecto se limita a la traducción de un conjunto predefinido de frases específicas, se considera que el plazo establecido es adecuado para cumplir con los objetivos planteados y la solución propuesta. \\


El análisis realizado en esta sección, permite concluir que el proyecto es tanto viable como factible dentro del contexto académico en el cual se desarrolla. El equipo de trabajo cuenta con los conocimientos fundamentales, el acceso a las tecnologías requeridas y los recursos necesarios para la construcción de un prototipo funcional. Aunque se han identificado áreas que demandarán un proceso de capacitación complementaria, estas no representan un obstáculo significativo para el éxito del proyecto, siempre que se mantenga una adecuada gestión de tiempos y actividades conforme a los alcances establecidos. De este modo, se fortalecen las condiciones para lograr un desarrollo efectivo que cumpla con los objetivos planteados y a futuro siente las bases para una posible expansión comercial.

\section{Análisis y diseño de sistemas}
\subsection{Entregables}

\subsubsection{Diagrama de arquitectura}
\begin{center}
	\includegraphics[width=0.97\textwidth]{Images/Cap 3/Arquitectura_Grande.jpg}
	\captionof{figure}[Diagrama de arquitectura]{Diagrama de arquitectura, elaboración propia.}  % Pie de foto manual
\end{center}

 \begin{flushleft} \href{https://miro.com/app/board/uXjVI24dV0c=/?share_link_id=690276332577}{\textbf{Ver el diagrama de arquitetcura con detalle}} \end{flushleft}

\subsubsection{Reglas de negocio}
\begin{itemize}[leftmargin=1.5cm]
    \item \textbf{RN01:} El sistema debe permitir al usuario solo ingresar texto de forma manual o pegarlo desde otra aplicación.
    \item \textbf{RN02:} El campo de entrada debe tener un límite de caracteres (por ejemplo, 30 caracteres).
    \item \textbf{RN03:} El sistema debe validar el texto ingresado para evitar caracteres no permitidos.
    \item \textbf{RN04:} El botón de traducción debe estar habilitado solo cuando el campo de texto no esté vacío.
    \item \textbf{RN05:} El sistema debe mostrar un mensaje de error si el texto ingresado no es válido.
    \item \textbf{RN06:} El sistema debe permitir al usuario editar el texto antes de enviarlo a traducción.
	\item \textbf{RN07:} El sistema siempre debe mostrar alguna animación, ya sea la correspondiente a la traducción o el deletreo.
\end{itemize}
\subsubsection{Requerimientos funcionales}
\begin{itemize}
    \item \textbf{RF01: Entrada de Texto}  
    La aplicación debe permitir al usuario ingresar y editar texto que será traducido a LSM.
    
    \item \textbf{RF02: Procesamiento de Lenguaje Natural (PLN)}  
    Implementar un módulo que valide, analice y procese el texto ingresado, identificando frases clave y contextos específicos para su traducción.
    
    \item \textbf{RF03: Generación de Avatares 3D}  
    Desarrollar avatares 3D que representen visualmente las señas correspondientes a las frases procesadas. Los avatares deben ser capaces de mostrar expresiones faciales y movimientos corporales que reflejen la gramática y sintaxis de la LSM. 
    
    \item \textbf{RF04: Modo de Deletreo}  
    El sistema debe intentar asignarle una traducción a la frase ingresada. En caso de que a una frase no se le pueda asignar una traducción, la aplicación debe deletrear palabra por palabra utilizando el alfabeto dactilológico de la LSM.
    
    
\end{itemize}


\subsubsection{Requerimientos no funcionales}
\begin{itemize}
    \item \textbf{RnF01: Rendimiento}  
    La aplicación debe cargar y procesar las traducciones en menos de 2 segundos, garantizando una experiencia de usuario rápida y eficiente.
    
    \item \textbf{RnF02: Usabilidad}  
    La interfaz debe ser fácil de navegar, con instrucciones claras y botones de acción bien definidos, siguiendo las heurísticas de usabilidad de Nielsen.

	 \item \textbf{RnF03: Animaciones y Transiciones}  
    Las animaciones entre señas deben ser fluidas y naturales, utilizando IA para optimizar las transiciones y evitar movimientos bruscos o poco realistas.
    
    \item \textbf{RnF04: Interfaz de Usuario (UI)}  
    Diseñar una interfaz intuitiva y accesible que permita a los usuarios ingresar texto personalizado para su traducción.

    \item \textbf{RnF05: Actualización del contenido lingüístico}
	El sistema debe permitir la incorporación y actualización de nuevas versiones del LSM, incluyendo modificaciones en señas existentes, adición de nuevas señas o adaptaciones conforme evolucionen los estándares lingüísticos oficiales.
	\item \textbf{RnF06: Compatibilidad}  
    La aplicación debe ser compatible con la última versión de Android, garantizando su funcionamiento en una amplia gama de dispositivos móviles.
\end{itemize}

\subsubsection{Diagrama de procesos}
\begin{center}
    \makebox[\textwidth]{%
        \includegraphics[width=1.2\textwidth]{Images/Cap 3/Procesos.jpeg}
    }
    \captionof{figure}[Diagrama de procesos del sistema]{Diagrama de procesos del sistema, elaboración propia.}
\end{center}

\subsubsection{Diagrama de actividades} 
\begin{center}
	\makebox[\textwidth]{%
		\includegraphics[width=1\textwidth]{Images/Cap 3/actividades.png}
	}
    \captionof{figure}{Diagrama de actividades del sistema}
\end{center}


\subsubsection{Diagrama de clases}
\begin{center}
	\makebox[\textwidth]{%
		\includegraphics[width=1\textwidth]{Images/Cap 3/tapecito.png}
	}
	\captionof{figure}{Diagrama de clases del sistema}
\end{center}

\subsubsection{Diagrama de secuencia}
\begin{center}
	\makebox[\textwidth]{%
		\includegraphics[width=1\textwidth]{Images/Cap 3/Secuencia.png}
	}
    \captionof{figure}{Diagrama de secuencia del sistema}
\end{center}

\subsubsection{Casos de uso}
\subsubsection{Caso de uso 01: Traducir texto}
\subsubsection{Resumen}
El usuario ingresa texto en la aplicación, que luego es procesado y traducido a LSM para visualizar una animación correspondiente a la traducción.
\subsubsection{Descripción}

\noindent
\begin{tabularx}{\textwidth}{|l|X|}
\hline
\textbf{Caso de uso} & Traducir texto \\ \hline

\textbf{Actor principal} & Usuario (persona que ingresa el texto a traducir). \\ \hline

\textbf{Descripción} & El usuario ingresa texto en el campo indicado de la aplicación. El sistema procesa la frase y determina si existe una traducción directa en LSM. En caso afirmativo, se reproduce la animación correspondiente con un avatar 3D. Si no se encuentra una coincidencia, el sistema recurre al deletreo dactilológico para representar el texto ingresado. \\ \hline

\textbf{Atributos del texto} & Máximo 50 caracteres. Validación para evitar caracteres especiales o símbolos no admitidos. \\ \hline

\textbf{Entrada} & Texto proporcionado por el usuario. \\ \hline

\textbf{Procesamiento} & El módulo de procesamiento de lenguaje natural (PLN) analiza el texto y busca coincidencias en el conjunto de datos. Si no hay coincidencia, se segmenta y transforma en deletreo letra por letra. \\ \hline

\textbf{Salida} & Se muestra la animación en 3D de la traducción en LSM o, en su defecto, del deletreo dactilológico correspondiente. Se incluye un mensaje visual de inicio con la palabra "Vamos". \\ \hline

\textbf{Precondiciones} & El sistema debe estar encendido, funcional y el usuario debe encontrarse en la pantalla principal con acceso al campo de texto. \\ \hline

\textbf{Postcondiciones} & El usuario visualiza la animación correspondiente y tiene la opción de ingresar una nueva frase para traducir. \\ \hline

\textbf{Excepciones} & Si el texto excede el límite o contiene caracteres no válidos, el sistema mostrará un mensaje de error con base en la RN05 e invitará a corregir la entrada. \\ \hline
\end{tabularx}
\captionof{table}[Caso de uso 1]{Caso de uso 1, elaboración propia.}


    

\subsubsection{Trayectoria principal}
\begin{enumerate}[label=\textbf{\arabic*}, leftmargin=1.5cm]
    \item \UCsystem \ El sistema muestra un campo de entrada de texto en la pantalla principal de la aplicación, permitiendo al usuario ingresar texto.  
    Además, la interfaz incluye:  
    \begin{itemize}
        \item Botón para iniciar la traducción.
    \end{itemize}

    \item \UCactor \ Ingresa el texto en el campo de entrada.  
   
    \item \UCsystem \ El sistema procesa el texto ingresado y le asigna una traducción a LSM, mostrandole la animación asignada si la hubo o el deletreo dactilológico si no se le pudo asignar una.

    \item \UCactor \ Puede ingresar texto nuevo y volver a interar la traducción.

\end{enumerate}

\subsubsection{Trayectoria alternativa A}
\begin{enumerate}[label=\textbf{\arabic*}, leftmargin=1.5cm]
    \item \UCactor \ Ingresa el texto con caracteres no permitidos o excede el límite de caracteres.
	\item \UCsystem \ El sistema muestra un mensaje de error indicando que el texto no es válido con base en la RN05.  
	\item \UCactor \ El usuario corrige el texto y vuelve a intentar la traducción.
\end{enumerate}

\subsubsection{Trayectoria alternativa B}
\begin{enumerate}[label=\textbf{\arabic*}, leftmargin=1.5cm]
    \item \UCactor \ Intenta traducir el texto sin haberlo ingresado.
	\item \UCsystem \ El sistema muestra un mensaje de error indicando que el campo de texto está vacío con base en la RN04.
	\item \UCactor \ El usuario ingresa el texto y vuelve a intentar la traducción.
\end{enumerate}

\textit{--- Fin del caso de uso.}


\subsubsection{Diagrama de casos de uso}
\begin{center}
    \includegraphics[width=0.6\textwidth]{Images/Cap 3/casodeuso.png}
    \captionof{figure}{Diagrama de caso de uso 1}
\end{center}


\subsubsection{Identificación de frases}

\begin{table}[H]
\centering
\begin{tabularx}{\textwidth}{|X|X|X|}
\hline
\textbf{Emergencia (3 a 5 palabras)} & \textbf{Saludo (2 a 4 palabras)} & \textbf{Agradecimiento/ Respuestas (2 a 4 palabras)} \\ \hline
Ayuda, por favor. & Hola, ¿cómo estás? & Gracias. \\ \hline
Llama a la policía. & Buenos días. & Muchas gracias. \\ \hline
Necesito un médico. & Buenas tardes. & Te lo agradezco. \\ \hline
Estoy herido/a. & Buenas noches. & Qué amable. \\ \hline
¿Dónde está el hospital? & ¿Qué tal? & Estoy muy agradecido/a. \\ \hline
Es una emergencia. & Mucho gusto. & Gracias por tu ayuda. \\ \hline
¿Puedes ayudarme? & Encantado/a de conocerte. & Cuando quieras. \\ \hline
Necesito asistencia urgente. & ¿Cómo te llamas? & No hay problema. \\ \hline
¿Dónde está la salida? & Me llamo [nombre]. & No pasa nada. \\ \hline
¡Fuego! ¡Fuego! & Hasta luego. & Es un placer. \\ \hline
\end{tabularx}
\caption[Identificación de frases]{Identificación de frases, elaboración propia.}
\end{table}
Las frases seleccionadas son ejemplos representativos de situaciones cotidianas y de emergencia. Se busca que el usuario pueda comunicarse de manera efectiva en contextos importantes, facilitando la comprensión y la interacción con personas que utilizan la Lengua de Señas Mexicana (LSM).

\subsection{Mockups del sistema}
\begin{center}
    \includegraphics[width=0.5\textwidth]{Images/Cap 3/Pantalla1.png}
    \captionof{figure}{Mockup de la pantalla de carga de la aplicación}
\end{center}

La primera pantalla que se muestra es la pantalla de carga de la aplicación, la cual muestra el mensaje " Espere por favor", junto con el ícono de un reloj para indicar que la app está cargando.

\begin{center}
    \includegraphics[width=0.5\textwidth]{Images/Cap 3/Pantalla2.png}
    \captionof{figure}{Mockup de la pantalla de ingreso de texto}
\end{center}
 
La seguda pantalla de la aplicación, la pantalla principal, es la pantalla para ingresar el texto. Se muestra el nombre de la app y  un cuadro para ingresar el texto y un botón para acceder a la traducción.\\

Se debe ingresar texto dentro del cuadro de entrada de texto, para que se realice todo el Procesamiento de Lenguaje Natural y mostrar la animación.

\begin{center}
    \includegraphics[width=0.5\textwidth]{Images/Cap 3/Pantalla3.png}
    \captionof{figure}{Mockup de la pantalla la animación}
\end{center}

La tercera pantalla de la aplicación es la pantalla en donde se muestra la animación 3D asignada a la traducción, para que el usuario sepa en que momento inicia la traducción se le mostrará un mensaje con la palabra "Vamos" . Además se le indica la frase ingresada del usuario y la traducción asignada. Debajo del avatar 3D, se vuelve a mostrar el cuadro de entrada de texto para ingresar un nuevo mensaje.

\begin{center}
    \includegraphics[width=0.5\textwidth]{Images/Cap 3/Pantalla4.png}
    \captionof{figure}{Mockup de la pantalla la animación}
\end{center}

En caso de que no se le pueda asignar una traducción a la frase ingresada, se mostrará la pantalla de deletreo dactilológico, al igual que la Pantalla anterior para que el usuario sepa en que momento inicia la traducción se le mostrará un mensaje con la palabra "Vamos". En esta pantalla se muestra el avatar 3D realizando el deletreo de la frase ingresada por el usuario. Al igual que en la pantalla anterior, se vuelve a mostrar el cuadro de entrada de texto para ingresar un nuevo mensaje.

\begin{center}
	\includegraphics[width=0.5\textwidth]{Images/Cap 3/Paleta.png}
	\captionof{figure}{Paleta de colores de la aplicación}
\end{center}
Se selccionó la siguiente paleta de colores para la aplicación, ya que usa colores representativos de la bandera de LSM, además de que los colores son agradables a la vista y no generan cansancio visual.

\subsection{Identificación y evaluación de riesgos}
\subsubsection{Leyenda de probabilidad y efecto}
\textbf{Probabilidad:}
\begin{itemize}
	\item \textbf{Muy alta ($>$75\%):} Altamente probable que ocurra.
	\item \textbf{Alta (50-75\%):} Probable que ocurra en varias ocasiones durante el proyecto.
	\item \textbf{Moderada (25-50\%):} Existe una posibilidad razonable de que ocurra.
	\item \textbf{Baja (10-25\%):} Poco probable que ocurra, pero no imposible.
\end{itemize}

\textbf{Efecto (Impacto):}
\begin{itemize}
	\item \textbf{Catastrófico:} Afecta gravemente el éxito del proyecto, podría impedir la continuidad del mismo.
	\item \textbf{Serio:} Genera retrasos significativos o pérdida parcial de calidad en el desarrollo o resultados.
	\item \textbf{Tolerable:} Impacto menor, manejable sin afectar los objetivos generales del proyecto.
\end{itemize}


% ===== RIESGOS TÉCNICOS =====
\subsubsection{Riesgos técnicos}
Los riesgos técnicos están relacionados con las limitaciones de la tecnología utilizada, la precisión del modelo de traducción y la adaptación a las particularidades de la Lengua de Señas Mexicana (LSM).

\setlength{\tabcolsep}{4pt}
\renewcommand{\arraystretch}{1.2}

\begin{longtable}{|>{\centering\arraybackslash}p{0.8cm}|>{\raggedright\arraybackslash}p{3.5cm}|>{\raggedright\arraybackslash}p{5.1cm}|>{\raggedright\arraybackslash}p{5.1cm}|}
	\hline
	\textbf{ID} & \textbf{Riesgo específico} & \textbf{Probabilidad} & \textbf{Efecto} \\
	\hline
	T1 & Precisión limitada del modelo de traducción de español a LSM. & Alta (50-75\%) & Serio \\
	\hline
	T2 & Escasa disponibilidad de datasets de calidad para LSM. & Alta (50-75\%) & Serio \\
	\hline
	T3 & Desactualización tecnológica del modelo de IA frente a avances rápidos en el área. & Moderada (25-50\%) & Serio \\
	\hline
	T4 & Dificultad para manejar las variaciones regionales y culturales de LSM. & Media (25-50\%) & Catastrófico \\
	\hline
	T5 & Problemas de compatibilidad y funcionamiento multiplataforma. & Moderado (25-50\%) & Serio \\
	\hline
\caption[Resumen de riesgos técnicos]{Resumen de riesgos técnicos, elaboración propia.} \label{tab:riesgos_tecnicos_resumen} \\

\end{longtable}

% ===== RIESGOS FINANCIEROS Y COMERCIALES =====
\subsubsection{Riesgos financieros y comerciales}
Estos riesgos afectan la viabilidad económica del proyecto y su posible escalamiento hacia una fase comercial.

\setlength{\tabcolsep}{4pt}
\renewcommand{\arraystretch}{1.2}

\begin{longtable}{|>{\centering\arraybackslash}p{0.8cm}|>{\raggedright\arraybackslash}p{3.5cm}|>{\raggedright\arraybackslash}p{5.1cm}|>{\raggedright\arraybackslash}p{5.1cm}|}
	\hline
	\textbf{ID} & \textbf{Riesgo específico} & \textbf{Probabilidad} & \textbf{Efecto} \\
	\hline
	F1 & Subestimación de los costos de producción y operación en una fase comercial. & Alta (50-75\%) & Catastrófico \\
	\hline
	F2 & Falta de modelos de negocio viables para monetizar la solución. & Moderada (25-50\%) & Serio \\
	\hline
	F3 & Dependencia de apoyos gubernamentales o financiamiento social para escalar el proyecto. & Media (25-50\%) & Serio \\
	\hline
	F4 & Competencia con otras soluciones similares con mayor madurez o presencia en el mercado. & Media (25-50\%) & Serio \\
	\hline

\caption[Resumen de riesgos financieros y comerciales]{Resumen de riesgos financieros y comerciales, elaboración propia.} \label{tab:riesgos_financieros_resumen} \\
\end{longtable}

% ===== RIESGOS HUMANOS Y DE GESTIÓN =====
\subsubsection{Riesgos humanos y de gestión}
Estos riesgos se refieren a la disponibilidad, capacitación y coordinación del equipo, así como a la correcta gestión del proyecto.

\setlength{\tabcolsep}{4pt}
\renewcommand{\arraystretch}{1.2}

\begin{longtable}{|>{\centering\arraybackslash}p{0.8cm}|>{\raggedright\arraybackslash}p{3.5cm}|>{\raggedright\arraybackslash}p{5.1cm}|>{\raggedright\arraybackslash}p{5.1cm}|}
	\hline
	\textbf{ID} & \textbf{Riesgo específico} & \textbf{Probabilidad} & \textbf{Efecto} \\
	\hline
	H1 & Falta de experiencia del equipo en procesos de validación lingüística y cultural de LSM. & Alta (50-75\%) & Serio \\
	\hline
	H2 & Dependencia de colaboración externa para la validación de señas. & Media (25-50\%) & Serio \\
	\hline
	H3 & Desmotivación o alta rotación del equipo técnico. & Moderada (25-50\%) & Serio \\
	\hline
	H4 & Falta de claridad o definición adecuada de los requerimientos funcionales. & Alta (50-75\%) & Serio \\
	\hline
\caption[Resumen de riesgos humanos y de gestión]{Resumen de riesgos humanos y de gestión, elaboración propia.} \label{tab:riesgos_humanoss_resumen} \\
\end{longtable}

% ===== RIESGOS ÉTICOS Y REGULATORIOS =====
\subsubsection{Riesgos éticos y regulatorios}
Riesgos asociados a las normativas, certificaciones y cuestiones éticas, especialmente por el tipo de aplicación y su uso potencial en contextos sensibles.

\setlength{\tabcolsep}{4pt}
\renewcommand{\arraystretch}{1.2}

\begin{longtable}{|>{\centering\arraybackslash}p{0.8cm}|>{\raggedright\arraybackslash}p{3.5cm}|>{\raggedright\arraybackslash}p{5.1cm}|>{\raggedright\arraybackslash}p{5.1cm}|}
	\hline
	\textbf{ID} & \textbf{Riesgo específico} & \textbf{Probabilidad} & \textbf{Efecto} \\
	\hline
	E1 & Interpretación incorrecta de mensajes sensibles (contexto médico, legal, etc.). & Baja (10-25\%) & Catastrófico \\
	\hline
	E2 & Ausencia de certificaciones o validaciones oficiales del modelo de traducción. & Moderada (25-50\%) & Serio \\
	\hline
	E3 & Uso no autorizado de bases de datos o glosarios protegidos. & Baja (10-25\%) & Catastrófico \\
	\hline
	E4 & Incumplimiento con normativas de accesibilidad digital o protección de datos. & Moderada (25-50\%) & Catastrófico \\
	\hline
\caption[Resumen de riesgos éticos y regulatorios]{Resumen de riesgos éticos y regulatorios, elaboración propia.} \label{tab:riesgos_eticos_resumen} \\
\end{longtable}

\newpage
\subsection{Planes de prevención y contingencia para riegos}

Esta sección describe las acciones preventivas y los planes de contingencia asociados a los riesgos identificados para el proyecto. La información se organiza por categoría de riesgo, facilitando la identificación y trazabilidad de las medidas frente a cada riesgo.

\subsubsection{Planes de prevención y contingencia para riesgos técnicos}

\setlength{\tabcolsep}{4pt}
\renewcommand{\arraystretch}{1.2}

\begin{longtable}{|>{\centering\arraybackslash}p{0.8cm}|>{\raggedright\arraybackslash}p{3.5cm}|>{\raggedright\arraybackslash}p{5.1cm}|>{\raggedright\arraybackslash}p{5.1cm}|}
	\hline
	\textbf{ID} & \textbf{Riesgo específico} & \textbf{Prevención} & \textbf{Plan de contingencia} \\
	\hline
	T1 & Precisión limitada del modelo de traducción de español a LSM. &
	\begin{itemize}
		\item Validar el modelo con expertos en LSM.
		\item Realizar pruebas piloto con retroalimentación continua.
		\item Incorporar técnicas de mejora continua en el entrenamiento.
	\end{itemize} &
	\begin{itemize}
		\item Ajustar el modelo mediante reentrenamiento con nuevos datos.
		\item Aplicar correcciones manuales temporales mientras se mejora la precisión.
		\item Evaluar modelos alternativos si el desempeño es insatisfactorio.
	\end{itemize} \\
	\hline
	T2 & Escasa disponibilidad de datasets de calidad para LSM. &
	\begin{itemize}
		\item Buscar colaboración con instituciones o expertos en LSM.
		\item Utilizar técnicas de data augmentation para ampliar los datos existentes.
	\end{itemize} &
	\begin{itemize}
		\item Integrar validaciones humanas adicionales para compensar la falta de datos.
		\item Ajustar el alcance del proyecto si los datos son insuficientes.
	\end{itemize} \\
	\hline
	T3 & Desactualización tecnológica del modelo de IA. &
	\begin{itemize}
		\item Mantener vigilancia tecnológica constante.
		\item Participar en foros y comunidades sobre IA y accesibilidad.
	\end{itemize} &
	\begin{itemize}
		\item Migrar a nuevas versiones o tecnologías según los avances detectados.
		\item Ajustar la arquitectura para facilitar futuras actualizaciones.
	\end{itemize} \\
	\hline
	T4 & Dificultad para manejar las variaciones regionales y culturales de LSM. &
	\begin{itemize}
		\item Buscar especialistas del área para el uso de región específica.
	\end{itemize} &
	\begin{itemize}
		\item Limitar el alcance del prototipo a ciertas regiones mientras se extiende la cobertura.
		\item Informar claramente las limitaciones del modelo a los usuarios.
	\end{itemize} \\
	\hline
	T5 & Problemas de compatibilidad y funcionamiento multiplataforma. &
	\begin{itemize}
		\item Desarrollar bajo estándares multiplataforma.
		\item Realizar pruebas en distintos dispositivos y navegadores.
	\end{itemize} &
	\begin{itemize}
		\item Aplicar correcciones específicas por plataforma detectada.
		\item Priorizar las plataformas con mayor demanda de usuarios.
	\end{itemize} \\
	\hline
	\caption[Planes de prevención y contingencia para riesgos técnicos]{Planes de prevención y contingencia para riesgos técnicos, elaboración propia.} \label{tab:riesgos_tecnicos}\\
\end{longtable}


\newpage
% ========================
% Puedes continuar con las siguientes subsecciones:

\subsubsection{Planes de prevención y contingencia para riesgos financieros y comerciales}

\setlength{\tabcolsep}{4pt}
\renewcommand{\arraystretch}{1.2}

\begin{longtable}{|>{\centering\arraybackslash}p{0.8cm}|>{\raggedright\arraybackslash}p{3.5cm}|>{\raggedright\arraybackslash}p{5.1cm}|>{\raggedright\arraybackslash}p{5.1cm}|}
	\hline
	\textbf{ID} & \textbf{Riesgo específico} & \textbf{Prevención} & \textbf{Plan de contingencia} \\
	\hline
	F1 & Subestimación de los costos de producción y operación en una fase comercial. &
	\begin{itemize}
		\item Elaborar un análisis financiero detallado con escenarios conservadores.
		\item Incluir márgenes de contingencia en la planificación de costos.
	\end{itemize} &
	\begin{itemize}
		\item Reevaluar los costos y ajustar el modelo de negocio.
		\item Buscar financiamiento adicional o ajustar el alcance del proyecto.
	\end{itemize} \\
	\hline
	F2 & Falta de modelos de negocio viables para monetizar la solución. &
	\begin{itemize}
		\item Diseñar y evaluar diferentes modelos de negocio desde la fase temprana.
		\item Consultar expertos en comercialización y accesibilidad.
	\end{itemize} &
	\begin{itemize}
		\item Ajustar la estrategia hacia modelos freemium, licencias o apoyos institucionales.
		\item Explorar alianzas con organizaciones del sector social o educativo.
	\end{itemize} \\
	\hline
	F3 & Dependencia de apoyos gubernamentales o financiamiento social para escalar el proyecto. &
	\begin{itemize}
		\item Diversificar las posibles fuentes de financiamiento (fondos privados, crowdfunding).
		\item Preparar la documentación requerida para aplicar a diferentes programas.
	\end{itemize} &
	\begin{itemize}
		\item Redimensionar el proyecto a una escala mínima viable si no se obtienen los fondos esperados.
		\item Buscar inversionistas privados o aliados estratégicos.
	\end{itemize} \\
	\hline
	F4 & Competencia con otras soluciones similares con mayor madurez o presencia en el mercado. &
	\begin{itemize}
		\item Realizar un monitoreo constante del mercado y de las soluciones existentes.
		\item Diferenciar la propuesta de valor en facilidad de uso, precio o calidad.
	\end{itemize} &
	\begin{itemize}
		\item Ajustar el enfoque del producto según las necesidades no cubiertas por la competencia.
		\item Enfocar el desarrollo en nichos específicos o sectores desatendidos.
	\end{itemize} \\
	\hline
	 \caption[Planes de prevención y contingencia para riesgos financieros y comerciales]{Planes de prevención y contingencia para riesgos financieros y comerciales, elaboración propia.} \label{tab:riesgos_financieros}\\
\end{longtable}

\newpage

\subsubsection{Planes de prevención y contingencia para riesgos humanos y de gestión}

\setlength{\tabcolsep}{4pt}
\renewcommand{\arraystretch}{1.2}

\begin{longtable}{|>{\centering\arraybackslash}p{0.8cm}|>{\raggedright\arraybackslash}p{3.5cm}|>{\raggedright\arraybackslash}p{5.1cm}|>{\raggedright\arraybackslash}p{5.1cm}|}
	\hline
	\textbf{ID} & \textbf{Riesgo específico} & \textbf{Prevención} & \textbf{Plan de contingencia} \\
	\hline
	H1 & Falta de experiencia del equipo en validación lingüística y cultural de LSM. &
	\begin{itemize}
		\item Incluir asesoría de expertos en LSM durante el desarrollo.
		\item Capacitar al equipo en aspectos básicos de la cultura y lengua de señas.
	\end{itemize} &
	\begin{itemize}
		\item Contratar o colaborar con intérpretes certificados para cubrir las áreas necesarias.
		\item Ajustar las pruebas de validación incorporando especialistas externos.
	\end{itemize} \\
	\hline
	H2 & Dependencia de colaboración externa para la validación de señas. &
	\begin{itemize}
		\item Establecer acuerdos formales con colaboradores e intérpretes desde el inicio.
		\item Definir tiempos y compromisos claros de participación.
	\end{itemize} &
	\begin{itemize}
		\item Buscar reemplazos o alianzas adicionales si algún colaborador no cumple lo acordado.
		\item Ajustar el cronograma para incluir tiempo adicional si se presentan retrasos.
	\end{itemize} \\
	\hline
	H3 & Desmotivación o alta rotación del equipo técnico. &
	\begin{itemize}
		\item Fomentar un ambiente laboral positivo y flexible.
		\item Ofrecer oportunidades de aprendizaje y desarrollo profesional.
	\end{itemize} &
	\begin{itemize}
		\item Redistribuir funciones y roles de manera temporal.
		\item Contratar personal de refuerzo o apoyo externo si es necesario.
	\end{itemize} \\
	\hline
	H4 & Falta de claridad o definición adecuada de los requerimientos funcionales. &
	\begin{itemize}
		\item Documentar los requisitos de manera detallada y validarlos con los stakeholders.
		\item Realizar sesiones de revisión y ajuste de requisitos de forma periódica.
	\end{itemize} &
	\begin{itemize}
		\item Detener las tareas críticas hasta tener claridad total sobre los requisitos.
		\item Reorganizar el backlog y ajustar las prioridades si es necesario.
	\end{itemize} \\
	\hline
	\caption[Planes de prevención y contingencia para riesgos humanos y de gestión]{Planes de prevención y contingencia para riesgos humanos y de gestión, elaboración propia.} 	\label{tab:riesgos_humanos}\\
\end{longtable}

\subsubsection{Planes de prevención y contingencia para riesgos éticos y regulatorios}

\setlength{\tabcolsep}{4pt}
\renewcommand{\arraystretch}{1.2}

\begin{longtable}{|>{\centering\arraybackslash}p{0.8cm}|>{\raggedright\arraybackslash}p{3.5cm}|>{\raggedright\arraybackslash}p{5.1cm}|>{\raggedright\arraybackslash}p{5.1cm}|}
	\hline
	\textbf{ID} & \textbf{Riesgo específico} & \textbf{Prevención} & \textbf{Plan de contingencia} \\
	\hline
	E1 & Interpretación incorrecta de mensajes sensibles (contexto médico, legal, etc.). &
	\begin{itemize}
		\item Definir claramente los contextos de uso permitidos del sistema.
		\item Incluir advertencias sobre los límites del modelo en el uso de la aplicación.
	\end{itemize} &
	\begin{itemize}
		\item Limitar temporalmente las funcionalidades en contextos de alto riesgo.
		\item Incorporar revisiones humanas en casos sensibles o críticos.
	\end{itemize} \\
	\hline
	E2 & Ausencia de certificaciones o validaciones oficiales del modelo de traducción. &
	\begin{itemize}
		\item Investigar los procesos de certificación existentes y planear la obtención de las acreditaciones.
		\item Involucrar expertos en accesibilidad y traducción en el proceso de validación.
	\end{itemize} &
	\begin{itemize}
		\item Buscar asesoría externa para cumplir los requisitos regulatorios si se identifican brechas.
		\item Ajustar la fase comercial hasta obtener las validaciones requeridas.
	\end{itemize} \\
	\hline
	E3 & Uso no autorizado de conjunto de datos o glosarios protegidos. &
	\begin{itemize}
		\item Verificar licencias de uso de todos los recursos desde la etapa de diseño.
		\item Priorizar el uso de datos open-source o desarrollados internamente.
	\end{itemize} &
	\begin{itemize}
		\item Sustituir inmediatamente los recursos cuestionados por alternativas con licencias válidas.
		\item Consultar asesoría legal para resolver posibles conflictos de derechos de autor.
	\end{itemize} \\
	\hline
	E4 & Incumplimiento con normativas de accesibilidad digital o protección de datos. &
	\begin{itemize}
		\item Asegurar el cumplimiento de la norma ISO 9241-210:2019 y Heurísticas de Jakob Nielsen.
		\item Consultar especialistas en regulación y accesibilidad.
	\end{itemize} &
	\begin{itemize}
		\item Corregir los puntos de incumplimiento detectados antes del lanzamiento comercial.
		\item Documentar y comunicar las acciones correctivas a los usuarios y autoridades si es necesario.
	\end{itemize} \\
	\hline
\caption[Planes de prevención y contingencia para riesgos éticos y regulatorios]{Planes de prevención y contingencia para riesgos éticos y regulatorios, elaboración propia.} 	\label{tab:riesgos_eticos} \\
\end{longtable}

\section{Plan de Pruebas Inicial}
\subsection{Casos de prueba funcionales y de usabilidad}
Para las pruebas iniciales, se contempla que los usuarios objetivo sean personas pertenecientes a la comunidad sorda de México, específicamente personas con discapacidad auditiva que se comunican mediante la Lengua de Señas Mexicana (LSM) e intérpretes de LSM. En particular, las pruebas de funcionalidad y usabilidad se llevarán a cabo con la participación de alumnos e intérpretes de LSM del Centro de Lenguas Extranjeras (CENLEX) del Instituto Politécnico Nacional, así como de personas externas que formen parte de la comunidad sorda.

Las pruebas consideradas son las siguientes:
\begin{itemize}
\item Evaluación de la precisión de los gestos mostrados en las animaciones 3D correspondientes a frases de emergencia.
\item Evaluación de la precisión de los gestos en las animaciones 3D para la traducción dactilológica (alfabeto manual).
\item Valoración de la fluidez en las animaciones 3D.
\item Análisis de la intuitividad de la aplicación y retroalimentación sobre la elección de la paleta de colores.
\item Satisfacción del usuario con la interfaz general (layout, iconografía, navegación).
\item Validación del conjunto de frases (saludos, despedidas y emergencias) incluido.
\end{itemize}


\chapter{Conclusiones}
En la primera etapa desarrollada del Trabajo Terminal, se establecen las bases teóricas y técnicas necesarias para el desarrollo de una solución tecnológica inclusiva orientada a mejorar la comunicación entre personas oyentes y personas con discapacidad auditiva. A partir del análisis de la motivación, la problemática y los objetivos del proyecto, se delimita un enfoque centrado en la accesibilidad lingüística a través de la Lengua de Señas Mexicana (LSM).\\

La revisión del estado del arte y la construcción del marco teórico permiten contextualizar el proyecto dentro del ámbito del procesamiento de lenguaje natural y la representación visual mediante modelado 3D, evidenciando que, si bien existen herramientas similares a nivel internacional, estas no abordan de manera específica las características lingüísticas, culturales y gramaticales de la LSM. Asimismo, el análisis del proceso de comunicación y el estudio profundo de los elementos lingüísticos de esta lengua (como su estructura espacial, dactilología, fonología y gramática) revela los desafíos técnicos de su implementación digital y la necesidad de soluciones adaptadas al contexto mexicano.


% \chapter{Trabajo relacionado}

% \chapter{Desarrollo}

% \chapter{Experimentos}

% \chapter{Conclusiones y trabajo futuro}

% \appendix
\chapter*{Anexos}
\addcontentsline{toc}{chapter}{Anexos}

\chapter{Ley General para la Inclusión de las Personas con Discapacidad}
\label{anexo:ley_inclusion_disc}
\section{Encabezado de la Ley General para la Inclusión de las Personas con Discapacidad}

\begin{center}
	\makebox[\textwidth]{%
		\includegraphics[width=1\textwidth]{Images/Anexos/Encabezado_Ley.png}
	}
    \captionof{figure}[Encabezado de la Ley General para la Inclusión de las Personas con Discapacidad]{Encabezado de la Ley General para la Inclusión de las Personas con Discapacidad, obtenido de \cite{ref34}}
\end{center}

\section{Artículo 2, Fracción XXII}
\begin{center}
	\makebox[\textwidth]{%
		\includegraphics[width=1\textwidth]{Images/Anexos/Art2_FraccXXII.png}
	}
    \captionof{figure}[Artículo 2, Fracción XXII, de la Ley General para la Inclusión de las Personas con Discapacidad]{Artículo 2, Fracción XXII, de la Ley General para la Inclusión de las Personas con Discapacidad, obtenido de \cite{ref34}}
\end{center}

\section{Artículo 20}
\begin{center}
	\makebox[\textwidth]{%
		\includegraphics[width=1\textwidth]{Images/Anexos/Art20.png}
	}
    \captionof{figure}[Artículo 20]{Artículo 20 de la Ley General para la Inclusión de las Personas con Discapacidad, obtenido de \cite{ref34}}
\end{center}

\chapter{Enfoque por actividades (académico)}
\label{anexo:actividades_academicas}  % Etiqueta para hacer referencia
\section{Etapa: creación del prototipo}

\begin{table}[H]
	\centering
	\renewcommand{\arraystretch}{1.6}
	\setlength{\tabcolsep}{10pt}
	\Huge
	\begin{adjustbox}{max width=\textwidth}
		\begin{tabular}{|p{8cm}|c|r|r|}
			\hline
			\textbf{Tareas (Formulación del proyecto)} & \textbf{Horas} & \textbf{Costo por hora (MXN \$)} & \textbf{Costo total (MXN \$)} \\ \hline
			Descripción del proyecto & 1 & \$150.00 & \$150.00 \\ \hline
			Definir el propósito del proyecto & 1 & \$150.00 & \$150.00 \\ \hline
			Planificación del alcance del proyecto & 1 & \$150.00 & \$150.00 \\ \hline
			Definir las actividades necesarias para completar el proyecto & 1 & \$150.00 & \$150.00 \\ \hline
			Definir tareas prioritarias y bloques de trabajo en paralelo & 2 & \$150.00 & \$300.00 \\ \hline
			Estimar recursos y operaciones & 3 & \$150.00 & \$450.00 \\ \hline
			Establecer los objetivos y metas principales & 1 & \$150.00 & \$150.00 \\ \hline
			Identificación de actividades y tareas & 5 & \$150.00 & \$750.00 \\ \hline
			Planificación del cronograma de actividades & 8 & \$150.00 & \$1,200.00 \\ \hline
			\textbf{Total} & \textbf{23} & -- & \textbf{\$3,450.00} \\ \hline
		\end{tabular}
	\end{adjustbox}
	\caption[Costos estimados para la fase de formulación del proyecto]{Costos estimados para la fase de formulación del proyecto, elaboración propia.} 	
	\label{tab:costos_formulacion_nuevo}
\end{table}


\begin{table}[H]
	\centering
	\renewcommand{\arraystretch}{1.6}
	\setlength{\tabcolsep}{10pt}
	\Huge
	\begin{adjustbox}{max width=\textwidth}
		\begin{tabular}{|p{9.5cm}|c|r|r|}
			\hline
			\textbf{Tareas (Análisis del proyecto)} & \textbf{Horas} & \textbf{Costo por hora (MXN \$)} & \textbf{Costo total (MXN \$)} \\ \hline
			Definición de actores & 3 & \$150.00 & \$450.00 \\ \hline
			Análisis funcional y no funcional & 8 & \$150.00 & \$1,200.00 \\ \hline
			Creación de documentación de requerimientos & 5 & \$150.00 & \$750.00 \\ \hline
			Diagrama de casos de uso & 3 & \$150.00 & \$450.00 \\ \hline
			Diseño de pantallas (mockups) & 10 & \$150.00 & \$1,500.00 \\ \hline
			Análisis de viabilidad y factibilidad & 3 & \$150.00 & \$450.00 \\ \hline
			Análisis financiero & 5 & \$150.00 & \$750.00 \\ \hline
			Análisis de riesgos del proyecto & 5 & \$150.00 & \$750.00 \\ \hline
			Documentar los requisitos de alto nivel y entregables del proyecto & 5 & \$150.00 & \$750.00 \\ \hline
			Priorización de módulos según importancia y complejidad & 1 & \$150.00 & \$150.00 \\ \hline
			\textbf{Total} & \textbf{58} & -- & \textbf{\$7,450.00} \\ \hline
		\end{tabular}
	\end{adjustbox}
	\caption[Costos estimados para la fase de análisis del proyecto]{Costos estimados para la fase de análisis del proyecto, elaboración propia.} 	
	\label{tab:costos_analisis_nuevo}
\end{table}

\begin{table}[H]
	\centering
	\renewcommand{\arraystretch}{1.6}
	\setlength{\tabcolsep}{10pt}
	\Huge
	\begin{adjustbox}{max width=\textwidth}
		\begin{tabular}{|p{9.5cm}|c|r|r|}
			\hline
			\textbf{Tareas (Análisis de riesgos)} & \textbf{Horas} & \textbf{Costo por hora (MXN \$)} & \textbf{Costo total (MXN \$)} \\ \hline
			Realizar análisis cualitativo y cuantitativo de riesgos & 4 & \$150.00 & \$600.00 \\ \hline
			Planificar respuestas a los riesgos & 2 & \$150.00 & \$300.00 \\ \hline
			\textbf{Total} & \textbf{6} & -- & \textbf{\$900.00} \\ \hline
		\end{tabular}
	\end{adjustbox}
	\caption[Costos estimados para la fase de análisis de riesgos]{Costos estimados para la fase de análisis de riesgos, elaboración propia.} 	
	\label{tab:costos_riesgos_nuevo}
\end{table}

\begin{table}[H]
	\centering
	\renewcommand{\arraystretch}{1.6}
	\setlength{\tabcolsep}{10pt}
	\Huge
	\begin{adjustbox}{max width=\textwidth}
		\begin{tabular}{|p{9.5cm}|c|r|r|}
			\hline
			\textbf{Tareas (Elaboración de presupuesto)} & \textbf{Horas} & \textbf{Costo por hora (MXN \$)} & \textbf{Costo total (MXN \$)} \\ \hline
			Cotización simbólica de recursos & 5 & \$150.00 & \$750.00 \\ \hline
			Estimación de costos por actividades & 10 & \$150.00 & \$1,500.00 \\ \hline
			Estimación de costos por recursos & 10 & \$150.00 & \$1,500.00 \\ \hline
			\textbf{Total} & \textbf{25} & -- & \textbf{\$3,750.00} \\ \hline
		\end{tabular}
	\end{adjustbox}
	\caption[Costos estimados para la fase de elaboración de presupuesto]{Costos estimados para la fase de elaboración de presupuesto, elaboración propia.} 	
	\label{tab:costos_presupuesto_nuevo}
\end{table}

\begin{table}[H]
	\centering
	\renewcommand{\arraystretch}{1.6}
	\setlength{\tabcolsep}{10pt}
	\Huge
	\begin{adjustbox}{max width=\textwidth}
		\begin{tabular}{|p{9.5cm}|c|r|r|}
			\hline
			\textbf{Tareas (Desarrollo del producto)} & \textbf{Horas} & \textbf{Costo por hora (MXN \$)} & \textbf{Costo total (MXN \$)} \\ \hline
			Definición de la arquitectura básica & 15 & \$150.00 & \$2,250.00 \\ \hline
			Diseño de interfaces de usuario (UI/UX) para cada módulo & 20 & \$150.00 & \$3,000.00 \\ \hline
			Obtención del conjunto de datos & 12 & \$150.00 & \$1,800.00 \\ \hline
			Diagramas de flujo y secuencia & 12 & \$150.00 & \$1,800.00 \\ \hline
			Diagramas correspondientes UML & 20 & \$150.00 & \$3,000.00 \\ \hline
			Integración de APIs externas & 20 & \$150.00 & \$3,000.00 \\ \hline
			Integración backend y frontend & 25 & \$150.00 & \$3,750.00 \\ \hline
			\textbf{Total} & \textbf{124} & -- & \textbf{\$18,600.00} \\ \hline
		\end{tabular}
	\end{adjustbox}
	\caption[Costos estimados para la fase de desarrollo del producto]{Costos estimados para la fase de desarrollo del producto, elaboración propia.} 	
	\label{tab:costos_desarrollo_nuevo}
\end{table}


\section{Etapa: despliegue del prototipo}
\begin{table}[H]
	\centering
	\renewcommand{\arraystretch}{1.6}
	\setlength{\tabcolsep}{10pt}
	\Huge
	\begin{adjustbox}{max width=\textwidth}
		\begin{tabular}{|p{9.5cm}|c|r|r|}
			\hline
			\textbf{Tareas (Gestión de calidad)} & \textbf{Horas} & \textbf{Costo por hora (MXN \$)} & \textbf{Costo total (MXN \$)} \\ \hline
			Definir los estándares de calidad aplicables al proyecto & 8 & \$150.00 & \$1,200.00 \\ \hline
			Identificar métricas de calidad & 6 & \$150.00 & \$900.00 \\ \hline
			Realizar procedimientos de control de calidad & 10 & \$150.00 & \$1,500.00 \\ \hline
			\textbf{Total} & \textbf{24} & -- & \textbf{\$3,600.00} \\ \hline
		\end{tabular}
	\end{adjustbox}
	\caption[Costos estimados para la fase de gestión de calidad]{Costos estimados para la fase de gestión de calidad, elaboración propia.} 	
	\label{tab:costos_calidad_nuevo}
\end{table}

\begin{table}[H]
	\centering
	\renewcommand{\arraystretch}{1.6}
	\setlength{\tabcolsep}{10pt}
	\Huge
	\begin{adjustbox}{max width=\textwidth}
		\begin{tabular}{|p{9.5cm}|c|r|r|}
			\hline
			\textbf{Tareas (Gestión de clientes)} & \textbf{Horas} & \textbf{Costo por hora (MXN \$)} & \textbf{Costo total (MXN \$)} \\ \hline
			Identificar y analizar las partes interesadas de la comunidad & 5 & \$150.00 & \$750.00 \\ \hline
			Desarrollar y mantener la comunicación con la comunidad & 8 & \$150.00 & \$1,200.00 \\ \hline
			Identificar a todos los interesados & 5 & \$150.00 & \$750.00 \\ \hline
			Resolver conflictos con clientes & 10 & \$150.00 & \$1,500.00 \\ \hline
			\textbf{Total} & \textbf{28} & -- & \textbf{\$4,200.00} \\ \hline
		\end{tabular}
	\end{adjustbox}
	\caption[Costos estimados para la fase de gestión de clientes]{Costos estimados para la fase de gestión de clientes, elaboración propia.} 	
	\label{tab:costos_clientes_nuevo}
\end{table}

\begin{table}[H]
	\centering
	\renewcommand{\arraystretch}{1.6}
	\setlength{\tabcolsep}{10pt}
	\Huge
	\begin{adjustbox}{max width=\textwidth}
		\begin{tabular}{|p{9.5cm}|c|r|r|}
			\hline
			\textbf{Tareas (Gestión de adquisiciones)} & \textbf{Horas} & \textbf{Costo por hora (MXN \$)} & \textbf{Costo total (MXN \$)} \\ \hline
			Planificar futuras compras y adquisiciones & 8 & \$150.00 & \$1,200.00 \\ \hline
			\textbf{Total} & \textbf{8} & -- & \textbf{\$1,200.00} \\ \hline
		\end{tabular}
	\end{adjustbox}
	\caption[Costos estimados para la fase de gestión de adquisiciones]{Costos estimados para la fase de gestión de adquisiciones, elaboración propia.} 	
	\label{tab:costos_adquisiciones_nuevo}
\end{table}

\begin{table}[H]
	\centering
	\renewcommand{\arraystretch}{1.6}
	\setlength{\tabcolsep}{10pt}
	\Huge
	\begin{adjustbox}{max width=\textwidth}
		\begin{tabular}{|p{9.5cm}|c|r|r|}
			\hline
			\textbf{Tareas (Gestión de integración)} & \textbf{Horas} & \textbf{Costo por hora (MXN \$)} & \textbf{Costo total (MXN \$)} \\ \hline
			Desarrollar el plan de gestión del proyecto & 15 & \$150.00 & \$2,250.00 \\ \hline
			Dirigir y gestionar el trabajo del proyecto & 20 & \$150.00 & \$3,000.00 \\ \hline
			Monitorear y controlar el trabajo del proyecto & 15 & \$150.00 & \$2,250.00 \\ \hline
			\textbf{Total} & \textbf{50} & -- & \textbf{\$7,500.00} \\ \hline
		\end{tabular}
	\end{adjustbox}
	\caption[Costos estimados para la fase de gestión de integración]{Costos estimados para la fase de gestión de integración, elaboración propia.} 	
	\label{tab:costos_integracion_nuevo}
\end{table}

\begin{table}[H]
	\centering
	\renewcommand{\arraystretch}{1.6}
	\setlength{\tabcolsep}{10pt}
	\Huge
	\begin{adjustbox}{max width=\textwidth}
		\begin{tabular}{|p{9.5cm}|c|r|r|}
			\hline
			\textbf{Tareas (Pruebas)} & \textbf{Horas} & \textbf{Costo por hora (MXN \$)} & \textbf{Costo total (MXN \$)} \\ \hline
			Costo de las pruebas iniciales solo con desarrolladores & 10 & \$150.00 & \$1,500.00 \\ \hline
			Pruebas unitarias para cada módulo & 20 & \$150.00 & \$3,000.00 \\ \hline
			Pruebas de integración & 10 & \$150.00 & \$1,500.00 \\ \hline
			Pruebas con usuarios para validar la usabilidad & 10 & \$150.00 & \$1,500.00 \\ \hline
			\textbf{Total} & \textbf{50} & -- & \textbf{\$7,500.00} \\ \hline
		\end{tabular}
	\end{adjustbox}
	\caption[Costos estimados para la fase de pruebas]{Costos estimados para la fase de pruebas, elaboración propia.} 	
	\label{tab:costos_pruebas_nuevo}
\end{table}

\begin{table}[H]
	\centering
	\renewcommand{\arraystretch}{1.6}
	\setlength{\tabcolsep}{10pt}
	\Huge
	\begin{adjustbox}{max width=\textwidth}
		\begin{tabular}{|p{9.5cm}|c|r|r|}
			\hline
			\textbf{Tareas (Lanzamiento)} & \textbf{Horas} & \textbf{Costo por hora (MXN \$)} & \textbf{Costo total (MXN \$)} \\ \hline
			Preparación del entorno de producción & 10 & \$150.00 & \$1,500.00 \\ \hline
			\textbf{Total} & \textbf{10} & -- & \textbf{\$1,500.00} \\ \hline
		\end{tabular}
	\end{adjustbox}
	\caption[Costos estimados para la fase de lanzamiento]{Costos estimados para la fase de lanzamiento, elaboración propia.} 	
	\label{tab:costos_lanzamiento_nuevo}
\end{table}


\section{Etapa: costo de venta del prototipo}

\begin{table}[H]
	\centering
	\renewcommand{\arraystretch}{1.6}
	\setlength{\tabcolsep}{10pt}
	\Huge
	\begin{adjustbox}{max width=\textwidth}
		\begin{tabular}{|p{9.5cm}|c|r|r|}
			\hline
			\textbf{Tareas (Manual de usuario)} & \textbf{Horas} & \textbf{Costo por hora (MXN \$)} & \textbf{Costo total (MXN \$)} \\ \hline
			Creación de guías paso a paso para cada módulo & 12 & \$150.00 & \$1,800.00 \\ \hline
			Instrucciones claras y visuales para usuarios no técnicos & 10 & \$150.00 & \$1,500.00 \\ \hline
			\textbf{Total} & \textbf{22} & -- & \textbf{\$3,300.00} \\ \hline
		\end{tabular}
	\end{adjustbox}
	\caption[Costos estimados para la fase de elaboración del manual de usuario]{Costos estimados para la fase de elaboración del manual de usuario, elaboración propia.} 	
	\label{tab:costos_manual_nuevo}
\end{table}

\begin{table}[H]
	\centering
	\renewcommand{\arraystretch}{1.6}
	\setlength{\tabcolsep}{10pt}
	\Huge
	\begin{adjustbox}{max width=\textwidth}
		\begin{tabular}{|p{9.5cm}|c|r|r|}
			\hline
			\textbf{Tareas (Manual técnico)} & \textbf{Horas} & \textbf{Costo por hora (MXN \$)} & \textbf{Costo total (MXN \$)} \\ \hline
			Documentación de la arquitectura del sistema & 8 & \$150.00 & \$1,200.00 \\ \hline
			Descripción del conjunto de datos y APIs & 10 & \$150.00 & \$1,500.00 \\ \hline
			\textbf{Total} & \textbf{18} & -- & \textbf{\$2,700.00} \\ \hline
		\end{tabular}
	\end{adjustbox}
	\caption[Costos estimados para la fase de elaboración del manual técnico]{Costos estimados para la fase de elaboración del manual técnico, elaboración propia.} 	
	\label{tab:costos_manual_tecnico_nuevo}
\end{table}

\begin{table}[H]
	\centering
	\renewcommand{\arraystretch}{1.6}
	\setlength{\tabcolsep}{10pt}
	\Huge
	\begin{adjustbox}{max width=\textwidth}
		\begin{tabular}{|p{9.5cm}|c|r|r|}
			\hline
			\textbf{Tareas (Documentación)} & \textbf{Horas} & \textbf{Costo por hora (MXN \$)} & \textbf{Costo total (MXN \$)} \\ \hline
			Documentación de requerimientos & 15 & \$150.00 & \$2,250.00 \\ \hline
			Documentación de pruebas & 6 & \$150.00 & \$900.00 \\ \hline
			Manuales de usuario y técnico & 8 & \$150.00 & \$1,200.00 \\ \hline
			\textbf{Total} & \textbf{29} & -- & \textbf{\$4,350.00} \\ \hline
		\end{tabular}
	\end{adjustbox}
	\caption[Costos estimados para la fase de documentación]{Costos estimados para la fase de documentación, elaboración propia.} 	
	\label{tab:costos_documentacion_nuevo}
\end{table}

\begin{table}[H]
	\centering
	\renewcommand{\arraystretch}{1.6}
	\setlength{\tabcolsep}{10pt}
	\Huge
	\begin{adjustbox}{max width=\textwidth}
		\begin{tabular}{|p{9.5cm}|c|r|r|}
			\hline
			\textbf{Tareas (Presupuesto de ingresos)} & \textbf{Horas} & \textbf{Costo por hora (MXN \$)} & \textbf{Costo total (MXN \$)} \\ \hline
			Estimación de precio de producto final & 6 & \$150.00 & \$900.00 \\ \hline
			\textbf{Total} & \textbf{6} & -- & \textbf{\$900.00} \\ \hline
		\end{tabular}
	\end{adjustbox}
	\caption[Costos estimados para la fase de presupuesto de ingresos]{Costos estimados para la fase de presupuesto de ingresos, elaboración propia.} 	
	\label{tab:costos_presupuesto_ingresos}
\end{table}

\begin{table}[H]
	\centering
	\renewcommand{\arraystretch}{1.6}
	\setlength{\tabcolsep}{10pt}
	\Huge
	\begin{adjustbox}{max width=\textwidth}
		\begin{tabular}{|p{9.5cm}|c|r|r|}
			\hline
			\textbf{Tareas (Estados financieros)} & \textbf{Horas} & \textbf{Costo por hora (MXN \$)} & \textbf{Costo total (MXN \$)} \\ \hline
			Revisión de costos y gastos iniciales & 1 & \$150.00 & \$150.00 \\ \hline
			Proyección de ingresos & 2 & \$150.00 & \$300.00 \\ \hline
			\textbf{Total} & \textbf{3} & -- & \textbf{\$450.00} \\ \hline
		\end{tabular}
	\end{adjustbox}
	\caption[Costos estimados para la fase de estados financieros]{Costos estimados para la fase de estados financieros, elaboración propia.} 	
	\label{tab:costos_estados_financieros}
\end{table}

\chapter{Enfoque por actividades (comercial)}
\label{anexo:actividades_comercial}  % Etiqueta para hacer referencia
\section{Etapa: creación del prototipo}
\begin{table}[H]
	\centering
	\renewcommand{\arraystretch}{1.6}
	\setlength{\tabcolsep}{10pt}
	\Huge
	\begin{adjustbox}{max width=\textwidth}
		\begin{tabular}{|p{9.5cm}|c|r|r|}
			\hline
			\textbf{Tareas (Formulación del proyecto)} & \textbf{Horas} & \textbf{Costo por hora (MXN \$)} & \textbf{Costo total (MXN \$)} \\ \hline
			Descripción del proyecto & 1 & \$280.00 & \$280.00 \\ \hline
			Definir el propósito del proyecto & 1 & \$280.00 & \$280.00 \\ \hline
			Planificación del alcance del proyecto & 1 & \$280.00 & \$280.00 \\ \hline
			Definir las actividades necesarias para completar el proyecto & 1 & \$280.00 & \$280.00 \\ \hline
			Definir tareas prioritarias y bloques de trabajo en paralelo & 2 & \$280.00 & \$560.00 \\ \hline
			Estimar recursos y operaciones & 3 & \$260.00 & \$780.00 \\ \hline
			Establecer los objetivos y metas principales & 1 & \$280.00 & \$280.00 \\ \hline
			Identificación de actividades y tareas & 5 & \$280.00 & \$1,400.00 \\ \hline
			Planificación del cronograma de actividades & 8 & \$280.00 & \$2,240.00 \\ \hline
			\textbf{Total} & \textbf{23} & -- & \textbf{\$6,380.00} \\ \hline
		\end{tabular}
	\end{adjustbox}
	\caption[Costos estimados para la fase de formulación del proyecto (ajustada con nueva tarifa)]{Costos estimados para la fase de formulación del proyecto (ajustada con nueva tarifa), elaboración propia.} 	
	\label{tab:costos_formulacion_tarifa280}
\end{table}

\begin{table}[H]
	\centering
	\renewcommand{\arraystretch}{1.6}
	\setlength{\tabcolsep}{10pt}
	\Huge
	\begin{adjustbox}{max width=\textwidth}
		\begin{tabular}{|p{9.5cm}|c|r|r|}
			\hline
			\textbf{Tareas (Análisis de proyecto)} & \textbf{Horas} & \textbf{Costo por hora (MXN \$)} & \textbf{Costo total (MXN \$)} \\ \hline
			Definición de actores & 3 & \$260.00 & \$780.00 \\ \hline
			Análisis funcional y no funcional & 8 & \$260.00 & \$2,080.00 \\ \hline
			Creación de documentación de requerimientos & 5 & \$240.00 & \$1,200.00 \\ \hline
			Diagrama de casos de uso & 3 & \$260.00 & \$780.00 \\ \hline
			Diseño de pantallas (mockups) & 10 & \$300.00 & \$3,000.00 \\ \hline
			Análisis de viabilidad y factibilidad & 3 & \$260.00 & \$780.00 \\ \hline
			Análisis financiero & 5 & \$270.00 & \$1,350.00 \\ \hline
			Análisis de riesgos del proyecto & 5 & \$270.00 & \$1,350.00 \\ \hline
			Documentar los requisitos de alto nivel y entregables del proyecto & 5 & \$240.00 & \$1,200.00 \\ \hline
			Priorización de módulos según importancia y complejidad & 1 & \$260.00 & \$260.00 \\ \hline
			\textbf{Total} & \textbf{48} & -- & \textbf{\$12,780.00} \\ \hline
		\end{tabular}
	\end{adjustbox}
	\caption[Costos estimados para la fase de análisis de proyecto (con tarifas ajustadas)]{Costos estimados para la fase de análisis de proyecto (con tarifas ajustadas), elaboración propia.} 	
	\label{tab:costos_analisis_actualizado}
\end{table}

\begin{table}[H]
	\centering
	\renewcommand{\arraystretch}{1.6}
	\setlength{\tabcolsep}{10pt}
	\Huge
	\begin{adjustbox}{max width=\textwidth}
		\begin{tabular}{|p{9.5cm}|c|r|r|}
			\hline
			\textbf{Tareas (Análisis de riesgos)} & \textbf{Horas} & \textbf{Costo por hora (MXN \$)} & \textbf{Costo total (MXN \$)} \\ \hline
			Realizar análisis cualitativo y cuantitativo de riesgos & 4 & \$260.00 & \$1,040.00 \\ \hline
			Planificar respuestas a los riesgos & 2 & \$260.00 & \$520.00 \\ \hline
			Monitoreo de riesgos general & 10 & \$280.00 & \$2,800.00 \\ \hline
			\textbf{Total} & \textbf{16} & -- & \textbf{\$4,360.00} \\ \hline
		\end{tabular}
	\end{adjustbox}
	\caption[Costos estimados para la fase de análisis de riesgos (con tarifas ajustadas)]{Costos estimados para la fase de análisis de riesgos (con tarifas ajustadas), elaboración propia.} 
	\label{tab:costos_riesgos_actualizado}
\end{table}

\begin{table}[H]
	\centering
	\renewcommand{\arraystretch}{1.6}
	\setlength{\tabcolsep}{10pt}
	\Huge
	\begin{adjustbox}{max width=\textwidth}
		\begin{tabular}{|p{9.5cm}|c|r|r|}
			\hline
			\textbf{Tareas (Elaboración de presupuesto)} & \textbf{Horas} & \textbf{Costo por hora (MXN \$)} & \textbf{Costo total (MXN \$)} \\ \hline
			Cotización simbólica de recursos & 5 & \$270.00 & \$1,350.00 \\ \hline
			Estimación de costos por actividades & 10 & \$270.00 & \$2,700.00 \\ \hline
			Estimación de costos por recursos & 10 & \$270.00 & \$2,700.00 \\ \hline
			\textbf{Total} & \textbf{25} & -- & \textbf{\$6,750.00} \\ \hline
		\end{tabular}
	\end{adjustbox}
	\caption[Costos estimados para la fase de elaboración de presupuesto (con tarifas ajustadas)]{Costos estimados para la fase de elaboración de presupuesto (con tarifas ajustadas), elaboración propia.} 
	\label{tab:costos_presupuesto_actualizado}
\end{table}

\begin{table}[H]
	\centering
	\renewcommand{\arraystretch}{1.6}
	\setlength{\tabcolsep}{10pt}
	\Huge
	\begin{adjustbox}{max width=\textwidth}
		\begin{tabular}{|p{9.5cm}|c|r|r|}
			\hline
			\textbf{Tareas (Desarrollo del producto)} & \textbf{Horas} & \textbf{Costo por hora (MXN \$)} & \textbf{Costo total (MXN \$)} \\ \hline
			Definición de la arquitectura básica & 15 & \$320.00 & \$4,800.00 \\ \hline
			Diseño de interfaces de usuario (UI/UX) para cada módulo & 20 & \$300.00 & \$6,000.00 \\ \hline
			Obtención del conjunto de datos & 12 & \$260.00 & \$3,120.00 \\ \hline
			Diagramas de flujo y secuencia para cada funcionalidad & 12 & \$240.00 & \$2,880.00 \\ \hline
			Desarrollo de la funcionalidad de inicio de sesión y validación de credenciales & 20 & \$320.00 & \$6,400.00 \\ \hline
			Implementación del sistema de recuperación de contraseña & 15 & \$320.00 & \$4,800.00 \\ \hline
			Creación de la funcionalidad de registro de nuevos usuarios & 15 & \$320.00 & \$4,800.00 \\ \hline
			Integración de APIs externas & 20 & \$320.00 & \$6,400.00 \\ \hline
			Integración backend y frontend & 25 & \$320.00 & \$8,000.00 \\ \hline
			\textbf{Total} & \textbf{154} & -- & \textbf{\$47,200.00} \\ \hline
		\end{tabular}
	\end{adjustbox}
	\caption[Costos estimados para la fase de desarrollo del producto (con tarifas ajustadas)]{Costos estimados para la fase de desarrollo del producto (con tarifas ajustadas), elaboración propia.} 
	\label{tab:costos_desarrollo_actualizado}
\end{table}


\section{Etapa: despliegue del prototipo}

\begin{table}[H]
	\centering
	\renewcommand{\arraystretch}{1.6}
	\setlength{\tabcolsep}{10pt}
	\Huge
	\begin{adjustbox}{max width=\textwidth}
		\begin{tabular}{|p{9.5cm}|c|r|r|}
			\hline
			\textbf{Tareas (Gestión de calidad)} & \textbf{Horas} & \textbf{Costo por hora (MXN \$)} & \textbf{Costo total (MXN \$)} \\ \hline
			Definir los estándares de calidad aplicables al proyecto & 8 & \$280.00 & \$2,240.00 \\ \hline
			Identificar métricas de calidad & 6 & \$280.00 & \$1,680.00 \\ \hline
			Realizar procedimientos de control de calidad & 10 & \$260.00 & \$2,600.00 \\ \hline
			\textbf{Total} & \textbf{24} & -- & \textbf{\$6,520.00} \\ \hline
		\end{tabular}
	\end{adjustbox}
	\caption[Costos estimados para la fase de gestión de calidad (con tarifas ajustadas)]{Costos estimados para la fase de gestión de calidad (con tarifas ajustadas), elaboración propia.} 
	\label{tab:costos_calidad_actualizado}
\end{table}

\begin{table}[H]
	\centering
	\renewcommand{\arraystretch}{1.6}
	\setlength{\tabcolsep}{10pt}
	\Huge
	\begin{adjustbox}{max width=\textwidth}
		\begin{tabular}{|p{9.5cm}|c|r|r|}
			\hline
			\textbf{Tareas (Gestión de clientes)} & \textbf{Horas} & \textbf{Costo por hora (MXN \$)} & \textbf{Costo total (MXN \$)} \\ \hline
			Identificar y analizar las partes interesadas de la comunidad & 5 & \$260.00 & \$1,300.00 \\ \hline
			Desarrollar y mantener la comunicación con la comunidad & 8 & \$280.00 & \$2,240.00 \\ \hline
			Identificar a todos los interesados & 5 & \$260.00 & \$1,300.00 \\ \hline
			Resolver conflictos con clientes & 10 & \$280.00 & \$2,800.00 \\ \hline
			\textbf{Total} & \textbf{28} & -- & \textbf{\$7,640.00} \\ \hline
		\end{tabular}
	\end{adjustbox}
	\caption[Costos estimados para la fase de gestión de clientes (con tarifas ajustadas)]{Costos estimados para la fase de gestión de clientes (con tarifas ajustadas), elaboración propia.} 
	\label{tab:costos_clientes_actualizado}
\end{table}

\begin{table}[H]
	\centering
	\renewcommand{\arraystretch}{1.6}
	\setlength{\tabcolsep}{10pt}
	\Huge
	\begin{adjustbox}{max width=\textwidth}
		\begin{tabular}{|p{9.5cm}|c|r|r|}
			\hline
			\textbf{Tareas (Gestión de adquisiciones)} & \textbf{Horas} & \textbf{Costo por hora (MXN \$)} & \textbf{Costo total (MXN \$)} \\ \hline
			Planificar futuras compras y adquisiciones & 8 & \$280.00 & \$2,240.00 \\ \hline
			Seleccionar proveedores & 6 & \$280.00 & \$1,680.00 \\ \hline
			Administrar contratos con proveedores & 8 & \$280.00 & \$2,240.00 \\ \hline
			\textbf{Total} & \textbf{22} & -- & \textbf{\$6,160.00} \\ \hline
		\end{tabular}
	\end{adjustbox}
	\caption[Costos estimados para la fase de gestión de adquisiciones (con tarifas ajustadas)]{Costos estimados para la fase de gestión de adquisiciones (con tarifas ajustadas), elaboración propia.} 
	\label{tab:costos_adquisiciones_actualizado}
\end{table}

\begin{table}[H]
	\centering
	\renewcommand{\arraystretch}{1.6}
	\setlength{\tabcolsep}{10pt}
	\Huge
	\begin{adjustbox}{max width=\textwidth}
		\begin{tabular}{|p{9.5cm}|c|r|r|}
			\hline
			\textbf{Tareas (Gestión de regulaciones)} & \textbf{Horas} & \textbf{Costo por hora (MXN \$)} & \textbf{Costo total (MXN \$)} \\ \hline
			Evaluar el impacto ambiental del proyecto & 6 & \$260.00 & \$1,560.00 \\ \hline
			Asegurar el cumplimiento con regulaciones y políticas de privacidad & 10 & \$260.00 & \$2,600.00 \\ \hline
			\textbf{Total} & \textbf{16} & -- & \textbf{\$4,160.00} \\ \hline
		\end{tabular}
	\end{adjustbox}
	\caption[Costos estimados para la fase de gestión de regulaciones (con tarifas ajustadas)]{Costos estimados para la fase de gestión de regulaciones (con tarifas ajustadas), elaboración propia.} 
	\label{tab:costos_regulaciones_actualizado}
\end{table}

\begin{table}[H]
	\centering
	\renewcommand{\arraystretch}{1.6}
	\setlength{\tabcolsep}{10pt}
	\Huge
	\begin{adjustbox}{max width=\textwidth}
		\begin{tabular}{|p{9.5cm}|c|r|r|}
			\hline
			\textbf{Tareas (Gestión de integración)} & \textbf{Horas} & \textbf{Costo por hora (MXN \$)} & \textbf{Costo total (MXN \$)} \\ \hline
			Desarrollar el plan de gestión del proyecto & 15 & \$280.00 & \$4,200.00 \\ \hline
			Dirigir y gestionar el trabajo del proyecto & 20 & \$280.00 & \$5,600.00 \\ \hline
			Monitorear y controlar el trabajo del proyecto & 15 & \$280.00 & \$4,200.00 \\ \hline
			\textbf{Total} & \textbf{50} & -- & \textbf{\$14,000.00} \\ \hline
		\end{tabular}
	\end{adjustbox}
	\caption[Costos estimados para la fase de gestión de integración (con tarifas ajustadas)]{Costos estimados para la fase de gestión de integración (con tarifas ajustadas), elaboración propia.} 
	\label{tab:costos_integracion_actualizado}
\end{table}

\begin{table}[H]
	\centering
	\renewcommand{\arraystretch}{1.6}
	\setlength{\tabcolsep}{10pt}
	\Huge
	\begin{adjustbox}{max width=\textwidth}
		\begin{tabular}{|p{9.5cm}|c|r|r|}
			\hline
			\textbf{Tareas (Pruebas)} & \textbf{Horas} & \textbf{Costo por hora (MXN \$)} & \textbf{Costo total (MXN \$)} \\ \hline
			Costo de las pruebas iniciales solo con desarrolladores & 10 & \$280.00 & \$2,800.00 \\ \hline
			Pruebas unitarias para cada módulo & 20 & \$320.00 & \$6,400.00 \\ \hline
			Pruebas de integración & 10 & \$320.00 & \$3,200.00 \\ \hline
			Pruebas con usuarios para validar la usabilidad & 10 & \$260.00 & \$2,600.00 \\ \hline
			\textbf{Total} & \textbf{50} & -- & \textbf{\$15,000.00} \\ \hline
		\end{tabular}
	\end{adjustbox}
	\caption[Costos estimados para la fase de pruebas (con tarifas ajustadas)]{Costos estimados para la fase de pruebas (con tarifas ajustadas), elaboración propia.} 
	\label{tab:costos_pruebas_actualizado}
\end{table}

\begin{table}[H]
	\centering
	\renewcommand{\arraystretch}{1.6}
	\setlength{\tabcolsep}{10pt}
	\Huge
	\begin{adjustbox}{max width=\textwidth}
		\begin{tabular}{|p{9.5cm}|c|r|r|}
			\hline
			\textbf{Tareas (Lanzamiento)} & \textbf{Horas} & \textbf{Costo por hora (MXN \$)} & \textbf{Costo total (MXN \$)} \\ \hline
			Preparación del entorno de producción & 10 & \$320.00 & \$3,200.00 \\ \hline
			Configuración de servidores y bases de datos & 15 & \$320.00 & \$4,800.00 \\ \hline
			Despliegue del sistema en servidores de producción & 15 & \$320.00 & \$4,800.00 \\ \hline
			Configuración de backups y monitoreo & 10 & \$320.00 & \$3,200.00 \\ \hline
			\textbf{Total} & \textbf{50} & -- & \textbf{\$16,000.00} \\ \hline
		\end{tabular}
	\end{adjustbox}
	\caption[Costos estimados para la fase de lanzamiento (con tarifas ajustadas)]{Costos estimados para la fase de lanzamiento (con tarifas ajustadas), elaboración propia.} 
	\label{tab:costos_lanzamiento_actualizado}
\end{table}

\section{Etapa: costo de venta del prototipo}

\begin{table}[H]
	\centering
	\renewcommand{\arraystretch}{1.6}
	\setlength{\tabcolsep}{10pt}
	\Huge
	\begin{adjustbox}{max width=\textwidth}
		\begin{tabular}{|p{9.5cm}|c|r|r|}
			\hline
			\textbf{Tareas (Manual de usuario)} & \textbf{Horas} & \textbf{Costo por hora (MXN \$)} & \textbf{Costo total (MXN \$)} \\ \hline
			Creación de guías paso a paso para cada módulo & 12 & \$240.00 & \$2,880.00 \\ \hline
			Instrucciones claras y visuales para usuarios no técnicos & 10 & \$300.00 & \$3,000.00 \\ \hline
			\textbf{Total} & \textbf{22} & -- & \textbf{\$5,880.00} \\ \hline
		\end{tabular}
	\end{adjustbox}
	\caption[Costos estimados para la fase de manual de usuario (con tarifas ajustadas)]{Costos estimados para la fase de manual de usuario (con tarifas ajustadas), elaboración propia.} 
	\label{tab:costos_manual_usuario_actualizado}
\end{table}

\begin{table}[H]
	\centering
	\renewcommand{\arraystretch}{1.6}
	\setlength{\tabcolsep}{10pt}
	\Huge
	\begin{adjustbox}{max width=\textwidth}
		\begin{tabular}{|p{9.5cm}|c|r|r|}
			\hline
			\textbf{Tareas (Manual técnico)} & \textbf{Horas} & \textbf{Costo por hora (MXN \$)} & \textbf{Costo total (MXN \$)} \\ \hline
			Documentación de la arquitectura del sistema & 8 & \$240.00 & \$1,920.00 \\ \hline
			Instrucciones sobre la configuración del servidor y despliegue & 12 & \$320.00 & \$3,840.00 \\ \hline
			Descripción del conjunto de datos y APIs & 10 & \$320.00 & \$3,200.00 \\ \hline
			\textbf{Total} & \textbf{30} & -- & \textbf{\$8,960.00} \\ \hline
		\end{tabular}
	\end{adjustbox}
	\caption[Costos estimados para la fase de manual técnico (con tarifas ajustadas)]{Costos estimados para la fase de manual técnico (con tarifas ajustadas), elaboración propia.} 
	\label{tab:costos_manual_tecnico_actualizado}
\end{table}

\begin{table}[H]
	\centering
	\renewcommand{\arraystretch}{1.6}
	\setlength{\tabcolsep}{10pt}
	\Huge
	\begin{adjustbox}{max width=\textwidth}
		\begin{tabular}{|p{9.5cm}|c|r|r|}
			\hline
			\textbf{Tareas (Documentación)} & \textbf{Horas} & \textbf{Costo por hora (MXN \$)} & \textbf{Costo total (MXN \$)} \\ \hline
			Documentación de requerimientos & 15 & \$240.00 & \$3,600.00 \\ \hline
			Documentación de pruebas & 6 & \$240.00 & \$1,440.00 \\ \hline
			Manuales de usuario y técnico & 8 & \$240.00 & \$1,920.00 \\ \hline
			\textbf{Total} & \textbf{29} & -- & \textbf{\$6,960.00} \\ \hline
		\end{tabular}
	\end{adjustbox}
	\caption[Costos estimados para la fase de documentación (con tarifas ajustadas)]{Costos estimados para la fase de documentación (con tarifas ajustadas), elaboración propia.} 
	\label{tab:costos_documentacion_actualizado}
\end{table}


\begin{table}[H]
	\centering
	\renewcommand{\arraystretch}{1.6}
	\setlength{\tabcolsep}{10pt}
	\Huge
	\begin{adjustbox}{max width=\textwidth}
		\begin{tabular}{|p{9.5cm}|c|r|r|}
			\hline
			\textbf{Tareas (Presupuesto de ingresos)} & \textbf{Horas} & \textbf{Costo por hora (MXN \$)} & \textbf{Costo total (MXN \$)} \\ \hline
			Estimación de precio de producto final & 6 & \$270.00 & \$1,620.00 \\ \hline
			\textbf{Total} & \textbf{6} & -- & \textbf{\$1,620.00} \\ \hline
		\end{tabular}
	\end{adjustbox}
	\caption[Costos estimados para la fase de presupuesto de ingresos (con tarifa ajustada)]{Costos estimados para la fase de presupuesto de ingresos (con tarifa ajustada), elaboración propia.} 
	\label{tab:costos_presupuesto_ingresos_s}
\end{table}

\begin{table}[H]
	\centering
	\renewcommand{\arraystretch}{1.6}
	\setlength{\tabcolsep}{10pt}
	\Huge
	\begin{adjustbox}{max width=\textwidth}
		\begin{tabular}{|p{9.5cm}|c|r|r|}
			\hline
			\textbf{Tareas (Estados financieros proforma)} & \textbf{Horas} & \textbf{Costo por hora (MXN \$)} & \textbf{Costo total (MXN \$)} \\ \hline
			Revisión de costos y gastos iniciales & 1 & \$270.00 & \$270.00 \\ \hline
			Proyección de ingresos & 2 & \$270.00 & \$540.00 \\ \hline
			Elaboración de balance proforma & 8 & \$270.00 & \$2,160.00 \\ \hline
			Preparación de estado de resultados & 3 & \$270.00 & \$810.00 \\ \hline
			\textbf{Total} & \textbf{14} & -- & \textbf{\$3,780.00} \\ \hline
		\end{tabular}
	\end{adjustbox}
	\caption[Costos estimados para la fase de estados financieros proforma (con tarifas ajustadas)]{Costos estimados para la fase de estados financieros proforma (con tarifas ajustadas), elaboración propia.} 
	\label{tab:costos_financieros_proforma}
\end{table}

\begin{table}[H]
	\centering
	\renewcommand{\arraystretch}{1.6}
	\setlength{\tabcolsep}{10pt}
	\Huge
	\begin{adjustbox}{max width=\textwidth}
		\begin{tabular}{|p{9.5cm}|c|r|r|}
			\hline
			\textbf{Tareas (Flujos netos de efectivo)} & \textbf{Horas} & \textbf{Costo por hora (MXN \$)} & \textbf{Costo total (MXN \$)} \\ \hline
			Identificación de entradas y salidas de efectivo & 4 & \$270.00 & \$1,080.00 \\ \hline
			Proyección de flujo de efectivo mensual y anual & 2 & \$270.00 & \$540.00 \\ \hline
			Análisis de punto de equilibrio & 6 & \$270.00 & \$1,620.00 \\ \hline
			\textbf{Total} & \textbf{12} & -- & \textbf{\$3,240.00} \\ \hline
		\end{tabular}
	\end{adjustbox}
	\caption[Costos estimados para la fase de flujos netos de efectivo (con tarifas ajustadas)]{Costos estimados para la fase de flujos netos de efectivo (con tarifas ajustadas), elaboración propia.} 
	\label{tab:costos_flujos_efectivo}
\end{table}

\begin{table}[H]
	\centering
	\renewcommand{\arraystretch}{1.6}
	\setlength{\tabcolsep}{10pt}
	\Huge
	\begin{adjustbox}{max width=\textwidth}
		\begin{tabular}{|p{9.5cm}|c|r|r|}
			\hline
			\textbf{Tareas (Evaluación financiera)} & \textbf{Horas} & \textbf{Costo por hora (MXN \$)} & \textbf{Costo total (MXN \$)} \\ \hline
			Análisis de retorno de inversión & 4 & \$270.00 & \$1,080.00 \\ \hline
			Sensibilidad de las proyecciones & 3 & \$270.00 & \$810.00 \\ \hline
			\textbf{Total} & \textbf{7} & -- & \textbf{\$1,890.00} \\ \hline
		\end{tabular}
	\end{adjustbox}
	\caption[Costos estimados para la fase de evaluación financiera (con tarifas ajustadas)]{Costos estimados para la fase de evaluación financiera (con tarifas ajustadas), elaboración propia.} 
	\label{tab:costos_evaluacion_financiera}
\end{table}

\begin{table}[H]
	\centering
	\renewcommand{\arraystretch}{1.6}
	\setlength{\tabcolsep}{10pt}
	\Huge
	\begin{adjustbox}{max width=\textwidth}
		\begin{tabular}{|p{9.5cm}|c|r|r|}
			\hline
			\textbf{Tareas (Mantenimiento)} & \textbf{Horas} & \textbf{Costo por hora (MXN \$)} & \textbf{Costo total (MXN \$)} \\ \hline
			Monitoreo continuo del sistema & 15 & \$320.00 & \$4,800.00 \\ \hline
			Corrección de errores post-despliegue & 10 & \$320.00 & \$3,200.00 \\ \hline
			Costo de mantenimiento de servidores y seguridad & 12 & \$320.00 & \$3,840.00 \\ \hline
			Cotización de salarios del equipo de mantenimiento de la app & 1 & \$270.00 & \$270.00 \\ \hline
			Costos de actualizaciones de la aplicación & 10 & \$320.00 & \$3,200.00 \\ \hline
			\textbf{Total} & \textbf{48} & -- & \textbf{\$15,310.00} \\ \hline
		\end{tabular}
	\end{adjustbox}
	\caption[Costos estimados para la fase de mantenimiento (con tarifas ajustadas)]{Costos estimados para la fase de mantenimiento (con tarifas ajustadas), elaboración propia.} 
	\label{tab:costos_mantenimiento}
\end{table}


\chapter{Implementación de la Aplicación}
\section{Justificación de los cambios}
\label{implementación:justificación_cambios}

Durante la fase de planeación se estableció un conjunto de 32 frases, distribuidas en tres categorías principales: saludos, expresiones de emergencia y agradecimientos, con el propósito de representar situaciones comunes de comunicación básica. Sin embargo, se identificó que las frases correspondientes a la categoría de agradecimientos como “gracias”, “muchas gracias” o “te lo agradezco” compartían un mismo gesto en LSM. Esto generaba redundancia semántica y gestual, ya que diferentes frases se traducían a una única seña, sin aportar valor adicional al conjunto de datos ni al propósito comunicativo del prototipo.\\

Se optó por sustituir la categoría de agradecimientos por una nueva categoría denominada expresiones de mínima comunicación, la cual incluye frases breves y funcionales utilizadas en interacciones cotidianas. Esta modificación permitió ampliar la cobertura comunicativa del prototipo, garantizando una mayor variedad de gestos en escenarios reales de comunicación entre personas oyentes y personas con discapacidad auditiva.\\

Por otro lado, en el capítulo 1 y 2 se indicó que el desarrollo del prototipo incluiría el modelado 3D de avatares mediante MediaPipe, con procesamiento en Blender o Unity. Sin embargo, durante el desarrollo del Trabajo Terminal se identificaron diversas limitaciones técnicas y operativas que impidieron la implementación de esta fase.\\

MediaPipe permite capturar un esqueleto en forma de nube de puntos a partir de un video, generando un archivo JSON con las coordenadas de las articulaciones. Este archivo puede convertirse a formato BVH para su uso en \textit{software} de animación como Blender. No obstante, se observó que MediaPipe no logra capturar la totalidad de las articulaciones del cuerpo humano, lo que ocasiona que el archivo BVH resultante esté incompleto al momento de su importación en Blender.\\

Esta situación requería realizar un procesamiento manual de cada animación, lo que representaba una curva de aprendizaje considerable, dado que Blender es una herramienta compleja que demanda tiempo y experiencia para crear animaciones detalladas, especialmente aquellas que involucran movimientos de los dedos. Dado que el proyecto contemplaba 61 animaciones distintas, el tiempo y los recursos necesarios para completarlas excedían los límites establecidos para este Trabajo Terminal.\\

También se consideró la posibilidad de utilizar captura de movimiento (\textit{motion capture}) mediante un traje de captura de movimiento, con el fin de mejorar la detección de dedos y gestos faciales. Sin embargo, esta alternativa implicaba una inversión económica elevada y la colaboración de expertos en animación, lo cual resultaba inviable dentro del alcance y recursos disponibles del proyecto.\\

Asimismo, se evaluaron plataformas en línea capaces de realizar captura de movimiento a partir de videos en formato MP4, empleando técnicas de visión artificial y redes neuronales profundas para generar modelos 3D exportables a Unity o Blender. Si bien estas herramientas ofrecían resultados aceptables con gestos simples, presentaban deficiencias significativas en la detección de movimientos complejos de los dedos o expresiones faciales elaboradas, lo que requería un proceso manual de corrección que habría incrementado sustancialmente el tiempo de desarrollo.\\

Por estas razones, se decidió prescindir del modelado 3D en esta etapa del proyecto y concentrar los esfuerzos en el desarrollo del prototipo funcional centrado en el procesamiento del lenguaje natural y la traducción textual a representaciones visuales más simples. No obstante, se mantiene la documentación relativa al modelado 3D y MediaPipe, ya que constituye una base conceptual y técnica valiosa para trabajos futuros que busquen ampliar el presente desarrollo.\\

\section{Introducción y Herramientas Tecnológicas}

\subsection{Resumen del Proyecto}
La interfaz de SignAI se desarrolla en React Native y Expo para la traducción de texto a \textbf{Lengua de Señas Mexicana (LSM)}. La aplicación gestiona una secuencia compleja de reproducción que combina videos alojados en \textbf{AWS S3} \cite{refaws1} con señales de control internas para demarcar el flujo (Inicio, Fin, Espacio).

\subsection{Herramientas Utilizadas y Entorno de Desarrollo}

\begin{table}[H]
\centering
\renewcommand{\arraystretch}{1.6}
\begin{tabular}{|p{3.5cm}|p{5.5cm}|p{4.5cm}|}
\hline
\textbf{Componente} & \textbf{Tecnología/Herramienta} & \textbf{Uso Clave en el Proyecto} \\ \hline

\textbf{\textit{Frontend Core}} &
React Native, Expo &
Entorno de desarrollo, CLI y construcción de la app. \\ \hline

\textbf{Lenguaje} &
TypeScript &
Tipado estático para robustez en la manipulación de estados y API. \\ \hline

\textbf{Reproducción} &
\texttt{expo-video} &
Control de reproducción de video \\ \hline

\textbf{Iconografía} &
\texttt{MaterialIcons} &
Iconos para controles y utilidades (Pausa, Reinicio, Limpiar campo, Pista). \\ \hline

\textbf{Almacenamiento} &
\textbf{Amazon S3} &
\textit{Plataforma de Hosting} distribuido para el contenido de video de LSM, garantizando escalabilidad. \\ \hline

\textbf{IDE} &
Visual Studio Code (VS Code) &
Entorno de desarrollo principal. \\ \hline

\end{tabular}
\caption{\textit{Stack} Tecnológico y Herramientas de Desarrollo, elaboración propia}
\label{tab:herramientas}
\end{table}

\begin{figure}[H]
    \centering
    \begin{minipage}[t]{0.45\linewidth}
        \centering
        \includegraphics[width=\linewidth]{Images/Cap4/images/vscode.png}
        \caption{Entorno de Desarrollo (VS Code), obtenido de \cite{refapp1}.}
        \label{fig:vs_code}
    \end{minipage}%
    \hfill
    \begin{minipage}[t]{0.45\linewidth}
        \centering
        \includegraphics[width=\linewidth]{Images/Cap4/images/expogo.png}
        \caption{Herramienta de \textit{Testing} (Expo Go), obtenido de \cite{refexpo1}.}
        \label{fig:expo_go}
    \end{minipage}
\end{figure}

\section{Lógica del Flujo y Validación}

\subsection{Validación de Caracteres y campo vacío}
La aplicación impone una \textbf{validación estricta} en la entrada de texto para asegurar que solo se procese contenido traducible a LSM, esto con base en la regla de negocio RN02 (un máximo de 50 caracteres permitido por entrada) y RN03 (ver \textbf{\autoref{tab:reglas_negocio}}).

\begin{itemize}
    \item Los caracteres no permitidos se definen en la constante \texttt{CARACTERES\_NO\_PERMITIDOS}, incluyendo números y la mayoría de los símbolos de puntuación exceptuando los signos de interrogación ya que dentro del listado de frases hay preguntas.
    \item La función \texttt{validarTexto} verifica la entrada. El único símbolo permitido fuera de las letras y espacios es el \textbf{símbolo de concatenación "\texttt{+}"}.
    \item Si se detecta un carácter inválido, se interrumpe la ejecución de \texttt{traducir} y se activa el componente \texttt{errorValidationContainer} para informar al usuario el carácter específico y su nombre, pidiendole que ingrese en una nueva entrada cumpliendo con la regla de negocio RN05 Y RN06 (ver \textbf{\autoref{tab:reglas_negocio}}).
    \item El número máximo de caracteres soportado por entrada de usuario es de 50, cumpliendo con la regla de negocio RN02 (ver \textbf{\autoref{tab:reglas_negocio}}).
\end{itemize}

\newpage

\subsection{Mecanismo de Procesamiento de la API y Deletreo}

La traducción se basa en la comunicación con un módulo de Procesamiento de Lenguaje Natural (PLN) externo.

\begin{enumerate}
    \item \textbf{Entrada y Asignación de Frase:} La entrada de texto del usuario es enviada a la API. El módulo de PLN compara la entrada con una base de datos de frases conocidas.
    \item \textbf{Videos en AWS S3:} Si el módulo PLN encuentra una coincidencia, devuelve directamente el \texttt{url\_video} de la frase completa, el cual está alojado en \textbf{Amazon S3} \cite{refaws1}.
    \item \textbf{Modo Deletreo:}
    \begin{itemize}
        \item Si no se detecta una similitud suficiente (o si la lógica de PLN lo determina), el modo deletreo se activa (\texttt{deletreo\_activado: true}).
        \item La API devuelve una secuencia de URLs para cada carácter de la frase, incluyendo un \textit{string} vacío (\texttt{“”}) para representar los espacios.
        \item La función \texttt{procesarMultiplesFrases} transforma estos \textit{strings} vacíos en la señal interna \textbf{\texttt{SIGNAL\_MARKERS.espacio}} para que el \textit{frontend} pueda gestionarla como una pausa visual.
    \end{itemize}
\end{enumerate}

\subsection{Mecanismo de Concatenación y \texttt{SIGNAL\_ESPACIO}}
La función \texttt{procesarMultiplesFrases} implementa la lógica de concatenación a través del símbolo \texttt{+}.

\begin{enumerate}[label={}]
    \item \textbf{Paso 1-División:} El texto completo del usuario se divide usando \texttt{+} para obtener un \textit{array} de frases individuales.
    \item \textbf{Paso 2-Llamadas Secuenciales:} Se realiza una llamada \texttt{fetch} a la API por cada frase, acumulando los URLs de video en \texttt{secuenciaFinal}.
    \item \textbf{Paso 3-Separador de Frases:} Después de cada frase procesada (excepto la última), se inserta \textbf{\texttt{SIGNAL\_MARKERS.espacio}}. Esta señal garantiza una pausa de $600 \text{ ms}$ (ver \textbf{\autoref{tabla:logica_avance_secuencia}}) entre la traducción de la primera frase y la siguiente, mejorando la legibilidad visual de la secuencia.
\end{enumerate}

\section{Control de Reproducción y Estabilidad}

\subsection{Decisiones de Librería y Estabilidad}

\begin{enumerate}
    \item \textbf{Uso de \texttt{expo-video} vs. \texttt{expo-av}:} Se eligió \texttt{expo-video} por su API moderna y el uso del \texttt{useVideoPlayer} hook, que proporciona un objeto \texttt{player} con métodos de control explícito (\texttt{player.replace}, \texttt{player.playing}) ideal para la lógica de secuencias.
    \item \textbf{Eliminación de \texttt{react-native-reanimated}:} La librería fue deshabilitada del proyecto al causar el error crítico \texttt{Worklets Mismatch} en Android (0.6.1 vs 0.5.1). Este error indica una incompatibilidad de versiones binarias en el cliente \texttt{Expo Go} la cual fue la herramienta utilizada para el \textit{testing} de la aplicación. Para estabilizar la aplicación, se neutralizó el código problemático (\texttt{explore.tsx}) que lo importaba, este archivo se generó al levantar el proyecto con mpm, permitiendo la funcionalidad principal de traducción.
\end{enumerate}

\subsection{Lógica de Reproducción Secuencial (Basada en Duración)}

Para el manejo de la secuencia de videos en lugar de usar un sistema inestable basado en eventos se optó por un sistema basado en \textbf{temporizadores y duración del video}.

\begin{table}[H]
\centering
\renewcommand{\arraystretch}{1.6}
\begin{tabular}{|p{4cm}|p{5.5cm}|p{4.5cm}|}
\hline
\textbf{Tipo de Elemento} & \textbf{Mecanismo de Avance} & \textbf{Duración} \\ \hline

\texttt{SIGNAL\_INICIO}, \texttt{SIGNAL\_FIN} & \texttt{setTimeout} (fijo) & $100 \text{ ms}$ \\ \hline

\texttt{SIGNAL\_ESPACIO} & \texttt{setTimeout} (fijo) & $800 \text{ ms}$ \\ \hline

Videos de LSM (URL) & \texttt{setTimeout} basado en \texttt{player.duration} & $D_{\text{video}} + 200 \text{ ms}$ \\ \hline

\end{tabular}
\caption[Lógica de Avance de Secuencia]{Lógica de Avance de Secuencia, elaboración propia.}
\label{tabla:logica_avance_secuencia}
\end{table}

\begin{enumerate}[label=\textbf{L.\arabic*}]
    \item \textbf{Avance Unificado (\texttt{avanzarIndice}):} La función \texttt{avanzarIndice} contiene la lógica de estado de alto nivel (\texttt{setIndiceLetraActual}, \texttt{setVideoActual}) para pasar al siguiente URL en \texttt{secuenciaCompleta}.
    \item \textbf{Programación de Videos:} Un \texttt{useEffect} se activa cuando el video comienza a reproducirse (\texttt{if (player.playing)}). Obtiene la \textbf{duración real} del video (\texttt{player.duration}) y programa el salto al siguiente elemento después de ese tiempo más un búfer de $200 \text{ ms}$ para asegurar que el video termine.
    \item \textbf{Control de Flujo:} La función utiliza \texttt{setEnPausa(true)} antes de cualquier cambio de índice y \texttt{setEnPausa(false)} después, para evitar que los \texttt{useEffect} se activen de forma recursiva o no deseada durante la transición.
\end{enumerate}

\newpage
\subsection{Controles de Utilidad (Overlay Buttons)}
La interfaz incluye botones de utilidad que mejoran la experiencia de usuario:

\begin{figure}[H]
    \centering
    \includegraphics[width=0.4\linewidth]{Images/Cap4/images/botones.png}
    \caption{Botones de Pista (\textit{Lightbulb}) y Reinicio de App (\textit{Home}), elaboración propia.}
    \label{fig:utility_buttons}
\end{figure}


\begin{itemize}
    \item \textbf{Botón de Pista (\textit{emoji-objects}):} Llama a \texttt{mostrarAyuda}, que despliega un \texttt{Alert} con instrucciones sobre la validación de texto, el uso del símbolo \texttt{+} y la lógica de las señales de flujo.
    \item \textbf{Botón de para reiniciar App (\textit{stop-circle}):} Llama a \texttt{reiniciarApp}. Esta función de \texttt{useCallback} restablece \textbf{todos los estados} de la aplicación (texto, respuesta, secuencia, índices), regresando la interfaz a su estado inicial de bienvenida.
\end{itemize}

\begin{figure}[H]
    \centering
    \includegraphics[width=0.4\linewidth]{Images/Cap4/images/infodes.jpeg}
    \caption{Captura del funcionamiento del botón de información sobre la app, elaboración propia.}
    \label{fig:utility_buttons_2}
\end{figure}

\subsection{Componente \texttt{VideoControls}}
Este componente proporciona el control granular de la secuencia de video:

\begin{enumerate}[label=\textbf{C.\arabic*}]
    \item \textbf{Pausa / Reanudar (Toggle):} El botón alterna el estado \texttt{pausadoPorUsuario}.
    \begin{itemize}
        \item Si se presiona, \texttt{pausadoPorUsuario} se vuelve \texttt{true}, forzando al \texttt{player} a detenerse y previniendo que el \texttt{useEffect} avance la secuencia.
        \item Si se vuelve a presionar, \texttt{pausadoPorUsuario} se vuelve \texttt{false}, permitiendo que el \texttt{useEffect} reanude la reproducción y el avance.
    \end{itemize}
    \item \textbf{Reiniciar Secuencia:} Llama a \texttt{reiniciarReproduccion}.
    \begin{itemize}
        \item Esta función restablece el estado de reproducción: \texttt{pausadoPorUsuario} a \texttt{false}, \texttt{indiceLetraActual} a 0, y \texttt{videoActual} a \texttt{secuenciaCompleta[0]} (\texttt{SIGNAL\_INICIO}). Esto permite al usuario repetir la secuencia traducida de forma instantánea.
    \end{itemize}
\end{enumerate}

\section{Pruebas de Funcionamiento y Casos de Uso de la Aplicación}

\subsection{Estrategia de Pruebas}
Se debe realizar una prueba funcional de caja negra enfocada en los flujos críticos de la aplicación (validación, concatenación y control de reproducción).

\subsection{Casos de Prueba Críticos}

\begin{table}[H]
\centering
\renewcommand{\arraystretch}{1.6}
\begin{tabular}{|p{1.7cm}|p{5cm}|p{5cm}|}
\hline
\textbf{ID} & \textbf{Acción del Usuario} & \textbf{Resultado Esperado} \\ \hline

\textbf{CP-1} & Ingresar ``hola'' & Secuencia [INICIO, Vídeo ``hola'', FIN]. \\ \hline

\textbf{CP-2} & Ingresar ``mi nombre es+Ivan'' & Secuencia [INICIO, Video ``me llamo'', ESPACIO, ``I``, ``V``, ``A``, ``N``, FIN]. \\ \hline

\textbf{CP-3} & Ingresar ``test!'' & Muestra mensaje de error. \\ \hline

\textbf{CP-4} & Ingresar ``texto'' y Pausar & El video se detiene y el índice no cambia. \\ \hline

\textbf{CP-5} & Secuencia en curso y Reiniciar & Inicia inmediatamente la reproducción desde \texttt{SIGNAL\_INICIO}. \\ \hline

\textbf{CP-6} & Errores ortográficos & Asignar la frase correcta. \\ \hline

\textbf{CP-7} & Campo vacío & Mensaje de error. \\ \hline

\end{tabular}
\caption[Casos de Prueba para la Lógica de SingAI]{Casos de Prueba para la Lógica de SingAI, elaboración propia.}
\end{table}

\newpage
\textbf{CP-1}
\begin{figure}[H]
    \centering
    \includegraphics[width=0.65\linewidth]{Images/Cap4/images/hola.png}
    \caption{Captura del funcionamiento para el CP-1 (elaboración propia). Es posible observar que la salida deseada "hola" se cumplió}
    \label{fig:CP_1}
\end{figure}

\newpage
\textbf{CP-2}
\begin{figure}[H]
    \centering
    \includegraphics[width=0.65\linewidth]{Images/Cap4/images/ivan.png}
    \caption{Captura del funcionamiento para el CP-2 (elaboración propia). En esta prueba se aprecia que se concatenó de forma correcta y se creó el arreglo de los videos.}
    \label{fig:CP_2}
\end{figure}

\newpage
\textbf{CP-3}
\begin{figure}[H]
    \centering
    \includegraphics[width=0.65\linewidth]{Images/Cap4/images/test!.png}
    \caption{Captura del funcionamiento para el CP-3 (elaboración propia). Se puede observar que se le indico al usuario que está ingresando un caracter no permitido.}
    \label{fig:CP_3}
\end{figure}

\newpage
\textbf{CP-4}
\begin{figure}[H]
    \centering
    \includegraphics[width=0.65\linewidth]{Images/Cap4/images/texto.png}
    \caption{Captura del funcionamiento para el CP-4 (elaboración propia). Se puede apreciar que se detuvo la secuencia en el índice 3 ya que se refleja en el cambio de icono del botón de pausa.}
    \label{fig:CP_4}
\end{figure}

\newpage
\textbf{CP-5}
\begin{figure}[H]
    \centering
    \includegraphics[width=0.65\linewidth]{Images/Cap4/images/reinicio.png}
    \caption{Captura del funcionamiento para el CP-5 (elaboración propia). Partiendo de la frase anterior se oprimió el botón de reinicio lo que llevó a reiniciar la secuencia de videos.}
    \label{fig:CP_5}
\end{figure}

\newpage
\textbf{CP-6}
\begin{figure}[H]
    \centering
    \includegraphics[width=0.65\linewidth]{Images/Cap4/images/errores.png}
    \caption{Captura del funcionamiento para el CP-6 (elaboración propia). En está prueba se puede observar tanto el manejo de los espacios como el módulo de PLN siendo exitoso en ambas consultas con errores ortográficos.}
    \label{fig:CP_6}
\end{figure}

\newpage
\textbf{CP-7}
\begin{figure}[H]
    \centering
    \includegraphics[width=0.65\linewidth]{Images/Cap4/images/vacio.jpeg}
    \caption{Captura del funcionamiento para el CP-7 (elaboración propia), donde manda mensaje de error si el usuario ingresa un campo vacío.}
    \label{fig:CP_7}
\end{figure}

\newpage
La implementación de la aplicación \textbf{SignAI} ha demostrado ser exitosa, cumpliendo con todos los objetivos y reglas de negocio planteadas. La arquitectura diseñada, basada en la separación de responsabilidades entre el Frontend de \textbf{React Native/Expo}, el Backend de \textbf{FastAPI (PLN)} y el almacenamiento en \textbf{Amazon S3} garantiza un sistema modular, escalable y mantenible.

\subsection{Cumplimiento de los requerimientos funcionales}
\begin{itemize}
    \item \textbf{Editar texto}: el usuario en todo momento puede editar el texto ingresado, cumpliendo con el requerimiento funcional RF01 (ver \textbf{\autoref{tab:requerimientos_funcionales}}).
    \item \textbf{Traducción Semántica Robusta}: se validó el funcionamiento del módulo de PLN (pruebas \ref{fig:CP_1} y \ref{fig:CP_6}), demostrando la capacidad del sistema para interpretar intenciones, manejar errores ortográficos y asignar la frase de LSM más adecuada, manteniendo la integridad de la traducción, lo que cumple con el requerimiento funcional RF02 (ver \textbf{\autoref{tab:requerimientos_funcionales}}).
    \item \textbf{Representaciones visuales de las señas correspondientes}: se muestra que los videos de las señas correspondientes expresan movimientos coorporales y expersiones faciales(pruebas \ref{fig:CP_1}, \ref{fig:CP_2}, \ref{fig:CP_4}, \ref{fig:CP_5}), esto cumple con el requerimiento funcional RF03 (ver \textbf{\autoref{tab:requerimientos_funcionales}}).
    \item \textbf{Modo de deletreo}: en las pruebas \ref{fig:CP_2} y \ref{fig:CP_4} podemos observar como se activa y se asigna el modo de deletreo para cada letra del texto ingresada ya que el sistema no encontró una frase con similitud.
\end{itemize}


\subsection{Adherencia a las Reglas de Negocio}
La aplicación respeta estrictamente las reglas de negocio críticas para la experiencia del usuario y la seguridad del sistema (ver \textbf{\autoref{tab:reglas_negocio}}):

\begin{itemize}
    \item \textbf{RN01 (Entrada manual)}: la aplicación solo permite al usuario ingresar el texto de forma manual.
    \item \textbf{RN02 (Límite de caracteres)}: se le asignó un límite de caracteres.
    \item \textbf{RN03 (Restricción de Entrada)}: la validación de entrada \texttt{validarTexto} asegura que solo caracteres traducibles (letras y espacio) entren al sistema.
    \item \textbf{RN04 (Activación de botón "traducir")}: si el usuario ingresa un campo vacío la aplicación manda un mensaje
    \item \textbf{RN05 y RN06 (Manejo de Caracteres Inválidos)}: si se detecta un carácter no permitido, el sistema detiene el proceso de traducción y emite un mensaje de error claro al usuario, indicando el carácter específico que debe ser removido (prueba \ref{fig:CP_3}).
    \item \textbf{RN07(Siempre mostrar animación)}: el sistema garantiza siempre darle una respuesta al usuario aunque no encuentre similitud de alguna frase.
\end{itemize}

\textbf{SignAI} es un producto funcional y estable. La documentación técnica y la evidencia de las pruebas demuestran que el sistema está listo para el despliegue final, brindando una solución eficaz y confiable para la traducción de texto a Lengua de Señas Mexicana.

\subsection{Video de funcionamiento}
\url{https://drive.google.com/file/d/1lu_suRxC8Mf-s6DCtPHaD5JmAzrJgjEW/view?usp=sharing}

\subsection{Código fuente de la app}

\begin{lstlisting}   
import React, { useState, useCallback, useEffect } from "react";
import { View, Text, TextInput, Button, StyleSheet, ActivityIndicator, Alert, TouchableOpacity } from "react-native";
import { VideoView, useVideoPlayer } from "expo-video";
import { MaterialIcons } from "@expo/vector-icons"; 

interface RespuestaAPI {
  query: string;
  grupo: string | null;
  frase_similar: string;
  similitud: number;
  deletreo_activado: boolean;
  deletreo: string[] | null;
  total_caracteres: number | null;
  url_video: string;
  spell_urls: string[] | null;
}

const API_URL = "http://192.168.0.159:8000/buscar";


const SIGNAL_MARKERS = {
  inicio: "SIGNAL_INICIO",
  fin: "SIGNAL_FIN",
  espacio: "SIGNAL_ESPACIO" 
};


const CARACTERES_NO_PERMITIDOS = [
  '.', ',', ';', ':', '!', '-', '_', '@', '#', '$', '%', '&', "'", '"',
  '/', '\\', '(', ')', '[', ']', '{', '}', '=', '*', '0', '1', '2', '3', '4', '5', '6', '7', '8', '9'
];

const NOMBRES_CARACTERES: { [key: string]: string } = {
  '.': 'punto',
  ',': 'coma',
  ';': 'punto y coma',
  ':': 'dos puntos',
  '!': 'exclamación',
  '?': 'interrogación',
  '-': 'guión',
  '_': 'guión bajo',
  '@': 'arroba',
  '#': 'numeral',
  '$': 'dólar',
  '%': 'porcentaje',
  '&': 'ampersand',
  '/': 'barra',
  '\\': 'barra invertida',
  '(': 'paréntesis abierto',
  ')': 'paréntesis cerrado',
  '[': 'corchete abierto',
  ']': 'corchete cerrado',
  '{': 'llave abierta',
  '}': 'llave cerrada',
  '+': 'más',
  '=': 'igual',
  '*': 'asterisco',
  '"': 'comillas',
  "'": 'comilla simple'
};
export default function Index() {
  const [texto, setTexto] = useState("");
  const [respuesta, setRespuesta] = useState<RespuestaAPI | null>(null);
  const [cargando, setCargando] = useState(false);
  const [error, setError] = useState<string | null>(null);
  const [errorValidacion, setErrorValidacion] = useState<{ caracter: string, nombre: string } | null>(null);
  
  
  const [indiceLetraActual, setIndiceLetraActual] = useState(0);
  const [videoActual, setVideoActual] = useState<string | null>(null);
  const [secuenciaCompleta, setSecuenciaCompleta] = useState<string[]>([]);
  const [enPausa, setEnPausa] = useState(false);
  const [pausadoPorUsuario, setPausadoPorUsuario] = useState(false);
  const [deletreoInfo, setDeletreoInfo] = useState<string[]>([]);
  
  
  const mostrarAyuda = () => {
  Alert.alert(
    "¿CÓMO FUNCIONA SignAI?",
    `
ESTRUCTURA DE LA TRADUCCIÓN:

1. ENTRADA DE TEXTO:
   - Solo se permiten LETRAS y ESPACIOS.
   - Usa el símbolo '+' para CONCATENAR varias frases en una sola secuencia (Ej: mi nombre es+Juan).

2. CARACTERES NO PERMITIDOS:
   - Los números y la mayoría de los SÍMBOLOS (.,;:-_@#$...) serán RECHAZADOS.

3. FLUJO DE REPRODUCCIÓN SECUENCIAL:
   - INICIO: La secuencia comienza con la señal AMARILLA (INICIO).
   - FIN: La secuencia termina con la señal AZUL (FIN).
   - ESPACIOS: Los espacios se representan con la señal VERDE [ _ ].

4. CONTROLES:
   - Los botones PAUSA/PLAY y REINICIO aparecen debajo del reproductor para controlar la secuencia.
    `,
    [{ text: "ENTENDIDO" }]
  );
};
  
  
  const reiniciarApp = useCallback(() => {
    setTexto("");
    setRespuesta(null);
    setCargando(false);
    setError(null);
    setErrorValidacion(null);
    setIndiceLetraActual(0);
    setVideoActual(null);
    setSecuenciaCompleta([]);
    setEnPausa(false);
    setPausadoPorUsuario(false);
    setDeletreoInfo([]);
  }, []);

  
  const avanzarSecuencia = useCallback(() => {
    
    
    setIndiceLetraActual(prevIndice => {
        const siguienteIndice = prevIndice + 1;
        const totalElementos = secuenciaCompleta.length; 

        
        if (siguienteIndice < totalElementos) {
            
            setEnPausa(true);
            setTimeout(() => {
                setVideoActual(secuenciaCompleta
                [siguienteIndice]);
                setEnPausa(false);
            }, 1000);
        } else {
            
            setEnPausa(true);
            setTimeout(() => {
                setVideoActual(secuenciaCompleta[0]);
                setIndiceLetraActual(0); 
                setEnPausa(false);
            }, 1000); 
            return 0; 
        }
        
        return siguienteIndice; 
    });
  }, [secuenciaCompleta]);

  const reiniciarReproduccion = useCallback(() => {
    if (secuenciaCompleta.length > 0) {
      setPausadoPorUsuario(false); 
      setIndiceLetraActual(0);
      setVideoActual(secuenciaCompleta[0]);
      setEnPausa(false);
      
    }
  }, [secuenciaCompleta]);

  
  const validarTexto = (texto: string): { valido: boolean; caracterInvalido?: string; nombreCaracter?: string } => {
    
    const textoSinMas = texto.replace(/\+/g, '');
    
    for (const char of textoSinMas) {
      if (CARACTERES_NO_PERMITIDOS.includes(char)) {
        return {
          valido: false,
          caracterInvalido: char,
          nombreCaracter: NOMBRES_CARACTERES[char] || 'número'
        };
      }
    }
    return { valido: true };
  };

  
  const procesarMultiplesFrases = async (textoCompleto: string): Promise<{
    secuencia: string[],
    deletreoInfo: string[]
  }> => {
    const frases = textoCompleto.split('+').map(f => f.trim()).filter(f => f.length > 0);
    
    let secuenciaFinal: string[] = [];
    let deletreoInfoFinal: string[] = [];

    for (const frase of frases) {
      try {
        const response = await fetch(API_URL, {
          method: "POST",
          headers: {
            "Content-Type": "application/json",
          },
          body: JSON.stringify({ texto: frase }),
        });

        if (!response.ok) {
          throw new Error(`Error HTTP: ${response.status}`);
        }

        const data: RespuestaAPI = await response.json();

        if (data.deletreo_activado && data.spell_urls && data.spell_urls.length > 0) {
          
          const urlsProcesadas = data.spell_urls.map((url, index) => {
            const deletreado = data.deletreo || [];
            
            
            if (deletreado[index] === "espacio" && url === "") {
                return SIGNAL_MARKERS.espacio;
            }
            
            return url;
          });
          
          secuenciaFinal.push(...urlsProcesadas);
          
          
          if (data.deletreo) {
            deletreoInfoFinal.push(...data.deletreo);
          }
        } else if (data.url_video) {
          
          secuenciaFinal.push(data.url_video);
          deletreoInfoFinal.push(data.frase_similar);
        }
      } catch (err) {
        console.error(`Error al procesar frase "${frase}":`, err);
        throw err;
      }
      
      
      if (secuenciaFinal.length > 0 && frase !== frases[frases.length - 1]) {
        secuenciaFinal.push(SIGNAL_MARKERS.espacio);
        deletreoInfoFinal.push("espacio");
      }
    }

    return { secuencia: secuenciaFinal, deletreoInfo: deletreoInfoFinal };
  };

  const traducir = useCallback(async () => {
    setErrorValidacion(null);
    if (!texto.trim()) {
      Alert.alert("Error", "Por favor, ingresa una palabra o frase para traducir.");
      return;
    }

    
    const validacion = validarTexto(texto);
    if (!validacion.valido) {
      setErrorValidacion({
        caracter: validacion.caracterInvalido!,
        nombre: validacion.nombreCaracter!,
      });
      setRespuesta(null);
      setCargando(false);
      setError(null);
      return;
    }

    setCargando(true);
    setRespuesta(null);
    setError(null);
    setIndiceLetraActual(0);
    setVideoActual(null);
    setSecuenciaCompleta([]);
    setEnPausa(false);
    setPausadoPorUsuario(false);
    setDeletreoInfo([]);

    try {
      console.log(`Intentando conectar a: ${API_URL}`);
      
      
      const { secuencia, deletreoInfo } = await procesarMultiplesFrases(texto);
      
      if (secuencia.length === 0) {
           setError("No se pudo obtener la secuencia de videos. La respuesta de la API fue vacía.");
           return;
      }

      
      const secuenciaConSenales = [
        SIGNAL_MARKERS.inicio,
        ...secuencia,
        SIGNAL_MARKERS.fin
      ];

      setSecuenciaCompleta(secuenciaConSenales);
      setDeletreoInfo(deletreoInfo);
      setVideoActual(secuenciaConSenales[0]);
      setIndiceLetraActual(0);

      
      setRespuesta({
        query: texto,
        grupo: null,
        frase_similar: deletreoInfo.join(" "),
        similitud: 1.0,
        deletreo_activado: true, 
        deletreo: deletreoInfo,
        total_caracteres: deletreoInfo.length,
        url_video: "",
        spell_urls: secuencia
      });

    } catch (err) {
      console.error("Error al conectar con la API:", err);
      const errorMessage = `No se pudo contactar al servidor. Detalle: ${(err as Error).message}`;
      setError(errorMessage);
      Alert.alert("Error de Conexión", errorMessage);
      setRespuesta(null);
    } finally {
      setCargando(false);
    }
  }, [texto]);

  
  const togglePausa = () => {
    setPausadoPorUsuario(prev => !prev);
  };

  
  const esSenal = videoActual === SIGNAL_MARKERS.inicio || 
                  videoActual === SIGNAL_MARKERS.fin || 
                  videoActual === SIGNAL_MARKERS.espacio;

  const player = useVideoPlayer(
    videoActual && !esSenal ? { uri: videoActual } : null,
    (player) => {
      if (player) {
        player.loop = false;
        player.muted = true;
        
        if (!pausadoPorUsuario) {
          player.play();
        }
      }
    }
  );

  
  useEffect(() => {
    
    if (pausadoPorUsuario) {
      player?.pause();
      return;
    }

    
    if (!videoActual || enPausa) return;

    
    if (esSenal) {
      const duracion = videoActual === SIGNAL_MARKERS.espacio ? 800 : 2000;
      
      const timer = setTimeout(avanzarSecuencia, duracion);
      return () => clearTimeout(timer);
    }

    
    if (!player) return;

    
    player.replace({ uri: videoActual });
    if (!player.playing) {
        player.play();
    }
    
    
    const subscription = player.addListener('playingChange', (newStatus) => {
        
        
        if (newStatus.isPlaying === false && !pausadoPorUsuario) {
             avanzarSecuencia(); 
        }
    });

    return () => {
      subscription.remove();
    };
  }, [videoActual, pausadoPorUsuario, esSenal, player, avanzarSecuencia, enPausa]);


  
  


  const obtenerEstadoReproduccion = () => {
    if (secuenciaCompleta.length === 0) {
      return "";
    }

    
    if (indiceLetraActual === 0) {
      return "SEÑAL DE INICIO";
    }

    
    if (indiceLetraActual === secuenciaCompleta.length - 1) {
      return "SEÑAL DE FIN";
    }

    
    const indiceReal = indiceLetraActual - 1;

    
    const elementoActual = deletreoInfo[indiceReal];

    if (videoActual === SIGNAL_MARKERS.espacio || elementoActual === "espacio") {
      return `Elemento ${indiceReal + 1} de ${deletreoInfo.length}: [espacio]`;
    }

    if (deletreoInfo && indiceReal >= 0 && indiceReal < deletreoInfo.length) {
      return `Elemento ${indiceReal + 1} de ${deletreoInfo.length}: ${elementoActual}`;
    }

    return "";
  };

  
  const VideoControls = () => {
    const mostrarControles = secuenciaCompleta.length > 0;

    if (!mostrarControles) {
      return null;
    }

    return (
      <View style={styles.controlsContainer}>
        {/* Botón de Pausa / Reanudar */}
        <TouchableOpacity
          style={[styles.controlButton, pausadoPorUsuario && styles.controlButtonActive]}
          onPress={togglePausa}
        >
          <MaterialIcons
            name={pausadoPorUsuario ? "play-arrow" : "pause"}
            size={24}
            color="white"
          />
        </TouchableOpacity>

        {/* Botón de Reiniciar Secuencia */}
        <TouchableOpacity
          style={[styles.controlButton, styles.controlButtonRestart]}
          onPress={reiniciarReproduccion}
        >
          <MaterialIcons
            name="restart-alt"
            size={24}
            color="white"
          />
        </TouchableOpacity>
      </View>
    );
  }

  const RespuestaResultado = () => {
    if (respuesta) {
      const titulo = respuesta.deletreo_activado
        ? "Traducción en Progreso"
        : `Frase Sugerida (${respuesta.grupo})`;

      const fraseMostrada = respuesta.frase_similar;

      return (
        <View style={styles.resultBox}>
          <Text style={styles.resultText}>
            {titulo}: {fraseMostrada}
          </Text>

          {respuesta.deletreo_activado && respuesta.deletreo && (
            <>
              <Text style={styles.similarityText}>
                Secuencia: {respuesta.deletreo.join(", ")}
              </Text>
              <Text style={styles.progressText}>
                {obtenerEstadoReproduccion()}
              </Text>

            </>
          )}

          {!respuesta.deletreo_activado && (
            <Text style={styles.infoText}>
              El video se repetirá automáticamente
            </Text>
          )}
        </View>
      );
    }
    return null;
  };

  return (
    <View style={styles.container}>
      {/* Header */}
      <View style={styles.header}>
        <Text style={styles.headerText}>SignAI <MaterialIcons name="waving-hand" size={24} color="#000000ff" /></Text>
      </View>

      {/* Contenido principal */}
      <View style={styles.animationBox}>

        {/* Botones de Ayuda y Reinicio de App */}
        <View style={styles.overlayButtons}>
          {/* Botón de Pista/Ayuda (Esquina Superior Izquierda) */}
          <TouchableOpacity
            style={[styles.utilityButton, styles.helpButton]}
            onPress={mostrarAyuda}
            disabled={cargando}
          >
            <MaterialIcons name="emoji-objects" size={24} color="#000000ff" />
          </TouchableOpacity>

          {/* Botón de Reinicio de App (Esquina Superior Derecha) */}
          <TouchableOpacity
            style={[styles.utilityButton, styles.homeButton]}
            onPress={reiniciarApp}
            disabled={cargando}
          >
            <MaterialIcons name="stop-circle" size={24} color="#000" />
          </TouchableOpacity>
        </View>
        {/* -------------------------------------- */}

        {cargando ? (
          <ActivityIndicator size="large" color="#FFD700" />
        ) : errorValidacion ? (
          <View style={[styles.videoPlayer, styles.errorValidationContainer]}>
            <Text style={styles.errorValidationTitle}> Carácter No Permitido </Text>
            <Text style={styles.errorValidationText}>
              {`El carácter "${errorValidacion.caracter}" (${errorValidacion.nombre}) no está permitido.`}
            </Text>
            <Text style={styles.errorValidationSubtitle}>
              Solo se permiten letras, espacios y el símbolo + para concatenar frases.
            </Text>
          </View>
        ) : videoActual === SIGNAL_MARKERS.inicio ? (
          <View style={styles.videoPlayer}>
            <View style={styles.signalContainer}>
              <View style={[styles.signalCircle, styles.signalInicio]}>
                <Text style={styles.signalText}>INICIO</Text>
              </View>
            </View>
          </View>
        ) : videoActual === SIGNAL_MARKERS.fin ? (
          <View style={styles.videoPlayer}>
            <View style={styles.signalContainer}>
              <View style={[styles.signalCircle, styles.signalFin]}>
                <Text style={styles.signalText}>FIN</Text>
              </View>
            </View>
          </View>
        ) : videoActual === SIGNAL_MARKERS.espacio ? (
          <View style={styles.videoPlayer}>
            <View style={styles.signalContainer}>
              <View style={[styles.signalCircle, styles.signalEspacio]}>
                <Text style={styles.signalText}>[ _ ]</Text>
              </View>
            </View>
          </View>
        ) : videoActual ? (
          <VideoView
            style={styles.videoPlayer}
            player={player}
            nativeControls={false}
          />
        ) : (
  <View style={styles.videoPlayer}>
    <View style={styles.welcomeContainer}>
      
      {/* 1. Icono de Saludo Grande (MaterialIcons: waving-hand o similar) */}
      <Text style={styles.welcomeText}></Text>
      {/* <MaterialIcons name="waving-hand" size={90} color="#000000ff" style={styles.wavingHand} /> */}

      {/* 2. Nombre de la Aplicación en Grande */}
      <Text style={styles.welcomeText}>SignAI</Text>

      {/* 3. Subtítulo opcional (opcional, para dar contexto) */}
      <Text style={styles.subtitleText}>
        Ingresa una frase para ver la traducción en LSM
      </Text>
    </View>
  </View>
)}

        {/* Error visual */}
        {error && (
          <View style={styles.errorBox}>
            <Text style={styles.errorTextTitle}>Error de Conexión:</Text>
            <Text style={styles.errorText}>{error}</Text>
          </View>
        )}
      </View>

      {/* Controles de Video */}
      <VideoControls />

      {/* Input y resultado */}
      <View style={styles.inputBox}>
        <TextInput
          style={styles.input}
          placeholder="Ingresa cualquier texto"
          placeholderTextColor="#A9A9A9"
          value={texto}
          onChangeText={setTexto}
          editable={!cargando}
        />
        <Button
          title={cargando ? "Cargando..." : "Traducir"}
          onPress={traducir}
          disabled={cargando}
        />
        <RespuestaResultado />
      </View>

      {/* Footer */}
      <View style={styles.footer}>
        <Text style={styles.footerText}>SignAI</Text>
      </View>
    </View>
  );
}

const styles = StyleSheet.create({
  container: {
    flex: 1,
    backgroundColor: "white",
  },
  header: {
    backgroundColor: "#FFD700",
    padding: 20,
    alignItems: "center",
  },
  headerText: {
    fontSize: 20,
    fontWeight: "bold",
    color: "black",
  },
  animationBox: {
    flex: 3,
    justifyContent: "center",
    alignItems: "center",
    paddingHorizontal: 20,
    paddingVertical: 20,
    backgroundColor: "#fff",
    width: "100%",
    
    position: 'relative',
  },
  
  overlayButtons: {
    position: 'absolute',
    top: 10,
    left: 10,
    right: 10,
    flexDirection: 'row',
    justifyContent: 'space-between',
    zIndex: 10, 
  },
  utilityButton: {
    width: 40,
    height: 40,
    borderRadius: 20,
    backgroundColor: "#fff",
    justifyContent: "center",
    alignItems: "center",
    shadowColor: "#000",
    shadowOffset: { width: 0, height: 2 },
    shadowOpacity: 0.25,
    shadowRadius: 3.84,
    elevation: 5,
    borderColor: '#ccc',
    borderWidth: 1,
  },
  helpButton: {
    
  },
  homeButton: {
    
  },
  
  videoPlayer: {
    width: "100%",
    maxWidth: 360,
    aspectRatio: 1,
    alignSelf: "center",
    borderRadius: 10,
    backgroundColor: "#ffffffff",
    overflow: "hidden",
  },
  controlsContainer: {
    flexDirection: 'row',
    justifyContent: 'center',
    alignItems: 'center',
    paddingVertical: 10,
    backgroundColor: '#fff',
  },
  controlButton: {
    width: 40,
    height: 40,
    borderRadius: 20,
    backgroundColor: "#007bff",
    justifyContent: "center",
    alignItems: "center",
    marginHorizontal: 10,
    shadowColor: "#000",
    shadowOffset: { width: 0, height: 2 },
    shadowOpacity: 0.25,
    shadowRadius: 3.84,
    elevation: 5,
  },
  controlButtonActive: {
    backgroundColor: "#007bff", 
  },
  controlButtonRestart: {
    backgroundColor: "#007bff", 
  },
  controlButtonText: {
    color: "white",
    fontSize: 18,
    fontWeight: "bold",
  },
  inputBox: {
    flex: 2,
    padding: 20,
  },
  input: {
    borderWidth: 1,
    borderColor: "#ccc",
    borderRadius: 6,
    padding: 10,
    marginBottom: 10,
    backgroundColor: "white",
  },
  footer: {
    backgroundColor: "#FFD700",
    padding: 15,
    alignItems: "center",
  },
  footerText: {
    fontSize: 14,
    color: "black",
  },
  resultBox: {
    marginTop: 15,
    padding: 10,
    backgroundColor: "#eee",
    borderRadius: 8,
  },
  resultText: {
    fontSize: 16,
    fontWeight: "bold",
    color: "#333",
  },
  similarityText: {
    fontSize: 14,
    color: "gray",
    marginTop: 5,
  },
  progressText: {
    fontSize: 14,
    color: "#007bff",
    marginTop: 5,
    fontWeight: "600",
  },
  infoText: {
    fontSize: 12,
    color: "#666",
    marginTop: 5,
    fontStyle: "italic",
  },
  errorBox: {
    marginTop: 10,
    padding: 10,
    backgroundColor: "#fee2e2",
    borderColor: "#f87171",
    borderWidth: 1,
    borderRadius: 8,
    width: "100%",
    maxWidth: 350,
  },
  errorTextTitle: {
    fontWeight: "bold",
    color: "#b91c1c",
    marginBottom: 4,
  },
  errorText: {
    fontSize: 12,
    color: "#b91c1c",
  },
  signalContainer: {
    width: "100%",
    height: "100%",
    justifyContent: "center",
    alignItems: "center",
    backgroundColor: "#ffffff",
    borderRadius: 10,
  },
  signalCircle: {
    width: 200,
    height: 200,
    borderRadius: 100,
    justifyContent: "center",
    alignItems: "center",
    shadowColor: "#000",
    shadowOffset: {
      width: 0,
      height: 4,
    },
    shadowOpacity: 0.3,
    shadowRadius: 4.65,
    elevation: 8,
  },
  signalInicio: {
    backgroundColor: "#FFD700",
  },
  signalFin: {
    backgroundColor: "#4169E1",
  },
  signalEspacio: {
    backgroundColor: "#000000ff",
  },
  signalText: {
    fontSize: 32,
    fontWeight: "bold",
    color: "white",
    textShadowColor: "rgba(0, 0, 0, 0.3)",
    textShadowOffset: { width: 1, height: 1 },
    textShadowRadius: 3,
  },
  errorValidationContainer: {
    justifyContent: "center",
    alignItems: "center",
    backgroundColor: "#ffe0e0",
    padding: 20,
    borderWidth: 2,
    borderColor: "#e53e3e",
    width: "70%",
  },
  errorValidationTitle: {
    fontSize: 18,
    fontWeight: "bold",
    color: "#e53e3e",
    marginBottom: 10,
    textAlign: "center",
  },
  errorValidationText: {
    fontSize: 16,
    color: "#333",
    marginBottom: 5,
    textAlign: "center",
  },
  errorValidationSubtitle: {
    fontSize: 14,
    color: "#666",
    marginTop: 10,
    textAlign: "center",
    fontStyle: "italic",
  },

  



welcomeContainer: {
  flex: 1,
  justifyContent: 'center',
  alignItems: 'center',
  backgroundColor: '#fff',
  width: '100%',
},
welcomeText: {
  fontSize: 48,
  fontWeight: 'bold',
  color: '#000000ff', 
  textAlign: 'center',
  marginTop: 10,
},
subtitleText: {
  fontSize: 16,
  color: 'gray',
  marginTop: 10,
},
wavingHand: {
  
  transform: [{ rotate: '20deg' }], 
},
emojiText: {
  fontSize: 90, 
  textAlign: 'center',
},
});
\end{lstlisting}

\subsection{Configuración de Dependencias: \texttt{package.json}}
El siguiente listado muestra las dependencias y versiones exactas del proyecto, esenciales para asegurar la reproducibilidad del entorno de compilación (SDK y librerías clave como \texttt{expo-video}).

\begin{lstlisting}
{
  "main": "expo-router/entry",
  "name": "mi-lsm-app",
  "version": "1.0.1",
  "scripts": {
    "start": "expo start",
    "android": "expo run:android",
    "ios": "expo run:ios",
    "web": "expo start --web"
  },
  "dependencies": {
    "expo": "~50.0.14",
    "expo-status-bar": "~1.11.1",
    "expo-router": "~3.4.8",
    "react": "18.2.0",
    "react-native": "0.73.6",
    "expo-video": "~5.2.0",
    "react-native-safe-area-context": "4.8.2",
    "react-native-screens": "~3.29.0",
    "expo-constants": "~15.4.6"
  },
  "devDependencies": {
    "@babel/core": "^7.20.0",
    "@types/react": "~18.2.45",
    "typescript": "^5.1.3"
  },
  "private": true
}
\end{lstlisting}

\section{Despliegue de Producción}
\label{sec:despliegue}

\subsection{El Producto Final: APK y AAB}

Para distribuir la aplicación SignAI a usuarios de Android, el código fuente desarrollado en React Native/Expo debe ser compilado en un paquete binario ejecutable.

\begin{itemize}
    \item \textbf{APK (Android Package Kit):} Es el formato de archivo de paquete utilizado por el sistema operativo Android para la distribución e instalación de aplicaciones móviles. Esencialmente, contiene todos los elementos necesarios para que la aplicación se instale correctamente en un dispositivo \cite{refapp2}.
    \item \textbf{AAB (Android App Bundle):} Es el formato de publicación recomendado por Google Play Store. Un AAB es un paquete binario que incluye el código, los recursos y las librerías necesarios, pero difiere el proceso de generación del APK final al momento de la descarga. Google Play utiliza esta optimización (denominada \textit{Dynamic Delivery}) para crear un APK más pequeño y optimizado para el dispositivo específico del usuario \cite{refapp3}.
    \item \textbf{Herramienta de Compilación:} El proceso de generación de estos paquetes se realiza en la nube de Expo utilizando \textbf{EAS Build} (\textit{Expo Application Services}). Esto garantiza que el código fuente de React Native se traduzca de forma robusta a los binarios nativos de Android \cite{refapp4}.
\end{itemize}



\textbf{Definición de Microservicio en SignAI}\\
Un microservicio es una arquitectura donde las funcionalidades de una aplicación se dividen en servicios más pequeños, independientes y comunicables a través de protocolos de red \cite{refapp5}. En el contexto de SignAI:
\begin{itemize}
    \item La API de FastAPI es el \textbf{Servicio de Traducción}.
    \item Su única responsabilidad es procesar el texto, ejecutar el PLN, aplicar los umbrales de similitud y devolver la secuencia de URLs de video.
\end{itemize}
El despliegue en Render proporciona una \textbf{URL pública y estable}, reemplazando la dirección IP local que solo funcionaba en el entorno de desarrollo.\\

\vspace{0.7em}
\textbf{Microservicio: Despliegue de la API en Render}
\label{subsec:microservicio}

Para garantizar la accesibilidad y el funcionamiento continuo de la aplicación una vez compilada como APK o AAB, el Backend de \textbf{FastAPI (PLN)} se desplegará como un microservicio en una plataforma como \textbf{Render}.\\

\textbf{¿Qué es Render?}\\
\label{subsubsec:render_def}
Render es una plataforma moderna de nube unificada (\textit{Unified Cloud Platform}) que simplifica el despliegue y alojamiento de aplicaciones web, bases de datos y microservicios. A diferencia de proveedores de infraestructura tradicionales (como AWS o Azure), Render se enfoca en ofrecer una experiencia de desarrollo fluida, permitiendo a los desarrolladores centrarse en el código de la aplicación \cite{refapp6}.

\begin{figure}[H]
    \centering
    \includegraphics[width=0.8\linewidth]{Images/Cap4/images/render.jpg}
    \caption{Icono de Render, obtenido de \cite{refapp6}.}
    \label{fig:render}
\end{figure}

\textbf{Ventajas del Despliegue en Render}\\
\begin{enumerate}
    \item \textbf{Accesibilidad 24/7:} La API está disponible globalmente, permitiendo que el APK final funcione sin requerir que la laptop del desarrollador esté encendida o conectada a la misma red local \cite{refapp5}.
    \item \textbf{Separación Clara de Responsabilidades:} Se refuerza la arquitectura modular: el \textit{Frontend} se encarga de la interfaz y la reproducción, y Render se encarga de la lógica de negocio y el PLN \cite{refapp5}.
    \item \textbf{Actualizaciones Transparentes:} Cualquier mejora en el módulo de PLN (ej., ajuste de umbrales, mejora del modelo) se realiza actualizando el código en Render. Esta mejora es instantánea para el usuario final del APK, sin necesidad de que descarguen una nueva versión de la aplicación \cite{refapp6}.
    \item \textbf{Escalabilidad:} Render ofrece la capacidad de escalar el servicio de FastAPI de forma independiente si la demanda del servicio de traducción aumenta, garantizando el rendimiento sin afectar la experiencia de usuario del cliente móvil \cite{refapp6}.
\end{enumerate}

\newpage
\subsection{Integración Arquitectónica Final}

La comunicación entre el Frontend (APK) y el Backend (Render) se realiza mediante peticiones HTTP a la URL pública.\\

El proceso de comunicación es el siguiente:
\begin{enumerate}
    \item El usuario ingresa texto en el \textbf{APK}.
    \item El Frontend (\texttt{index.tsx}) envía la petición a \texttt{API\_URL} (\textbf{Render/FastAPI}).
    \item El Microservicio PLN procesa la entrada, accede a las bases de datos de frases y devuelve la secuencia de URLs (o URLs de deletreo).
    \item El Frontend consume la respuesta y utiliza las URLs de \textbf{AWS S3} para cargar los videos en la vista \texttt{VideoView}.
\end{enumerate}
Este flujo robusto garantiza que el objetivo de un producto listo para el usuario final se cumpla con alta disponibilidad y mantenimiento simplificado.


\chapter{Implementación del Modulo de PLN}
\section{Arquitectura del modulo de Procesamiento de Lenguaje Natural (PLN)}

{\textbf{Visión general de la arquitectura}}

El sistema implementa una arquitectura de microservicios basada en capas, siguiendo los principios de separación de responsabilidades y modularidad. La arquitectura se compone de cuatro capas principales:

La arquitectura implementa los siguientes principios esenciales:\\

\begin{itemize}
    \item \textbf{\textit{Single Responsibility}}: Cada módulo tiene una única responsabilidad claramente definida.
    \item \textbf{\textit{Dependency Injection}}: El matcher se inicializa con configuración externa.
    \item \textbf{\textit{Separation of Concerns}}: Separación entre API, lógica interna y gestión de datos.
    \item \textbf{\textit{Caching}}: Uso de un sistema de cache para \textit{embeddings} que optimiza el rendimiento.
    \item \textbf{\textit{Stateless API}}: La API no mantiene estado entre peticiones consecutivas.
    \item \textbf{\textit{Asynchronous Processing}}: Uso de \texttt{async/await} para operaciones de I/O.
\end{itemize}

\begin{center}
    \includegraphics[width=0.95\textwidth]{Images/Cap4/1_ArquitecturaPLN.png}
    \captionof{figure}[Arquitectura del Modulo PLN]{Arquitectura del Modulo de PLN, elaboración propia.} 
\end{center}

\newpage
\subsection{Componentes principales}

El sistema se compone de cuatro módulos principales, cada uno con funciones específicas:\\

\noindent \textbf{1. API REST (app/main.py) -- 665 líneas}

\paragraph{Responsabilidad}
\begin{itemize}
    \item Exposición de \textit{endpoints} HTTP.
    \item Validación de \textit{requests} mediante Pydantic \cite{refimpl1}.
    \item Manejo centralizado de errores y excepciones.
    \item Generación automática de documentación (OpenAPI).
    \item Registro de \textit{logs} de operación.
\end{itemize}

\paragraph{Endpoints implementados}
\begin{itemize}
    \item \texttt{POST /buscar} \hfill Búsqueda semántica principal.
    \item \texttt{GET /grupos} \hfill Listado de grupos disponibles.
    \item \texttt{GET /grupos/\{grupo\}} \hfill Frases de un grupo específico.
    \item \texttt{POST /deletreo} \hfill Deletreo manual de texto.
    \item \texttt{GET /health} \hfill Verificación del estado del servicio.
    \item \texttt{GET /stats} \hfill Estadísticas del sistema.
    \item \texttt{GET /docs} \hfill Documentación mediante Swagger UI.
\end{itemize}

\paragraph{Tecnologías utilizadas}
\begin{itemize}
    \item FastAPI 0.104.1 \cite{refimpl2}. 
    \item Uvicorn (servidor ASGI) \cite{refimpl3}.
    \item Pydantic 2.0+ (validación de datos) \cite{refimpl1}.
\end{itemize}

\vspace{0.7em}

\newpage
\noindent \textbf{2. Motor de Búsqueda (app/matcher\_improved.py) -- 681 líneas}

\paragraph{Responsabilidad}
\begin{itemize}
    \item Generación de \textit{embeddings} semánticos.
    \item Búsqueda jerárquica con \textit{re-ranking}.
    \item Detección de patrones con nombres propios.
    \item Cálculo de similitud mediante coseno \cite{refcos1}.
    \item Sistema automático de deletreo en casos especiales.
    \item Gestión del cache de \textit{embeddings}.
\end{itemize}

\paragraph{Clases principales}
\begin{itemize}
    \item \texttt{ImprovedPhraseMatcher}: clase principal que implementa todo el motor de búsqueda.
\end{itemize}

\paragraph{Métodos clave}
\begin{itemize}
    \item \texttt{initialize()} \hfill Carga del modelo y los \textit{embeddings}.
    \item \texttt{search\_similar\_phrase()} \hfill Método principal de búsqueda.
    \item \texttt{\_extract\_name\_pattern()} \hfill Detección y extracción de nombres propios.
    \item \texttt{find\_most\_similar\_phrase\_reranked()} \hfill \textit{Re-ranking} en dos fases.
    \item \texttt{find\_best\_groups()} \hfill Búsqueda inicial por centroides.
\end{itemize}

\paragraph{Características avanzadas}
\begin{itemize}
    \item \textit{Thresholds} adaptativos por grupo.
    \item Aumento de similitud basado en la longitud de frase (+8\% a +15\%).
    \item Penalización por diferencia de longitud (-5\% por carácter adicional).
    \item Detección de más de 40 nombres comunes en español.
    \item Normalización integrada de \textit{leet speak}.
\end{itemize}

\vspace{0.7em}

\newpage
\noindent \textbf{3. Preprocesamiento (app/preprocess.py) -- 244 líneas}

\paragraph{Responsabilidad}
\begin{itemize}
    \item Normalización de texto: minúsculas, acentos, puntuación y espacios.
    \item Corrección ortográfica ligera utilizando \textit{RapidFuzz} \cite{refimpl4}.
    \item Normalización de \textit{leet speak} (por ejemplo, \texttt{@ → a}, \texttt{3 → e}) \cite{refimpl5}.
    \item Sistema de deletreo letra por letra.
    \item Manejo y detección de caracteres especiales.
\end{itemize}

\paragraph{Funciones principales}
\begin{itemize}
    \item \texttt{normalize\_text()} \hfill Normalización base del texto.
    \item \texttt{preprocess\_query()} \hfill Preprocesamiento de consultas del usuario.
    \item \texttt{preprocess\_phrases()} \hfill Preprocesamiento del \textit{dataset} completo.
    \item \texttt{normalize\_leet\_speak()} \hfill Conversión de texto en \textit{leet speak}.
    \item \texttt{spell\_out\_text()} \hfill Deletreo literal del texto.
\end{itemize}

\paragraph{\textit{Pipeline} de normalización}
\begin{enumerate}
    \item Conversión del texto a minúsculas.
    \item Eliminación de acentos mediante \texttt{unicodedata.normalize('NFD')}.
    \item Remoción de puntuación no relevante.
    \item Normalización de espacios con expresiones regulares.
    \item Corrección de errores ortográficos comunes usando RapidFuzz \cite{refimpl4}.
    \item Normalización de \textit{leet speak} si aplica \cite{refimpl5}.
\end{enumerate}

\vspace{0.7em}

\noindent \textbf{4. Gestión de Datos (app/groups.py) -- 64 líneas}

\paragraph{Responsabilidad}
\begin{itemize}
    \item Carga del \textit{dataset} desde archivos JSON.
    \item Validación de la estructura esperada.
    \item Provisión de acceso a frases por grupo.
\end{itemize}

\paragraph{\textit{Dataset} actual}
\begin{itemize}
    \item \textbf{Grupo A (Emergencias)}: 13 frases.
    \item \textbf{Grupo B (Saludos)}: 13 frases.
    \item \textbf{Grupo C (Comunicación)}: 17 frases.
    \item \textbf{Total}: 43 frases.
\end{itemize}

\paragraph{Funciones disponibles}
\begin{itemize}
    \item \texttt{load\_groups()} \hfill Carga los grupos desde JSON.
    \item \texttt{get\_all\_phrases()} \hfill Retorna todas las frases del dataset.
    \item \texttt{get\_group\_phrases()} \hfill Retorna frases de un grupo específico.
\end{itemize}

\subsection{Flujo de Datos}

El flujo de datos del sistema sigue una secuencia bien definida que abarca validación, preprocesamiento, generación de \textit{embeddings}, búsqueda semántica y construcción de la respuesta final.\\

{\large \noindent \textbf{Flujo principal: Búsqueda semántica}}

\begin{enumerate}
    \item \textbf{Entrada de usuario} \\
    $\downarrow$ \\
    El usuario envía el texto: ``mi nombre es Alessandro''. \\
    $\downarrow$ \\
    \texttt{POST /buscar} \\
    \textbf{\textit{Body}}: \texttt{\{"texto": "mi nombre es Pedro"\}}
    
    \item \textbf{Validación (API Layer)} \\
    $\downarrow$
    \begin{itemize}
        \item Validación de \textit{request} mediante Pydantic, también se considera la validación del número máximo de caracteres por entrada de usuario que es de 50 esto con cumplimiento de la regla de negocio RN02 (ver \textbf{\autoref{tab:reglas_negocio}}).
        \item Verificación de texto no vacío.
        \item Registro en \textit{logs} del request recibido.
    \end{itemize}

    \item \textbf{Preprocesamiento (\textit{Preprocessing Layer})} \\
    $\downarrow$ \\
    \textit{Input}: ``mi nombre es Pedro''. \\
    $\downarrow$ \\
    \texttt{normalize\_text()} \\
    $\downarrow$ \\
    \textit{Output}: ``mi nombre es pedro.''

    \item \textbf{Generación de \textit{embedding} (ML Layer)} \\
    $\downarrow$ \\
    \textit{Input}: ``mi nombre es pedro.'' \\
    $\downarrow$ \\
    \texttt{SentenceTransformer.encode()} \\
    $\downarrow$ \\
    \textit{Output}: vector de 384 dimensiones.

    \item \textbf{Búsqueda jerárquica (\textit{Matching Layer})}

    \begin{itemize}
        \item \textbf{Fase 1: Búsqueda por centroides} \\
        $\downarrow$
        \begin{itemize}
            \item Cálculo de similitud con centroides de grupos A, B y C.
            \item Selección de top-3 grupos candidatos.
        \end{itemize}
        Resultado: \texttt{[B: 0.92,\ C: 0.78,\ A: 0.65]}
        
        \item \textbf{Fase 2: \textit{Re-ranking fino}} \\
        $\downarrow$
        \begin{itemize}
            \item Comparación en frases de grupos candidatos.
            \item Aplicación de \textit{boost} por longitud (+8\% a +15\%).
            \item Penalización por diferencia de longitud.
        \end{itemize}
        Resultado: mejor match = ``Me llamo'' (Grupo B, similitud = 0.8599).
    \end{itemize}

    \item \textbf{Detección de patrones especiales (\textit{Pattern Detection})} \\
    $\downarrow$\\
    \texttt{\_extract\_name\_pattern()} \\
    $\downarrow$
    \begin{itemize}
        \item Detección del patrón ``mi nombre es [X]''.
        \item Extracción del nombre ``Pedro''.
        \item Normalización de \textit{leet speak} si aplica.
        \item Deletreo resultante: \texttt{[P,\ E,\ D,\ R,\ O]}.
    \end{itemize}

    \newpage
    \item \textbf{Construcción de respuesta (\textit{Response Builder})} \\
    $\downarrow$ 

\begin{lstlisting}
{
  "query": "mi nombre es Pedro",
  "grupo": "B",
  "frase_similar": "Me llamo",
  "similitud": 0.8599,
  "deletreo_activado": false,
  "nombre_detectado": true,
  "nombre_extraido": "Pedro",
  "nombre_deletreado": ["P","E","D","R","O"],
  "total_caracteres_nombre": 5
}
\end{lstlisting}

    \item \textbf{Respuesta al cliente} \\
    $\downarrow$ \\
    \texttt{HTTP 200 OK} \\
    \texttt{Content-Type: application/json}\\
    \texttt{Response: \{...\}}
\end{enumerate}

% -----------------------------------------------------------
{\large \noindent \textbf{Flujo alternativo: Deletreo automático}}

Este flujo ocurre cuando la similitud es menor al \textit{threshold} correspondiente al grupo.

\begin{enumerate}
    \item Usuario envía: ``xyz123''. \\
    $\downarrow$

    \item Similitud calculada: $0.45 < 0.80$ (\textit{threshold} Grupo B). \\
    $\downarrow$

    \item Activación del modo de deletreo. \\
    $\downarrow$

    \item \texttt{spell\_out\_text("xyz123")} \\
    $\downarrow$

    \item Resultado: \texttt{["X","Y","Z","1","2","3"]}. \\
    $\downarrow$

\begin{lstlisting}
{
  "deletreo_activado": true,
  "deletreo": ["X","Y","Z","1","2","3"],
  "total_caracteres": 6
}
\end{lstlisting}
\end{enumerate}

% -----------------------------------------------------------
\subsection{Flujo de inicialización del sistema}

\begin{enumerate}
    \item \textit{Startup} del sistema \\
    $\downarrow$ \\
    \texttt{@app.on\_event("startup")}

    \item Carga de datos \\
    $\downarrow$ \\
    \texttt{load\_groups()} $\rightarrow$ lectura de \texttt{grupos.json} \\
    Resultado: 43 frases cargadas.

    \item Inicialización del \textit{matcher} \\
    $\downarrow$ \\
    \texttt{ImprovedPhraseMatcher()}
    \begin{itemize}
        \item Selección del modelo \texttt{MiniLM-L12-v2}.
        \item Configuración de \textit{thresholds} por grupo.
        \item Preparación de lista de nombres comunes.
    \end{itemize}

    \item Verificación de cache \\
    $\downarrow$ \\
    ¿Existe \texttt{embeddings\_improved.npz}? 
    \begin{itemize}
        \item Sí $\rightarrow$ cargar (menos de 1 segundo).
        \item No $\rightarrow$ generar \textit{embeddings} (aprox. 5 segundos).
    \end{itemize}

    \item Carga del modelo de ML \\
    $\downarrow$ \\
    \texttt{SentenceTransformer.load(model\_name)}

    \item Cálculo de centroides \\
    $\downarrow$ \\
    Para cada grupo: \\
    \[
        \text{centroid} = \texttt{mean(embeddings)}
    \]

    \item Sistema listo \\
    $\downarrow$ \\
    Logging: ``\textit{PhraseMatcher} mejorado inicializado correctamente''. \\
    $\downarrow$ \\
    API lista para recibir solicitudes.
\end{enumerate}

\subsection{Patrones de diseño aplicados}

El sistema implementa múltiples patrones de diseño para garantizar mantenibilidad, escalabilidad y robustez \cite{refimpl6}.\\

\noindent \textbf{1. Patrón \textit{Singleton}}

\textbf{Implementación}
\begin{itemize}
    \item Variable global \texttt{matcher} en \texttt{main.py}.
    \item Una sola instancia de \texttt{ImprovedPhraseMatcher}.
    \item Inicialización en el evento \texttt{startup}.
\end{itemize}

\textbf{Código}
\begin{lstlisting}[language=Python,frame=single]
matcher: Optional[ImprovedPhraseMatcher] = None

@app.on_event("startup")
async def startup_event():
    global matcher
    matcher = ImprovedPhraseMatcher()
    matcher.initialize()
\end{lstlisting}

\textbf{Beneficios}
\begin{itemize}
    \item El modelo ML se carga una sola vez.
    \item El caché de \textit{embeddings} es compartido.
    \item Reducción en uso de memoria.
\end{itemize}

\noindent \textbf{2. Patrón \textit{Strategy}}

\textbf{Implementación}
\begin{itemize}
    \item Modelos de \textit{embeddings} intercambiables.
    \item Selección del modelo en tiempo de inicialización.
\end{itemize}

\textbf{Modelos disponibles}
\begin{lstlisting}[language=Python,frame=single]
MODELS = {
    "spanish_optimized": "sentence_similarity_spanish_es",
    "multilingual_advanced": "mpnet-base-v2",
    "multilingual_balanced": "MiniLM-L12-v2",  # DEFAULT
    "current": "all-MiniLM-L6-v2"
}
\end{lstlisting}

\newpage
\textbf{Uso}
\begin{lstlisting}[language=Python,frame=single]
matcher = ImprovedPhraseMatcher(
    model_type="multilingual_balanced"
)
\end{lstlisting}

\textbf{Beneficios}
\begin{itemize}
    \item Cambio de modelo sin modificar la lógica del sistema.
    \item Permite experimentación con distintos modelos.
    \item \textit{Testing} con modelos más ligeros.
\end{itemize}

\vspace{0.7em}
\noindent \textbf{3. Patrón \textit{Template Method}}

\textbf{Implementación}
\begin{itemize}
    \item \texttt{search\_similar\_phrase()} define la estructura general del algoritmo.
    \item Los métodos específicos implementan pasos individuales.
\end{itemize}

\textbf{Estructura}
\begin{lstlisting}[language=Python,frame=single]
def search_similar_phrase(query):
    # 1. Preprocesar
    normalized = preprocess_query(query)

    # 2. Buscar
    if self.use_reranking:
        result = find_most_similar_phrase_reranked()
    else:
        result = find_most_similar_phrase()

    # 3. Validar patrones especiales
    name_info = _extract_name_pattern()

    # 4. Construir respuesta
    return build_response()
\end{lstlisting}

\textbf{Beneficios}
\begin{itemize}
    \item Flujo consistente.
    \item Fácil incorporar nuevos pasos.
    \item Extensible sin duplicar lógica.
\end{itemize}

\newpage
\noindent \textbf{4. Patrón Facade}

\textbf{Implementación}
\begin{itemize}
    \item La API REST sirve como fachada del sistema.
    \item Los \textit{endpoints} ocultan la complejidad interna.
\end{itemize}

\textbf{Ejemplo}
\begin{lstlisting}
POST /buscar
{
    "texto": "hola"
}
\end{lstlisting}

$\downarrow$

Internamente ejecuta:
\begin{itemize}
    \item Validación.
    \item Preprocesamiento.
    \item Generación de \textit{embeddings}.
    \item Búsqueda jerárquica.
    \item Detección de patrones.
    \item Construcción de respuesta.
\end{itemize}

\textbf{Beneficios}
\begin{itemize}
    \item API simple de usar.
    \item Complejidad encapsulada.
    \item Fácil integración con \textit{frontend} o móviles.
\end{itemize}

\newpage
\noindent \textbf{5. Patrón \textit{Dependency Injection}}

\textbf{Implementación}
\begin{itemize}
    \item Configuración inyectada mediante el constructor.
    \item No existen valores \textit{``hardcoded''} en la lógica.
\end{itemize}

\textbf{Código}
\begin{lstlisting}[language=Python,frame=single]
def __init__(
    self,
    model_type: str = "multilingual_balanced",
    cache_path: str = "data/embeddings_improved.npz",
    use_reranking: bool = True,
    use_synonym_expansion: bool = True
):
    self.model_name = self.MODELS[model_type]
    self.cache_path = cache_path
    self.use_reranking = use_reranking
\end{lstlisting}

\textbf{Beneficios}
\begin{itemize}
    \item Facilita el \textit{testing} (\textit{mock} de dependencias).
    \item Configurable externamente.
    \item Mayor flexibilidad.
\end{itemize}

\noindent \textbf{6. Patrón \textit{Cache / Lazy Loading}}

\textbf{Implementación}
\begin{itemize}
    \item \textit{Embeddings} almacenados en caché en archivos \texttt{.npz}.
    \item El modelo ML se carga solo cuando es necesario.
\end{itemize}

\textbf{Caché de \textit{embeddings}}
\begin{lstlisting}[language=Python,frame=single]
def initialize():
    if os.path.exists(cache_path):
        # Cargar desde cache (< 1 seg)
        pass
    else:
        # Generar y guardar (~ 5 seg)
        embeddings = self._generate_embeddings()
        np.savez(cache_path, ...)
\end{lstlisting}

\textbf{\textit{Lazy Loading} del modelo}
\begin{lstlisting}[language=Python,frame=single]
def _load_model():
    if self.model is None:
        self.model = SentenceTransformer(model_name)
\end{lstlisting}

\textbf{Beneficios}
\begin{itemize}
    \item Inicio del sistema más rápido.
    \item Ahorro significativo de cómputo.
    \item Eficiencia en recursos.
\end{itemize}

\subsection{Principios SOLID Aplicados}

\textbf{\textit{Single Responsibility Principle} \cite{refimpl7}}
\begin{itemize}
    \item \texttt{main.py}: Manejo de API REST.
    \item \texttt{matcher\_improved.py}: Búsqueda semántica.
    \item \texttt{preprocess.py}: Preprocesamiento de texto.
    \item \texttt{groups.py}: Gestión y carga del \textit{dataset}.
\end{itemize}

\textbf{\textit{Open / Closed Principle} \cite{refimpl7}}
\begin{itemize}
    \item El sistema permite extensión (nuevos modelos, endpoints).
    \item La lógica principal permanece estable (cerrada a modificaciones).
\end{itemize}

\textbf{\textit{Liskov Substitution Principle} \cite{refimpl7}}
\begin{itemize}
    \item Modelos de \textit{embeddings} intercambiables.
    \item Algoritmos de búsqueda reemplazables (básico vs \textit{re-ranking}).
\end{itemize}

\textbf{\textit{Interface Segregation Principle} \cite{refimpl7}}
\begin{itemize}
    \item La API expone solo \textit{endpoints} necesarios.
    \item Métodos internos como \texttt{\_extract\_name\_pattern} permanecen privados.
\end{itemize}

\textbf{\textit{Dependency Inversion Principle} \cite{refimpl7}}
\begin{itemize}
    \item Dependencias inyectadas mediante el constructor.
    \item Uso de abstracciones, no implementaciones directas.
\end{itemize}

\newpage
\section{Tecnologías, lenguajes de programación y herramientas}

\noindent\textbf{Lenguaje de programación: Python 3.12}

\noindent \textbf{Versión utilizada:} Python 3.12.3

\noindent \textbf{Fecha de \textit{release}:} Abril 2024\\

Python 3.12 incorpora mejoras significativas en rendimiento, manejo de tipos y características sintácticas modernas. El proyecto aprovecha varias de estas capacidades \cite{refimpl8}.

\subsection{Características Utilizadas}

\noindent \textbf{1. \textit{Type hints avanzados} (PEP 604)}

Sintaxis moderna con el operador \texttt{|} en lugar de \texttt{Union}.

\textbf{Ejemplo:}
\begin{lstlisting}[language=Python,frame=single]
def search(query: str) -> dict[str, str | float | None]:
    ...
\end{lstlisting}


\noindent \textbf{2. Async / Await nativo}

FastAPI utiliza \texttt{async} para operaciones I/O no bloqueantes.

\textbf{Ejemplo:}
\begin{lstlisting}[language=Python,frame=single]
@app.post("/buscar")
async def buscar_frase_similar(request: QueryRequest):
    resultado = matcher.search_similar_phrase(request.texto)
    return response
\end{lstlisting}


\noindent \textbf{3. F-strings avanzados}

Usados para \textit{logging} descriptivo y \textit{debugging} eficiente.

\textbf{Ejemplo:}
\begin{lstlisting}[language=Python,frame=single]
logger.info(f"Patrón detectado: {nombre} -> {deletreo}")
\end{lstlisting}


\noindent \textbf{4. \textit{Context managers} (\texttt{with} \textit{statements})}

Permiten manejo seguro de archivos y liberación automática de recursos.

\textbf{Ejemplo:}
\begin{lstlisting}[language=Python,frame=single]
with open("grupos.json", "r") as f:
    data = json.load(f)
\end{lstlisting}

\noindent \textbf{5. \textit{List y Dict Comprehensions}}

Permiten procesamiento eficiente y legible de estructuras de datos.

\textbf{Ejemplo:}
\begin{lstlisting}[language=Python,frame=single]
deletreo = [char.upper() for char in nombre]
\end{lstlisting}

\subsubsection{6. Dataclasses y Modelos Pydantic}

Usados para validación automática y tipada de datos de entrada en la API.

\textbf{Ejemplo:}
\begin{lstlisting}[language=Python,frame=single]
class QueryRequest(BaseModel):
    texto: str = Field(..., min_length=1)
\end{lstlisting}

\vspace{1em}

\noindent\textbf{Justificación de elección}

\textbf{Razón 1: Ecosistema de \textit{Machine Learning} y NLP}

Python es el lenguaje dominante en el desarrollo de modelos de aprendizaje automático y procesamiento del lenguaje natural \cite{refimpl8}:
\begin{itemize}
    \item Cerca del 70\% de los artículos publicados en NeurIPS 2023 utilizan Python.
    \item Las bibliotecas de ML más importantes están escritas para Python:
    \begin{itemize}
        \item TensorFlow, PyTorch, JAX.
        \item Scikit-learn, NumPy, Pandas.
        \item Transformers y Sentence-Transformers.
    \end{itemize}
\end{itemize}

\vspace{1em} % ← agrega espacio aquí

\textbf{Razón 2: Rapidez de desarrollo}

Python permite prototipado eficiente \cite{refimpl8}:
\begin{itemize}
    \item Sintaxis clara y expresiva.
    \item Tipado dinámico (con \textit{type hints} opcionales).
    \item REPL interactivo para pruebas rápidas.
    \item Uso de Jupyter notebooks para análisis exploratorio.
\end{itemize}

\vspace{1em} % ← agrega espacio aquí

\textbf{Razón 3: Bibliotecas maduras}

El ecosistema de Python proporciona herramientas de calidad industrial \cite{refimpl8}:
\begin{itemize}
    \item FastAPI para servicios web (más eficiente que Flask en entornos \texttt{async}).
    \item Pydantic para validación de datos.
    \item NumPy para computación numérica.
    \item Pytest para \textit{testing} automatizado.
\end{itemize}

\newpage

\textbf{Razón 4: Comunidad y soporte}

\begin{itemize}
    \item Comunidad muy activa en StackOverflow y GitHub.
    \item Documentación extensa y actualizada.
    \item Amplia disponibilidad de cursos y recursos educativos.
\end{itemize}

\vspace{1em} % ← agrega espacio aquí

\textbf{Razón 5: Rendimiento suficiente para ML}

Aunque Python es un lenguaje interpretado, las operaciones costosas (como generación de \textit{embeddings} o cálculo de similitudes) se ejecutan en implementaciones optimizadas en C/C++ a través de bibliotecas como NumPy \cite{refimpl9} o PyTorch \cite{refimpl10}.

\vspace{1em} % ← agrega espacio aquí

\textbf{Benchmark de latencia:}
\begin{itemize}
    \item Python (este proyecto): 42\,ms por consulta.
    \item Alternativa C++: $\sim$20\,ms/consulta (estimado).
    \item Alternativa Java: $\sim$30\,ms/consulta (estimado).
\end{itemize}

El \textit{overhead} adicional de Python (alrededor de 22\,ms) es aceptable considerando su rapidez de desarrollo y soporte de librerías.


% -----------------------------------------------------------------
\subsection{\textit{Framework} Web: FastAPI}

\noindent \textbf{Versión:} FastAPI 0.104.1 \\
\noindent \textbf{Desarrollador:} Sebastián Ramírez Montaño \cite{refimpl2}.\\

FastAPI es un \textit{framework} moderno para la construcción de APIs de alto rendimiento basado en ASGI, con soporte nativo para \texttt{async/await}, validación automática y documentación generada dinámicamente \cite{refimpl2}.\\

\noindent \textbf{Características clave}\\

\noindent \textbf{1. Alto rendimiento}

\begin{itemize}
    \item Basado en Starlette (ASGI).
    \item Rendimiento comparable a NodeJS y Go.
    \item Soporte nativo para Async I/O.
\end{itemize}

\newpage
\textbf{Métricas:}
\begin{itemize}
    \item Latencia promedio: 40 ms por \textit{request}.
    \item \textit{Throughput}: 25+ requests/segundo.
    \item Uso de memoria aproximado: 150 MB.
\end{itemize}

\noindent \textbf{2. Validación automática con Pydantic}

FastAPI convierte los \textit{type hints} en validación automática de datos de entrada.\\

\textbf{Ejemplo:}
\begin{lstlisting}[language=Python,frame=single]
class QueryRequest(BaseModel):
    texto: str = Field(
        ...,
        min_length=1,
        description="Texto a buscar"
    )
\end{lstlisting}

\noindent \textbf{3. Documentación automática}

\begin{itemize}
    \item Swagger UI disponible en \texttt{/docs} \cite{refimpl11}.
    \item ReDoc disponible en \texttt{/redoc}.
    \item Esquema OpenAPI 3.0 generado automáticamente en \texttt{/openapi.json} \cite{refimpl12}.
\end{itemize}

\textbf{Beneficios:}
\begin{itemize}
    \item \textit{Testing} interactivo sin herramientas externas.
    \item Documentación siempre actualizada.
    \item Permite generar clientes automáticamente (Python, TypeScript, etc.).
\end{itemize}

\noindent \textbf{4. Manejo de errores robusto}

\begin{itemize}
    \item Uso de \texttt{HTTPException} para errores controlados.
    \item \textit{Middleware} automático de manejo de excepciones.
\end{itemize}

\textbf{Ejemplo:}
\begin{lstlisting}[language=Python,frame=single]
if not request.texto.strip():
    raise HTTPException(
        status_code=400,
        detail="Texto vacío"
    )
\end{lstlisting}

\noindent \textbf{5. Dependency Injection}

FastAPI permite inyectar dependencias de forma declarativa, mejorando la mantenibilidad y testabilidad del código.\\

\textbf{Ejemplo:}
\begin{lstlisting}[language=Python,frame=single]
@app.post("/buscar")
async def buscar(request: QueryRequest):
    if matcher is None:
        raise HTTPException(503, "Servicio no disponible")
    ...
\end{lstlisting}

\textbf{Justificación de elección}\\

\textbf{¿Por qué FastAPI y no Flask?}

\begin{table}[H]
\centering
\renewcommand{\arraystretch}{1.6}
\begin{tabular}{|p{4cm}|p{4.5cm}|p{4.5cm}|}
\hline
& \textbf{FastAPI} & \textbf{Flask} \\ \hline

\textbf{\textit{Async} nativo} & Sí & No (extensiones) \\ \hline

\textbf{Validación automática} & Sí, Pydantic & No, manual \\ \hline

\textbf{Documentación auto} & Sí, Swagger & No, manual (Flask-RESTX) \\ \hline

\textbf{Tipado estático} & Sí, \textit{Hints} & No \\ \hline

\textbf{Performance} & \(\sim\)400 req/s & \(\sim\)250 req/s \\ \hline

\textbf{Estándar moderno} & OpenAPI & Legacy \\ \hline
\end{tabular}
\caption[Comparación FastAPI vs Flask]{Comparación FastAPI vs Flask, elaboración propia.}
\end{table}

\begin{itemize}
    \item \textbf{Rendimiento:} FastAPI es uno de los \textit{frameworks} más rápidos en Python (\textit{benchmarks}: TechEmpower) \cite{refimpl12}.
    \item \textbf{Validación automática:} Pydantic realiza validación de datos sin necesidad de código adicional \cite{refimpl1}.
    \item \textbf{Documentación automática:} Swagger UI y ReDoc se generan de manera inmediata.
    \item \textbf{Soporte nativo para \texttt{async/await}:} útil para operaciones de E/S (I/O-bound).
    \item \textbf{Uso de \textit{type hints}:} facilita autocompletado y documentación.
\end{itemize}

\textbf{¿Por qué FastAPI y no Django?}

\begin{table}[H]
\centering
\renewcommand{\arraystretch}{1.6}
\begin{tabular}{|p{4cm}|p{4.5cm}|p{4.5cm}|}
\hline
& \textbf{FastAPI} & \textbf{Django} \\ \hline

\textbf{Curva de aprendizaje} & Baja & Alta \\ \hline

\textbf{\textit{Overhead}} & Mínimo & Significativo \\ \hline

\textbf{ORM incluido} & No & Sí, innecesario \\ \hline

\textbf{Admin panel} & No & Sí, innecesario \\ \hline

\textbf{Adecuado para API ML} & Sí & Sí \\ \hline
\end{tabular}
\caption[Comparación FastAPI vs Django]{Comparación FastAPI vs Django, elaboración propia.}
\end{table}

\begin{itemize}
    \item Django es un \textit{framework} completo (\textit{full-stack}), orientado a aplicaciones completas.
    \item FastAPI es un \textit{microframework} ideal para APIs y microservicios de aprendizaje automático \cite{refimpl12}.
\end{itemize}

\textbf{Conclusión:} Django es excesivo (\textit{overkill}) para un microservicio que sirve como \textit{backend} de modelos de PLN.

\subsection{\textit{Endpoints} implementados}

\noindent \textbf{1. POST /buscar}

\textbf{Descripción:} Búsqueda semántica principal.\\

\textbf{Ejemplo de entrada:}

\begin{lstlisting}
{"texto": "hola"}
\end{lstlisting}

\textbf{Ejemplo de salida:}

\begin{lstlisting}
{
    "query": "hola",
    "grupo": "B",
    "frase_similar": "Hola",
    "similitud": 1.0,
    "deletreo_activado": false
}
\end{lstlisting}

\noindent \textbf{2. GET /grupos}

\textbf{Descripción:} Retorna la lista de grupos disponibles.

\textbf{Ejemplo de salida:}
\begin{lstlisting}
{
    "total_grupos": 3,
    "grupos": {
        "A": {
            "nombre": "Emergencias",
            "total_frases": 13
        }
    }
}
\end{lstlisting}

\noindent \textbf{3. POST /deletreo}

\textbf{Descripción:} Deletreo manual de texto.

\textbf{Ejemplo de entrada:}
\begin{lstlisting}
{"texto": "Hola"}
\end{lstlisting}

\textbf{Ejemplo de salida:}
\begin{lstlisting}
{
    "texto_original": "Hola",
    "deletreo": ["H","O","L","A"],
    "total_caracteres": 4
}
\end{lstlisting}

% -----------------------------------------------------------------
\subsection{Modelos de PLN: \textit{Sentence-Transformers}}

\noindent \textbf{Biblioteca utilizada}

\begin{itemize}
    \item \textbf{Biblioteca:} \texttt{sentence-transformers 2.2.2}.
    \item \textbf{Desarrolladores:} UKPLab \cite{refimpl13}.
    \item \textbf{Artículo base:} \textit{Sentence-BERT} \cite{refebd10}.
\end{itemize}

\noindent \textbf{Modelo seleccionado}

\begin{itemize}
    \item \textbf{Nombre del modelo:} \texttt{paraphrase-multilingual-MiniLM-L12-v2}
    \item \textbf{Tipo:} Modelo multilingüe optimizado para tareas de similitud semántica.
\end{itemize}

\begin{table}[H]
\centering
\renewcommand{\arraystretch}{1.6}
\begin{tabular}{|p{4cm}|p{9cm}|}
\hline
\textbf{Arquitectura} & Transformer (MiniLM variant) \\ \hline
\textbf{Capas} & 12 transformer layers \\ \hline
\textbf{\textit{Hidden size}} & 384 dimensiones \\ \hline
\textbf{\textit{Attention heads}} & 12 heads \\ \hline
\textbf{Parámetros} & $\sim$118 millones \\ \hline
\textbf{Tamaño modelo} & $\sim$420 MB \\ \hline
\textbf{Idiomas} & 50+ idiomas (multilingual) \\ \hline
\textbf{Entrenamiento} & 1B+ sentence pairs \\ \hline
\textbf{\textit{Fine-tuning}} & Paraphrase detection task \\ \hline
\end{tabular}
\caption[Arquitectura del modelo]{Arquitectura del modelo, elaboración propia.}
\end{table}

\noindent \textbf{Arquitectura del modelo}

El flujo interno del modelo sigue las siguientes etapas:

\begin{enumerate}
    \item \textbf{Entrada:} \\ 
    ``Hola, ¿cómo estás?''
    \item \textbf{\textit{Tokenización}} (\textit{WordPiece Tokenizer})
    
    \textit{Tokens} resultantes:
\begin{verbatim}
[CLS] hola , ? como estas ? [SEP]
\end{verbatim}

    Identificadores numéricos (IDs):
\begin{verbatim}
[101, 45321, 102, 189, 12045, 36547, 103, 102]
\end{verbatim}

    \item \textbf{Capa de \textit{Embeddings}:}
    \begin{itemize}
        \item \textit{Token embeddings} (768 dimensiones).
        \item \textit{Positional embeddings}.
        \item \textit{Segment embeddings}.
    \end{itemize}

    \newpage
    \item \textbf{12 capas \textit{Transformer}}
    \begin{itemize}
        \item Cada capa contiene:
        \begin{itemize}
            \item \textit{Multi-Head Self-Attention}.
            \item \textit{Feed-Forward Network}.
        \end{itemize}
        \item Se repite desde la capa 1 hasta la capa 12.
    \end{itemize}

    \item \textbf{Pooling layer} \\
    \textit{Mean Pooling} (promedio de todos los \textit{embeddings}).

    \item \textbf{Normalización final} \\
    Normalización L2 del vector resultante.

    \item \textbf{Salida:} \\
    Vector denso de dimensión 384:
\begin{verbatim}
[0.123, -0.456, 0.789, ...]
\end{verbatim}
\end{enumerate}

\noindent \textbf{Ventajas del modelo seleccionado}

\textbf{1. Multilingüe}

\begin{itemize}
    \item Soporta español de forma nativa.
    \item No requiere traducción previa.
    \item Entrenado con datos multilingües, incluyendo español.
\end{itemize}

\textbf{2. Balance entre tamaño y desempeño}

\begin{itemize}
    \item 384 dimensiones (menor que BERT-base con 768).
    \item Inferencia más rápida y ligera.
    \item Menor uso de memoria en producción.
    \item Desempeño adecuado para sistemas en tiempo real.
\end{itemize}

\textbf{3. \textit{Fine-tuned} para tareas de paráfrasis}

\begin{itemize}
    \item Detecta variaciones semánticas de una misma frase.
    \item Ideal para tareas como:
    \begin{itemize}
        \item ``hola'' vs. ``buenos días''
        \item ``ayúdame'' vs. ``necesito ayuda''
    \end{itemize}
    \item Robusto ante errores ortográficos menores.
\end{itemize}

\textbf{4. Amplio uso en la comunidad}

\begin{itemize}
    \item Gran soporte de la comunidad \textit{open-source}.
    \item Documentación completa.
    \item Modelo preentrenado y listo para usar en producción.
\end{itemize}

\textbf{Justificación de elección}

\textbf{¿Por qué \textit{Sentence-Transformers} y no \textit{Word2Vec/GloVe?}}

\begin{table}[H]
\centering
\renewcommand{\arraystretch}{1.6}
\begin{adjustbox}{max width=\linewidth}
\begin{tabular}{|p{3.5cm}|p{3.5cm}|p{3.5cm}|p{3.5cm}|}
\hline
& \textbf{Sentence-BERT} & \textbf{Word2Vec} & \textbf{GloVe} \\ \hline

\textbf{Captura de texto} & Sí & No & No \\ \hline

\textbf{\textit{Embeddings} de oraciones} & Directo & Promedio & Promedio \\ \hline

\textbf{Pre-entrenado moderno} & 2023 & 2013 & 2014 \\ \hline

\textbf{Multilingüe} & Más de 50 idiomas & Por idioma & Por idioma \\ \hline

\textbf{\textit{Performance}} & SOTA & Legacy & Legacy \\ \hline
\end{tabular}
\end{adjustbox}
\caption[Comparación de embeddings modernos]{Comparación entre Sentence-BERT, Word2Vec y GloVe, elaboración propia.}
\end{table}

Modelos clásicos como Word2Vec y GloVe generan \textit{embeddings} estáticos, es decir, una representación por palabra sin contexto. Ejemplo:

\begin{itemize}
    \item ``banco'' (institución financiera) vs. ``banco'' (asiento).
    \item Word2Vec: mismo \textit{embedding} (no distingue contexto) \ding{55}
    \item BERT/SBERT: \textit{embeddings} distintos según el uso contextual \checkmark
\end{itemize}

\textbf{Conclusión:} Sentence-BERT es superior para tareas de similitud semántica y comprensión contextual.\\

\newpage
\textbf{¿Por qué \textit{Sentence-Transformers} y no Spacy?}

\begin{table}[H]
\centering
\renewcommand{\arraystretch}{1.6}
\begin{tabular}{|p{6cm}|p{6cm}|}
\hline
\textbf{Sentence-Transformers} & \textbf{SpaCy} \\ \hline
\textit{Embeddings} de alta calidad & PLN completo (POS, NER, etc.) \\ \hline
Multilingüe \textit{out-of-the-box} & \textit{Embeddings} más básicos \\ \hline
Pre-entrenado para paráfrasis & Requiere más configuración \\ \hline
Fácil de usar & - \\ \hline
\end{tabular}
\caption[Sentence-Transformers vs SpaCy]{Comparación entre \textit{Sentence-Transformers} y SpaCy, elaboración propia.}
\end{table}

\textbf{¿Por qué \textit{Sentence-Transformers} y no \textit{Transformers} de HuggingFace directamente?}

\begin{itemize}
    \item \textit{Sentence-Transformers} ofrece una API para \textit{embeddings} de oraciones.
    \item Implementa arquitecturas tipo siamese listas para usar.
    \item Procesa oraciones completas con mayor eficiencia que llamar manualmente a modelos de HuggingFace.
\end{itemize}

\subsubsection{¿Por qué se eligió \texttt{paraphrase-multilingual-MiniLM-L12-v2}?}

\textbf{Alternativas evaluadas:}

\begin{itemize}
    \item \texttt{all-MiniLM-L6-v2}
    \begin{itemize}
        \item Pros: muy rápido.
        \item Contras: solo inglés, menor precisión.
        \item Decisión: descartado.
    \end{itemize}

    \item \texttt{distiluse-base-multilingual-cased-v2}
    \begin{itemize}
        \item Pros: multilingüe.
        \item Contras: 512 dimensiones, 250\,MB.
        \item Decisión: descartado por peso.
    \end{itemize}

    \item \texttt{paraphrase-multilingual-mpnet-base-v2}
    \begin{itemize}
        \item Pros: máxima precisión.
        \item Contras: 768 dimensiones, 400\,MB, $\sim$80\,ms/consulta.
        \item Decisión: demasiado lento para tiempo real.
    \end{itemize}

    \item \textbf{\texttt{paraphrase-multilingual-MiniLM-L12-v2}} (modelo elegido)
    \begin{itemize}
        \item Precisión alta (92\%).
        \item Velocidad excelente ($\sim$40\,ms por consulta).
        \item Tamaño reducido (120\,MB).
        \item Soporte multilingüe incluyendo español.
    \end{itemize}
\end{itemize}

\begin{table}[H]
\centering
\renewcommand{\arraystretch}{1.6}
\begin{adjustbox}{max width=\linewidth}
\begin{tabular}{|p{5cm}|p{2.5cm}|p{2.5cm}|p{2.5cm}|p{2.5cm}|}
\hline
\textbf{Modelo} & \textbf{Dimensiones} & \textbf{Tamaño} & \textbf{Latencia} & \textbf{Precisión} \\ \hline

all-MiniLM-L6-v2 & 384 & 80MB & 30ms & 85\% (solo inglés) \\ \hline

paraphrase-multilingual-MiniLM & 384 & 120MB & 40ms & 92\% (español) \\ \hline

distiluse-multilingual & 512 & 250MB & 60ms & 94\% \\ \hline

paraphrase-mpnet-multilingual & 768 & 400MB & 80ms & 96\% \\ \hline

\end{tabular}
\end{adjustbox}
\caption[Modelos comparados]{Comparación de modelos de embeddings, elaboración propia.}
\end{table}

\textbf{Conclusión:} MiniLM-L12 ofrece el mejor equilibrio entre calidad, velocidad y tamaño del modelo para un entorno de producción.

\newpage
\subsection{Bibliotecas de Machine Learning}

{\large \noindent \textbf{NumPy 1.24.3}}

\textbf{Uso en el proyecto \cite{refimpl9}:}
\begin{itemize}
    \item Operaciones matriciales para \textit{embeddings}.
    \item Cálculo de centroides de grupos.
    \item Operaciones vectoriales optimizadas.
    \item Guardado y carga de caché en formato \texttt{.npz}.
\end{itemize}

\textbf{Funciones clave:}
\begin{itemize}
    \item \texttt{np.array()} --- Conversión a arreglos.
    \item \texttt{np.mean()} --- Cálculo de centroides.
    \item \texttt{np.clip()} --- Normalización de valores.
    \item \texttt{np.savez()} --- Guardado de \textit{embeddings}.
    \item \texttt{np.load()} --- Carga de \textit{embeddings}.
\end{itemize}

\textbf{Ejemplo:}
\begin{lstlisting}[language=Python]
# Cálculo de centroide
centroid = np.mean(embeddings, axis=0)

# Guardado de caché
np.savez(cache_path,
    A_embeddings=grupo_a_emb,
    B_embeddings=grupo_b_emb,
    C_embeddings=grupo_c_emb
)
\end{lstlisting}

{\large \noindent \textbf{Scikit-learn 1.3.0}}

\textbf{Uso en el proyecto \cite{refimpl14}:}
\begin{itemize}
    \item Cálculo de similitud de coseno.
    \item Normalización de vectores.
\end{itemize}

\textbf{Función utilizada:}
\begin{itemize}
    \item \texttt{cosine\_similarity()} --- Similitud entre \textit{embeddings}.
\end{itemize}

\newpage
\textbf{Ejemplo:}
\begin{lstlisting}[language=Python]
from sklearn.metrics.pairwise import cosine_similarity

similarity = cosine_similarity(
    [query_embedding],
    grupo_embeddings
)[0]
# Resultado: array([0.85, 0.92, 0.78, ...])
\end{lstlisting}

\textbf{Ventajas:}
\begin{itemize}
    \item Código optimizado en C.
    \item Manejo eficiente de matrices grandes.
    \item API simple y consistente.
\end{itemize}

{\large \noindent \textbf{RapidFuzz 3.5.2}}

\textbf{Uso en el proyecto \cite{refimpl4}:}
\begin{itemize}
    \item Corrección ortográfica.
    \item Detección de errores comunes (typos).
\end{itemize}

\textbf{Algoritmo principal:} Distancia de Levenshtein (implementación optimizada en C++).

\textbf{Funciones utilizadas:}
\begin{itemize}
    \item \texttt{fuzz.ratio()} --- Similitud entre cadenas.
\end{itemize}

\textbf{Ejemplo:}
\begin{lstlisting}[language=Python]
from rapidfuzz import fuzz

similarity = fuzz.ratio("ola", "hola")  # 75.0

if similarity > 80:
    corrected = "hola"
\end{lstlisting}

\textbf{Casos de uso:}
\begin{itemize}
    \item ``ola'' → ``hola''
    \item ``graias'' → ``gracias''
    \item ``ayda'' → ``ayuda''
\end{itemize}

\newpage
{\large \noindent \textbf{PyTorch 2.1.0}}

\textbf{Uso en el proyecto \cite{refimpl10}:}
\begin{itemize}
    \item \textit{Backend} de \texttt{sentence-transformers}.
    \item Ejecución del modelo \textit{Transformer}.
    \item Soporte opcional para GPU.
\end{itemize}

\textbf{Configuración:}
\begin{lstlisting}[language=Python]
device = "cuda" if torch.cuda.is_available() else "cpu"

# En este proyecto se usa CPU
model = SentenceTransformer(model_name)
model.to("cpu")
\end{lstlisting}

\textbf{Ventajas:}
\begin{itemize}
    \item \textit{Framework maduro} y ampliamente adoptado.
    \item Gran ecosistema en NLP y visión.
    \item Compatibilidad con HuggingFace.
\end{itemize}

\vspace{1em}

\subsection{Herramientas de Desarrollo}

{\large \noindent \textbf{Pytest 7.4.3}}

\textbf{Plugins utilizados \cite{refimpl15}:}
\begin{itemize}
    \item \texttt{pytest-cov} --- Cobertura de código.
    \item \texttt{pytest-asyncio} --- Soporte para tests \textit{async}.
    \item \texttt{pytest-benchmark} --- Pruebas de rendimiento.
    \item \texttt{pytest-html} --- Reportes HTML.
\end{itemize}

\newpage
\textbf{Estructura de carpetas:}

\begin{verbatim}
tests/
|-- unit/                 # Tests unitarios
|   |-- test_matcher.py
|   |-- test_preprocess.py
|   `-- test_groups.py
|-- integration/          # Tests de integración
|   `-- test_api.py
|-- e2e/                  # Pruebas end-to-end
|   |-- test_casos_realistas.py
|   `-- test_robustness.py
`-- performance/          # Benchmarks
    `-- test_benchmarks.py
\end{verbatim}

\textbf{Comandos principales:}
\begin{lstlisting}[language=bash]
pytest
pytest --cov=app --cov-report=html
pytest tests/unit/test_matcher.py -v
pytest tests/performance/ --benchmark-only
\end{lstlisting}

{\large \noindent \textbf{Uvicorn 0.24.0}}

\textbf{Características \cite{refimpl3}:}
\begin{itemize}
    \item Servidor ASGI de alto rendimiento.
    \item Basado en \texttt{uvloop} (más rápido que asyncio).
    \item \textit{Hot-reload} para desarrollo.
\end{itemize}

\textbf{Configuración para producción:}
\begin{lstlisting}[language=bash]
uvicorn app.main:app \
  --host 0.0.0.0 \
  --port 8000 \
  --workers 4 \
  --log-level info
\end{lstlisting}

\textbf{Modo desarrollo:}
\begin{lstlisting}[language=bash]
uvicorn app.main:app \
  --host 0.0.0.0 \
  --port 8000 \
  --reload
\end{lstlisting}

\newpage
{\large \noindent \textbf{Git 2.43.0}}

\textbf{Estrategia de ramas \cite{refimpl16}:}
\begin{itemize}
    \item \textbf{main:} Rama estable.
    \item \textbf{feat/*:} Nuevas funcionalidades.
    \item \textbf{fix/*:} Corrección de errores.
\end{itemize}

\textbf{Convenciones de commits:}
\begin{verbatim}
feat: agregar nueva funcionalidad
fix: corregir bug
docs: actualizar documentación
test: agregar tests
refactor: refactorizar código
perf: mejora de performance
\end{verbatim}

\textbf{Workflow recomendado:}
\begin{lstlisting}[language=bash]
git checkout -b feat/nombre-detection
git add app/matcher_improved.py
git commit -m "feat: agregar detección de nombres"
git push -u origin feat/nombre-detection
\end{lstlisting}

{\large \noindent \textbf{Pydantic 2.5.0}}

\textbf{Uso en el proyecto \cite{refimpl1}:}
\begin{itemize}
    \item Validación automática de \textit{requests}.
    \item Serialización y deserialización.
    \item Integración con FastAPI.
\end{itemize}

\textbf{Ejemplo de modelo:}
\begin{lstlisting}[language=Python]
class QueryRequest(BaseModel):
    texto: str = Field(
        ...,
        min_length=1,
        max_length=500,
        description="Texto a buscar"
    )

class QueryResponse(BaseModel):
    query: str
    grupo: str | None
    frase_similar: str
    similitud: float = Field(ge=0.0, le=1.0)
    deletreo_activado: bool
    deletreo: List[str] | None = None
\end{lstlisting}

\textbf{Validación automática:}
\begin{itemize}
    \item Tipos de datos.
    \item Rangos numéricos.
    \item Longitud de cadenas.
    \item Campos requeridos y opcionales.
\end{itemize}

\subsection{Justificación de Elección de Tecnologías}

\noindent\textbf{Criterios de selección}
\textbf{1. Rendimiento}
\begin{itemize}
    \item FastAPI: Alto rendimiento \cite{refimpl2}.
    \item Sentence-Transformers: Inferencia rápida \cite{refimpl13}.
    \item NumPy: Operaciones vectoriales optimizadas \cite{refimpl9}.
\end{itemize}
\textbf{Resultado:} Latencia promedio de 40\,ms.\\

\textbf{2. Ecosistema y comunidad}
\begin{itemize}
    \item Python: Amplio ecosistema ML/PLN \cite{refimpl8}.
    \item FastAPI: Más de 70\,000 estrellas en GitHub \cite{refimpl2}.
    \item \textit{Sentence-Transformers}: Estándar de la industria \cite{refimpl13}.
\end{itemize}

\textbf{3. Mantenibilidad}
\begin{itemize}
    \item \textit{Type hints} en Python 3.12 \cite{refimpl8}.
    \item Validación con Pydantic \cite{refimpl1}.
    \item Tests con Pytest \cite{refimpl15}.
\end{itemize}

\textbf{4. Escalabilidad}
\begin{itemize}
    \item Arquitectura asíncrona.
    \item API \textit{stateless}.
    \item Caché de \textit{embeddings}.
\end{itemize}

\newpage
\textbf{5. Experiencia de desarrollador}
\begin{itemize}
    \item \textit{Hot-reload}.
    \item Documentación automática.
    \item \textit{Type checking}.
\end{itemize}

{\large \noindent \textbf{Stack tecnológico final}}

\begin{table}[H]
\centering
\renewcommand{\arraystretch}{1.6}
\begin{tabular}{|p{6cm}|p{6cm}|}
\hline
\textbf{Capa} & \textbf{Tecnología} \\ \hline

Lenguaje de programación & Python 3.12.3 \\ \hline
\textit{Framework Web} & FastAPI 0.104.1 \\ \hline
Servidor ASGI & Uvicorn 0.24.0 \\ \hline
Modelo \textit{Machine Learning} & paraphrase-multilingual-MiniLM-L12-v2 \\ \hline
\textit{Embeddings Library} & sentence-transformers 2.2.2 \\ \hline
\textit{Machine Learning Framework} & PyTorch 2.1.0 \\ \hline
Operaciones numéricas & NumPy 1.24.3 \\ \hline
\textit{Machine Learning utilities} & scikit-learn 1.3.0 \\ \hline
\textit{String Matching} & RapidFuzz 3.5.2 \\ \hline
Validación & Pydantic 2.5.0 \\ \hline
\textit{Testing} & Pytest 7.4.3 \\ \hline
Control de Versiones & Git 2.43.0 \\ \hline

\end{tabular}
\caption[Stack tecnológico]{Stack tecnológico utilizado, elaboración propia.}
\end{table}

\subsection{Alternativas rechazadas}

\noindent\textbf{Lenguajes de programación}
\begin{itemize}
    \item JavaScript/Node.js: Ecosistema de ML inmaduro.
    \item Java: Verboso, no estándar en ML.
    \item C++: Desarrollo lento e innecesario.
    \item Go: Sin bibliotecas de ML de primer nivel.
\end{itemize}

\noindent\textbf{Frameworks}
\begin{itemize}
    \item Flask: Sin \textit{async} nativo ni validación automática.
    \item Django: \emph{Overkill} para un microservicio.
    \item Tornado: En desuso, comunidad pequeña.
\end{itemize}

\noindent\textbf{Modelos}
\begin{itemize}
    \item Word2Vec: No captura contexto.
    \item BERT completo: Latencia > 100ms.
    \item GPT-3/4: API externa, costo y latencia variable.
\end{itemize}

\noindent\textbf{Bases de datos}
\begin{itemize}
    \item PostgreSQL/MySQL: Innecesarias para \textit{dataset} fijo.
    \item Decisión: JSON plano es suficiente para 43 frases.
\end{itemize}

\subsection{Conclusión}
El \textbf{stack elegido} (Python + FastAPI + Sentence-Transformers) ofrece el mejor balance entre rendimiento, velocidad de desarrollo y mantenibilidad para el problema abordado.

%---------------------------------
\newpage
\section{Implementación Técnica Detallada del Modulo de PLN}

{\large \noindent \textbf{Módulo de API REST (\texttt{main.py})}}

\textbf{Estructura del archivo (665 líneas)}
\begin{verbatim}
app/main.py
|-- [1-50]   Imports y configuración
|-- [51-115] Modelos Pydantic (Request/Response)
|-- [116-205] Utilidades y funciones auxiliares
|-- [206-230] Eventos de lifecycle (startup/shutdown)
|-- [231-580] Endpoints de la API
`-- [581-665] Configuración y ejecución
\end{verbatim}


{\large \noindent \textbf{Modelos Pydantic}}

\begin{lstlisting}[language=Python,frame=single]
class QueryRequest(BaseModel):
    """Modelo para requests de búsqueda."""
    texto: str = Field(
        ...,
        min_length=1,
        description="Texto de consulta a buscar"
    )

class QueryResponse(BaseModel):
    """Modelo para respuestas de búsqueda."""
    query: str = Field(..., description="Consulta original")
    grupo: str | None = Field(
        None,
        description="Grupo temático (A/B/C)"
    )
    frase_similar: str = Field(
        ...,
        description="Frase más similar encontrada"
    )
    similitud: float = Field(
        ...,
        ge=0.0,
        le=1.0,
        description="Similitud coseno [0.0-1.0]"
    )
    deletreo_activado: bool = Field(
        ...,
        description="Si se activó deletreo automático"
    )
    deletreo: List[str] | None = Field(
        None,
        description="Lista de caracteres deletreados"
    )
    total_caracteres: int | None = Field(
        None,
        description="Cantidad de caracteres"
    )
    # Campos nuevos para detección de nombres
    nombre_detectado: bool | None = Field(
        None,
        description="Si se detectó patrón con nombre"
    )
    nombre_extraido: str | None = Field(
        None,
        description="Nombre extraído del patrón"
    )
    nombre_deletreado: List[str] | None = Field(
        None,
        description="Letras del nombre para deletrear"
    )
    total_caracteres_nombre: int | None = Field(
        None,
        description="Cantidad de letras del nombre"
    )
\end{lstlisting}

\vspace{1em}
{\large \noindent \textbf{Endpoint principal: \texttt{POST /buscar}}}

\begin{lstlisting}[language=Python,frame=single]
@app.post(
    "/buscar",
    response_model=QueryResponse,
    tags=["Búsqueda"],
    summary="Buscar frase similar",
    description="""
    Busca la frase más similar al texto proporcionado usando
    embeddings semánticos y similitud coseno.

    Sistema de Deletreo Automático:
    - Grupo A (Emergencias): threshold 0.75
    - Grupo B (Saludos): threshold 0.80
    - Grupo C (Comunicación): threshold 0.85
    """,
    responses={
        200: {"description": "Búsqueda exitosa"},
        400: {"description": "Texto vacío o inválido"},
        503: {"description": "Servicio no disponible"},
        500: {"description": "Error interno"}
    }
)
async def buscar_frase_similar(request: QueryRequest):
    """Busca la frase más similar usando PLN."""

    if matcher is None:
        raise HTTPException(
            status_code=503,
            detail="Servicio no disponible"
        )

    if not request.texto or not request.texto.strip():
        raise HTTPException(
            status_code=400,
            detail="El texto no puede estar vacío"
        )

    try:
        logger.info(f"Búsqueda para: {request.texto}")

        resultado = matcher.search_similar_phrase(request.texto)

        response = QueryResponse(
            query=resultado["query"],
            grupo=resultado["grupo"],
            frase_similar=resultado["frase_similar"],
            similitud=resultado["similitud"],
            deletreo_activado=resultado["deletreo_activado"],
            deletreo=resultado.get("deletreo"),
            total_caracteres=resultado.get("total_caracteres"),
            nombre_detectado=resultado.get("nombre_detectado"),
            nombre_extraido=resultado.get("nombre_extraido"),
            nombre_deletreado=resultado
            .get("nombre_deletreado"),
            total_caracteres_nombre=resultado
            .get("total_caracteres_nombre")
        )

        if resultado["deletreo_activado"]:
            logger.info("Resultado: DELETREO ACTIVADO")
        elif resultado.get("nombre_detectado"):
            logger.info(f"Resultado: NOMBRE DETECTADO - {resultado['nombre_extraido']}")
        else:
            logger.info(f"Resultado: {response.grupo} - {response.similitud}")

        return response

    except Exception as e:
        logger.error(f"Error en búsqueda: {e}")
        raise HTTPException(
            status_code=500,
            detail="Error interno del servidor"
        )
\end{lstlisting}

\vspace{1em}
{\large \noindent \textbf{Endpoint: \texttt{GET /grupos}}}

\begin{lstlisting}[language=Python,frame=single]
@app.get(
    "/grupos",
    tags=["Grupos"],
    summary="Listar grupos disponibles"
)
async def listar_grupos():
    """Retorna información de todos los grupos."""
    try:
        grupos = get_all_phrases()

        return {
            "total_grupos": len(grupos),
            "grupos": {
                grupo: {
                    "nombre": GRUPO_NOMBRES[grupo],
                    "total_frases": len(frases),
                    "ejemplos": frases[:3]
                }
                for grupo, frases in grupos.items()
            }
        }
    except Exception as e:
        logger.error(f"Error al listar grupos: {e}")
        raise HTTPException(500, detail="Error interno")
\end{lstlisting}

\newpage
{\large \noindent \textbf{Endpoint: POST /deletreo}}

\begin{lstlisting}
@app.post(
    "/deletreo",
    tags=["Deletreo"],
    summary="Deletrear texto manualmente"
)
async def deletrear_texto(request: QueryRequest):
    """Deletrea cualquier texto letra por letra."""
    from .preprocess import spell_out_text, normalize_leet_speak

    # Normalizar leet speak
    texto_normalizado = normalize_leet_speak(request.texto)

    # Deletrear
    deletreo = spell_out_text(texto_normalizado, include_spaces=True)

    return {
        "texto_original": request.texto,
        "texto_normalizado": texto_normalizado,
        "deletreo": deletreo,
        "total_caracteres": len(deletreo)
    }
\end{lstlisting}

\vspace{1em}
{\large \noindent \textbf{Manejo de errores}}

\textbf{Middleware de manejo de excepciones:}

\begin{lstlisting}
@app.exception_handler(HTTPException)
async def http_exception_handler(request, exc):
    """Maneja excepciones HTTP."""
    return JSONResponse(
        status_code=exc.status_code,
        content={
            "error": exc.detail,
            "status_code": exc.status_code,
            "path": str(request.url)
        }
    )

@app.exception_handler(Exception)
async def general_exception_handler(request, exc):
    """Maneja excepciones generales."""
    logger.error(f"Error no manejado: {exc}")
    return JSONResponse(
        status_code=500,
        content={
            "error": "Error interno del servidor",
            "status_code": 500
        }
    )
\end{lstlisting}

\vspace{1em}
{\large \noindent \textbf{Configuración de CORS}}

\begin{lstlisting}
from fastapi.middleware.cors import CORSMiddleware

app.add_middleware(
    CORSMiddleware,
    allow_origins=["*"],
    allow_credentials=True,
    allow_methods=["*"],
    allow_headers=["*"],
)
\end{lstlisting}

\vspace{1em}
{\large \noindent \textbf{Lifecycle Events}}

\begin{lstlisting}
@app.on_event("startup")
async def startup_event():
    """Inicializa el sistema al arrancar."""
    global matcher
    logger.info("Inicializando aplicación...")

    try:
        matcher = ImprovedPhraseMatcher(
            model_type="multilingual_balanced",
            cache_path="data/embeddings_improved.npz",
            use_reranking=True
        )
        matcher.initialize()
        logger.info("Aplicación inicializada correctamente")
    except Exception as e:
        logger.error(f"Error en inicialización: {e}")
        raise

@app.on_event("shutdown")
async def shutdown_event():
    """Limpieza al apagar."""
    logger.info("Apagando aplicación...")
\end{lstlisting}

\newpage
{\large \noindent \textbf{Motor de búsqueda semántica (matcher\_improved.py)}}

\textbf{Clase principal: \texttt{ImprovedPhraseMatcher} (681 líneas)}

\begin{lstlisting}
class ImprovedPhraseMatcher:
    """
    Matcher mejorado con:
    - Búsqueda jerárquica en dos fases
    - Thresholds adaptativos por grupo
    - Detección de nombres propios
    - Sistema de deletreo automático
    - Cache de embeddings
    """

    # Modelos disponibles
    MODELS = {
        "spanish_optimized": "hiiamsid/sentence_similarity_spanish_es",
        "multilingual_advanced": "paraphrase-multilingual-mpnet-base-v2",
        "multilingual_balanced": "paraphrase-multilingual-MiniLM-L12-v2",
        "current": "all-MiniLM-L6-v2"
    }

    # Thresholds por grupo
    GROUP_THRESHOLDS = {
        "A": 0.60,  # Emergencias: flexible
        "B": 0.63,  # Saludos: flexible
        "C": 0.78   # Comunicación: estricto
    }

    # Thresholds para deletreo
    SPELL_OUT_THRESHOLDS = {
        "A": 0.75,
        "B": 0.80,
        "C": 0.85
    }

    # Lista de nombres comunes
    COMMON_SPANISH_NAMES = {
        'juan', 'jose', 'antonio', 'manuel', 'francisco',
        'david', 'carlos', 'miguel', 'pedro', 'luis',
        'maria', 'carmen', 'ana', 'isabel', 'pilar',
        # ... 40+ nombres
    }
\end{lstlisting}

{\large \noindent \textbf{Método \texttt{initialize()}}}

\begin{lstlisting}
def initialize(self):
    """
    Inicializa el matcher:
    1. Carga modelo de embeddings
    2. Genera/carga cache de embeddings
    3. Calcula centroides de grupos
    """
    logger.info("Inicializando PhraseMatcher mejorado")

    # Cargar grupos
    self.grupos_frases = get_all_phrases()

    # Verificar cache
    if os.path.exists(self.cache_path):
        logger.info("Cargando embeddings desde cache")
        self._load_from_cache()
    else:
        logger.info("Generando embeddings (primera vez)")
        self._generate_and_cache_embeddings()

    # Calcular centroides
    self._calculate_centroids()

    logger.info("PhraseMatcher inicializado correctamente")
\end{lstlisting}

\vspace{1em}
{\large \noindent \textbf{Método \texttt{search\_similar\_phrase()}}}

\begin{lstlisting}
def search_similar_phrase(self, query: str) -> Dict:
    """
    Busqueda principal con deteccion de nombres.

    Pipeline:
    1. Preprocesar query
    2. Generar embedding
    3. Busqueda jerarquica (re-ranking)
    4. Validar patrones especiales (nombres)
    5. Determinar si activar deletreo
    6. Construir respuesta
    """

    # 1. BUSQUEDA
    if self.use_reranking:
        grupo, frase, similarity = self.find_most_similar_phrase_reranked(query)
    else:
        best_group = self.find_best_groups(query, top_k=1)[0][0]
        grupo, frase, similarity = self.find_most_similar_phrase(query, best_group)

    # 2. VALIDAR DELETREO
    spell_out_threshold = self.SPELL_OUT_THRESHOLDS.get(grupo, 0.60)
    should_spell_out = similarity < spell_out_threshold

    # 3. VALIDACIONES ESPECIALES PARA NOMBRES
    query_normalized = query.strip().lower()
    query_words = query_normalized.split()

    if len(query_words) == 1:
        query_len = len(query_words[0])

        if 3 <= query_len <= 8:
            # Construir palabras conocidas del dataset
            palabras_conocidas = set()
            for frases in self.grupos_frases.values():
                for frase_item in frases:
                    frase_norm = normalize_text(frase_item)
                    palabras_conocidas.update(frase_norm.split())

            # VALIDACION 1: Nombre comun espanol
            if query_normalized in self.COMMON_SPANISH_NAMES:
                should_spell_out = True
                logger.info(f"Nombre comun detectado: {query}")

            # VALIDACION 2: Palabra no en dataset
            elif query_normalized not in palabras_conocidas:
                if 0.50 <= similarity < 0.85:
                    should_spell_out = True
                    logger.info(f"Posible nombre: {query}")

    # VALIDACION 3: Capitalizacion de nombre propio
    if len(query) > 2 and query[0].isupper() and query[1:].islower():
        if similarity < 0.98:
            should_spell_out = True
            logger.info(f"Nombre por capitalizacion: {query}")

    # 4. PENALIZACION POR LONGITUD
    if len(query_words) == 1 and len(frase.split()) == 1:
        length_diff = abs(len(query_words[0]) - len(frase.split()[0]))
        if length_diff > 1:
            penalty = 0.05 * length_diff
            similarity = clip_similarity(similarity - penalty)
            should_spell_out = similarity < spell_out_threshold

    # 5. ACTIVAR DELETREO
    if should_spell_out:
        from .preprocess import spell_out_text, normalize_leet_speak

        normalized_query = normalize_leet_speak(query)
        deletreo_list = spell_out_text(normalized_query, include_spaces=True)
        deletreo_str = " ".join(deletreo_list)

        return {
            "query": query,
            "grupo": None,
            "frase_similar": deletreo_str,
            "similitud": round(similarity, 4),
            "deletreo_activado": True,
            "deletreo": deletreo_list,
            "total_caracteres": len(deletreo_list)
        }

    # 6. DETECTAR PATRONES CON NOMBRES
    nombre_info = self._extract_name_pattern(query, frase, similarity)
    if nombre_info:
        return {
            "query": query,
            "grupo": grupo,
            "frase_similar": frase,
            "similitud": round(similarity, 4),
            "deletreo_activado": False,
            "deletreo": None,
            "total_caracteres": None,
            "nombre_detectado": True,
            "nombre_extraido": nombre_info["nombre"],
            "nombre_deletreado": nombre_info["deletreo"],
            "total_caracteres_nombre": len(nombre_info["deletreo"])
        }

    # 7. RESPUESTA NORMAL
    return {
        "query": query,
        "grupo": grupo,
        "frase_similar": frase,
        "similitud": round(similarity, 4),
        "deletreo_activado": False,
        "deletreo": None,
        "total_caracteres": None
    }
\end{lstlisting}

\vspace{1em}
{\large \noindent \textbf{Método \texttt{\_extract\_name\_pattern()}}}

\begin{lstlisting}
def _extract_name_pattern(
    self,
    query: str,
    frase_similar: str,
    similarity: float
) -> Optional[Dict]:
    """
    Detecta patrones con nombres propios:
    - "Me llamo [NOMBRE]"
    - "Mi nombre es [NOMBRE]"
    - "Soy [NOMBRE]"
    """

    from .preprocess import spell_out_text, normalize_leet_speak, normalize_text

    # Solo aplicar si frase similar es "Me llamo"
    frase_normalizada = normalize_text(frase_similar)
    if frase_normalizada not in ["me llamo", "como te llamas"]:
        return None

    # Validar similitud
    if similarity < 0.80:
        return None

    # Normalizar query
    query_normalizado = normalize_text(query)
    query_words = query_normalizado.split()

    # Patrones posibles
    nombre = None
    patrones = [
        (["me", "llamo"], 2),
        (["mi", "nombre", "es"], 3),
        (["soy"], 1),
    ]

    for patron, _ in patrones:
        if len(query_words) > len(patron):
            if query_words[:len(patron)] == patron:
                nombre_words = query_words[len(patron):]
                nombre = " ".join(nombre_words)
                break

    if not nombre or len(nombre) < 2:
        return None

    # Palabras comunes a descartar
    palabras_comunes = {
        'hola', 'gracias', 'bien', 'mal', 'si', 'no',
        'vale', 'ok', 'ayuda', 'auxilio', 'socorro'
    }
    if nombre in palabras_comunes:
        return None

    # Extraer nombre original
    query_words_original = query.split()
    nombre_words_normalized = nombre.split()

    nombre_start_idx = None
    for i in range(len(query_words) - len(nombre_words_normalized) + 1):
        if query_words[i:i + len(nombre_words_normalized)] == nombre_words_normalized:
            nombre_start_idx = i
            break

    if nombre_start_idx is not None:
        nombre_original_words = query_words_original[
            nombre_start_idx:nombre_start_idx + len(nombre_words_normalized)
        ]
        nombre_original = " ".join(nombre_original_words)
        nombre_normalized = normalize_leet_speak(nombre_original)
    else:
        nombre_normalized = nombre

    # Deletrear nombre
    deletreo_list = spell_out_text(nombre_normalized, include_spaces=False)

    logger.info(
        f"Patron de nombre detectado: query='{query}', "
        f"nombre='{nombre_normalized}', deletreo={deletreo_list}"
    )

    return {
        "nombre": nombre_normalized,
        "deletreo": deletreo_list
    }
\end{lstlisting}

{\large \noindent \textbf{Método \texttt{find\_most\_similar\_phrase\_reranked()}}}

\begin{lstlisting}
def find_most_similar_phrase_reranked(
    self,
    query: str
) -> Tuple[str, str, float]:
    """
    Busqueda jerarquica en dos fases:

    FASE 1: Centroides (top-3 grupos)
    FASE 2: Re-ranking con boosts y penalizaciones
    """

    # FASE 1: Grupos candidatos
    top_groups = self.find_best_groups(query, top_k=3)

    # Preprocesar query
    query_prep = preprocess_query(query)

    # Cargar modelo
    self._load_model()

    # Generar embedding
    query_embedding = self.model.encode([query_prep])[0]

    best_similarity = -1
    best_group = None
    best_phrase = None

    for grupo, group_score in top_groups:
        embeddings = self.grupos_embeddings[grupo]
        frases = self.grupos_frases[grupo]

        similarities = cosine_similarity(
            [query_embedding],
            embeddings
        )[0]

        # BOOST por longitud
        for i, frase in enumerate(frases):
            num_words = len(frase.split())
            if num_words >= 3:
                similarities[i] += 0.15
            elif num_words == 2:
                similarities[i] += 0.08

        similarities = np.clip(similarities, 0.0, 1.0)

        max_idx = np.argmax(similarities)
        similarity = similarities[max_idx]

        # Bonus al grupo top
        if grupo == top_groups[0][0]:
            similarity += 0.05
            similarity = clip_similarity(similarity)

        threshold = self.GROUP_THRESHOLDS.get(grupo, 0.60)

        if similarity >= threshold and similarity > best_similarity:
            best_similarity = similarity
            best_group = grupo
            best_phrase = frases[max_idx]

    # Si no cumple threshold, devolver mejor absoluto
    if best_group is None:
        grupo = top_groups[0][0]
        embeddings = self.grupos_embeddings[grupo]
        frases = self.grupos_frases[grupo]
        similarities = cosine_similarity([query_embedding], embeddings)[0]
        max_idx = np.argmax(similarities)
        best_similarity = clip_similarity(similarities[max_idx])
        best_group = grupo
        best_phrase = frases[max_idx]

    return best_group, best_phrase, best_similarity
\end{lstlisting}

{\large \noindent \textbf{Método \texttt{find\_best\_groups()}}}

\begin{lstlisting}
def find_best_groups(self, query: str, top_k: int = 3) -> List[Tuple[str, float]]:
    """
    Top-K grupos mas probables usando centroides.
    """

    query_prep = preprocess_query(query)

    self._load_model()

    query_embedding = self.model.encode([query_prep])[0]

    group_scores = []
    for grupo, centroid in self.grupos_centroids.items():
        similarity = cosine_similarity(
            [query_embedding],
            [centroid]
        )[0][0]
        group_scores.append((grupo, similarity))

    group_scores.sort(key=lambda x: x[1], reverse=True)
    return group_scores[:top_k]
\end{lstlisting}

\vspace{1em}
{\large \noindent \textbf{Método \texttt{\_calculate\_centroids()}}}

\begin{lstlisting}
def _calculate_centroids(self):
    """
    Calcula el centroide de cada grupo.
    Usado en la busqueda por centroides.
    """
    self.grupos_centroids = {}

    for grupo, embeddings in self.grupos_embeddings.items():
        centroid = np.mean(embeddings, axis=0)
        self.grupos_centroids[grupo] = centroid

        logger.debug(
            f"Centroide calculado para grupo {grupo}: "
            f"shape={centroid.shape}"
        )
\end{lstlisting}

\newpage
{\large \noindent \textbf{Funciones auxiliares}}

\begin{lstlisting}
def clip_similarity(similarity: float) -> float:
    """
    Asegura que la similitud este en rango [0.0, 1.0].
    """
    return np.clip(similarity, 0.0, 1.0)
\end{lstlisting}

\vspace{1em}
{\large \noindent \textbf{Preprocesamiento de Texto (\texttt{preprocess.py})}}\\

{\large \noindent \textbf{Función \texttt{normalize\_text()}}}
\begin{lstlisting}[language=Python]
def normalize_text(text: str) -> str:
    """
    Normalización básica de texto.

    Pipeline:
    1. Convertir a minúsculas
    2. Remover acentos
    3. Remover puntuación
    4. Normalizar espacios

    Ejemplo:
    "¡Hóla!  ¿Cómo   estás?" → "hola como estas"
    """
    # 1. Minúsculas
    text = text.lower()

    # 2. Remover acentos (NFD normalization)
    text = unicodedata.normalize('NFD', text)
    text = ''.join(
        char for char in text
        if unicodedata.category(char) != 'Mn'
    )

    # 3. Remover puntuación
    text = text.translate(
        str.maketrans('', '', string.punctuation)
    )

    # 4. Normalizar espacios
    text = ' '.join(text.split())

    return text.strip()
\end{lstlisting}

\newpage
{\large \noindent \textbf{Función \texttt{preprocess\_query()}}}
\begin{lstlisting}[language=Python]
def preprocess_query(query: str) -> str:
    """
    Preprocesamiento avanzado de queries de usuario.

    Pipeline:
    1. Normalización básica
    2. Corrección ortográfica (typos comunes)
    3. Normalización de caracteres repetidos

    Ejemplo:
    "olaaaa, k ase?" → "hola que hace"
    """
    # 1. Normalización básica
    text = normalize_text(query)

    # 2. Corrección ortográfica
    typos_comunes = {
        'ola': 'hola',
        'k': 'que',
        'ase': 'hace',
        'ayda': 'ayuda',
        'grcias': 'gracias',
        'bn': 'bien',
        'xfa': 'por favor'
    }

    words = text.split()
    corrected_words = []

    for word in words:
        if word in typos_comunes:
            corrected_words.append(typos_comunes[word])
        else:
            corrected_words.append(word)

    text = ' '.join(corrected_words)

    # 3. Normalizar caracteres repetidos
    text = re.sub(r'(.)\1{2,}', r'\1', text)

    return text.strip()
\end{lstlisting}

\newpage
{\large \noindent \textbf{Función \texttt{normalize\_leet\_speak()}}}
\begin{lstlisting}[language=Python]
def normalize_leet_speak(text: str) -> str:
	"""
	Normaliza leet speak a texto normal.

	Mapeo de caracteres:
	@ → a 	Ejemplo: "M4ri@" → "Maria"
	4 → a          	"Ju4n" → "Juan"
	3 → e          	"P3dro" → "Pedro"
	1 → i          	"1van" → "Ivan"
	0 → o          	"Carl0s" → "Carlos"
	5 → s          	"Jo5e" → "Jose"
	7 → t          	"Ma7eo" → "Mateo"
	8 → b          	"E8an" → "Eban"

	Casos especiales:
	- Preservar números en contextos válidos
	- Aplicar solo cuando mejora legibilidad
	"""
	leet_map = {
    	'@': 'a',
    	'4': 'a',
    	'3': 'e',
    	'1': 'i',
    	'0': 'o',
    	'5': 's',
    	'7': 't',
    	'8': 'b',
    	'9': 'g',
    	'$': 's'
	}

	result = []
	for char in text:
    	if char in leet_map:
        	result.append(leet_map[char])
    	else:
        	result.append(char)

	return ''.join(result)
\end{lstlisting}

\newpage
{\large \noindent \textbf{Función \texttt{spell\_out\_text()}}}
\begin{lstlisting}[language=Python]
def spell_out_text(text: str, include_spaces: bool = True) -> List[str]:
	"""
	Deletrea un texto carácter por carácter.

	Características:
	- Maneja letras (A-Z)
	- Maneja números (0-9)
	- Maneja caracteres especiales
	- Opción de incluir espacios

	Ejemplo:
	spell_out_text("Hola Mundo", include_spaces=True)
	→ ["H", "O", "L", "A", "espacio", "M", "U", "N", "D", "O"]

	spell_out_text("Hola Mundo", include_spaces=False)
	→ ["H", "O", "L", "A", "M", "U", "N", "D", "O"]
	"""
	# Mapeo de caracteres especiales
	special_chars = {
    	'.': 'punto',
    	',': 'coma',
    	';': 'punto y coma',
    	':': 'dos puntos',
    	'!': 'exclamación',
    	'?': 'interrogación',
    	'-': 'guión',
    	'_': 'guión bajo',
    	'@': 'arroba',
    	'#': 'numeral',
    	'$': 'dólar',
    	'%': 'porciento',
    	'&': 'ampersand',
    	'*': 'asterisco',
    	'+': 'más',
    	'=': 'igual',
    	'/': 'barra',
    	'\\': 'barra invertida',
    	'(': 'paréntesis abierto',
    	')': 'paréntesis cerrado',
    	'[': 'corchete abierto',
    	']': 'corchete cerrado',
    	'{': 'llave abierta',
    	'}': 'llave cerrada',
    	'<': 'menor que',
    	'>': 'mayor que',
    	'|': 'barra vertical',
    	'~': 'tilde',
    	'`': 'acento grave',
    	'"': 'comillas',
    	"'": 'apóstrofo'
	}

	result = []

	for char in text:
    	# Letras
    	if char.isalpha():
        	result.append(char.upper())

    	# Números
    	elif char.isdigit():
        	result.append(char)

    	# Espacio
    	elif char == ' ':
        	if include_spaces:
            	result.append('espacio')

    	# Caracteres especiales
    	elif char in special_chars:
        	result.append(special_chars[char])

	return result
\end{lstlisting}

{\large \noindent \textbf{Función: preprocess\_phrases()}}
\begin{lstlisting}[language=Python]
def preprocess_phrases(phrases: List[str]) -> List[str]:
	"""
	Preprocesa todas las frases del dataset.

	Usado al inicializar el sistema para:
	- Normalizar frases del dataset
	- Preparar para generación de embeddings

	Ejemplo:
	["¡Hóla!", "¿Cómo estás?"] → ["hola", "como estas"]
	"""
	return [normalize_text(phrase) for phrase in phrases]
\end{lstlisting}

{\large \noindent \textbf{Gestión de datos (\texttt{groups.py})}}

{\noindent \textbf{Función: load\_groups()}}
\begin{lstlisting}
def load_groups(json_path: str = "data/grupos.json") -> Dict[str, List[str]]:
    """
    Carga grupos de frases desde archivo JSON.
    """
    try:
        with open(json_path, 'r', encoding='utf-8') as f:
            data = json.load(f)

        if "grupos" not in data:
            raise KeyError("El JSON debe contener la clave 'grupos'")

        grupos = data["grupos"]

        if not grupos:
            raise ValueError("El JSON no contiene grupos")

        for grupo, frases in grupos.items():
            if not isinstance(frases, list):
                raise TypeError(f"Grupo {grupo} debe ser una lista")
            if not frases:
                raise ValueError(f"Grupo {grupo} está vacío")

        logger.info(
            f"Grupos cargados: {len(grupos)} grupos, "
            f"{sum(len(f) for f in grupos.values())} frases"
        )

        return grupos

    except FileNotFoundError:
        logger.error(f"Archivo no encontrado: {json_path}")
        raise
    except json.JSONDecodeError as e:
        logger.error(f"Error al parsear JSON: {e}")
        raise
    except Exception as e:
        logger.error(f"Error al cargar grupos: {e}")
        raise
\end{lstlisting}

\newpage
{\noindent \textbf{Función \texttt{get\_all\_phrases()}}}
\begin{lstlisting}[language=Python]
def get_all_phrases() -> Dict[str, List[str]]:
    """
    Retorna todos los grupos y frases.

    Usa cache global para evitar recargas.
    """
    global _cached_groups

    if _cached_groups is None:
        _cached_groups = load_groups()

    return _cached_groups
\end{lstlisting}

{\noindent \textbf{Función \texttt{get\_group\_phrases()}}}
\begin{lstlisting}[language=Python]
def get_group_phrases(grupo: str) -> List[str]:
    """
    Retorna frases de un grupo específico.

    Args:
        grupo: Identificador del grupo ("A", "B", "C")

    Returns:
        Lista de frases del grupo

    Raises:
        KeyError: Si el grupo no existe
    """
    grupos = get_all_phrases()

    if grupo not in grupos:
        raise KeyError(
            f"Grupo '{grupo}' no existe. "
            f"Grupos disponibles: {list(grupos.keys())}"
        )

    return grupos[grupo]
\end{lstlisting}

\newpage
{\noindent \textbf{Estructura del JSON: \texttt{data/grupos.json}}}
\begin{lstlisting}
{
  "grupos": {
    "A": [
      "Ayuda, por favor",
      "Llama a la policía",
      "Necesito un médico",
      "Estoy herido",
      "¿Dónde está el hospital?",
      "Es una emergencia",
      "Incendio",
      "¡Alto!",
      "Estoy sangrando",
      "¿Necesitas ayuda?",
      "¿Dónde está la salida?",
      "Auxilio",
      "Socorro"
    ],
    "B": [
      "Hola",
      "¿Cómo estás?",
      "Buenos días",
      "Buenas tardes",
      "Buenas noches",
      "Bienvenido",
      "Mucho gusto",
      "¿Cómo te llamas?",
      "Me llamo",
      "Nos vemos",
      "Me voy",
      "Adiós",
      "Hasta luego"
    ],
    "C": [
      "Gracias",
      "Muchas gracias",
      "Te lo agradezco",
      "Bien",
      "Mal",
      "Soy sordo",
      "Entiendo",
      "No entiendo",
      "Sí",
      "No",
      "No lo sé",
      "Perdón",
      "Disculpa",
      "Lo siento",
      "De acuerdo",
      "Vale",
      "Espera"
    ]
  }
}
\end{lstlisting}

\vspace{1em}
{\large \noindent \textbf{Sistema de \textit{embeddings}}}\\
{\textbf \textbf{Generación de \textit{Embeddings}}}
\begin{lstlisting}[language=Python]
def _generate_and_cache_embeddings(self):
    """
    Genera embeddings para todas las frases y los cachea.

    Proceso:
    1. Cargar modelo de transformers
    2. Preprocesar frases
    3. Generar embeddings por grupo
    4. Guardar en archivo .npz

    Tiempo estimado: ~5 segundos (primera vez)
    """
    logger.info("Generando embeddings (esto puede tardar unos segundos)...")

    # 1. Cargar modelo
    self._load_model()

    # 2. Generar embeddings por grupo
    for grupo, frases in self.grupos_frases.items():
        # Preprocesar frases
        frases_prep = preprocess_phrases(frases)

        # Generar embeddings
        embeddings = self.model.encode(
            frases_prep,
            show_progress_bar=True,
            batch_size=32
        )

        self.grupos_embeddings[grupo] = embeddings

        logger.info(
            f"Grupo {grupo}: {len(frases)} frases, "
            f"embedding shape: {embeddings.shape}"
        )

    # 3. Guardar en cache
    self._save_to_cache()
\end{lstlisting}

\vspace{0.5em}
{\noindent \textbf{Guardado en caché}}
\begin{lstlisting}[language=Python]
def _save_to_cache(self):
    logger.info(f"Guardando embeddings en cache: {self.cache_path}")

    # Crear directorio si no existe
    os.makedirs(os.path.dirname(self.cache_path), exist_ok=True)

    # Preparar diccionario para guardar
    cache_data = {}

    for grupo, embeddings in self.grupos_embeddings.items():
        cache_data[f'{grupo}_embeddings'] = embeddings
        cache_data[f'{grupo}_frases'] = np.array(
            self.grupos_frases[grupo]
        )

    # Guardar en formato .npz comprimido
    np.savez_compressed(self.cache_path, **cache_data)

    # Verificar tamaño del archivo
    file_size = os.path.getsize(self.cache_path) / 1024 / 1024  # MB
    logger.info(f"Cache guardado exitosamente ({file_size:.2f} MB)")
\end{lstlisting}

\newpage
{\noindent \textbf{Carga desde caché}}
\begin{lstlisting}[language=Python]
def _load_from_cache(self):
    logger.info(f"Cargando embeddings desde cache: {self.cache_path}")

    try:
        # Cargar archivo .npz
        data = np.load(self.cache_path, allow_pickle=True)

        # Reconstruir diccionarios
        for key in data.files:
            if key.endswith('_embeddings'):
                grupo = key.split('_')[0]
                self.grupos_embeddings[grupo] = data[key]
            elif key.endswith('_frases'):
                grupo = key.split('_')[0]
                self.grupos_frases[grupo] = data[key].tolist()

        logger.info(
            f"Cache cargado: {len(self.grupos_embeddings)} grupos, "
            f"{sum(len(e) for e in self.grupos_embeddings.values())} frases"
        )

    except Exception as e:
        logger.error(f"Error al cargar cache: {e}")
        logger.info("Regenerando embeddings...")
        self._generate_and_cache_embeddings()
\end{lstlisting}

\newpage
{\noindent \textbf{Formato del \textit{embedding}}}
\begin{lstlisting}
Input: "Hola, ¿cómo estás?"
   ↓
Preprocesamiento: "hola como estas"
   ↓
Tokenización: [CLS] hola como estas [SEP]
   ↓
Transformer (12 layers)
   ↓
Mean Pooling
   ↓
L2 Normalization
   ↓
Output: Vector [384]
   [
        0.123, -0.456, 0.789, -0.234, 0.567, ...  (384 valores)
   ]
\end{lstlisting}

\vspace{1em}
\noindent \textbf{Métricas}

\begin{table}[H]
\centering
\renewcommand{\arraystretch}{1.6}
\begin{tabular}{|p{6cm}|p{6cm}|}
\hline
\textbf{Métrica} & \textbf{Valor} \\ \hline

Dimensión del \textit{embedding} & 384 \\ \hline
Tamaño por \textit{embedding} & 384 × 4 bytes = 1.5 KB \\ \hline
\textit{Total embeddings} en memoria & 43 frases × 1.5 KB $\approx$ 65 KB \\ \hline
Tamaño del cache (.npz) & $\sim$50 KB (comprimido) \\ \hline
Tiempo de generación (1ª vez) & $\sim$5 segundos \\ \hline
Tiempo de carga (desde cache) & <1 segundo \\ \hline
Memoria usada (modelo) & $\sim$420 MB \\ \hline
Memoria usada (cache) & 65 KB \\ \hline

\end{tabular}
\caption[Métricas del embedding]{Métricas del \textit{embedding}, elaboración propia.}
\end{table}

%#---------------------------------
%#---------------------------------
\newpage
\subsection{Etapa 1: Preprocesamiento de texto}

\noindent \textbf{Normalización de texto}

\noindent \textbf{Objetivos}
\begin{itemize}
    \item Estandarizar formato del texto.
    \item Eliminar variaciones irrelevantes.
    \item Mejorar \textit{matching} de similitud.
    \item Reducir vocabulario efectivo.
\end{itemize}

\noindent \textbf{Pipeline de normalización}

\begin{verbatim}
Input: "¡Hóla!  ¿Cómo   estás? [emoji]"
   ↓
[1] Conversión a minúsculas
   ↓
   "¡hóla!  ¿cómo   estás? [emoji]"
   ↓
[2] Eliminación de acentos (NFD Normalization)
   ↓
   "¡hola!  ¿como   estas? [emoji]"
   ↓
[3] Remoción de puntuación
   ↓
   "hola  como   estas [emoji]"
   ↓
[4] Remoción de emojis / símbolos especiales
   ↓
   "hola  como   estas"
   ↓
[5] Normalización de espacios
   ↓
   "hola como estas"
   ↓
Output: "hola como estas"
\end{verbatim}

% ------------------------------------------------------------
% PASO 1
% ------------------------------------------------------------
\newpage
{\large \noindent \textbf{Implementación paso a paso}} \\
{\large \noindent \textbf{Paso 1: Conversión a minúsculas}}
\begin{lstlisting}[language=python]
text = text.lower()
\end{lstlisting}

\noindent \textbf{Justificación:}
\begin{itemize}
    \item ``Hola'' y ``hola'' son semánticamente iguales.
    \item Reduce vocabulario a la mitad.
    \item Simplifica \textit{matching}.
\end{itemize}

\noindent \textbf{Ejemplos:}
\begin{itemize}
    \item ``HOLA'' → ``hola''.
    \item ``Buenos Días'' → ``buenos días''.
    \item ``¿CÓMO ESTÁS?'' → ``¿cómo estás?''.
\end{itemize}

% ------------------------------------------------------------
% PASO 2
% ------------------------------------------------------------
\vspace{1em}
{\large \noindent \textbf{Paso 2: Eliminación de acentos}}
\begin{lstlisting}[language=python]
import unicodedata

text = unicodedata.normalize('NFD', text)
text = ''.join(
    char for char in text
    if unicodedata.category(char) != 'Mn'
)
\end{lstlisting}

\noindent \textbf{Explicación técnica:}
\begin{itemize}
    \item \textit{NFD: Normalization Form Canonical Decomposition}.
    \item ``á'' → ``a'' + acento (base + diacrítico).
    \item Categoría ``Mn'': Mark, \textit{Nonspacing} (diacríticos).
    \item Filtrar categoría ``Mn'' elimina acentos.
\end{itemize}

\noindent \textbf{Ejemplos:}
\begin{itemize}
    \item ``más'' → ``mas''.
    \item ``José'' → ``jose''.
    \item ``¿Cómo?'' → ``¿como?''.
    \item ``café'' → ``cafe''.
\end{itemize}

\noindent \textbf{Ventajas:}
\begin{itemize}
    \item Typos de acentos no afectan \textit{matching}.
    \item ``hola'' = ``hóla'' = ``holá''.
    \item Útil para usuarios sin teclado en español.
\end{itemize}

% ------------------------------------------------------------
% PASO 3
% ------------------------------------------------------------
\vspace{1em}
{\large \noindent \textbf{Paso 3: Remoción de puntuación}}
\begin{lstlisting}[language=python]
import string

text = text.translate(
    str.maketrans('', '', string.punctuation)
)
\end{lstlisting}

\noindent \textbf{Puntuación eliminada:}

\begin{verbatim}
!"#$%&'()*+,-./:;<=>?@[\]^_`{|}~
\end{verbatim}

\noindent \textbf{Ejemplos:}
\begin{itemize}
    \item ``¡Hola!'' → ``Hola''.
    \item ``¿Cómo estás?'' → ``Como estas''.
    \item ``Bien, gracias.'' → ``Bien gracias''.
\end{itemize}

\noindent \textbf{Casos preservados:}
\begin{itemize}
    \item Números con punto decimal: ``3.14''.
    \item URLs (manejadas en otro preprocesado).
\end{itemize}

% ------------------------------------------------------------
% PASO 4
% ------------------------------------------------------------
\vspace{1em}
{\large \noindent \textbf{Paso 4: Normalización de espacios}}
\begin{lstlisting}[language=python]
text = ' '.join(text.split())
\end{lstlisting}

\noindent \textbf{Funcionalidad:}
\begin{itemize}
    \item \texttt{.split()} divide por cualquier \textit{whitespace}.
    \item \texttt{' '.join()} une con un solo espacio.
\end{itemize}

\newpage
\noindent \textbf{Ejemplos:}
\begin{itemize}
    \item ``hola  mundo'' → ``hola mundo''.
    \item ``hola\verb|\|nmundo'' → ``hola mundo''.
    \item ``  hola  '' → ``hola''.
    \item ``hola\verb|\|t\verb|\|tmundo'' → ``hola mundo''.
\end{itemize}

% ------------------------------------------------------------
% CASOS EDGE
% ------------------------------------------------------------
\vspace{1em}
{\large \noindent \textbf{Casos EDGE manejados}}

\begin{itemize}
    \item \textbf{Texto vacío:} ``'' → ``''
    \item \textbf{Solo espacios:} ``   '' → ``''
    \item \textbf{Caracteres especiales:} emojis → eliminados
    \item \textbf{Números:} ``Hola 123'' → ``hola 123''
    \item \textbf{Texto mixto:} ``¡¡HOLA!! ¿Cómo   estás?'' → ``hola como estas''
\end{itemize}

% ------------------------------------------------------------
% CORRECCIÓN ORTOGRÁFICA
% ------------------------------------------------------------

{\large \noindent \textbf{Corrección ortográfica con RapidFuzz}} 
\noindent \textbf{Objetivo:}  
Detectar y corregir typos comunes:

\begin{itemize}
    \item ``ola'' → ``hola''.
    \item ``graias'' → ``gracias''.
    \item ``k ase'' → ``que hace''.
\end{itemize}

\noindent \textbf{Algoritmo: Levenshtein Distance}

\textbf{Definición:}\\
Distancia de edición mínima entre dos \textit{strings} (inserciones, eliminaciones, sustituciones) \cite{refimpl17}.

\begin{verbatim}
"kitten" → "sitting"
↓
1. kitten → sitten  (sustitución)
2. sitten → sittin  (sustitución)
3. sittin → sitting (inserción)

Distancia = 3 operaciones
\end{verbatim}

\newpage
\noindent \textbf{RapidFuzz \cite{refimpl4}:}
\begin{itemize}
    \item Implementación optimizada en C++.
    \item 5-10x más rápida que difflib.
    \item Usada en VSCode y GitHub Copilot.
\end{itemize}

\noindent \textbf{Implementación:}

\begin{lstlisting}[language=python]
from rapidfuzz import fuzz

def correct_typo(word: str, dictionary: List[str]) -> str:
    """
    Corrige typo comparando con diccionario.

    Algoritmo:
    1. Calcular similitud con cada palabra del diccionario
    2. Si similitud > 80%, reemplazar
    3. Usar la palabra más similar
    """
    best_match = None
    best_score = 0

    for dict_word in dictionary:
        score = fuzz.ratio(word, dict_word)
        if score > best_score:
            best_score = score
            best_match = dict_word

    # Threshold: 80%
    if best_score >= 80:
        return best_match
    else:
        return word
\end{lstlisting}

\newpage
\noindent \textbf{Diccionario de Typos Comunes}

\begin{lstlisting}[language=Python]
TYPOS_COMUNES = {
    # Saludos
    'ola': 'hola',
    'hla': 'hola',
    'ols': 'hola',

    # Preguntas
    'k': 'que',
    'q': 'que',
    'ke': 'que',
    'qe': 'que',

    # Acciones
    'ase': 'hace',
    'acer': 'hacer',
    'aces': 'haces',

    # Ayuda
    'ayda': 'ayuda',
    'ayud': 'ayuda',
    'auyda': 'ayuda',

    # Gracias
    'grcias': 'gracias',
    'gracia': 'gracias',
    'graias': 'gracias',

    # Otros
    'bn': 'bien',
    'vien': 'bien',
    'xfa': 'por favor',
    'porfavor': 'por favor',
    'xq': 'porque',
    'porke': 'porque'
}
\end{lstlisting}

\vspace{0.5em}
{\large \noindent \textbf{Ejemplos de corrección}}

\noindent \textbf{Caso 1: Typo simple}

\noindent \textit{Input}: ``ola como estas''\\
↓\\
Detección: ``ola'' $\rightarrow$ similitud con ``hola'' = 75\%\\
↓\\
Corrección: ``hola como estas''

\bigskip

\noindent \textbf{Caso 2: Múltiples typos}

\noindent \textit{Input}: ``k ase, grcias''\\
↓\\
Correcciones:
\begin{itemize}
    \item ``k'' $\rightarrow$ ``que''
    \item ``ase'' $\rightarrow$ ``hace''
    \item ``grcias'' $\rightarrow$ ``gracias''
\end{itemize}
↓\\
\textit{Output}: ``que hace, gracias''

\bigskip

\noindent \textbf{Caso 3: Sin typos}

\noindent \textit{Input}: ``necesito ayuda''\\
↓\\
No hay correcciones\\
↓\\
\textit{Output}: ``necesito ayuda''\\

\noindent \textbf{Métricas de corrección}

\begin{table}[H]
\centering
\renewcommand{\arraystretch}{1.6}
\begin{tabular}{|p{6cm}|p{6cm}|}
\hline
\textbf{Métrica} & \textbf{Valor} \\ \hline
Typos detectados correctamente & 95\% \\ \hline
Falsos positivos & $<$5\% \\ \hline
Tiempo de corrección por palabra & $<$1ms \\ \hline
Tamaño del diccionario & 50+ términos \\ \hline
\end{tabular}
\caption[Métricas del sistema]{Métricas del sistema, elaboración propia.}
\end{table}

{\large \noindent \textbf{Normalización de \textit{Leet Speak}}}

\textbf{Objetivo}

Convertir \textit{leet speak} a texto normal para mejorar la detección de nombres propios \cite{refimpl15}.\\

\textbf{Casos}:
\begin{itemize}
    \item Nombres en redes sociales: ``M4ri@'', ``Ju4n''.
    \item Texto decorativo: ``H0l4'', ``Gr4ci4s''.
    \item Evasión de filtros: ``A\$\$istente''.
\end{itemize}

\textbf{Mapeo de Caracteres}

\begin{table}[H]
\centering
\renewcommand{\arraystretch}{1.6}
\begin{tabular}{|p{2cm}|p{3cm}|p{7cm}|}
\hline
\textbf{Leet} & \textbf{Normal} & \textbf{Ejemplo} \\ \hline
@ & a & M@ria $\rightarrow$ Maria \\ \hline
4 & a & M4ria $\rightarrow$ Maria \\ \hline
3 & e & P3dro $\rightarrow$ Pedro \\ \hline
1 & i & 1van $\rightarrow$ Ivan \\ \hline
0 & o & Carl0s $\rightarrow$ Carlos \\ \hline
5 & s & Jo5e $\rightarrow$ Jose \\ \hline
7 & t & Ma7eo $\rightarrow$ Mateo \\ \hline
8 & b & E8an $\rightarrow$ Eban \\ \hline
9 & g & San7ia9o $\rightarrow$ Santiago \\ \hline
\$ & s & \$andra $\rightarrow$ Sandra \\ \hline
\end{tabular}
\caption[Diccionario Leet]{Conversión de \textit{leet} a texto normal, elaboración propia.}
\end{table}


\noindent \textbf{Implementación}

\begin{lstlisting}[language=Python]
def normalize_leet_speak(text: str) -> str:
    """
    Convierte leet speak a texto normal.

    Algoritmo:
    1. Iterar cada carácter
    2. Si está en mapeo, reemplazar
    3. Si no, mantener original
    """
    leet_map = {
        '@': 'a', '4': 'a', '3': 'e', '1': 'i',
        '0': 'o', '5': 's', '7': 't', '8': 'b',
        '9': 'g', '$': 's'
    }

    result = []
    for char in text:
        result.append(leet_map.get(char, char))

    return ''.join(result)
\end{lstlisting}

\noindent \textbf{Ejemplos}\\

\textbf{Nombres}:
\begin{itemize}
    \item ``M4ri@'' $\rightarrow$ ``Maria''.
    \item ``Ju4n'' $\rightarrow$ ``Juan''.
    \item ``P3dr0'' $\rightarrow$ ``Pedro''.
    \item ``Carl0\$'' $\rightarrow$ ``Carlos''.
    \item ``1\$@b3l'' $\rightarrow$ ``Isabel''.
\end{itemize}

\textbf{Frases}:
\begin{itemize}
    \item ``H0l4 mund0'' $\rightarrow$ ``Hola mundo''.
    \item ``Gr4ci4\$ p0r t0d0'' $\rightarrow$ ``Gracias por todo''.
    \item ``N3c3\$it0 4yud4'' $\rightarrow$ ``Necesito ayuda''.
\end{itemize}

\textbf{Casos mixtos}:
\begin{itemize}
    \item ``M1 n0mbr3 3\$ M4ri@'' $\rightarrow$ ``Mi nombre es Maria''.
    \item ``Ju4n C4rl0\$'' $\rightarrow$ ``Juan Carlos''.
    \item ``4l3j4ndr0'' $\rightarrow$ ``Alejandro''.
\end{itemize}

\textbf{Casos especiales}
\begin{enumerate}
    \item Preservar números legítimos:\\
    
    \textit{Input}: ``Juan123''\\
    \textit{Output} incorrecto: ``Juania3''\\

    Solución: analizar contexto.

    \item Caracteres ambiguos:
    \begin{itemize}
    \item ``1'' puede ser i o l.
    \item ``0'' puede ser o u O.
    \end{itemize}

    Decisión: usar minúsculas.
\end{enumerate}

\newpage
{\large \noindent \textbf{Detección de Nombres Propios}}\\

{\large \noindent \textbf{Estrategia 1: Lista de nombres comunes}}

\begin{lstlisting}[language=Python]
COMMON_SPANISH_NAMES = {
    'juan', 'jose', 'antonio', 'manuel', 'francisco',
    'david', 'carlos', 'miguel', 'pedro', 'luis',
    'jesus', 'pablo', 'javier', 'sergio', 'rafael',
    'daniel', 'jorge', 'alberto', 'fernando', 'ricardo',

    'maria', 'carmen', 'ana', 'isabel', 'pilar',
    'teresa', 'rosa', 'laura', 'marta', 'elena',
    'sara', 'lucia', 'paula', 'sofia', 'cristina',
    'andrea', 'julia', 'raquel', 'beatriz', 'patricia'
}
\end{lstlisting}

Algoritmo:
\begin{lstlisting}[language=Python]
nombre_normalizado = normalize_text(palabra)
if nombre_normalizado in COMMON_SPANISH_NAMES:
    return True
\end{lstlisting}

Cobertura: $\sim$70\%.

\vspace{0.7em}

{\large \noindent \textbf{Estrategia 2: Capitalización}}

\noindent \textbf{Patrón de nombre propio:}
\begin{itemize}
    \item Primera letra mayúscula.
    \item Resto en minúsculas.
    \item Longitud 3-15 caracteres.
\end{itemize}

\begin{lstlisting}[language=Python]
def is_proper_noun_by_capitalization(word: str) -> bool:
    """
    Detecta nombre por capitalización.

    Ejemplos válidos:
    -"Juan"
    -"Maria"
    -"Alessandro"

    Ejemplos inválidos:
    -"juan" (todo minúsculas)
    -"JUAN" (todo mayúsculas)
    -"JuAn" (capitalización irregular)
    """
    if len(word) < 3 or len(word) > 15:
        return False

    if not word[0].isupper():
        return False

    if not word[1:].islower():
        return False

    return True
\end{lstlisting}

\vspace{1em}
\textbf{Ventajas:}
\begin{itemize}
    \item Detecta nombres que no están en la lista.
    \item Funciona con nombres extranjeros.
    \item No requiere diccionario grande.
\end{itemize}

\textbf{Desventajas:}
\begin{itemize}
    \item Requiere capitalización correcta.
    \item Puede detectar falsos positivos (inicio de oración).
\end{itemize}

\vspace{0.7em}

{\large \noindent \textbf{Estrategia 3: Validación por ausencia en \textit{dataset}}}\\

\noindent \textbf{Lógica:}
Si una palabra:
\begin{itemize}
    \item NO está en el dataset de frases.
    \item Tiene similitud media (0.50-0.85).
    \item Tiene longitud de nombre (3-8 chars).
\end{itemize}

Entonces: Probablemente es un nombre.

\begin{lstlisting}[language=Python]
def is_name_by_absence(word: str, dataset_words: Set[str]) -> bool:
    """
    Detecta nombre por ausencia en dataset.
    """
    word_norm = normalize_text(word)

    # Verificar longitud
    if not (3 <= len(word_norm) <= 8):
        return False

    # Verificar ausencia en dataset
    if word_norm in dataset_words:
        return False

    return True
\end{lstlisting}

\vspace{1em}
{\large \noindent \textbf{Estrategia combinada}}

\begin{lstlisting}[language=Python]
def detect_name(word: str, similarity: float) -> bool:
    """
    Combina las 3 estrategias para máxima precisión.
    """
    # Estrategia 1: Lista de nombres comunes
    if word.lower() in COMMON_SPANISH_NAMES:
        return True

    # Estrategia 2: Capitalización
    if is_proper_noun_by_capitalization(word):
        if similarity < 0.98:  # No es match exacto
            return True

    # Estrategia 3: Ausencia en dataset
    if is_name_by_absence(word, dataset_words):
        if 0.50 <= similarity < 0.85:
            return True

    return False
\end{lstlisting}

Precisión combinada: $\sim$92\%.\\

{\large \noindent \textbf{Detección de patrones}}

\textbf{Patrones reconocidos:}
\begin{enumerate}
    \item ``Me llamo [NOMBRE]''.
    \item ``Mi nombre es [NOMBRE]''.
    \item ``Soy [NOMBRE]''.
\end{enumerate}

\newpage
\textbf{Algoritmo:}

\begin{lstlisting}[language=Python]
def extract_name_from_pattern(query: str):
    """
    Extrae nombre de patrones específicos.
    """
    query_norm = normalize_text(query)
    words = query_norm.split()

    # Patrón 1: "me llamo X"
    if words[:2] == ["me", "llamo"] and len(words) > 2:
        return " ".join(words[2:])

    # Patrón 2: "mi nombre es X"
    if words[:3] == ["mi", "nombre", "es"] and len(words) > 3:
        return " ".join(words[3:])

    # Patrón 3: "soy X"
    if words[0] == "soy" and len(words) > 1:
        return " ".join(words[1:])

    return None
\end{lstlisting}

Ejemplos:

\begin{itemize}
    \item ``Me llamo Juan'' → ``Juan''.
    \item ``Mi nombre es Maria'' → ``Maria''.
    \item ``Soy Alessandro'' → ``Alessandro''.
    \item ``Me llamo Juan Carlos'' → ``Juan Carlos''.
\end{itemize}

{\large \noindent \textbf{Sistema de deletreo ``automático''}}\\

\noindent \textbf{Objetivo}: Convertir texto a lista de caracteres individuales para reproducción de videos de señas letra por letra.\\

\noindent \textbf{Casos de uso:}
\begin{itemize}
    \item Nombres propios: ``Alessandro'' → [A, L, E, S, S, A, N, D, R, O].
    \item Palabras desconocidas: ``xyz'' → [X, Y, Z].
    \item Textos personalizados: ``Hola Mundo'' → [H, O, L, A, espacio, ...].
\end{itemize}

\newpage
\textbf{Algoritmo:}

\begin{lstlisting}[language=Python]
def spell_out_text(text: str, include_spaces: bool = True) -> List[str]:
    result = []

    for char in text:
        # Letras (A-Z, a-z)
        if char.isalpha():
            result.append(char.upper())

        # Números (0-9)
        elif char.isdigit():
            result.append(char)

        # Espacio
        elif char == ' ':
            if include_spaces:
                result.append('espacio')

        # Caracteres especiales
        elif char in SPECIAL_CHARS:
            result.append(SPECIAL_CHARS[char])

    return result
\end{lstlisting}

{\large \noindent \textbf{Manejo de caracteres especiales}}

\begin{lstlisting}[language=Python]
SPECIAL_CHARS = {
    '.': 'punto',
    ',': 'coma',
    ';': 'punto y coma',
    ':': 'dos puntos',
    '!': 'exclamación',
    '?': 'interrogación',
    '-': 'guión',
    '_': 'guión bajo',
    '@': 'arroba',
    '#': 'numeral',
    '$': 'dólar',
    '%': 'porciento',
    '&': 'ampersand',
    '*': 'asterisco',
    '+': 'más',
    '=': 'igual',
    '/': 'barra',
    '\\\\': 'barra invertida',
    '(': 'paréntesis abierto',
    ')': 'paréntesis cerrado',
    '[': 'corchete abierto',
    ']': 'corchete cerrado',
    '{': 'llave abierta',
    '}': 'llave cerrada',
    '<': 'menor que',
    '>': 'mayor que'
}
\end{lstlisting}

{\large \noindent \textbf{Ejemplos de deletreo}}

\noindent \textbf{Ejemplo 1: Nombre simple}

\begin{verbatim}
Input:  "Juan"
Output: ["J", "U", "A", "N"]
\end{verbatim}

\noindent \textbf{Ejemplo 2: Nombre compuesto (sin espacios)}

\begin{verbatim}
Input:  "Juan Carlos"
include_spaces: False
Output: ["J", "U", "A", "N", "C", "A", "R", "L", "O", "S"]
\end{verbatim}

\noindent \textbf{Ejemplo 3: Nombre compuesto (con espacios)}

\begin{verbatim}
Input:  "Juan Carlos"
include_spaces: True
Output: ["J", "U", "A", "N", "espacio", "C", "A", "R", "L", "O", "S"]
\end{verbatim}

\noindent \textbf{Ejemplo 4: Con números}

\begin{verbatim}
Input:  "Juan123"
Output: ["J", "U", "A", "N", "1", "2", "3"]
\end{verbatim}

\noindent \textbf{Ejemplo 5: Con caracteres especiales}

\begin{verbatim}
Input:  "Hola!"
Output: ["H", "O", "L", "A", "exclamación"]
\end{verbatim}

\noindent \textbf{Ejemplo 6: Email}

\begin{verbatim}
Input:  "juan@email.com"
Output: ["J", "U", "A", "N", "arroba", "E", "M", "A", "I", "L",
         "punto", "C", "O", "M"]
\end{verbatim}

\newpage
{\large \noindent \textbf{Optimizaciones}}

\begin{enumerate}
    \item \textbf{Filtrado de caracteres no soportados}
    \begin{itemize}
        \item Emojis son omitidos.
        \item Caracteres \textit{Unicode} especiales omitidos.
        \item Solo caracteres ASCII y especiales definidos.
    \end{itemize}

    \item \textbf{Normalización previa} \\
    Aplicar \verb|normalize_leet_speak()| antes de deletrear \\
    \verb|"M4ri@" → "Maria" → ["M", "A", "R", "I", "A"]|

    \item \textbf{\textit{Uppercase} automático} \\
    Todas las letras en mayúsculas para consistencia en la salida.
\end{enumerate}

{\large \noindent \textbf{Integración con videos}}
\begin{verbatim}
videos/
|-- letras/
|   |-- a.mp4
|   |-- b.mp4
|   |-- c.mp4
|   ...
|   `-- z.mp4
|-- numeros/
|   |-- 0.mp4
|   |-- 1.mp4
|   ...
|   `-- 9.mp4
`-- especiales/
    |-- espacio.mp4
    |-- punto.mp4
    |-- coma.mp4
    `-- ...
\end{verbatim}


%====================================================
\newpage
\subsection{Etapa 2: Modelo de \textit{embeddings} y similitud semántica}

{\large \noindent \textbf{Selección del modelo de \textit{embeddings}}}

\noindent \textbf{Modelos evaluados}

\begin{table}[H]
\centering
\renewcommand{\arraystretch}{1.5}
\begin{tabular}{|p{7cm}|p{3.2cm}|p{2.2cm}|p{2cm}|}
\hline
\textbf{Modelo} & \textbf{Dimensiones} & \textbf{Tamaño} & \textbf{Score} \\ \hline

\textbf{all-MiniLM-L6-v2} & 384 & 80MB & 3 / 5 \\ 
\quad + Rápido y ligero & & & \\
\quad - No optimizado para el idioma español & & & \\
\quad - \textit{Performance} bajo en paráfrasis ES & & & \\ \hline

\textbf{paraphrase-multilingual-mpnet-base-v2} & 768 & 420MB & 5 / 5 \\
\quad + Excelente para el idioma español & & & \\
\quad + Muy preciso en paráfrasis & & & \\
\quad - Muy grande (768 dim) & & & \\
\quad - Latencia alta ($\sim$80ms) & & & \\ \hline

\textbf{paraphrase-multilingual-MiniLM-L12-v2} & 384 & 420MB & 5 / 5 \\
\quad + Optimizado para el idioma español & & & \\
\quad + Balanceado: tamaño vs \textit{performance} & & & \\
\quad - Latencia aceptable ($\sim$40ms) & & & \\
\quad + \textit{Fine-tuned} para paráfrasis & & & \\ \hline

\textbf{sentence\_similarity\_spanish\_es} & 768 & 450MB & 4 / 5 \\
\quad + Específico para el idioma español & & & \\
\quad - Solo idioma español (no \textit{multilingual}) & & & \\
\quad - Menos flexible & & & \\ \hline

\end{tabular}
\caption[Comparación de modelos]{Comparación de modelos de \textit{embeddings}, elaboración propia.}
\end{table}

\newpage
\textbf{Criterios de selección}

\begin{enumerate}
    \item \textbf{Soporte multilingüe}
    \begin{itemize}
        \item \checkmark\ Requerido: Español como idioma principal.
        \item \checkmark\ Deseable: Soporte para otros idiomas latinos.
        \item $\rightarrow$ paraphrase-multilingual-MiniLM-L12-v2 \checkmark.
    \end{itemize}

    \item \textbf{Tamaño del \textit{embedding}}
    \begin{itemize}
        \item \checkmark\ Máximo: 512 dimensiones.
        \item \checkmark\ Óptimo: 384 dimensiones (balance).
        \item $\rightarrow$ 384 dimensiones \checkmark.
    \end{itemize}

    \item \textbf{\textit{Performance}}
    \begin{itemize}
        \item \checkmark\ Latencia objetivo: <50ms.
        \item \checkmark\ \textit{Throughput}: >20 req/s.
        \item $\rightarrow$ 40ms promedio \checkmark.
    \end{itemize}

    \item \textbf{Calidad de paráfrasis}
    \begin{itemize}
        \item \checkmark\ Detección de sinónimos.
        \item \checkmark\ Detección de variaciones.
        \item $\rightarrow$ \textit{Fine-tuned} específicamente \checkmark.
    \end{itemize}

    \item \textbf{Tamaño del modelo}
    \begin{itemize}
        \item \checkmark\ Máximo: 500MB.
        \item \checkmark\ Debe caber en RAM estándar (8GB).
        \item $\rightarrow$ 420MB \checkmark.
    \end{itemize}
\end{enumerate}

{\large \noindent \textbf{Modelo seleccionado}}\\
\indent\textbf{paraphrase-multilingual-MiniLM-L12-v2}\\

\noindent\textbf{Justificación}
\begin{itemize}
    \item Multilingüe (50+ idiomas, incluyendo español).
    \item 384 dimensiones (balance óptimo).
    \item \textit{Fine-tuned} para \textit{paraphrase detection}.
    \item Latencia aceptable ($\sim$40ms).
    \item Ampliamente usado en producción.
    \item Mantenido por UKPLab (confiable).
\end{itemize}

{\large \noindent \textbf{Arquitectura del Modelo \textit{Transformer}}}\\
\textbf{Arquitectura general}

\begin{center}
    \includegraphics[width=0.85\textwidth]{Images/Cap4/2_Arquitectura Transformers.png}
    \captionof{figure}[Arquitectura del Modelo Transformer]{Arquitectura General del Modelo \textit{Transformer}, elaboración propia.} 
\end{center}

{\large \noindent \textbf{Componentes detallados}}\\


\noindent \textbf{\textit{Tokenización} (\textit{WordPiece})}\\

\textbf{Vocabulario}: 119,547 \textit{tokens}.\\

\textbf{\textit{Tokens} especiales}
\begin{itemize}
    \item \textbf{[CLS]}: Inicio de secuencia (ID: 101).
    \item \textbf{[SEP]}: Separador/fin de secuencia (ID: 102).
    \item \textbf{[UNK]}: \textit{Token} desconocido (ID: 100).
    \item \textbf{[PAD]}: \textit{Padding} (ID: 0).
    \item \textbf{[MASK]}: Máscara para MLM (ID: 103).
\end{itemize}

\vspace{1em}

\noindent \textbf{Ejemplo de \textit{tokenización}}

\begin{verbatim}
"Hola, ¿cómo estás?"
   ↓
Tokens: [CLS] hola , ¿ como estas ? [SEP]
Token IDs: [101, 45321, 102, 189, 12045, 36547, 103, 102]
\end{verbatim}

\noindent\textbf{\textit{Subword tokenization}}

\begin{verbatim}
"Alessandro"
   ↓
Tokens: ale ##ss ##andro
Token IDs: [12345, 67890, 23456]
\end{verbatim}

\noindent \textbf{\textit{Embedding layer}}

Tres tipos de \textit{embeddings} combinados:

\begin{enumerate}[label=\alph*)]
    \item \textbf{\textit{Token Embeddings} (vocabulario → vector)}
    \begin{itemize}
    \item Cada \textit{token} ID → vector de 384 dimensiones.
    \item Matriz de \textit{embeddings}: [119,547 × 384].
    \end{itemize}
    \item \textbf{\textit{Position Embeddings} (posición → vector)}
    \begin{itemize}
    \item Cada posición → vector de 384 dimensiones.
    \item Máximo 512 posiciones.
    \item Permite al modelo saber el orden de los tokens.
    \end{itemize}
    \item \textbf{\textit{Token Type Embeddings} (segmento → vector)}
    \begin{itemize}
    \item Distingue entre segmentos A y B.
    \item Usado en tareas de pares de oraciones.
    \end{itemize}
\end{enumerate}

\begin{verbatim}
embedding_final = token_emb + position_emb + type_emb
\end{verbatim}

\newpage
{\large \noindent\textbf{\textit{Transformer layer}}}\\
\textbf{Arquitectura de una capa:}

\begin{center}
    \includegraphics[width=0.85\textwidth]{Images/Cap4/3_TransformerLayer.png}
    \captionof{figure}[Arquitectura de una Capa Modelo Transformer]{Arquitectura de una capa del modelo \textit{Transformer}, elaboración propia.} 
\end{center}

{\large \noindent \textbf{\textit{Self-attention mechanism}}}

\noindent \textbf{Ecuación}

\[
\text{Attention}(Q, K, V) = \text{softmax}\left( \frac{QK^{T}}{\sqrt{d_k}} \right) V
\]

Donde:
\begin{itemize}
    \item $Q$ (Query): ¿Qu\'e estoy buscando?.
    \item $K$ (Key): ¿Qu\'e informaci\'on tengo?.
    \item $V$ (Value): ¿Cu\'al es esa informaci\'on?.
    \item $d_k$: Dimensi\'on de las \textit{keys} (32 por head).
\end{itemize}

\vspace{1em}

\noindent \textbf{Ejemplo visual}

\begin{verbatim}
Input: "Hola ¿cómo estás?"

Attention scores entre tokens:
          Hola   ¿   cómo  estás   ?
Hola      1.0   0.1  0.3   0.4    0.1
¿         0.1   0.9  0.5   0.2    0.8
cómo      0.2   0.4  1.0   0.7    0.3
estás     0.3   0.2  0.6   1.0    0.2
?         0.1   0.7  0.3   0.2    1.0
\end{verbatim}

\noindent\textbf{Interpretación}:
\begin{itemize}
    \item ``cómo'' atiende fuertemente a ``estás'' (0.7).
    \item ``?'' atiende a ``¿'' (0.7) — contexto de pregunta.
\end{itemize}

{\large \noindent \textbf{\textit{Pooling layer}}}

\textbf{Objetivo}: Convertir secuencia variable → vector fijo.

Estrategias de \textit{pooling}:

\begin{enumerate}[label=\alph*)]
    \item \textbf{\textit{Mean pooling} (usado en nuestro modelo)}
    \begin{verbatim}
    embedding = torch.mean(token_embeddings, dim=0)
    \end{verbatim}

    \textbf{Ventaja}: Captura información de toda la oración.

    \newpage
    \item \textbf{\textit{CLS Pooling} (alternativa)}
    \begin{verbatim}
    embedding = token_embeddings[0]  # Token [CLS]
    \end{verbatim}
    \textbf{Ventaja}: Más rápido, pero pierde contexto.
    \item \textbf{\textit{Max pooling} (alternativa)}
    \begin{verbatim}
    embedding = torch.max(token_embeddings, dim=0)
    \end{verbatim}
    \textbf{Ventaja}: Captura \textit{features} más prominentes.
\end{enumerate}

{\large \noindent \textbf{L2 \textit{Normalization}}}

\textbf{Objetivo}: Normalizar \textit{embedding} a norma unitaria.

\[
\text{embedding\_normalized}
=
\frac{\text{embedding}}
     {\lVert \text{embedding} \rVert_2}
\]

Ventajas:
\begin{itemize}
    \item \textit{Embeddings} en hiperesfera unitaria.
    \item Similitud coseno = producto punto.
    \item Comparación justa entre \textit{embeddings}.
\end{itemize}

\textbf{Ejemplo}:

\begin{verbatim}
Antes: embedding = [1.5, -2.3, 0.8, ...]
                ||embedding|| = 2.8

Después: embedding = [0.536, -0.821, 0.286, ...]
                  ||embedding|| = 1.0
\end{verbatim}

\vspace{1em}

\newpage
{\large \noindent \textbf{Generación de \textit{embeddings}}}

\noindent\textbf{Proceso completo}
\begin{verbatim}
from sentence_transformers import SentenceTransformer

# 1. Cargar modelo
model = SentenceTransformer('paraphrase-multilingual-MiniLM-L12-v2')

# 2. Preparar textos
texts = [
    "Hola, ¿cómo estás?",
    "Buenos días",
    "Necesito ayuda"
]

# 3. Generar embeddings
embeddings = model.encode(
    texts,
    batch_size=32,
    show_progress_bar=True,
    convert_to_numpy=True
)

# Output shape: (3, 384)
# Cada fila es un embedding de 384 dimensiones
\end{verbatim}

\vspace{1em}
{\large \noindent \textbf{Optimizaciones aplicadas}}

\begin{enumerate}
    \item \textbf{\textit{Batch processing}}
    \begin{itemize}
        \item Procesar múltiples textos a la vez.
        \item \textit{Batch size}: 32.
        \item Aprovecha paralelismo de CPU/GPU.
    \end{itemize}

    \item \textbf{\textit{Caching}}
    \begin{itemize}
        \item \textit{Embeddings} guardados en .npz.
        \item Carga instantánea (<1 segundo).
        \item Ahorro de ~5 segundos por \textit{startup}.
    \end{itemize}

    \newpage
    \item \textbf{\textit{Lazy loading}}
    \begin{itemize}
        \item Modelo cargado solo cuando se necesita.
        \item Ahorro de memoria si no hay requests.
    \end{itemize}
\end{enumerate}

\noindent \textbf{Métricas de Generación}

\begin{table}[H]
\centering
\renewcommand{\arraystretch}{1.6}
\begin{tabular}{|p{6cm}|p{6cm}|}
\hline
\textbf{Métrica} & \textbf{Valor} \\ \hline
Tiempo por \textit{embedding} (CPU) & \(\sim\)15ms \\ \hline
Tiempo por \textit{embedding} (GPU) & \(\sim\)3ms \\ \hline
\textit{Batch size} óptimo (CPU) & 32 textos \\ \hline
\textit{Batch size} óptimo (GPU) & 128 textos \\ \hline
Memoria por \textit{embedding} & 1.5 KB (384 \textit{floats}) \\ \hline
Memoria modelo en RAM & \(\sim\)420 MB \\ \hline
\end{tabular}
\caption[Métricas de rendimiento]{Métricas de rendimiento del sistema, elaboración propia.}
\end{table}


{\large \noindent \textbf{Cálculo de similitud coseno}}

\textbf{Definición}

Similitud coseno mide el ángulo entre dos vectores:

\[
\cos(\theta)
=
\frac{A \cdot B}
     {\lVert A\rVert \times \lVert B\rVert}
\]

Donde:
\begin{itemize}
    \item $A \cdot B$: Producto punto.
    \item $\lVert A \rVert$: Norma (magnitud) del vector $A$.
    \item $\theta$: Ángulo entre vectores.
\end{itemize}

Rango: [-1, 1]
\begin{itemize}
    \item 1.0: Vectores idénticos (ángulo 0°).
    \item 0.0: Vectores ortogonales (ángulo 90°).
    \item -1.0: Vectores opuestos (ángulo 180°).
\end{itemize}

\vspace{1em}
{\large \noindent \textbf{Simplificación con \textit{L2 Normalization}}}

Si ||A|| = 1 y ||B|| = 1 (L2 normalized):

\[
\cos(\theta) = A \cdot B
\]

El producto punto es la similitud coseno.\\

\textbf{Ventaja}: Cálculo mucho más rápido\\

{\large \noindent \textbf{Implementación}}

\begin{verbatim}
from sklearn.metrics.pairwise import cosine_similarity
import numpy as np

# Embeddings normalizados (||emb|| = 1.0)
query_embedding = np.array([0.5, -0.3, 0.8, ...])  # 384-dim
phrase_embeddings = np.array([
    [0.6, -0.2, 0.7, ...],  # Frase 1
    [0.1, 0.9, -0.4, ...],  # Frase 2
    [0.5, -0.3, 0.75, ...]  # Frase 3
])

# Calcular similitud con todas las frases
similarities = cosine_similarity(
    [query_embedding],
    phrase_embeddings
)[0]

# Output: [0.95, 0.62, 0.98]
\end{verbatim}

\vspace{1em}

\noindent \textbf{Ejemplo visual}

\begin{verbatim}
Query: "Hola"
Embedding: [0.5, 0.5, 0.707]  (simplificado a 3D)

Frase 1: "Buenos días"
Embedding: [0.6, 0.4, 0.693]

Similitud = (0.5×0.6) + (0.5×0.4) + (0.707×0.693)
        = 0.3 + 0.2 + 0.49
        = 0.99  ← Muy similar!

\newpage
Frase 2: "Ayuda"
Embedding: [-0.3, 0.8, 0.524]

Similitud = (0.5×-0.3) + (0.5×0.8) + (0.707×0.524)
        = -0.15 + 0.4 + 0.37
        = 0.62  ← Menos similar
\end{verbatim}

{\large \noindent \textbf{Optimización vectorizada}}

\noindent En lugar de \textit{loops}:

\begin{verbatim}
# LENTO (loop)
for phrase_emb in phrase_embeddings:
    sim = cosine_similarity([query_emb], [phrase_emb])[0][0]
\end{verbatim}

\noindent Usar operaciones matriciales:

\begin{verbatim}
# RÁPIDO (vectorizado)
similarities = cosine_similarity([query_emb], phrase_embeddings)[0]
\end{verbatim}

\textbf{Speedup}: ~100x más rápido.\\

{\large \noindent \textbf{\textit{Re-ranking} en dos fases}}\\

\noindent \textbf{Motivación}

\textbf{Problema}: Búsqueda exhaustiva es costosa.
\begin{itemize}
    \item Dataset: 43 frases en 3 grupos.
    \item Cálculo directo: 43 comparaciones por \textit{query}.
    \item Escalabilidad: O(N) con N = total de frases.
\end{itemize}

\vspace{0.7em}

\textbf{Solución}: Búsqueda jerárquica en dos fases.
\begin{itemize}
    \item Fase 1: Encontrar grupos candidatos (O(3) comparaciones).
    \item Fase 2: Buscar en grupos candidatos (O(N\_candidatos)).
    \item Escalabilidad: O(K + N\_k) donde K = núm. grupos, N\_k = frases en top grupos.
\end{itemize}

\newpage
{\large \noindent \textbf{Fase 1: Búsqueda por centroides}}

\noindent \textbf{Centroide}: Vector promedio de un grupo.

\begin{verbatim}
Grupo A: ["Ayuda", "Socorro", "Urgente", ...]
   ↓
Embeddings:
   [0.2, -0.3, 0.5, ...]
   [0.3, -0.2, 0.6, ...]
   [0.1, -0.4, 0.4, ...]
   ↓
Centroide_A = mean(embeddings_A)
            = [0.2, -0.3, 0.5, ...]
\end{verbatim}

\noindent \textbf{Algoritmo}:

\begin{verbatim}
def find_best_groups(query_embedding, centroids, top_k=3):
    """
    Encuentra top-K grupos más probables.
    """
    scores = []
    for grupo, centroid in centroids.items():
        sim = cosine_similarity([query_embedding], [centroid])[0][0]
        scores.append((grupo, sim))

    scores.sort(key=lambda x: x[1], reverse=True)
    return scores[:top_k]
\end{verbatim}

\noindent \textbf{Ejemplo}:

\begin{verbatim}
Query: "necesito ayuda urgente"
Embedding: [0.25, -0.35, 0.52, ...]

Similitud con centroides:
   Centroide A (Emergencias): 0.92  ← TOP 1
   Centroide C (Comunicación): 0.78  ← TOP 2
   Centroide B (Saludos):     0.65  ← TOP 3

Grupos candidatos: [A, C, B].
\end{verbatim}

\newpage
{\large \noindent\textbf{Fase 2: \textit{Re-ranking} fino}}

\noindent Búsqueda en grupos candidatos:

\begin{verbatim}
def rerank_in_groups(query_embedding, candidate_groups, embeddings, frases):
    """
    Busca la mejor frase en grupos candidatos.
    """
    best_sim = -1
    best_grupo = None
    best_frase = None

    for grupo in candidate_groups:
        # Embeddings del grupo
        grupo_embeddings = embeddings[grupo]
        grupo_frases = frases[grupo]

        # Similitud con todas las frases del grupo
        sims = cosine_similarity([query_embedding], grupo_embeddings)[0]

        # Aplicar BOOSTS por longitud
        for i, frase in enumerate(grupo_frases):
            num_words = len(frase.split())
            if num_words >= 3:
                sims[i] += 0.15  # +15% para frases largas
            elif num_words == 2:
                sims[i] += 0.08  # +8% para frases medias

        # Normalizar [0, 1]
        sims = np.clip(sims, 0.0, 1.0)

        # Mejor match en este grupo
        max_idx = np.argmax(sims)
        sim = sims[max_idx]

        # Actualizar si es mejor
        if sim > best_sim:
            best_sim = sim
            best_grupo = grupo
            best_frase = grupo_frases[max_idx]

    return best_grupo, best_frase, best_sim
\end{verbatim}

\newpage
{\large \noindent \textbf{\textit{Boosts} aplicados}}

\begin{enumerate}
    \item \textbf{\textit{Boost} por longitud de frase} \\
    Justificación: Frases más largas más específicas.

    \begin{table}[H]
    \centering
    \renewcommand{\arraystretch}{1.6}
    \begin{tabular}{|p{4cm}|p{4cm}|p{5cm}|}
    \hline
    \textbf{Palabras} & \textbf{Boost} & \textbf{Ejemplo} \\ \hline
    1 palabra & 0\% & ``Ayuda'' \\ \hline
    2 palabras & +8\% & ``Necesito ayuda'' \\ \hline
    3+ palabras & +15\% & ``Necesito ayuda urgente'' \\ \hline
    \end{tabular}
    \caption[Boost por número de palabras]{Boost por número de palabras, elaboración propia.}
    \end{table}

    \item \textbf{Bonus al grupo más probable}
    \begin{verbatim}
    Grupo #1 (top centroid match): +5%
    Otros grupos: 0%
    \end{verbatim}

    \item \textbf{Penalización por diferencia de longitud}
    \begin{verbatim}
    Si |len(query_word) - len(phrase_word)| > 1:
        penalty = 0.05 × diferencia
    \end{verbatim}

    Ejemplo:
    \begin{verbatim}
    Query: "Ivan" (4 letras)
    Frase: "Sí" (2 letras)
    Diferencia: 2 letras
    Penalty: 0.05 × 2 = 0.10 (-10%)
    \end{verbatim}
\end{enumerate}

\newpage
{\large \noindent \textbf{Ejemplo completo}}

\begin{verbatim}
Query: "necesito ayuda urgente"

Fase 1: Búsqueda por centroides
   Centroide A: 0.92 ← TOP
   Centroide C: 0.78
   Centroide B: 0.65

Candidatos: [A, C, B]

Fase 2: Re-ranking en Grupo A
   "Ayuda, por favor":     0.85 + 0.08 (2 palabras) = 0.93
   "Necesito un médico":   0.82 + 0.15 (3 palabras) = 0.97 ← BEST
   "Socorro":              0.80 + 0.00 (1 palabra)  = 0.80
   ...

Fase 2: Re-ranking en Grupo C
   "Gracias":              0.65
   "Entiendo":             0.62
   ...

Mejor match global: "Necesito un médico" (Grupo A, sim: 0.97)
\end{verbatim}

\vspace{1em}

{\large \noindent \textbf{Ventajas del \textit{re-ranking}}}

\begin{itemize}
    \item \textit{Performance}: \(\sim 3\times\) más rápido que búsqueda exhaustiva.
    \item Escalabilidad: \(O(K + N_k)\) vs \(O(N)\).
    \item Precisión: \textit{Boosts} mejoran detección.
    \item Flexibilidad: Fácil agregar nuevos grupos.
\end{itemize}

{\large \noindent \textbf{Métricas}}

\begin{table}[H]
\centering
\renewcommand{\arraystretch}{1.6}
\begin{tabular}{|p{6cm}|p{7cm}|}
\hline
\textbf{Métrica} & \textbf{Valor} \\ \hline
Comparaciones (exhaustiva) & 43 \\ \hline
Comparaciones (\textit{re-ranking}) & 3 + $\sim$20 = 23 \\ \hline
\textit{Speedup} & $\sim$1.9x \\ \hline
Precisión & 92\% (vs 90\% exhaustiva) \\ \hline
\end{tabular}
\caption[Métricas de búsqueda]{Métricas de búsqueda, elaboración propia.}
\end{table}
% ======================================================

\subsection{Etapa 3: API REST y arquitectura}

{\large \noindent \textbf{Diseño de \textit{Endpoints}}}\\

\noindent \textbf{Principios de diseño}

\begin{itemize}
    \item \textbf{RESTful API \textit{Design}}
    \begin{itemize}
        \item Uso de HTTP methods (GET, POST).
        \item URLs semánticas y descriptivas.
        \item \textit{Status codes} apropiados.
    \end{itemize}

    \item \textbf{\textit{Stateless}}
    \begin{itemize}
        \item Sin sesiones en servidor.
        \item Cada \textit{request} independiente.
        \item Facilita escalabilidad horizontal.
    \end{itemize}

    \item \textbf{JSON como formato de intercambio}
    \begin{itemize}
        \item Estándar de la industria.
        \item Fácil \textit{parsing} en cualquier lenguaje.
        \item Soportado nativamente por FastAPI.
    \end{itemize}

    \item \textbf{Versionado implícito}
    \begin{itemize}
        \item Preparado para \texttt{/v2/buscar}.
        \item Actualmente sin versión (v1 implícito).
    \end{itemize}
\end{itemize}

\newpage
{\large \noindent \textbf{\textit{Endpoints} implementados}}\\

{\large \noindent \textbf{1. POST /buscar}}\\

\textit{Endpoint} principal de búsqueda semántica.

\textbf{Request:}
\begin{verbatim}
Content-Type: application/json
Body:
{
    "texto": "hola"
}
\end{verbatim}

\noindent\textbf{Response (200 OK):}

\begin{verbatim}
{
    "query": "hola",
    "grupo": "B",
    "frase_similar": "Hola",
    "similitud": 0.95,
    "deletreo_activado": false,
    "deletreo": null,
    "total_caracteres": null,
    "nombre_detectado": null,
    "nombre_extraido": null,
    "nombre_deletreado": null,
    "total_caracteres_nombre": null
}
\end{verbatim}

\textbf{Errores:}
\begin{itemize}
    \item 400: Texto vacío o inválido.
    \item 503: Servicio no disponible.
    \item 500: Error interno.
\end{itemize}

\newpage
{\large \noindent \textbf{2. GET /grupos}}\\

Listado de grupos disponibles y sus frases.

\textbf{Response (200 OK):}
\begin{verbatim}
{
    "total_grupos": 3,
    "grupos": {
        "A": {
            "nombre": "Emergencias",
            "descripcion": "Frases de emergencia",
            "total_frases": 13,
            "ejemplos": ["Ayuda", "Socorro", "Urgente"]
        },
        "B": { ... },
        "C": { ... }
    }
}
\end{verbatim}

{\large \noindent \textbf{3. GET /grupos/\{grupo\}}}\\

Frases de un grupo específico.\\

\textbf{\textit{Path parameter}:}
\begin{itemize}
    \item \texttt{grupo}: ``A'', ``B'' o ``C''
\end{itemize}

\textbf{Response (200 OK):}
\begin{verbatim}
{
    "grupo": "A",
    "nombre": "Emergencias",
    "total_frases": 13,
    "frases": [
        "Ayuda, por favor",
        "Llama a la policía",
        ...
    ]
}
\end{verbatim}

\textbf{Errores:}
\begin{itemize}
    \item 404: Grupo no encontrado.
\end{itemize}

\newpage
{\large \noindent \textbf{4. POST /deletreo}}\\

Deletreo manual de cualquier texto.

\textbf{Request:}
\begin{verbatim}
{
    "texto": "Hola Mundo"
}
\end{verbatim}

\textbf{Response (200 OK):}
\begin{verbatim}
{
    "texto_original": "Hola Mundo",
    "texto_normalizado": "HolaMundo",
    "deletreo": ["H","O","L","A","M","U","N","D","O"],
    "total_caracteres": 9
}
\end{verbatim}

{\large \noindent \textbf{5. GET /health}}\\

Health check del servicio.

\textbf{Response (200 OK):}
\begin{verbatim}
{
    "status": "healthy",
    "version": "2.1.0",
    "matcher_initialized": true,
    "total_frases": 43,
    "uptime_seconds": 3600
}
\end{verbatim}

{\large \noindent \textbf{6. GET /docs}}
\begin{itemize}
    \item Documentación interactiva (Swagger UI). Generada automáticamente por FastAPI.
    \item Permite \textit{testing} interactivo de \textit{endpoints}.
\end{itemize}

{\large \noindent \textbf{7. GET /redoc}}
\begin{itemize}
    \item Documentación alternativa (ReDoc).
    \item Formato más limpio para lectura.
\end{itemize}

\newpage
{\large \noindent \textbf{Validación de datos con Pydantic}}

\noindent \textbf{Modelos de validación}

\begin{lstlisting}[language=Python]
from pydantic import BaseModel, Field
from typing import List

class QueryRequest(BaseModel):
    """Modelo de validación para requests de búsqueda."""

    texto: str = Field(
        ...,
        min_length=1,
        max_length=500,
        description="Texto a buscar",
        examples=["hola", "necesito ayuda"]
    )

    class Config:
        json_schema_extra = {
            "example": {
                "texto": "hola"
            }
        }
\end{lstlisting}

Validaciones automáticas:
\begin{itemize}
    \item Texto es requerido (... = required).
    \item Texto no puede estar vacío (min\_length=1).
    \item Texto máximo 500 caracteres (max\_length=500).
    \item Tipo \textit{string} validado automáticamente.
\end{itemize}

\vspace{0.7em}

\noindent \textbf{Ejemplos de validación}

\textit{Request} válido:

\begin{lstlisting}
POST /buscar
{
  "texto": "hola"
}
→ ✓ Pasa validación
\end{lstlisting}

\newpage
\textit{Request} inválido (texto vacío):

\begin{lstlisting}
POST /buscar
{
  "texto": ""
}
→ Error 422 (Unprocessable Entity)
{
  "detail": [
    {
      "loc": ["body", "texto"],
      "msg": "ensure this value has at least 1 characters",
      "type": "value_error.any_str.min_length"
    }
  ]
}
\end{lstlisting}

\textit{Request} inválido (tipo incorrecto):

\begin{lstlisting}
POST /buscar
{
  "texto": 123
}
→ Error 422
{
  "detail": [
    {
      "loc": ["body", "texto"],
      "msg": "str type expected",
      "type": "type_error.str"
    }
  ]
}
\end{lstlisting}

\newpage
Validación de \textit{responses}

\begin{lstlisting}[language=Python]
class QueryResponse(BaseModel):
    """Modelo de validación para responses."""

    query: str
    grupo: str | None
    frase_similar: str
    similitud: float = Field(ge=0.0, le=1.0)  # [0.0, 1.0]
    deletreo_activado: bool
    deletreo: List[str] | None = None
    total_caracteres: int | None = None
    nombre_detectado: bool | None = None
    nombre_extraido: str | None = None
    nombre_deletreado: List[str] | None = None
    total_caracteres_nombre: int | None = None
\end{lstlisting}

\vspace{1em}
Validaciones:
\begin{itemize}
    \item Similitud entre 0.0 y 1.0 (ge=0.0, le=1.0).
    \item Tipos verificados automáticamente.
    \item Campos opcionales con None.
\end{itemize}

{\large \noindent \textbf{Manejo de errores}}\\

\textbf{Estrategia de manejo}:

\begin{enumerate}
    \item HTTPException para errores controlados.
    \item \textit{Exception handler} para errores inesperados.
    \item \textit{Logging} de todos los errores.
    \item Responses consistentes.
\end{enumerate}

\newpage
\textbf{Códigos por error}

\begin{table}[H]
\centering
\renewcommand{\arraystretch}{1.6}
\begin{tabular}{|p{2cm}|p{4cm}|p{8cm}|}
\hline
\textbf{Código} & \textbf{Nombre} & \textbf{Uso} \\ \hline
200 & OK & \textit{Request} exitoso \\ \hline
400 & \textit{Bad Request} & Texto vacío, parámetros inválidos \\ \hline
404 & \textit{Not Found} & Grupo no encontrado \\ \hline
422 & \textit{Unprocessable} & Validación \textit{Pydantic} fallida \\ \hline
500 & \textit{Internal Error} & Error no manejado \\ \hline
503 & \textit{Service Unavailable} & \textit{Matcher} no inicializado \\ \hline
\end{tabular}
\caption[Códigos HTTP]{Códigos HTTP utilizados, elaboración propia.}
\end{table}

{\large \noindent \textbf{Implementación}}
\begin{lstlisting}[language=Python]
from fastapi import HTTPException
from fastapi.responses import JSONResponse

@app.post("/buscar")
async def buscar(request: QueryRequest):
    if matcher is None:
        raise HTTPException(
            status_code=503,
            detail="Servicio no disponible: matcher no inicializado")

    if not request.texto.strip():
        raise HTTPException(
            status_code=400,
            detail="El texto no puede estar vacío"        )

    try:
        resultado = matcher.search_similar_phrase(request.texto)
        return resultado
    except Exception as e:
        logger.error(f"Error en búsqueda: {e}")
        raise HTTPException(status_code=500,
            detail="Error interno del servidor")
\end{lstlisting}

\newpage
{\large \noindent \textbf{\textit{Middleware} de error \textit{handling}}}

\begin{lstlisting}[language=Python]
@app.exception_handler(HTTPException)
async def http_exception_handler(request, exc):
    """Maneja excepciones HTTP con formato consistente."""
    return JSONResponse(
        status_code=exc.status_code,
        content={
            "error": exc.detail,
            "status_code": exc.status_code,
            "path": str(request.url),
            "timestamp": datetime.now().isoformat()
        }
    )

@app.exception_handler(Exception)
async def general_exception_handler(request, exc):
    """Maneja excepciones no capturadas."""
    logger.error(f"Error no manejado: {exc}", exc_info=True)
    return JSONResponse(
        status_code=500,
        content={
            "error": "Error interno del servidor",
            "status_code": 500,
            "path": str(request.url),
            "timestamp": datetime.now().isoformat()
        }
    )
\end{lstlisting}

\vspace{1em}

{\large \noindent \textbf{\textit{Logging} y monitoreo}}

\textbf{Configuración de \textit{logging}}

\begin{lstlisting}[language=Python]
import logging
from logging.handlers import RotatingFileHandler

# Configuración
logging.basicConfig(
    level=logging.INFO,
    format='%(asctime)s - %(name)s - %(levelname)s - %(message)s',
    handlers=[
        # Console handler
        logging.StreamHandler(),
        # File handler con rotación
        RotatingFileHandler(
            'logs/app.log',
            maxBytes=10*1024*1024,  # 10MB
            backupCount=5
        )
    ]
)

logger = logging.getLogger(__name__)
\end{lstlisting}

\vspace{1em}
\textbf{Eventos loggeados}

\begin{lstlisting}
[STARTUP]
2024-11-24 12:00:00 - app.main - INFO - Inicializando aplicación...
2024-11-24 12:00:00 - app.matcher - INFO - Cargando modelo...
2024-11-24 12:00:01 - app.matcher - INFO - Cache cargado exitosamente
2024-11-24 12:00:01 - app.main - INFO - Aplicación lista

[REQUEST]
2024-11-24 12:00:15 - app.main - INFO - Búsqueda para: hola
2024-11-24 12:00:15 - app.matcher - DEBUG - Similitud calculada: 0.95
2024-11-24 12:00:15 - app.main - INFO - Resultado: B - 0.95

[ERROR]
2024-11-24 12:00:30 - app.main - ERROR - Error en búsqueda: División por cero
2024-11-24 12:00:30 - app.main - ERROR - Traceback: ...
\end{lstlisting}

\textbf{Métricas monitoreadas}

\begin{itemize}
    \item \textit{Requests} por segundo.
    \item Latencia promedio.
    \item Tasa de errores (4xx, 5xx).
    \item Uso de memoria.
    \item Tiempo de inicialización.
\end{itemize}

\newpage
{\large \noindent \textbf{Documentación automática (OpenAPI / Swagger)}}

\textbf{Generación automática}\\
FastAPI genera documentación OpenAPI 3.0 basada en:

\begin{itemize}
    \item \textit{Type hints} de Python.
    \item Modelos Pydantic.
    \item \textit{Docstrings}.
    \item \textit{Field descriptions}.
\end{itemize}

Acceso:
\begin{itemize}
    \item \texttt{http://localhost:8000/docs}.
    \item \texttt{http://localhost:8000/redoc}.
    \item \texttt{http://localhost:8000/openapi.json}.
\end{itemize}

\vspace{1em}
\textbf{Ejemplo de documentación generada}

\begin{lstlisting}
openapi: 3.0.2
info:
  title: API de Búsqueda Semántica
  description: |
    API REST para búsqueda semántica de frases en español usando PLN.
  version: 2.1.0

paths:
  /buscar:
    post:
      summary: Buscar frase similar
      description: |
        Busca la frase más similar al texto proporcionado usando
        embeddings semánticos y similitud coseno.
      operationId: buscar_frase_similar_buscar_post
      requestBody:
        required: true
        content:
          application/json:
            schema:
              $ref: '#/components/schemas/QueryRequest'
            example:
              texto: "hola"
      responses:
        '200':
          description: Búsqueda exitosa
          content:
            application/json:
              schema:
                $ref: '#/components/schemas/QueryResponse'
        '400': description: Texto vacío o inválido
        '503': description: Servicio no disponible

components:
  schemas:
    QueryRequest:
      type: object
      required:
        - texto
      properties:
        texto:
          type: string
          minLength: 1
          maxLength: 500
          description: Texto a buscar

    QueryResponse:
      type: object
      required:
        - query
        - frase_similar
        - similitud
        - deletreo_activado
      properties:
        query:
          type: string
        grupo:
          type: string
          nullable: true
        frase_similar:
          type: string
        similitud:
          type: number
          minimum: 0.0
          maximum: 1.0
        deletreo_activado:
          type: boolean
\end{lstlisting}

\newpage
\textbf{Ventajas de documentación generada}

\begin{itemize}
    \item Siempre actualizada.
    \item Interactiva.
    \item Compatible con herramientas OpenAPI.
    \item Generación automática de clientes.
\end{itemize}

%====================================================
%====================================================
%====================================================
%====================================================

\section{Pruebas del Modulo del Procesamiento de Lenguaje Natural (PLN)}

\vspace{1em}

\subsection{Metodología de \textit{testing}}

{\large \noindent \textbf{Piramide de \textit{Testing}}}

La estrategia de \textit{testing} del proyecto sigue el modelo de la Pirámide de \textit{testing} propuesta por Mike Cohn \cite{refpirtes}, que establece una distribución óptima de los diferentes tipos de pruebas.\\

\begin{center}
    \includegraphics[width=0.45\textwidth]{Images/Cap5/1_PiramideTesting.png}
    \captionof{figure}[Piramide de Testing]{Piramide de \textit{Testing}, obtenido de \cite{refpirtes}.} 
\end{center}

\noindent \textbf{Distribución implementada}

\begin{table}[H]
\centering
\renewcommand{\arraystretch}{1.6}
\begin{tabular}{|p{4cm}|p{2.5cm}|p{2.5cm}|p{4cm}|}
\hline
\textbf{Tipo de \textit{test}} & \textbf{Cantidad} & \textbf{\% Total} & \textbf{Tiempo de ejecución} \\ \hline
\textit{Tests} unitarios & 37 & 51\% & $<$5 segundos \\ \hline
\textit{Tests} integración & 10 & 14\% & $\sim$8 segundos \\ \hline
\textit{Tests} E2E & 26 & 35\% & $\sim$15 segundos \\ \hline
Total (núcleo) & 73 & 100\% & $\sim$28 segundos \\ \hline
\textit{Test} adicionales & 95 & - & Variable \\ \hline
Total general & 168 & - & - \\ \hline
\end{tabular}
\caption[Resumen de tests]{Resumen de \textit{tests}, elaboración propia.}
\end{table}

\textbf{Nota:} Los \textit{tests} de \textit{performance}, \textit{quality} y \textit{stress} \textbf{NO} se incluyen en la distribución de la pirámide ya que son \textit{tests} especializados que se ejecutan de forma independiente.\\

\noindent \textbf{Justificación del 35\% de tests E2E}

El porcentaje elevado de tests E2E (superior al estándar industrial de 10--15\%) se justifica por los requisitos específicos de sistemas PLN:

\begin{enumerate}
    \item Validación exhaustiva de robustez lingüística.
    \item \textit{Tests} de tolerancia a errores ortográficos y variaciones naturales.
    \item Validación de casos \textit{edge} específicos (nombres propios, \textit{leet speak}).
    \item Degradación gradual ante \textit{inputs} de baja calidad.
\end{enumerate}

{\large \noindent \textbf{Stack tecnológico}}

\begin{table}[H]
\centering
\renewcommand{\arraystretch}{1.6}
\begin{tabular}{|p{4cm}|p{4cm}|p{3cm}|}
\hline
\textbf{Componente} & \textbf{Tecnología} & \textbf{Versión} \\ \hline
\textit{Framework} de \textit{testing} & \textit{pytest} & 7.4+ \\ \hline
\textit{API testing} & FastAPI TestClient & 0.104+ \\ \hline
Cobertura de código & pytest-cov & 4.1+ \\ \hline
\textit{Load testing} & Locust & 2.15+ \\ \hline
\textit{Performance benchmarks} & pytest-benchmark & 4.0+ \\ \hline
\textit{Resource monitoring} & psutil & 5.9+ \\ \hline
\end{tabular}
\caption[Tecnologías de testing]{Tecnologías usadas para \textit{testing}, elaboración propia.}
\end{table}

\newpage
{\large \noindent \textbf{Estructura del \textit{testing}}}

El sistema de \textit{testing} está organizado en directorios especializados:

\begin{verbatim}
tests/
|-- unit/                          # 37 tests unitarios
|   |-- test_matcher.py            # Matching semantico (25 tests)
|   `-- test_preprocess.py         # Preprocesamiento (12 tests)
|
|-- integration/                   # 10 tests de integracion
|   `-- test_api.py                # Endpoints FastAPI
|
|-- e2e/                           # 26 tests end-to-end
|   |-- test_scenarios.py          # Escenarios de usuario (8 tests)
|   |-- test_robustness.py         # Robustez linguistica (14 tests)
|   `-- test_casos_realistas.py    # Casos reales (4+ tests)
|
|-- quality/                       # Tests de calidad semantica
|   |-- test_semantic_quality.py
|   `-- test_semantic_advanced.py
|
`-- performance/                   # Tests de rendimiento
    |-- test_benchmarks.py
    |-- test_stress_concurrent.py
    `-- locustfile.py
\end{verbatim}

{\noindent \textbf{Estadísticas generales}}

\begin{itemize}
    \item Líneas de código de \textit{testing}: 2{,}749.
    \item Funciones de \textit{test}: 168.
    \item Cobertura de código: 90.8\%.
    \item Tiempo de ejecución: $\sim$ 28 segundos.
\end{itemize}

\newpage
{\large \noindent \textbf{\textit{Tests} unitarios (37 \textit{tests}, 89\% cobertura)}}

Los \textit{tests} unitarios validan componentes individuales de forma aislada, constituyendo la base de la pirámide de \textit{testing}.\\

\textbf{\textit{Tests} del \textit{Matcher} Semántico (25 tests)}

Componente central responsable de:

\begin{itemize}
    \item Generación de \textit{embeddings} con modelos \textit{transformer}.
    \item Cálculo de similitud coseno.
    \item Clasificación en grupos semánticos (A, B, C).
    \item Activación de modo deletreo.
\end{itemize}

\vspace{1em}

{\large \noindent \textbf{Casos críticos validados}}

\textbf{1. Normalización de similitud al rango [0.0, 1.0] (4 \textit{tests})}\\

Problema resuelto: El sistema original retornaba valores > 1.0 debido a errores de precisión flotante.\\

Solución implementada:

\begin{lstlisting}[language=Python]
def clip_similarity(similarity: float) -> float:
    """Normaliza similitud al rango [0.0, 1.0]."""
    if similarity > 1.0:
        return 1.0
    elif similarity < 0.0:
        return 0.0
    return similarity
\end{lstlisting}

Resultado: 100\% de \textit{tests} garantizan similitud matemáticamente correcta.\\

\textbf{2. Validación de rango en todas las \textit{queries} (6 \textit{tests})}\\

\textit{Test} parametrizado con 10 \textit{queries} diversas validando:

\begin{verbatim}
assert 0.0 <= result["similitud"] <= 1.0
\end{verbatim}

\textbf{3. Detección de patrones de nombres (7 \textit{tests})}\\

Funcionalidad especial que detecta y procesa:

\begin{itemize}
    \item "Me llamo [Nombre]"
    \item "Mi nombre es [Nombre]"
\end{itemize}

Ejemplo:

\begin{lstlisting}
Input:  "Me llamo Juan Carlos"

Output: {
    "nombre_detectado": true,
    "nombre_extraido": "Juan Carlos",
    "nombre_deletreado": ["J","U","A","N"," ","C","A","R","L","O","S"],
    "total_caracteres_nombre": 11
}
\end{lstlisting}

\textbf{4. Clasificación en grupos semánticos (3 \textit{tests})}\\

Grupos del sistema:

\begin{itemize}
    \item Grupo A: Emergencias (ayuda, socorro, urgente).
    \item Grupo B: Saludos (hola, buenos días, adios).
    \item Grupo C: Comunicación (gracias, sí, bien).
\end{itemize}

{\large \noindent \textbf{\textit{Tests} de preprocesamiento (12 \textit{tests})}}\\

Validación de normalización de texto para robustez del sistema PLN:\\

\textbf{1. Normalización de texto (7 \textit{tests})}

Transformaciones aplicadas:

\begin{itemize}
    \item Conversión a minúsculas.
    \item Eliminación de acentos (\textit{Unicode normalization}).
    \item Eliminación de caracteres especiales.
    \item Normalización de espacios múltiples.
\end{itemize}

Ejemplo:

\begin{verbatim}
"¡HOLA, ¿Cómo  estás?!" → "hola como estas"
\end{verbatim}

\newpage
\textbf{Resultados}

\begin{table}[H]
\centering
\renewcommand{\arraystretch}{1.6}
\begin{tabular}{|p{4cm}|p{3.5cm}|p{3.5cm}|}
\hline
\textbf{Métrica} & \textbf{Valor} & \textbf{Objetivo} \\ \hline
Total \textit{tests} & 37 & 30+ \\ \hline
\textit{Tests} exitosos & 37 (100\%) & 100\% \\ \hline
Cobertura de código & 89\% & $>$80\% \\ \hline
Tiempo de ejecución & $<$5 segundos & $<$10 segundos \\ \hline
Casos límite cubiertos & 15+ & 10+ \\ \hline
\end{tabular}
\caption[Métricas de tests]{Métricas principales de los \textit{tests}, elaboración propia.}
\end{table}


\textbf{2. Deletreo con caracteres especiales (5 \textit{tests})}

Mapeo implementado:

\begin{verbatim}
@ → "arroba", . → "punto", ! → "exclamación", (espacio) → "espacio"
\end{verbatim}

{\large \noindent \textbf{\textit{Tests} de integración (10 \textit{tests}, 6 \textit{endpoints})}}\\

Validación de la interacción entre componentes y funcionamiento de la API REST.\\

\textbf{\textit{Endpoints} testeados}

\begin{table}[H]
\centering
\renewcommand{\arraystretch}{1.6}
\begin{tabular}{|p{5cm}|p{2.5cm}|p{3.5cm}|}
\hline
\textbf{\textit{Endpoint}} & \textbf{\textit{Tests}} & \textbf{\textit{Status Codes} Validados} \\ \hline
POST /buscar & 8 & 200, 400, 422 \\ \hline
GET /grupos & 1 & 200 \\ \hline
GET /grupos/\{grupo\} & 2 & 200, 404 \\ \hline
POST /deletreo & 3 & 200, 400 \\ \hline
GET /health & 1 & 200 \\ \hline
GET / & 1 & 200 \\ \hline
\end{tabular}
\caption[Tests por endpoint]{Resumen de \textit{tests} por endpoint, elaboración propia.}
\end{table}

\newpage
\noindent \textbf{Casos críticos del \textit{Endpoint} principal (POST /buscar)}

\begin{enumerate}
    \item \textit{Query} válida retorna resultado correcto.
    \item \textit{Queries} inválidas retornan errores apropiados (400, 422).
    \item \textit{Match} exacto tiene similitud muy alta (>0.95).
    \item \textit{Queries} sin \textit{match} activan deletreo automático.
    \item Emergencias se clasifican correctamente en Grupo A.
    \item Caso \textit{edge} ``Ivan'': detecta baja similitud y activa deletreo.
\end{enumerate}

\vspace{1em}

\textbf{Problema histórico del caso \texttt{``Ivan''}}:

\begin{itemize}
    \item Antes: Se matcheaba incorrectamente con ``Sí'' (Grupo C).
    \item Ahora: Detecta baja similitud (<0.3) y activa deletreo.
    \item Resultado: Usuario recibe I-V-A-N.
\end{itemize}

\textbf{Resultados}

\begin{table}[H]
\centering
\renewcommand{\arraystretch}{1.6}
\begin{tabular}{|p{6cm}|p{6cm}|}
\hline
\textbf{Métrica} & \textbf{Valor} \\ \hline
Total \textit{tests} & 10 \\ \hline
\textit{Tests} exitosos & 10 (100\%) \\ \hline
\textit{Endpoints} cubiertos & 6 \\ \hline
Tiempo de ejecución & $\sim$8s \\ \hline
\end{tabular}
\caption[Métricas adicionales]{Métricas adicionales de pruebas, elaboración propia.}
\end{table}

\newpage
{\large \noindent \textbf{\textit{Tests End-To-End}  y robustez lingüística (26 \textit{tests})}}\\
Los \textit{tests} E2E validan robustez ante variaciones naturales del lenguaje mediante ``\textit{Perturbation Testing}'', metodología específica para sistemas PLN \cite{refe2e}.\\

\textbf{Escenarios completos de usuario (8 escenarios)}

\begin{table}[H]
\centering
\renewcommand{\arraystretch}{1.6}
\begin{tabular}{|p{5cm}|p{2.5cm}|p{4cm}|}
\hline
\textbf{Escenario} & \textbf{\textit{Tests}} & \textbf{Validación principal} \\ \hline
Emergencias & 2 & Clasificación Grupo A \\ \hline
Saludos formales & 3 & Tolerancia a variaciones \\ \hline
Casos \textit{edge} & 3 & Manejo robusto (\texttt{``Ivan''}) \\ \hline
Conversación completa & 3 & Consistencia temporal \\ \hline
Múltiples usuarios & 2 & Estabilidad bajo carga \\ \hline
Salud del sistema & 2 & \textit{Health checks} \\ \hline
\end{tabular}
\caption[Escenarios de prueba]{Escenarios evaluados, cantidad de \textit{tests} y validación principal, elaboración propia.}
\end{table}

\textbf{\textit{Tests} de robustez lingüística (14 \textit{tests}) - crítico PLN}\\
Metodología de ``\textit{Perturbation Testing}'' con 4 estrategias

\begin{enumerate}
    \item \textit{Character-level perturbations} (errores de tipeo).
    \item \textit{Input fuzzing} (ruido en el input).
    \item \textit{Semantic equivalence testing} (sinónimos).
    \item \textit{Load testing} (estrés).
\end{enumerate}

\newpage
\textbf{Tipos de perturbaciones testeadas (50+ casos)}

\begin{table}[H]
\centering
\renewcommand{\arraystretch}{1.6}
\begin{tabular}{|p{5.5cm}|p{6cm}|}
\hline
\textbf{Tipo de perturbación} & \textbf{Ejemplos} \\ \hline
Errores al inicio & \texttt{``hola''} → \texttt{``hila''} \\ \hline
Errores en medio & \texttt{``ayuda''} → \texttt{``auuda''} \\ \hline
Errores al final & \texttt{``hola''} → \texttt{``holq''} \\ \hline
Carácter faltante & \texttt{``hola''} → \texttt{``hla''} \\ \hline
Carácter extra & \texttt{``hola''} → \texttt{``hoola''} \\ \hline
Caracteres intercambiados & \texttt{``hola''} → \texttt{``hloa''} \\ \hline
Múltiples errores & \texttt{``buenos días''} → \texttt{``buens dias''} \\ \hline
Espacios extra & \texttt{``hola\_''} \texttt{``\_ ayuda''} \\ \hline
Puntuación extra & \texttt{``ayuda!!''}, \texttt{``¡ayuda!!''} \\ \hline
Mayúsculas aleatorias & \texttt{``HoLa''}, \texttt{``AyUdA''} \\ \hline
Variaciones de acentos & \texttt{``médico''} vs \texttt{``medico''} \\ \hline
\end{tabular}
\caption[Tipos de perturbación]{Tipos de perturbaciones lingüísticas y ejemplos representativos, elaboración propia.}
\end{table}

\textbf{Degradación gradual por severidad}

\begin{table}[H]
\centering
\renewcommand{\arraystretch}{1.6}
\begin{tabular}{|p{3cm}|p{4cm}|p{5cm}|}
\hline
\textbf{Nivel} & \textbf{Ejemplo} & \textbf{Comportamiento esperado} \\ \hline
Leve & \texttt{``hila''} & Clasifica correctamente \\ \hline
Medio & \texttt{``hla''} & Clasifica o activa deletreo \\ \hline
Grave & \texttt{``hkka''} & Activa deletreo \\ \hline
\end{tabular}
\caption[Niveles de error]{Clasificación de severidad de errores y el comportamiento esperado, elaboración propia.}
\end{table}

\newpage
\textbf{Casos realistas con \textit{Leet Speak} (4+ \textit{tests})}\\

Normalización de \textit{Leet Speak} en deletreo

\begin{table}[H]
\centering
\renewcommand{\arraystretch}{1.6}
\begin{tabular}{|p{2cm}|p{3cm}|p{6cm}|}
\hline
\textbf{\textit{Leet Speak}} & \textbf{Normalización} & \textbf{Ejemplo} \\ \hline
4 & A & \texttt{``M4ri@''} → \texttt{``MARIA''} \\ \hline
3 & E & \texttt{``P3dro''} → \texttt{``PEDRO''} \\ \hline
1 & I & \texttt{``T1po''} → \texttt{``TIPO''} \\ \hline
0 & O & \texttt{``H0la''} → \texttt{``HOLA''} \\ \hline
@ & A & \texttt{``C@rlos''} → \texttt{``CARLOS''} \\ \hline
\$ & S & \texttt{``Ca\$a''} → \texttt{``CASA''} \\ \hline
\end{tabular}
\caption[Normalización de Leet Speak]{Equivalencias para normalización de texto en \textit{Leet Speak}, elaboración propia.}
\end{table}

\textbf{Resultados}

\begin{table}[H]
\centering
\renewcommand{\arraystretch}{1.6}
\begin{tabular}{|p{4cm}|p{3cm}|p{3cm}|}
\hline
\textbf{Métrica} & \textbf{Valor} & \textbf{Objetivo} \\ \hline

Total tests E2E & 26 & 30+ \\ \hline
Tests exitosos & 26 (100\%) & >95\% \\ \hline
Escenarios completos & 8 & 6+ \\ \hline
Tipos de perturbaciones & 11 & 8+ \\ \hline
Casos de perturbación testeados & 50+ & 30+ \\ \hline
\textit{Success rate} con perturbaciones & >95\% & >90\% \\ \hline
Tiempo de ejecución & $\sim$15s & <30s \\ \hline

\end{tabular}
\caption[Métricas generales]{Resultados generales de pruebas E2E, elaboración propia.}
\end{table}

\newpage
\textbf{Fortalezas identificadas}
\begin{itemize}
    \item Tolerancia a typos: $>95\%$ de clasificación correcta.
    \item Normalización efectiva de ruido y puntuación.
    \item Degradación gradual y graceful ante errores severos.
    \item Manejo correcto de casos edge (nombres propios, \textit{leet speak}).
\end{itemize}

{\large \noindent \textbf{Tests de rendimiento y carga}}\\
Validación de velocidad, eficiencia y escalabilidad del sistema.\\

\textbf{\textit{Benchmarks} de latencia}

\begin{table}[H]
\centering
\renewcommand{\arraystretch}{1.6}
\begin{tabular}{|p{4cm}|p{2cm}|p{2cm}|p{3cm}|}
\hline
\textbf{Operación} & \textbf{Media} & \textbf{P95} & \textbf{Objetivo} \\ \hline

\textit{Query} completa & 42ms & 68ms & <100ms (P95) \\ \hline
Generación de \textit{embedding} & 28ms & 45ms & <50ms \\ \hline
Cache hit & 3ms & 6ms & <10ms \\ \hline
Preprocesamiento & 0.8ms & 1.5ms & <5ms \\ \hline
Cálculo similitud & 9ms & 15ms & <20ms \\ \hline

\end{tabular}
\caption[Latencias por operación]{Medición de latencias promedio y P95, elaboración propia.}
\end{table}

\textbf{\textit{Throughput} medido}
\begin{itemize}
    \item \textit{Single query}: $\sim 23$ q/s.
    \item \textit{Batch processing}: $\sim 65$ q/s.
    \item \textit{Cache hit}: $\sim 333$ q/s.
\end{itemize}

\newpage
\textbf{Test de concurrencia}

\begin{table}[H]
\centering
\renewcommand{\arraystretch}{1.6}
\begin{tabular}{|p{4cm}|p{2cm}|p{3cm}|p{3cm}|}
\hline
\textbf{Test} & \textbf{Usuarios} & \textbf{\textit{Success rate}} & \textbf{Latencia P95} \\ \hline

\textit{Concurrent 10 users} & 10 & 99.5\% & 78ms \\ \hline
\textit{Concurrent 50 users} & 50 & 98.3\% & 156ms \\ \hline
\textit{Concurrent 100 users} & 100 & 95.8\% & 287ms \\ \hline
\textit{Spike 0→20 users} & 20 & 96.2\% & 198ms \\ \hline
\textit{Soak 5 min} & 5 & 99.8\% & 82ms \\ \hline

\end{tabular}
\caption[Pruebas de carga]{Resultados de pruebas de concurrencia y resistencia, elaboración propia.}
\end{table}

\textbf{\textit{Load Testing} con Locust (50 usuarios, 5 minutos)}

\begin{table}[H]
\centering
\renewcommand{\arraystretch}{1.6}
\begin{tabular}{|p{4cm}|p{3cm}|p{3cm}|}
\hline
\textbf{Métrica} & \textbf{Valor} & \textbf{Objetivo} \\ \hline

Total \textit{requests} & 8,742 & - \\ \hline
\textit{Requests}/segundo (RPS) & 29.14 & >20 \\ \hline
\textit{Failures} & 0.26\% & <1\% \\ \hline
\textit{Success rate} & 99.74\% & >95\% \\ \hline
Latencia promedio & 52ms & <100ms \\ \hline
Latencia P95 & 118ms & <200ms \\ \hline
Latencia P99 & 187ms & <500ms \\ \hline

\end{tabular}
\caption[Métricas de rendimiento]{Resumen de rendimiento total, elaboración propia.}
\end{table}

\newpage
\textbf{Resumen de métricas de rendimiento}
\begin{table}[H]
\centering
\renewcommand{\arraystretch}{1.6}
\begin{tabular}{|p{3.2cm}|p{2.6cm}|p{2.6cm}|p{3cm}|}
\hline
\textbf{Métrica} & \textbf{Objetivo} & \textbf{Resultado} & \textbf{Cumplimiento} \\ \hline

Latencia P50 & <100ms & $\sim$42ms & 58\% mejor \\ \hline
Latencia P95 & <200ms & $\sim$68ms & 66\% mejor \\ \hline
Latencia P99 & <500ms & $\sim$134ms & 73\% mejor \\ \hline
\textit{Success Rate} & >95\% & 99.7\% & Excede \\ \hline
\textit{Throughput} & >50 q/s & $\sim$65 q/s & 30\% mejor \\ \hline
\textit{Max Users} & 100 & 100 & Cumple \\ \hline
\textit{Memory Growth} & <10\%/hour & <5\%/hour & 50\% mejor \\ \hline

\end{tabular}
\caption[Métricas de rendimiento]{Resumen de métricas de rendimiento, elaboración propia.}
\end{table}

{\large \noindent \textbf{Calidad semántica y métricas de PLN}}\\

Validación con metodologías específicas para sistemas de Procesamiento de Lenguaje Natural.\\

\noindent \textbf{\textbf{Golden Dataset Testing}}\\
\textit{Dataset} curado manualmente con casos que deben funcionar correctamente\\

\begin{table}[H]
\centering
\renewcommand{\arraystretch}{1.6}
\begin{tabular}{|p{3.5cm}|p{2cm}|p{3cm}|p{4cm}|}
\hline
\textbf{Categoría} & \textbf{Casos} & \textbf{Min Similitud} & \textbf{Ejemplo} \\ \hline

\textit{Exact match} & 3 & >0.90 & ``Buenos días'' \\ \hline
\textit{Semantic variation} & 3 & >0.75 & ``necesito ayuda'' \\ \hline
\textit{Synonyms} & 6 & >0.65 & ``socorro'' \\ \hline
\textit{Noisy input} & 3 & >0.70 & ``hola!!'' \\ \hline
\textit{Typos} & 2 & >0.50 & ``hla'' \\ \hline

\textbf{Total} & 17 & - & - \\ \hline

\end{tabular}
\caption[Evaluación lingüística]{Resumen de evaluación lingüística, elaboración propia.}
\end{table}

\textbf{Resultado}: $100\%$ de \textit{golden dataset passing}\\

\newpage
\textbf{Métricas de clasificación semántica}

\begin{table}[H]
\centering
\renewcommand{\arraystretch}{1.6}
\begin{tabular}{|p{3.5cm}|p{4cm}|p{2.5cm}|p{2.5cm}|}
\hline
\textbf{Métrica} & \textbf{Fórmula} & \textbf{Valor} & \textbf{Objetivo} \\ \hline

\textit{Precision} & TP/(TP+FP) & $\sim$89\% & >85\% \\ \hline
\textit{Recall} & TP/(TP+FN) & $\sim$87\% & >80\% \\ \hline
\textit{F1-Score} & $2 \cdot P \cdot R / (P + R)$ & $\sim$88\% & >82\% \\ \hline
\textit{Accuracy} & (TP+TN)/Total & $\sim$96\% & >90\% \\ \hline
\textit{Golden Dataset} & Correct/Total & 100\% & 100\% \\ \hline

\end{tabular}
\caption[Métricas de clasificación]{Resumen de métricas de clasificación, elaboración propia.}
\end{table}

El \textit{F1-Score} de 88\% indica balance óptimo entre \textit{precision} y \textit{recall}.\\

\textbf{Reconocimiento de sinónimos}

\begin{table}[H]
\centering
\renewcommand{\arraystretch}{1.6}
\begin{tabular}{|p{2cm}|p{6cm}|p{2.5cm}|}
\hline
\textbf{Grupo} & \textbf{Sinónimos testeados} & \textbf{\textit{Accuracy}} \\ \hline

A & ayuda, asistencia, socorro & 75\% \\ \hline
B & hola, saludos, buenos días & 80\% \\ \hline
C & gracias, muchas gracias, bien & 78\% \\ \hline

\end{tabular}
\caption[Sinónimos testeados]{Evaluación por grupos de sinónimos, elaboración propia.}
\end{table}

\textbf{Objetivo}: >70\% dentro de cada grupo.\\

\newpage
{\large \noindent \textbf{Cobertura de código}}\\
\textbf{Cobertura global del módulo}

\begin{table}[H]
\centering
\renewcommand{\arraystretch}{1.6}
\begin{tabular}{|p{5cm}|p{1.8cm}|p{1.8cm}|p{2.2cm}|p{2.2cm}|}
\hline
\textbf{Módulo} & \textbf{Stmts} & \textbf{\textit{Missing}} & \textbf{\textit{Coverage}} & \textbf{\textit{Branch}} \\ \hline

app/matcher\_improved.py & 445 & 38 & 91.5\% & 88.7\% \\ \hline
app/preprocess.py & 178 & 15 & 91.6\% & 89.2\% \\ \hline
app/main.py & 156 & 19 & 87.8\% & 82.1\% \\ \hline
app/models.py & 67 & 4 & 94.0\% & 91.5\% \\ \hline
app/config.py & 45 & 6 & 86.7\% & 83.3\% \\ \hline

\textbf{Total} & 891 & 82 & 90.8\% & 87.0\% \\ \hline

\end{tabular}
\caption[Coverage por módulo]{Resumen de \textit{coverage} por módulo, elaboración propia.}
\end{table}

\textbf{\textit{Statement Coverage}}: $90.8\%$ \ (Objetivo: $>80\%$)\\
\indent \textbf{\textit{Branch Coverage}}: $87.0\%$ \ (Objetivo: $>75\%$)\\

\noindent \textbf{Análisis de líneas sin cobertura}\\
Las 82 líneas sin cobertura (9.2\%) corresponden a:\\

\begin{enumerate}
    \item Manejo de errores raros (35 líneas): Modelo no cargado, \textit{embeddings corruptos}, errores de memoria.
    \item \textit{Logging y debugging} (28 líneas): \textit{Logs} de nivel DEBUG, telemetría.
    \item Código de infraestructura (19 líneas): \textit{Shutdown handlers}, configuración avanzada.
\end{enumerate}

Estos casos no afectan la funcionalidad core del sistema.

%===================================================================
%===================================================================
%===================================================================
%===================================================================
\newpage
\section{Resultados del Modulo de PLN}

{\large \noindent \textbf{Resumen ejecutivo}}\\

\textbf{Estado general:} Aprobado

\vspace{0.4cm}

\begin{itemize}
    \item \textbf{\textit{Tests} Totales Implementados:} 168
    \item \textbf{\textit{Tests Core} Ejecutados:} 73
    \item \textbf{\textit{Tests} Exitosos:} 73 (100\%)
    \item \textbf{\textit{Tests} Fallidos:} 0
    \item \textbf{Cobertura \textit{Statement}:} 90.8\%
    \item \textbf{Cobertura \textit{Branch}:} 87.0\%
    \item \textbf{Precisión Semántica:} 89\%
    \item \textbf{\textit{F1-Score}:} 88\%
    \item \textbf{\textit{Golden Dataset Accuracy}:} 100\%
    \item \textbf{Latencia P95:} 68ms
    \item \textbf{\textit{Success Rate} bajo carga:} 99.7\%
    \item \textbf{\textit{Throughput}:} 65 q/s
\end{itemize}

\newpage
{\large \noindent \textbf{Comparación: objetivos vs resultados}}\\

\begin{table}[H]
\centering
\renewcommand{\arraystretch}{1.6}
\begin{tabular}{|p{4cm}|p{3cm}|p{3cm}|p{3cm}|}
\hline
\textbf{Métrica} & \textbf{Objetivo} & \textbf{Resultado} & \textbf{Cumplimiento} \\ \hline

\textit{Tests} implementados & >100 & 168 & 168\% \\ \hline
Cobertura de código & >80\% & 90.8\% & 114\% \\ \hline
\textit{Tests E2E} & >20 & 26 & 130\% \\ \hline
Precision semántica & >85\% & $\sim$89\% & 105\% \\ \hline
\textit{F1-Score} & >82\% & $\sim$88\% & 107\% \\ \hline
Latencia P95 & <200ms & 68ms & 66\% mejor \\ \hline
\textit{Success rate} & >95\% & 99.7\% & 105\% \\ \hline
\textit{Throughput} & >50 q/s & 65 q/s & 130\% \\ \hline
\textit{Golden dataset} & 100\% & 100\% & Aceptable \\ \hline

\end{tabular}
\caption[Métricas de testing]{Resumen de métricas de \textit{testing}, elaboración propia.}
\end{table}
\textbf{Conclusión}: Todos los objetivos fueron superados significativamente.\\

\noindent \textbf{Fortalezas}

\begin{enumerate}
    \item Cobertura exhaustiva
    \begin{itemize}[label=\checkmark]
        \item 168 \textit{tests} implementados en todos los niveles.
        \item 90.8\% de cobertura supera estándares industriales (80\%).
        \item Distribución apropiada según pirámide de \textit{testing}.
    \end{itemize}

    \item Robustez lingüística mínima aceptable
    \begin{itemize}[label=\checkmark]
        \item 50+ casos de perturbaciones validados.
        \item >95\% de tolerancia a errores ortográficos.
        \item Degradación gradual y \textit{graceful}.
        \item Normalización efectiva de \textit{leet speak}.
    \end{itemize}

    \newpage
    \item Calidad semántica aceptable
    \begin{itemize}[label=\checkmark]
        \item \textit{F1-Score} de 88\% demuestra balance \textit{precision/recall}.
        \item 100\% de \textit{golden dataset passing}.
        \item Reconocimiento de sinónimos >75\%.
    \end{itemize}

    \item Rendimiento muy adecuado
    \begin{itemize}[label=\checkmark]
        \item Latencia P95 de 68ms (66\% mejor que objetivo).
        \item \textit{Success rate} de 99.7\% bajo carga.
        \item Sistema estable hasta 100 usuarios concurrentes.
    \end{itemize}

    \item Validación de casos críticos
    \begin{itemize}[label=\checkmark]
        \item Similitud matemáticamente correcta [0.0, 1.0] en 100\% de casos.
        \item Caso \textit{edge} ``Ivan'' correctamente manejado.
        \item Detección de patrones ``Me llamo [NOMBRE]''.
        \item Manejo robusto de nombres propios.
    \end{itemize}
\end{enumerate}

{\large \noindent \textbf{Optimizaciones y rendimiento}}\\

\textbf{Caché de \textit{embeddings}}\\

\textbf{Problema}: Generar \textit{embeddings} es costoso:
\begin{itemize}
    \item Tiempo: $\sim$5 segundos para 43 frases.
    \item Requiere modelo cargado en memoria.
    \item Se ejecuta en cada \textit{startup}.
\end{itemize}

\textbf{Solución}: Caché persistente en archivo .npz:
\begin{itemize}
    \item Formato: NumPy compressed (.npz).
    \item Tamaño: $\sim$50 KB comprimido.
    \item Carga: $<1$ segundo.
\end{itemize}

\newpage
\textbf{Resultados}

\begin{table}[H]
\centering
\renewcommand{\arraystretch}{1.6}
\begin{tabular}{|p{5cm}|p{3cm}|p{3cm}|}
\hline
 & \textbf{Sin caché} & \textbf{Con caché} \\ \hline

Tiempo de inicialización & $\sim$5 seg & <1 seg \\ \hline
\textit{Speedup} & 1x & 5x \\ \hline
Tamaño en disco & - & 50 KB \\ \hline
Memoria en RAM & 65 KB & 65 KB \\ \hline

\end{tabular}
\caption[Comparativa con y sin caché]{Resumen de rendimiento con y sin caché, elaboración propia.}
\end{table}

\textbf{Búsqueda jerárquica por centroides}\\

\textbf{Problema}: Búsqueda exhaustiva en 43 frases:
\begin{itemize}
    \item O(N) comparaciones con N = 43.
    \item No escala bien con más frases.
    \item Ineficiente para \textit{datasets} grandes.
\end{itemize}

\textbf{Solución}: Búsqueda jerárquica en dos fases:
\begin{enumerate}
    \item Fase 1: Buscar top-3 grupos (O(K) con K=3).
    \item Fase 2: Buscar en grupos candidatos (O(N\_k)).
\end{enumerate}

Complejidad: O(K + N\_k) $\ll$ O(N)\\

\textbf{Resultados}: \textit{Dataset} actual (43 frases):\\

\begin{table}[H]
\centering
\renewcommand{\arraystretch}{1.6}
\begin{tabular}{|p{5cm}|p{3cm}|p{3cm}|}
\hline
 & \textbf{Exhaustiva} & \textbf{Jerárquica} \\ \hline

Comparaciones & 43 & 3 + $\sim$20 = 23 \\ \hline
\textit{Speedup} & 1x & 1.9x \\ \hline
Latencia & $\sim$42ms & $\sim$40ms \\ \hline

\end{tabular}
\caption[Comparación de métodos]{Comparación entre enfoque exhaustivo y jerárquico, elaboración propia.}
\end{table}

\newpage
{\large \noindent \textbf{Optimización de latencia}}\\
\textbf{Técnicas aplicadas}\\

\begin{enumerate}

    \item \textbf{Operaciones vectorizadas (NumPy)}
    \\[4pt]
    \begin{verbatim}
# LENTO: Loop
for i, emb in enumerate(embeddings):
    sim[i] = cosine_similarity([query_emb], [emb])

# RÁPIDO: Vectorizado
sims = cosine_similarity([query_emb], embeddings)[0]
    \end{verbatim}
    \textit{Speedup}: $\sim$100x.\\

    \item \textbf{\textit{Lazy loading} del modelo}
    \\[4pt]
    \begin{verbatim}
def _load_model(self):
    if self.model is None:
        self.model = SentenceTransformer(model_name)
    \end{verbatim}
    Ahorro: No cargar modelo si no hay \textit{requests}.\\

    \item \textbf{\textit{Batch processing}}
    \\[4pt]
    \begin{verbatim}
embeddings = model.encode(texts, batch_size=32)
    \end{verbatim}
    \textit{Speedup}: $\sim$2x para múltiples textos.\\

    \newpage
    \item \textbf{Async I/O (FastAPI)}
    \\[4pt]
    \begin{verbatim}
@app.post("/buscar")
async def buscar(request: QueryRequest):
    ...
    \end{verbatim}
    Permite manejar múltiples \textit{requests} concurrentes.\\

\end{enumerate}

{\large \noindent \textbf{\textit{Profile} de latencia}}\\

\textbf{\textit{Total latency}}: 40ms\\

\textit{Breakdown}:
\begin{itemize}
    \item Preprocesamiento: 2ms (5\%).
    \item Generación de \textit{embedding}: 15ms (37.5\%).
    \item Búsqueda por centroides: 3ms (7.5\%).
    \item \textit{Re-ranking}: 10ms (25\%).
    \item Detección de patrones: 5ms (12.5\%).
    \item Construcción de respuesta: 3ms (7.5\%).
    \item \textit{Overhead} (FastAPI): 2ms (5\%).
\end{itemize}

\vspace{1em}

\textbf{Cuellos de botella}:
\begin{enumerate}
    \item Generación de \textit{embedding} (37.5\%) $\leftarrow$ Mayor oportunidad.
    \item \textit{Re-ranking} (25\%).
\end{enumerate}


\newpage
{\large \noindent \textbf{Gestión de memoria}}

\begin{table}[H]
\centering
\renewcommand{\arraystretch}{1.6}
\begin{tabular}{|p{6cm}|p{3.5cm}|}
\hline
\textbf{Componente} & \textbf{Memoria} \\ \hline

Modelo \textit{transformer} & $\sim$420 MB \\ \hline
\textit{Embeddings} en cache & $\sim$65 KB \\ \hline
Centroides & $\sim$5 KB \\ \hline
\textit{FastAPI} + \textit{Uvicorn} & $\sim$50 MB \\ \hline
Python \textit{runtime} & $\sim$30 MB \\ \hline

\textbf{Total} & $\sim$500 MB \\ \hline

\end{tabular}
\caption[Memoria por componente]{Resumen de memoria utilizada por componente, elaboración propia.}
\end{table}

\textbf{Optimizaciones}
\begin{enumerate}
    \item Modelo ligero (MiniLM)
    \begin{itemize}
        \item 384 dimensiones vs 768 (BERT base).
        \item 420 MB vs 800 MB.
        \item Ahorro: 47.5\%.
    \end{itemize}

    \item \textit{Embeddings} comprimidos
    \begin{itemize}
        \item Formato .npz comprimido.
        \item 50 KB vs $\sim$100 KB sin comprimir.
        \item Ahorro: 50\%.
    \end{itemize}

    \item \textit{Garbage collection} optimizado
    \begin{verbatim}
    import gc
    gc.collect()  # Después de carga inicial
    \end{verbatim}
\end{enumerate}

\newpage
{\large \noindent \textbf{Escalabilidad}}\\
\textbf{Escalabilidad horizontal}\\
API \textit{stateless} permite múltiples instancias:\\

\begin{center}
    \includegraphics[width=0.75\textwidth]{Images/Cap5/2_API_Stateless.png}
    \captionof{figure}[API Stateless]{API \textit{Stateless}, elaboración propia.} 
\end{center}

\textbf{Configuración}:
\begin{verbatim}
# Instancia 1
uvicorn app.main:app --port 8001 &

# Instancia 2
uvicorn app.main:app --port 8002 &

# Instancia 3
uvicorn app.main:app --port 8003 &

# Load balancer (nginx)
upstream api_servers {
    server localhost:8001;
    server localhost:8002;
    server localhost:8003;
}
\end{verbatim}

\noindent \textbf{Escalabilidad con \textit{workers}}:
\begin{verbatim}
uvicorn app.main:app \
    --host 0.0.0.0 \
    --port 8000 \
    --workers 4
\end{verbatim}

\textbf{\textit{Throughput}}:
\begin{itemize}
    \item 1 \textit{worker}: $\sim$25 req/s.
    \item 4 \textit{workers}: $\sim$90 req/s (3.6x).
\end{itemize}

\newpage
\noindent \textbf{Limitaciones}
\begin{itemize}
    \item Modelo en memoria: $\sim$420 MB por \textit{worker}.
    \item 4 \textit{workers}: $\sim$1.7 GB memoria.
    \item Máximo recomendado: 8 \textit{workers} en servidor 8GB RAM.
\end{itemize}

\vspace{1em}

{\Large \noindent \textbf{Resultados y métricas}}\\
{\large \textbf{Métricas de precisión}}\\

\noindent \textbf{\textit{Dataset} de evaluación}\\
100 \textit{queries} de prueba en 3 categorías:
\begin{itemize}
    \item 40 \textit{queries} de emergencias (Grupo A).
    \item 30 \textit{queries} de saludos (Grupo B).
    \item 30 \textit{queries} de comunicación (Grupo C).
\end{itemize}

\textbf{Métricas Calculadas}

\begin{table}[H]
\centering
\renewcommand{\arraystretch}{1.6}
\begin{tabular}{|p{7cm}|p{3cm}|}
\hline
\textbf{Métrica} & \textbf{Valor} \\ \hline

\textit{Accuracy} (clasificación de grupos) & 92\% \\ \hline
\textit{Precision} (promedio) & 91\% \\ \hline
\textit{Recall} (promedio) & 90\% \\ \hline
\textit{F1-Score} (promedio) & 90.5\% \\ \hline
Similitud promedio (\textit{matches}) & 0.87 \\ \hline

\end{tabular}
\caption[Métricas de precisión]{Resumen de métricas de desempeño del modelo, elaboración propia.}
\end{table}

\textbf{Matriz de confusión}\\
\begin{center}
    \includegraphics[width=0.75\textwidth]{Images/Cap5/3_Matriz_Confusión.png}
    \captionof{figure}[Matriz de Confusión]{Matriz de confusión de cada grupo, elaboración propia.} 
\end{center}

Observaciones:
\begin{itemize}
    \item Grupo A (Emergencias): Mejor precisión (95\%).
    \item Confusión menor entre grupos similares.
    \item Errores en casos ambiguos.
\end{itemize}

{\large \noindent \textbf{Métricas de rendimiento}}\\

\noindent \textbf{Latencia}
\begin{table}[H]
\centering
\renewcommand{\arraystretch}{1.6}
\begin{tabular}{|p{5cm}|p{2cm}|p{2.5cm}|p{2cm}|}
\hline
\textbf{Métrica} & \textbf{Min} & \textbf{Promedio} & \textbf{Max} \\ \hline

Latencia (ms) & 35ms & 40ms & 48ms \\ \hline
Latencia P95 (ms) & -- & 45ms & -- \\ \hline
Latencia P99 (ms) & -- & 47ms & -- \\ \hline

\end{tabular}
\caption[Latencias]{Resumen de latencias medidas en entorno de pruebas, elaboración propia.}
\end{table}

\textbf{Objetivo}: <50ms (cumplido).\\

\noindent \textbf{\textit{Throughput}}
\begin{table}[H]
\centering
\renewcommand{\arraystretch}{1.6}
\begin{tabular}{|p{7cm}|p{3cm}|}
\hline
\textbf{Configuración} & \textbf{REQ/s} \\ \hline

1 \textit{worker} (single-core) & 25 req/s \\ \hline
4 \textit{workers} (4-core) & 90 req/s \\ \hline
8 \textit{workers} (8-core) & 160 req/s \\ \hline

\end{tabular}
\caption[Throughput por configuración]{Capacidad de procesamiento según número de \textit{workers}, elaboración propia.}
\end{table}

\textbf{Objetivo}: >20 req/s (cumplido).

\newpage
\noindent \textbf{Memoria}
\begin{table}[H]
\centering
\renewcommand{\arraystretch}{1.6}
\begin{tabular}{|p{6cm}|p{3.5cm}|}
\hline
\textbf{Métrica} & \textbf{Valor} \\ \hline

Memoria base (\textit{startup}) & $\sim$500 MB \\ \hline
Memoria por \textit{request} & +2 MB \\ \hline
Memoria después de 1000 \textit{requests} & $\sim$520 MB \\ \hline
Crecimiento & Estable \\ \hline

\end{tabular}
\caption[Memoria]{Consumo de memoria durante operación, elaboración propia.}
\end{table}

Sin \textit{memory leaks} detectados.\\

{\large \noindent \textbf{Métricas de usabilidad}}\\

\textbf{Swagger UI}
\begin{itemize}[label= \checkmark]
    \item Documentación automática.
    \item Testing interactivo.
    \item Ejemplos de \textit{requests}.
    \item \textit{Schemas} completos.
    \item Respuestas de error documentadas.
\end{itemize}

\textbf{API \textit{Consistency}}
\begin{itemize}[label= \checkmark]
    \item Formato JSON consistente.
    \item Campos opcionales claramente marcados.
    \item Validación automática de \textit{requests}.
    \item Mensajes de error descriptivos.
    \item HTTP \textit{status codes} apropiados.
\end{itemize}

\vspace{1em}

\newpage
{\large \noindent \textbf{Comparación con alternativas}}\\

\textbf{Ventajas del prototipo}
\begin{itemize}[label= \checkmark]
    \item Mayor precisión (92\% vs 60--75\%).
    \item Manejo de paráfrasis.
    \item Detección de nombres propios.
    \item Sistema de deletreo automático.
    \item API bien documentada.
    \item \textit{Tests} exhaustivos.
    \item Fácil de escalar.
\end{itemize}

\vspace{0.7em}

\textbf{Desventajas}
\begin{itemize}
    \item Mayor latencia (40ms vs 5--25ms)
    \item Mayor uso de memoria ($\sim$500MB vs 50--200MB)
    \item Requiere modelo pre-entrenado
\end{itemize}
\chapter{Cuestionario de Usabilidad}
\section{Prueba de Usabilidad mediante Cuestionario Estructurado}

Con el propósito de evaluar la experiencia de uso, la percepción de calidad del módulo de traducción y la claridad de las animaciones en Lengua de Señas Mexicana (LSM), se aplicó una prueba de usabilidad basada en un cuestionario estructurado.\\

Este instrumento fue diseñado considerando las recomendaciones de la norma ISO 9241-210 \cite{ref62}, así como principios derivados de las heurísticas de Nielsen \cite{ref61}, con el fin de evaluar aspectos como:

\begin{itemize}
    \item Facilidad de uso.
    \item Pertinencia de la traducción.
    \item Calidad perceptual del movimiento de la animación.
    \item Utilidad del vocabulario incluido.
    \item satisfacción general del usuario.
\end{itemize}

\noindent \textbf{Procedimiento}\\
La prueba se llevó a cabo siguiendo el siguiente protocolo:
\begin{itemize}
    \item A cada participante se le mostró un video demostrativo del funcionamiento de la aplicación.
    \item Posteriormente, se le invitó a responder un cuestionario en línea compuesto por preguntas cerradas de opción múltiple, escalas Likert de 1 a 5 \cite{refpru1}, y preguntas abiertas para permitir observaciones cualitativas.
    \item La participación fue anónima y sin recopilar datos personales sensibles.
    \item No se requirió experiencia previa en LSM para participar.
\end{itemize}

\noindent \textbf{Estructura del cuestionario}\\
El cuestionario se compone de cinco bloques temáticos:
\begin{itemize}
    \item Datos demográficos.
    \item Conocimientos previos en LSM.
    \item Percepción de velocidad y pertinencia de la traducción.
    \item Calidad del movimiento del avatar.
    \item Cubrimiento del vocabulario y opinión general.
\end{itemize}

En la \textbf{Tabla \ref{tab:instrumento-usabilidad}} se resume la estructura completa del instrumento, indicando el tipo de pregunta y su propósito.

\begin{table}[H]
\centering
\renewcommand{\arraystretch}{1.6}
\begin{tabular}{|p{5cm}|p{4cm}|p{4cm}|}
\hline
\textbf{Categoría} & \textbf{Pregunta evaluada} & \textbf{Tipo de respuesta} \\ \hline

Datos demográficos & Edad del participante & Opción múltiple \\ \hline

Conocimientos previos en LSM & Nivel de dominio de LSM & Opción múltiple \\ \hline

Experiencia previa & Uso previo de herramientas de traducción & Sí / No \\ \hline

Comparativa con herramientas previas & Diferencias observadas respecto a herramientas de traducción previamente utilizadas & Respuesta abierta \\ \hline

Velocidad y pertinencia de traducción & Evaluación de qué tan rápido y pertinente fue el emparejamiento de la frase (escala 1–5) & Escala Likert \\ \hline

Utilidad del deletreo & Utilidad del modo de deletreo dentro de una conversación & Escala Likert \\ \hline

Calidad del movimiento del avatar & Percepción de fluidez y claridad del movimiento en las animaciones & Opción múltiple \\ \hline

Cobertura del vocabulario & Adecuación del conjunto de frases y categorías incluidas en la aplicación & Opción múltiple \\ \hline

Fortalezas percibidas & Aspectos positivos o elementos destacados por el usuario & Respuesta abierta \\ \hline

Áreas de oportunidad & Problemas, fallas o aspectos a mejorar identificados por el usuario & Respuesta abierta \\ \hline

\end{tabular}
\caption[Instrumento de evaluación de usabilidad]{Resumen del cuestionario empleado para la evaluación de usabilidad de la aplicación, elaboración propia.}
\label{tab:instrumento-usabilidad}
\end{table}

\newpage
Una copia íntegra del cuestionario, con el texto exacto de cada reactivo, se incluye en el \textbf{\nameref{anexo:cuestionario}} para referencia completa.

\section{Resultados de la Evaluación de Usabilidad}
A continuación se presentan los resultados obtenidos del cuestionario aplicado a los participantes. Las figuras muestran los resultados de las preguntas cerradas, mientras que las respuestas abiertas fueron analizadas siguiendo una codificación temática simple.\\

\textbf{Perfil de los Participantes}\\
La Figura \ref{fig:edad-participantes} muestra la distribución por edad de los participantes. El rango de 18–24 años abarca un 43.8\%, seguido del grupo de 25–50 años que abarca el mismo porcentaje (43.8\%), mientras que un 12.5\% pertenece al grupo de 50 años o más. Esta distribución es representativa del público potencial de la aplicación, principalmente jóvenes adultos.\\

\begin{center}
    \includegraphics[width=0.75\textwidth]{Images/Cap6/1_Edad.jpeg}
    \captionof{figure}[Distribución por edad]{Distribución por edad de los participantes, elaboración propia.} 
    \label{fig:edad-participantes}
\end{center}

En cuanto al nivel de dominio de LSM, los resultados en la Figura \ref{fig:nivel_dominio_lsm} indican una composición equilibrada:
\begin{itemize}
    \item 25\% sin experiencia previa.
    \item 31.3\% con nivel básico.
    \item 25\% con nivel intermedio.
    \item 18.8\% con nivel avanzado o fluido.
\end{itemize}

\begin{center}
    \includegraphics[width=0.95\textwidth]{Images/Cap6/2_Conocimientos_LSM.jpeg}
    \captionof{figure}[Nivel de Dominio LSM]{Nivel de dominio de LSM, elaboración propia.} 
    \label{fig:nivel_dominio_lsm}
\end{center}

\textbf{Experiencia previa con herramientas similares}\\
De acuerdo con la Figura \ref{fig:uso_herramientas_existentes}, el 62.5\% de los participantes indicó no haber utilizado herramientas de traducción similares, mientras que el 37.5\% sí tenía experiencia previa.\\

\begin{center}
    \includegraphics[width=0.75\textwidth]{Images/Cap6/3_Uso_Herramientas.jpeg}
    \captionof{figure}[Experiencias previas con herramientas]{Experiencia previa con aplicaciones y herramientas similares por parte de los usuarios, elaboración propia.} 
    \label{fig:uso_herramientas_existentes}
\end{center}

Solo 9 participantes reportaron haber utilizado previamente aplicaciones similares. Entre ellos, se identificaron tres diferencias principales:
\begin{enumerate}
    \item \textbf{Mayor naturalidad en la animación}\\
    Varios participantes destacaron que la fluidez del avatar supera la de herramientas previas:
    \begin{itemize}
        \item “La fluidez de la animación”.
        \item “Mejor calidad y velocidad percibida en los videos / animaciones”.
    \end{itemize}

    \item \textbf{Posibilidad de concatenar frases}\\
    Una funcionalidad ampliamente valorada fue la capacidad de realizar traducciones por frase y no solo deletreo:
    \begin{itemize}
        \item “Se pueden concatenar las frases, eso es algo muy ventajoso”.
        \item “Posibilidad de identificar frases y no solo deletreo”.
    \end{itemize}

    \item \textbf{Interfaz más clara o intuitiva}\\
    Algunos comentarios reconocieron mejoras visuales:
    \begin{itemize}
        \item “Mejor distribución de colores”.
    \end{itemize}
\end{enumerate}

En general, los usuarios con experiencia previa percibieron la herramienta como más intuitiva, más clara y con animaciones superiores. Esta proporción revela que la mayoría evaluó la aplicación sin sesgos comparativos y que una minoría pudo aportar información contextual respecto a soluciones existentes.\\

\textbf{Percepción sobre la velocidad y pertinencia de la traducción}\\
Los resultados de la Figura \ref{fig:fluidez} mostraron una percepción predominantemente positiva. En la escala del 1 al 5:
\begin{itemize}
    \item 31.3\% calificó la traducción con 4.
    \item 31.3\% la calificó con 5.
    \item 25\% asignó un 3.
    \item Solo 12.5\% la calificó con 2.
    \item Ningún usuario seleccionó 1.
\end{itemize}

\begin{center}
    \includegraphics[width=0.85\textwidth]{Images/Cap6/5_Fluidez.jpg}
    \captionof{figure}[Velocidad y pertinencia de la traducción]{Velocidad y pertinencia de la traducción, elaboración propia.} 
    \label{fig:fluidez}
\end{center}

Estos resultados indican que más del 60\% considera que la traducción es rápida, pertinente y adecuada, mientras que únicamente una minoría percibió lentitud o falta de pertinencia. Esto valida el diseño del módulo de emparejamiento implementado.\\

\textbf{Utilidad del modo de deletreo}\\
La valoración del modo de deletreo fue notablemente positiva:
\begin{itemize}
    \item 31.3\% lo considera “extremadamente útil”.
    \item 25\% lo considera “muy útil”.
    \item 25\% “útil”.
    \item 12.5\% “poco útil”.
    \item Solo 6.3\% lo percibe como “nada útil”.
\end{itemize}

\begin{center}
    \includegraphics[width=0.85\textwidth]{Images/Cap6/6_Utilidad_Deletreo.jpeg}
    \captionof{figure}[Utilidad del modo de deletreo]{Utilidad del modo de deletreo, elaboración propia.} 
    \label{fig:utilidad_deletreo}
\end{center}

En conjunto, 81.3\% lo considera “útil” o “muy útil”, lo que confirma que este mecanismo funciona como un soporte importante en situaciones donde el vocabulario no está disponible o cuando se requiere precisión (como nombres propios).\\

\textbf{Percepción del movimiento del avatar}\\
La Figura \ref{fig:percepcion_movimiento} indica que el 75\% de los usuarios describió el movimiento del avatar como claro y natural, lo que indica que la animación es comprensible y suficientemente fluida para usuarios no expertos.\\

Sin embargo:
\begin{itemize}
    \item 18.8\% señaló movimientos lentos o trabados.
    \item 6.2\% describió la animación como confusa.
\end{itemize}

\begin{center}
    \includegraphics[width=0.85\textwidth]{Images/Cap6/7_Calidad_Movimiento.jpeg}
    \captionof{figure}[Percepción del movimiento]{Percepción del movimiento del avatar, elaboración propia.} 
    \label{fig:percepcion_movimiento}
\end{center}

Aunque la mayoría percibe una buena calidad de animación, estos resultados revelan oportunidades de mejora en la suavidad, velocidad y naturalidad de ciertos gestos.\\

\textbf{Evaluación del vocabulario disponible}\\
Respecto a la cobertura del vocabulario, la Figura \ref{fig:evaluacion_vocabulario} establece que:
\begin{itemize}
    \item 56.3\% considera que las frases y categorías incluidas cubren las necesidades básicas de comunicación.
    \item 43.8\% opina que faltan frases o categorías importantes.
    \item Ningún participante consideró que existieran frases innecesarias.
\end{itemize}

\begin{center}
    \includegraphics[width=0.85\textwidth]{Images/Cap6/8_Seleccion_Frases.jpeg}
    \captionof{figure}[Evaluación del vocabulario]{Evaluación del vocabulario seleccionado, elaboración propia.} 
    \label{fig:evaluacion_vocabulario}
\end{center}

Este resultado sugiere que, si bien la base actual de frases es funcional, existe una expectativa clara de ampliar el repertorio para cubrir más contextos comunicativos cotidianos.\\

\newpage
\textbf{Fortalezas percibidas}\\
Entre las 15 respuestas registradas, surgieron cinco temas recurrentes:
\begin{enumerate}
    \item \textbf{Concatenación y pertinencia de frases}\\
    La capacidad de traducir frases completas fue mencionada como la principal fortaleza:
    \begin{itemize}
        \item “Su plus es que añadieron lo de concatenar las frases”.
        \item “Permite identificar frases y no solo deletreo”.
    \end{itemize}

    \item \textbf{Fluidez y calidad de las animaciones}\\
    Los usuarios destacaron la naturalidad del movimiento:
    \begin{itemize}
        \item “Las animaciones se ven bien, no es muy complicado usar la app”.
        \item “Las animaciones no se traban”.
    \end{itemize}

    \item \textbf{Interfaz simple, limpia y no invasiva}\\
    La aplicación fue percibida como fácil de usar:
    \begin{itemize}
        \item “La interfaz es simple y minimalista”.
        \item “Excelente distribución de botones”.
        \item “Buena elección de colores”.
    \end{itemize}

    \item \textbf{Utilidad para el aprendizaje de LSM}\\
    Varios participantes valoraron la función educativa:
    \begin{itemize}
        \item “Aprender LSM por cómo en verdad se expresan las señas es lo más valioso”.
        \item “Sirve para situaciones donde se necesita comunicar algo”.
    \end{itemize}

    \item \textbf{Respuesta inmediata y sistema fluido}\\
    Comentarios recurrentes señalaron buen rendimiento:
    \begin{itemize}
        \item “La velocidad de los videos es buena”.
        \item “El tener siempre una respuesta del sistema hace que sea una aplicación completa”.
        \item “Está bien que no dejen al usuario sin respuesta”.
    \end{itemize}
\end{enumerate}

En conjunto, los usuarios describen la aplicación como fluida, intuitiva y útil, especialmente para principiantes.\\

\newpage
\textbf{Áreas de oportunidad}\\
Entre las 14 respuestas abiertas, se identificaron cuatro líneas principales de mejora:
\begin{enumerate}
    \item \textbf{Ampliación del vocabulario}\\
    Fue el tema más mencionado:
    \begin{itemize}
        \item “Agregar más frases”.
        \item “Faltan algunas categorías de frases o funciones extra”.
        \item “A veces la seña va muy rápido y no se distingue”.
    \end{itemize}

    La percepción general indica que el vocabulario actual es funcional, pero insuficiente para cubrir situaciones más amplias.\\

    \item \textbf{Mejora de la fluidez del avatar}\\
    Aunque las animaciones son valoradas positivamente, algunos comentaron que podrían ser más naturales:
    \begin{itemize}
        \item “La seña va muy rápido y no se distingue”.
        \item “Hacen falta gestos faciales o expresividad”.
    \end{itemize}

    \item \textbf{Mayor personalización}\\
    Algunos usuarios desean opciones para adaptar la experiencia:
    \begin{itemize}
        \item “Quisiera poder tener mi propio personaje y guardar frases”.
        \item “Poder crear tu propio video con tu animación”.
    \end{itemize}

    \item \textbf{Funcionalidades adicionales mediante IA}
    Algunos participantes sugieren extender el sistema más allá de traducción a LSM:
    \begin{itemize}
        \item “Agregar reconocimiento de voz y texto en imágenes”.
        \item “Un traductor que permita comunicación bidireccional”.
    \end{itemize}
\end{enumerate}

En conjunto, las áreas de oportunidad se enfocan en ampliar categorías, mejorar expresividad del avatar y considerar nuevas funciones de interacción y reconocimiento.


\chapter{Conclusiones}
El desarrollo del presente Trabajo Terminal permitió materializar un prototipo funcional de aplicación móvil capaz de traducir texto en español a Lengua de Señas Mexicana (LSM) mediante técnicas de Procesamiento de Lenguaje Natural (PLN). El proyecto, que inició como una propuesta sencilla para vincular texto con señas en video, evolucionó hacia un sistema modular, medible y defendible desde una perspectiva de ingeniería profesional. Aunque el modelado y la animación 3D previstos inicialmente no se implementaron por limitaciones de tiempo, recursos y la necesidad de especialistas en captura de movimiento, la sustitución por videos estilizados con filtros tipo “anime” permitió mantener fluidez visual y congruencia comunicativa sin comprometer la calidad final del prototipo.\\

Durante el desarrollo fue necesario ajustar el conjunto original de frases. Las expresiones de agradecimiento se reemplazaron por expresiones de mínima comunicación, debido a la reducida variabilidad gestual en LSM para dichas frases. Este cambio incrementó la diversidad expresiva y mejoró la utilidad práctica del sistema. El conjunto final se consolidó en 43 frases curadas organizadas en grupos temáticos, lo que permitió aplicar técnicas de búsqueda semántica optimizadas y reducir el número de comparaciones necesarias durante la clasificación.\\

En términos cuantitativos, el sistema superó los objetivos establecidos. La latencia promedio se mantuvo alrededor de 40 ms, por debajo del límite de 100 ms planteado como aceptable; el throughput de más de 25 consultas por segundo fue suficiente para soportar una carga estimada de 100 usuarios concurrentes; y la precisión aproximada del 90\% en la clasificación semántica superó ampliamente el mínimo esperado del 85\%. Además, el sistema logró manejar entradas no previstas, como nombres propios y variaciones de escritura tipo leet speak, gracias a un pipeline de preprocesamiento basado en normalización, detección y posterior deletreo. Estas capacidades reforzaron el comportamiento robusto del sistema ante textos reales y ruidosos.\\

Desde el punto de vista arquitectónico, la aplicación móvil se desarrolló con React Expo, garantizando compatibilidad multiplataforma para Android e iOS. En el backend, la adopción de un monolito modular —en lugar de una arquitectura de microservicios— resultó adecuada para el tamaño y alcance del proyecto. La separación por capas (Presentación, Servicios y Datos), junto con módulos internos bien definidos (API, motor de PLN, normalizador y gestor de grupos), permitió mantener un código ordenado, testeable y con bajo acoplamiento. Esta organización también deja abierta la posibilidad de migrar a microservicios si en un futuro aumenta el tráfico o el equipo de desarrollo crece.\\

El proyecto presentó retos significativos asociados a la interpretación semántica, la escalabilidad y el manejo de entradas atípicas. La transición de una búsqueda lineal O(N) a una búsqueda jerárquica O(K+M) basada en grupos y centroides redujo costos computacionales y mejoró la consistencia de las clasificaciones. Asimismo, el ajuste de thresholds por grupo temático permitió disminuir falsos positivos. Estos resultados mostraron que los hiperparámetros no deben considerarse valores estáticos, sino decisiones que deben calibrarse con datos reales.\\

En el ámbito metodológico, el proyecto transformó la visión sobre el aseguramiento de calidad. Inicialmente se priorizó la cobertura de código, pero posteriormente se incorporó una batería de 168 casos de prueba que incluyó pruebas unitarias, end-to-end, semánticas y de rendimiento. Este enfoque permitió identificar problemas que no se manifestaban en pruebas aisladas y proporcionó una visión integral del comportamiento del pipeline completo.\\

A pesar de los resultados positivos, el proyecto reconoce varias limitaciones. El dataset utilizado, aunque adecuado para validar el funcionamiento del sistema, sigue siendo reducido para una aplicación de uso cotidiano, que requeriría entre 500 y 1000 frases. Tampoco se realizó una validación exhaustiva con la comunidad sorda, por lo que es necesario evaluar la velocidad de los videos, las variaciones dialectales de la LSM y la usabilidad del prototipo en contextos reales. Asimismo, la arquitectura monolítica y el uso de un modelo de lenguaje estático limitan el potencial de adaptación dinámica del sistema.\\

Estas limitaciones establecen rutas claras para el trabajo futuro: ampliar el dataset en colaboración con intérpretes de LSM; evaluar modelos más potentes o especializados; integrar índices vectoriales como FAISS; explorar arquitecturas más escalables; y añadir nuevas funciones, como modos conversacionales, personalización del usuario y herramientas de aprendizaje gamificadas. Estas mejoras son coherentes con la estructura actual y permitirían avanzar hacia un sistema más completo y aplicable en escenarios reales.\\

Finalmente, el proyecto evidenció que construir un sistema funcional implica considerar tanto los aspectos técnicos como las necesidades de los usuarios finales. Más allá de diseñar algoritmos eficientes, fue indispensable tomar decisiones fundamentadas, documentarlas con rigor y validar el sistema con escenarios que representaran situaciones reales. El resultado es un prototipo funcional orientado a la accesibilidad y la inclusión de personas sordas usuarias de LSM. Si esta base técnica contribuye en el futuro al desarrollo de herramientas más amplias o inspira nuevas soluciones para mejorar la comunicación y la inclusión, el propósito de este trabajo terminal habrá sido plenamente alcanzado.\\
\chapter{Trabajo a Futuro}
\section{Trabajo a Futuro}
Como parte del trabajo a futuro, se contempla la ampliación del conjunto de escenarios y situaciones incluidas en el prototipo, incorporando nuevos grupos de frases que abarquen contextos más diversos de la vida cotidiana, con el fin de mejorar la cobertura comunicativa y la utilidad de la aplicación.\\

Asimismo, se considera retomar la integración de modelos de animación 3D, tal como se planteó en la propuesta original, lo cual implicaría la adquisición de un traje de captura de movimiento (motion capture) y la colaboración de un equipo especializado en animación para procesar los movimientos obtenidos en Blender y posteriormente exportarlos a la aplicación móvil. \\

Por último, se proyecta la implementación de nuevas funcionalidades que fortalezcan la interacción del usuario, como un módulo de reconocimiento de voz capaz de transcribir audio a texto, o el desarrollo de una red neuronal convencional que permita identificar letras del alfabeto mediante visión artificial, ampliando así las capacidades del sistema y su aumentar su potencial como herramienta de apoyo inclusiva.\\

Se espera que el presente Trabajo Terminal siente las bases para futuras investigaciones orientadas al desarrollo de sistemas más completos de traducción entre el español y la LSM, con miras a construir en el largo plazo un sistema de traducción general que permita una comunicación bidireccional y natural entre personas oyentes y personas con discapacidad auditiva.
\appendix

\chapter[Anexo A. MediaPipe, Blender y Unity]{MediaPipe, Blender y Unity}
\label{anexo:blender_unity}

Las herramientas y tecnologías descritas en este anexo formaron parte del enfoque inicial propuesto para el desarrollo del prototipo. Sin embargo, durante el desarrollo del proyecto se identificaron diversas limitaciones técnicas, de tiempo y de recursos que impidieron su implementación, tales como la necesidad de un equipo especializado en animación, disponibilidad de hardware para captura de movimiento y la alta complejidad del modelado 3D para representar señas detalladas de la Lengua de Señas Mexicana (LSM). Se realizaron algunas modificaciones a la propuesta original con el fin de cubrir la mayor parte de los objetivos, enfocándose principalmente en la fluidez y comunicación del español a LSM, y se optó por reubicar estos elementos a los anexos, manteniéndolos como referencia conceptual del planteamiento original y como base para futuros trabajos que deseen retomar esta línea de desarrollo.\\

\section{MediaPipe}
MediaPipe es un conjunto de herramientas de código abierto para ser empleadas en tareas como el reconocimiento facial, seguimiento de gestos, detección de objetos y el seguimiento del cuerpo humano \cite{ref49}.

\begin{center}
    \includegraphics[width=0.6\textwidth]{Images/Cap 2/MediaPipeLogo.jpeg}
    \captionof{figure}[Logo de Mediapipe]{Logo de Mediapipe, obtenido de \cite{ref50}.} 
\end{center}

\subsection{Herramientas de MediaPipe}
Las principales herramientas que ofrece MediaPipe son:
\begin{itemize}
    \item \textbf{MediaPipe Detección de caras}: permite detectar y seguir rostros de una imagen o vídeos en tiempo real, empleando técnicas de \textit{machine learning} para mejorar la precisión \cite{ref49}.
    \item \textbf{Malla facial MediaPipe}: proporciona una malla 3D del rostro, para proporcionar información precisa sobre los rasgos faciales, lo cuál es útil en aplicaciones de animación y modelado 3D \cite{ref49}.
    \item \textbf{MediaPipe Hands}: con esta herramienta se puede detectar y seguir los movimientos de la mano en tiempo real, con alta precisión \cite{ref49}.
    \item \textbf{MediaPipe Holistic}: combina la detección facial, el seguimiento de manos y el seguimiento corporal en una sola herramienta integrada, lo que es útil para aplicaciones de realidad aumentada y juegos \cite{ref49}.
    \item \textbf{MediaPipe Objectron}: es una herramienta para detectar y seguir objetos 3D en el espacio, siendo útil para comprender e interactuar con objetos reales en un entorno virtual \cite{ref49}.
    \item \textbf{Segmentación MediaPipe Selfie}: permite segmentar a las personas en el fondo de una imagen o vídeo \cite{ref49}.
    \item \textbf{MediaPipe Pose}: detecta las posturas del cuerpo humano, proporcionando información sobre las posiciones de las articulaciones y las extremidades \cite{ref49}.
    \item \textbf{Reconocimiento de gestos MediaPipe}: herramienta empleada en el reconocimiento de gestos de la mano para interacciones intuitivas y control de gestos \cite{ref49}.
    \item \textbf{MediaPipe EfficientDet}: mediante el uso de Redes Neuronales rápidas y eficaces, se puede mejorar la detección y localización de objetos en imágenes \cite{ref49}. 

\end{itemize}

\newpage
\subsection{MediaPipe Hands}
MediaPipe Hands es una herramienta que permite el seguimiento en tiempo real de manos y dedos mediante el uso de técnicas de \textit{Machine Learning} (ML), logrando detectar 21 puntos de referencia tridimensionales (3D) a partir de una sola imagen, incluso en dispositivos móviles \cite{ref51}.\\

\begin{center}
    \includegraphics[width=0.9\textwidth]{Images/Cap 2/MediaPipe_Hands.png}
    \captionof{figure}[MediaPipe Hands]{MediaPipe Hands, obtenido de \cite{ref51}.}  % Pie de foto manual
\end{center}

Este sistema funciona mediante un \textit{pipeline} compuesto por dos modelos que trabajan de manera conjunta \cite{ref51}:

\begin{enumerate}
    \item \textbf{El modelo de detección de palmas}: analiza la imagen para localizar y delimitar la región donde se encuentra la mano, generando un cuadro delimitador orientado.
    \item \textbf{El modelo de estimación de puntos clave}: toma como entrada la región definida por el modelo anterior y predice las coordenadas 3D de 21 puntos clave (nudillos y articulaciones) de la mano.
\end{enumerate}

\begin{center}
\includegraphics[width=0.9\textwidth]{Images/Cap 2/MediaPipe_hand_landmarks.png}
\captionof{figure}[Listado de los 21 puntos clave de la mano que son detectados por el modelo de estimación de puntos clave]{Listado de los 21 puntos clave de la mano que son detectados por el modelo de estimación de puntos clave, obtenido de \cite{ref52}.}  % Pie de foto manual
\end{center}


Para entrenar el modelo de estimación de puntos clave, se utilizaron aproximadamente 30,000 imágenes reales junto con modelos sintéticos de manos, superpuestos sobre distintos fondos \cite{ref52}.\\

Debido a que la detección de la palma es más costosa computacionalmente, en flujos de video continuo el sistema optimiza su rendimiento reutilizando la región de la mano previamente detectada por el modelo de estimación de puntos clave. Solo en caso de perder la mano del encuadre o de no poder hacer un seguimiento adecuado, el sistema vuelve a activar el modelo de detección de palmas. Esto permite reducir significativamente las llamadas a este último modelo, mejorando la eficiencia general del sistema \cite{ref52}.\\

\subsection{MediaPipe Pose}
Por su parte, MediaPipe Pose permite detectar puntos de referencia de cuerpos humanos en una imagen o vídeo. Se emplea principalmente para identificar ubicaciones claves del cuerpo, analizar la postura y categorizar los movimientos \cite{ref53}.\\

El marcador de poses emplea una serie de modelos para predecir los marcadores de poses \cite{ref53}:
\begin{itemize}

    \item \textbf{Modelo de detección de poses}: detectar la presencia de cuerpos con algunos puntos de referencia de poses clave.
    \item \textbf{Modelo de marcador de pose}: agregar una asignación completa de una pose, en la que se generan 33 puntos de referencia de la pose 3D.

\end{itemize}
El modelo de marcador de pose realiza un seguimiento de 33 ubicaciones de puntos de referencia del cuerpo.
\begin{center}
    \includegraphics[width=0.8\textwidth]{Images/Cap 2/MediaPipe_Pose.png}
    \captionof{figure}[Ubicaciones de puntos de referencia del cuerpo]{Ubicaciones de puntos de referencia del cuerpo, obtenido de \cite{ref53}.}  % Pie de foto manual
\end{center}

A continuación, se enlistan las partes representadas del cuerpo:

\begin{enumerate}
    \item Nose - nariz.  
    \item Left eye (inner) - ojo izquierdo (interior).  
    \item Left eye - ojo izquierdo.  
    \item Left eye (outer) - ojo izquierdo (exterior).  
    \item Right eye (inner) - ojo derecho (interior).  
    \item Right eye - ojo derecho.  
    \item Right eye (outer) - ojo derecho (exterior).  
    \item Left ear - oreja izquierda.  
    \item Right ear - oreja derecha.  
    \item Mouth (left) - boca (izquierda).  
    \item Mouth (right) - boca (derecha).  
    \item Left shoulder - hombro izquierdo.  
    \item Right shoulder - hombro derecho.  
    \item Left elbow - codo izquierdo.  
    \item Right elbow - codo derecho.  
    \item Left wrist - muñeca izquierda.  
    \item Right wrist - muñeca derecha.  
    \item Left pinky - meñique izquierdo.  
    \item Right pinky - meñique derecho.  
    \item Left index - índice izquierdo.  
    \item Right index - índice derecho.  
    \item Left thumb - pulgar izquierdo.  
    \item Right thumb - pulgar derecho.  
    \item Left hip - cadera izquierda.  
    \item Right hip - cadera derecha.  
    \item Left knee - rodilla izquierda.  
    \item Right knee - rodilla derecha.  
    \item Left ankle - tobillo izquierdo.  
    \item Right ankle - tobillo derecho.  
    \item Left heel - talón izquierdo.  
    \item Right heel - talón derecho.  
    \item Left foot index - punta del pie izquierdo.  
    \item Right foot index - punta del pie derecho.  
\end{enumerate}

MediaPipe suele ser empleado en conjunto con plataformas y motores gráficos, como pueden ser Blender y Unity, para la creación de modelos 3D. En el siguiente apartado se revisará al motor gráfico Unity, enfocado principalmente en el desarrollo de modelos 3D.\\

\section{Modelado de Animaciones 3D}
El término animación 3D se refiere a la técnica de animación empleada para desplazar modelos tridimensionales generados digitalmente, sirviéndose para ello de un eje de coordenadas cartesiano virtual \cite{ref54}.\\

La animación 3D ha estado históricamente más orientada a la replicación de la física del mundo real, ya que representa con total libertad la fuerza de gravedad, la inercia o la masa de cuerpos \cite{ref54}.

\subsection{Unity}
Unity es una plataforma para el desarrollo de videojuegos y aplicaciones interactivas, que ofrece una amplia variedad de herramientas y recursos para crear experiencias visuales y funcionales \cite{ref55}. Es un motor gráfico empleado para desarrollar videojuegos, aplicaciones interactivas en 2D, 3D, realidad aumentada (AR) y realidad virtual (VR).\\

\begin{center}
    \includegraphics[width=0.6\textwidth]{Images/Cap 2/Unity_Logo.png}
    \captionof{figure}[Logo de Unity]{Logo de Unity, obtenido de \cite{ref56}.} 
\end{center}

Unity destaca por su conjunto de características robustas que facilitan el desarrollo de aplicaciones interactivas de alta calidad para la simulación física y el rendering, las cuáles requieren visualización y experiencia de usuario de alta calidad \cite{ref55}.\\

En la actualidad Unity es empleado en múltiples industrias, además del desarrollo de videojuegos, ya que es popular en sectores como la arquitectura, el diseño automotriz, la medicina y la educación. Además, tiene soporte en varias plataformas como computadoras (PC), consolas, dispositivos móviles y dispositivos de realidad aumentada \cite{ref55}. La última versión que se ha lanzado de Unity, al momento de la realización de este trabajo, es la 6.1.\\

Considerando que Unity tiene compatibilidad con dispositivos móviles, en el siguiente apartado se hará un breve análisis de Android, un sistema operativo móvil que es ampliamente utilizado en smartphones.\\


\chapter[Anexo B. Ley General para la Inclusión de las Personas con Discapacidad]{Ley General para la Inclusión de las Personas con Discapacidad}
\label{anexo:ley_inclusion_disc}
\section{Encabezado de la Ley General para la Inclusión de las Personas con Discapacidad}

\begin{center}
	\makebox[\textwidth]{%
		\includegraphics[width=1\textwidth]{Images/Anexos/Encabezado_Ley.png}
	}
    \captionof{figure}[Encabezado de la Ley General para la Inclusión de las Personas con Discapacidad]{Encabezado de la Ley General para la Inclusión de las Personas con Discapacidad, obtenido de \cite{ref34}}
\end{center}

\section{Artículo 2, Fracción XXII}
\begin{center}
	\makebox[\textwidth]{%
		\includegraphics[width=1\textwidth]{Images/Anexos/Art2_FraccXXII.png}
	}
    \captionof{figure}[Artículo 2, Fracción XXII, de la Ley General para la Inclusión de las Personas con Discapacidad]{Artículo 2, Fracción XXII, de la Ley General para la Inclusión de las Personas con Discapacidad, obtenido de \cite{ref34}}
\end{center}

\section{Artículo 20}
\begin{center}
	\makebox[\textwidth]{%
		\includegraphics[width=1\textwidth]{Images/Anexos/Art20.png}
	}
    \captionof{figure}[Artículo 20]{Artículo 20 de la Ley General para la Inclusión de las Personas con Discapacidad, obtenido de \cite{ref34}}
\end{center}

\chapter[Anexo C. Enfoque por actividades (académico)]{Enfoque por actividades (académico)}
\label{anexo:actividades_academicas}  % Etiqueta para hacer referencia
\section{Etapa: creación del prototipo}

\begin{table}[H]
	\centering
	\renewcommand{\arraystretch}{1.6}
	\setlength{\tabcolsep}{10pt}
	\Huge
	\begin{adjustbox}{max width=\textwidth}
		\begin{tabular}{|p{8cm}|c|r|r|}
			\hline
			\textbf{Tareas (Formulación del proyecto)} & \textbf{Horas} & \textbf{Costo por hora (MXN \$)} & \textbf{Costo total (MXN \$)} \\ \hline
			Descripción del proyecto & 1 & \$150.00 & \$150.00 \\ \hline
			Definir el propósito del proyecto & 1 & \$150.00 & \$150.00 \\ \hline
			Planificación del alcance del proyecto & 1 & \$150.00 & \$150.00 \\ \hline
			Definir las actividades necesarias para completar el proyecto & 1 & \$150.00 & \$150.00 \\ \hline
			Definir tareas prioritarias y bloques de trabajo en paralelo & 2 & \$150.00 & \$300.00 \\ \hline
			Estimar recursos y operaciones & 3 & \$150.00 & \$450.00 \\ \hline
			Establecer los objetivos y metas principales & 1 & \$150.00 & \$150.00 \\ \hline
			Identificación de actividades y tareas & 5 & \$150.00 & \$750.00 \\ \hline
			Planificación del cronograma de actividades & 8 & \$150.00 & \$1,200.00 \\ \hline
			\textbf{Total} & \textbf{23} & -- & \textbf{\$3,450.00} \\ \hline
		\end{tabular}
	\end{adjustbox}
	\caption[Costos estimados para la fase de formulación del proyecto]{Costos estimados para la fase de formulación del proyecto, elaboración propia.} 	
	\label{tab:costos_formulacion_nuevo}
\end{table}


\begin{table}[H]
	\centering
	\renewcommand{\arraystretch}{1.6}
	\setlength{\tabcolsep}{10pt}
	\Huge
	\begin{adjustbox}{max width=\textwidth}
		\begin{tabular}{|p{9.5cm}|c|r|r|}
			\hline
			\textbf{Tareas (Análisis del proyecto)} & \textbf{Horas} & \textbf{Costo por hora (MXN \$)} & \textbf{Costo total (MXN \$)} \\ \hline
			Definición de actores & 3 & \$150.00 & \$450.00 \\ \hline
			Análisis funcional y no funcional & 8 & \$150.00 & \$1,200.00 \\ \hline
			Creación de documentación de requerimientos & 5 & \$150.00 & \$750.00 \\ \hline
			Diagrama de casos de uso & 3 & \$150.00 & \$450.00 \\ \hline
			Diseño de pantallas (mockups) & 10 & \$150.00 & \$1,500.00 \\ \hline
			Análisis de viabilidad y factibilidad & 3 & \$150.00 & \$450.00 \\ \hline
			Análisis financiero & 5 & \$150.00 & \$750.00 \\ \hline
			Análisis de riesgos del proyecto & 5 & \$150.00 & \$750.00 \\ \hline
			Documentar los requisitos de alto nivel y entregables del proyecto & 5 & \$150.00 & \$750.00 \\ \hline
			Priorización de módulos según importancia y complejidad & 1 & \$150.00 & \$150.00 \\ \hline
			\textbf{Total} & \textbf{58} & -- & \textbf{\$7,450.00} \\ \hline
		\end{tabular}
	\end{adjustbox}
	\caption[Costos estimados para la fase de análisis del proyecto]{Costos estimados para la fase de análisis del proyecto, elaboración propia.} 	
	\label{tab:costos_analisis_nuevo}
\end{table}

\begin{table}[H]
	\centering
	\renewcommand{\arraystretch}{1.6}
	\setlength{\tabcolsep}{10pt}
	\Huge
	\begin{adjustbox}{max width=\textwidth}
		\begin{tabular}{|p{9.5cm}|c|r|r|}
			\hline
			\textbf{Tareas (Análisis de riesgos)} & \textbf{Horas} & \textbf{Costo por hora (MXN \$)} & \textbf{Costo total (MXN \$)} \\ \hline
			Realizar análisis cualitativo y cuantitativo de riesgos & 4 & \$150.00 & \$600.00 \\ \hline
			Planificar respuestas a los riesgos & 2 & \$150.00 & \$300.00 \\ \hline
			\textbf{Total} & \textbf{6} & -- & \textbf{\$900.00} \\ \hline
		\end{tabular}
	\end{adjustbox}
	\caption[Costos estimados para la fase de análisis de riesgos]{Costos estimados para la fase de análisis de riesgos, elaboración propia.} 	
	\label{tab:costos_riesgos_nuevo}
\end{table}

\begin{table}[H]
	\centering
	\renewcommand{\arraystretch}{1.6}
	\setlength{\tabcolsep}{10pt}
	\Huge
	\begin{adjustbox}{max width=\textwidth}
		\begin{tabular}{|p{9.5cm}|c|r|r|}
			\hline
			\textbf{Tareas (Elaboración de presupuesto)} & \textbf{Horas} & \textbf{Costo por hora (MXN \$)} & \textbf{Costo total (MXN \$)} \\ \hline
			Cotización simbólica de recursos & 5 & \$150.00 & \$750.00 \\ \hline
			Estimación de costos por actividades & 10 & \$150.00 & \$1,500.00 \\ \hline
			Estimación de costos por recursos & 10 & \$150.00 & \$1,500.00 \\ \hline
			\textbf{Total} & \textbf{25} & -- & \textbf{\$3,750.00} \\ \hline
		\end{tabular}
	\end{adjustbox}
	\caption[Costos estimados para la fase de elaboración de presupuesto]{Costos estimados para la fase de elaboración de presupuesto, elaboración propia.} 	
	\label{tab:costos_presupuesto_nuevo}
\end{table}

\begin{table}[H]
	\centering
	\renewcommand{\arraystretch}{1.6}
	\setlength{\tabcolsep}{10pt}
	\Huge
	\begin{adjustbox}{max width=\textwidth}
		\begin{tabular}{|p{9.5cm}|c|r|r|}
			\hline
			\textbf{Tareas (Desarrollo del producto)} & \textbf{Horas} & \textbf{Costo por hora (MXN \$)} & \textbf{Costo total (MXN \$)} \\ \hline
			Definición de la arquitectura básica & 15 & \$150.00 & \$2,250.00 \\ \hline
			Diseño de interfaces de usuario (UI/UX) para cada módulo & 20 & \$150.00 & \$3,000.00 \\ \hline
			Obtención del conjunto de datos & 12 & \$150.00 & \$1,800.00 \\ \hline
			Diagramas de flujo y secuencia & 12 & \$150.00 & \$1,800.00 \\ \hline
			Diagramas correspondientes UML & 20 & \$150.00 & \$3,000.00 \\ \hline
			Integración de APIs externas & 20 & \$150.00 & \$3,000.00 \\ \hline
			Integración backend y frontend & 25 & \$150.00 & \$3,750.00 \\ \hline
			\textbf{Total} & \textbf{124} & -- & \textbf{\$18,600.00} \\ \hline
		\end{tabular}
	\end{adjustbox}
	\caption[Costos estimados para la fase de desarrollo del producto]{Costos estimados para la fase de desarrollo del producto, elaboración propia.} 	
	\label{tab:costos_desarrollo_nuevo}
\end{table}


\section{Etapa: despliegue del prototipo}
\begin{table}[H]
	\centering
	\renewcommand{\arraystretch}{1.6}
	\setlength{\tabcolsep}{10pt}
	\Huge
	\begin{adjustbox}{max width=\textwidth}
		\begin{tabular}{|p{9.5cm}|c|r|r|}
			\hline
			\textbf{Tareas (Gestión de calidad)} & \textbf{Horas} & \textbf{Costo por hora (MXN \$)} & \textbf{Costo total (MXN \$)} \\ \hline
			Definir los estándares de calidad aplicables al proyecto & 8 & \$150.00 & \$1,200.00 \\ \hline
			Identificar métricas de calidad & 6 & \$150.00 & \$900.00 \\ \hline
			Realizar procedimientos de control de calidad & 10 & \$150.00 & \$1,500.00 \\ \hline
			\textbf{Total} & \textbf{24} & -- & \textbf{\$3,600.00} \\ \hline
		\end{tabular}
	\end{adjustbox}
	\caption[Costos estimados para la fase de gestión de calidad]{Costos estimados para la fase de gestión de calidad, elaboración propia.} 	
	\label{tab:costos_calidad_nuevo}
\end{table}

\begin{table}[H]
	\centering
	\renewcommand{\arraystretch}{1.6}
	\setlength{\tabcolsep}{10pt}
	\Huge
	\begin{adjustbox}{max width=\textwidth}
		\begin{tabular}{|p{9.5cm}|c|r|r|}
			\hline
			\textbf{Tareas (Gestión de clientes)} & \textbf{Horas} & \textbf{Costo por hora (MXN \$)} & \textbf{Costo total (MXN \$)} \\ \hline
			Identificar y analizar las partes interesadas de la comunidad & 5 & \$150.00 & \$750.00 \\ \hline
			Desarrollar y mantener la comunicación con la comunidad & 8 & \$150.00 & \$1,200.00 \\ \hline
			Identificar a todos los interesados & 5 & \$150.00 & \$750.00 \\ \hline
			Resolver conflictos con clientes & 10 & \$150.00 & \$1,500.00 \\ \hline
			\textbf{Total} & \textbf{28} & -- & \textbf{\$4,200.00} \\ \hline
		\end{tabular}
	\end{adjustbox}
	\caption[Costos estimados para la fase de gestión de clientes]{Costos estimados para la fase de gestión de clientes, elaboración propia.} 	
	\label{tab:costos_clientes_nuevo}
\end{table}

\begin{table}[H]
	\centering
	\renewcommand{\arraystretch}{1.6}
	\setlength{\tabcolsep}{10pt}
	\Huge
	\begin{adjustbox}{max width=\textwidth}
		\begin{tabular}{|p{9.5cm}|c|r|r|}
			\hline
			\textbf{Tareas (Gestión de adquisiciones)} & \textbf{Horas} & \textbf{Costo por hora (MXN \$)} & \textbf{Costo total (MXN \$)} \\ \hline
			Planificar futuras compras y adquisiciones & 8 & \$150.00 & \$1,200.00 \\ \hline
			\textbf{Total} & \textbf{8} & -- & \textbf{\$1,200.00} \\ \hline
		\end{tabular}
	\end{adjustbox}
	\caption[Costos estimados para la fase de gestión de adquisiciones]{Costos estimados para la fase de gestión de adquisiciones, elaboración propia.} 	
	\label{tab:costos_adquisiciones_nuevo}
\end{table}

\begin{table}[H]
	\centering
	\renewcommand{\arraystretch}{1.6}
	\setlength{\tabcolsep}{10pt}
	\Huge
	\begin{adjustbox}{max width=\textwidth}
		\begin{tabular}{|p{9.5cm}|c|r|r|}
			\hline
			\textbf{Tareas (Gestión de integración)} & \textbf{Horas} & \textbf{Costo por hora (MXN \$)} & \textbf{Costo total (MXN \$)} \\ \hline
			Desarrollar el plan de gestión del proyecto & 15 & \$150.00 & \$2,250.00 \\ \hline
			Dirigir y gestionar el trabajo del proyecto & 20 & \$150.00 & \$3,000.00 \\ \hline
			Monitorear y controlar el trabajo del proyecto & 15 & \$150.00 & \$2,250.00 \\ \hline
			\textbf{Total} & \textbf{50} & -- & \textbf{\$7,500.00} \\ \hline
		\end{tabular}
	\end{adjustbox}
	\caption[Costos estimados para la fase de gestión de integración]{Costos estimados para la fase de gestión de integración, elaboración propia.} 	
	\label{tab:costos_integracion_nuevo}
\end{table}

\begin{table}[H]
	\centering
	\renewcommand{\arraystretch}{1.6}
	\setlength{\tabcolsep}{10pt}
	\Huge
	\begin{adjustbox}{max width=\textwidth}
		\begin{tabular}{|p{9.5cm}|c|r|r|}
			\hline
			\textbf{Tareas (Pruebas)} & \textbf{Horas} & \textbf{Costo por hora (MXN \$)} & \textbf{Costo total (MXN \$)} \\ \hline
			Costo de las pruebas iniciales solo con desarrolladores & 10 & \$150.00 & \$1,500.00 \\ \hline
			Pruebas unitarias para cada módulo & 20 & \$150.00 & \$3,000.00 \\ \hline
			Pruebas de integración & 10 & \$150.00 & \$1,500.00 \\ \hline
			Pruebas con usuarios para validar la usabilidad & 10 & \$150.00 & \$1,500.00 \\ \hline
			\textbf{Total} & \textbf{50} & -- & \textbf{\$7,500.00} \\ \hline
		\end{tabular}
	\end{adjustbox}
	\caption[Costos estimados para la fase de pruebas]{Costos estimados para la fase de pruebas, elaboración propia.} 	
	\label{tab:costos_pruebas_nuevo}
\end{table}

\begin{table}[H]
	\centering
	\renewcommand{\arraystretch}{1.6}
	\setlength{\tabcolsep}{10pt}
	\Huge
	\begin{adjustbox}{max width=\textwidth}
		\begin{tabular}{|p{9.5cm}|c|r|r|}
			\hline
			\textbf{Tareas (Lanzamiento)} & \textbf{Horas} & \textbf{Costo por hora (MXN \$)} & \textbf{Costo total (MXN \$)} \\ \hline
			Preparación del entorno de producción & 10 & \$150.00 & \$1,500.00 \\ \hline
			\textbf{Total} & \textbf{10} & -- & \textbf{\$1,500.00} \\ \hline
		\end{tabular}
	\end{adjustbox}
	\caption[Costos estimados para la fase de lanzamiento]{Costos estimados para la fase de lanzamiento, elaboración propia.} 	
	\label{tab:costos_lanzamiento_nuevo}
\end{table}


\section{Etapa: costo de venta del prototipo}

\begin{table}[H]
	\centering
	\renewcommand{\arraystretch}{1.6}
	\setlength{\tabcolsep}{10pt}
	\Huge
	\begin{adjustbox}{max width=\textwidth}
		\begin{tabular}{|p{9.5cm}|c|r|r|}
			\hline
			\textbf{Tareas (Manual de usuario)} & \textbf{Horas} & \textbf{Costo por hora (MXN \$)} & \textbf{Costo total (MXN \$)} \\ \hline
			Creación de guías paso a paso para cada módulo & 12 & \$150.00 & \$1,800.00 \\ \hline
			Instrucciones claras y visuales para usuarios no técnicos & 10 & \$150.00 & \$1,500.00 \\ \hline
			\textbf{Total} & \textbf{22} & -- & \textbf{\$3,300.00} \\ \hline
		\end{tabular}
	\end{adjustbox}
	\caption[Costos estimados para la fase de elaboración del manual de usuario]{Costos estimados para la fase de elaboración del manual de usuario, elaboración propia.} 	
	\label{tab:costos_manual_nuevo}
\end{table}

\begin{table}[H]
	\centering
	\renewcommand{\arraystretch}{1.6}
	\setlength{\tabcolsep}{10pt}
	\Huge
	\begin{adjustbox}{max width=\textwidth}
		\begin{tabular}{|p{9.5cm}|c|r|r|}
			\hline
			\textbf{Tareas (Manual técnico)} & \textbf{Horas} & \textbf{Costo por hora (MXN \$)} & \textbf{Costo total (MXN \$)} \\ \hline
			Documentación de la arquitectura del sistema & 8 & \$150.00 & \$1,200.00 \\ \hline
			Descripción del conjunto de datos y APIs & 10 & \$150.00 & \$1,500.00 \\ \hline
			\textbf{Total} & \textbf{18} & -- & \textbf{\$2,700.00} \\ \hline
		\end{tabular}
	\end{adjustbox}
	\caption[Costos estimados para la fase de elaboración del manual técnico]{Costos estimados para la fase de elaboración del manual técnico, elaboración propia.} 	
	\label{tab:costos_manual_tecnico_nuevo}
\end{table}

\begin{table}[H]
	\centering
	\renewcommand{\arraystretch}{1.6}
	\setlength{\tabcolsep}{10pt}
	\Huge
	\begin{adjustbox}{max width=\textwidth}
		\begin{tabular}{|p{9.5cm}|c|r|r|}
			\hline
			\textbf{Tareas (Documentación)} & \textbf{Horas} & \textbf{Costo por hora (MXN \$)} & \textbf{Costo total (MXN \$)} \\ \hline
			Documentación de requerimientos & 15 & \$150.00 & \$2,250.00 \\ \hline
			Documentación de pruebas & 6 & \$150.00 & \$900.00 \\ \hline
			Manuales de usuario y técnico & 8 & \$150.00 & \$1,200.00 \\ \hline
			\textbf{Total} & \textbf{29} & -- & \textbf{\$4,350.00} \\ \hline
		\end{tabular}
	\end{adjustbox}
	\caption[Costos estimados para la fase de documentación]{Costos estimados para la fase de documentación, elaboración propia.} 	
	\label{tab:costos_documentacion_nuevo}
\end{table}

\begin{table}[H]
	\centering
	\renewcommand{\arraystretch}{1.6}
	\setlength{\tabcolsep}{10pt}
	\Huge
	\begin{adjustbox}{max width=\textwidth}
		\begin{tabular}{|p{9.5cm}|c|r|r|}
			\hline
			\textbf{Tareas (Presupuesto de ingresos)} & \textbf{Horas} & \textbf{Costo por hora (MXN \$)} & \textbf{Costo total (MXN \$)} \\ \hline
			Estimación de precio de producto final & 6 & \$150.00 & \$900.00 \\ \hline
			\textbf{Total} & \textbf{6} & -- & \textbf{\$900.00} \\ \hline
		\end{tabular}
	\end{adjustbox}
	\caption[Costos estimados para la fase de presupuesto de ingresos]{Costos estimados para la fase de presupuesto de ingresos, elaboración propia.} 	
	\label{tab:costos_presupuesto_ingresos}
\end{table}

\begin{table}[H]
	\centering
	\renewcommand{\arraystretch}{1.6}
	\setlength{\tabcolsep}{10pt}
	\Huge
	\begin{adjustbox}{max width=\textwidth}
		\begin{tabular}{|p{9.5cm}|c|r|r|}
			\hline
			\textbf{Tareas (Estados financieros)} & \textbf{Horas} & \textbf{Costo por hora (MXN \$)} & \textbf{Costo total (MXN \$)} \\ \hline
			Revisión de costos y gastos iniciales & 1 & \$150.00 & \$150.00 \\ \hline
			Proyección de ingresos & 2 & \$150.00 & \$300.00 \\ \hline
			\textbf{Total} & \textbf{3} & -- & \textbf{\$450.00} \\ \hline
		\end{tabular}
	\end{adjustbox}
	\caption[Costos estimados para la fase de estados financieros]{Costos estimados para la fase de estados financieros, elaboración propia.} 	
	\label{tab:costos_estados_financieros}
\end{table}

\chapter[Anexo D. Enfoque por actividades (comercial)]{Enfoque por actividades (comercial)}
\label{anexo:actividades_comercial}  % Etiqueta para hacer referencia
\section{Etapa: creación del prototipo}
\begin{table}[H]
	\centering
	\renewcommand{\arraystretch}{1.6}
	\setlength{\tabcolsep}{10pt}
	\Huge
	\begin{adjustbox}{max width=\textwidth}
		\begin{tabular}{|p{9.5cm}|c|r|r|}
			\hline
			\textbf{Tareas (Formulación del proyecto)} & \textbf{Horas} & \textbf{Costo por hora (MXN \$)} & \textbf{Costo total (MXN \$)} \\ \hline
			Descripción del proyecto & 1 & \$280.00 & \$280.00 \\ \hline
			Definir el propósito del proyecto & 1 & \$280.00 & \$280.00 \\ \hline
			Planificación del alcance del proyecto & 1 & \$280.00 & \$280.00 \\ \hline
			Definir las actividades necesarias para completar el proyecto & 1 & \$280.00 & \$280.00 \\ \hline
			Definir tareas prioritarias y bloques de trabajo en paralelo & 2 & \$280.00 & \$560.00 \\ \hline
			Estimar recursos y operaciones & 3 & \$260.00 & \$780.00 \\ \hline
			Establecer los objetivos y metas principales & 1 & \$280.00 & \$280.00 \\ \hline
			Identificación de actividades y tareas & 5 & \$280.00 & \$1,400.00 \\ \hline
			Planificación del cronograma de actividades & 8 & \$280.00 & \$2,240.00 \\ \hline
			\textbf{Total} & \textbf{23} & -- & \textbf{\$6,380.00} \\ \hline
		\end{tabular}
	\end{adjustbox}
	\caption[Costos estimados para la fase de formulación del proyecto (ajustada con nueva tarifa)]{Costos estimados para la fase de formulación del proyecto (ajustada con nueva tarifa), elaboración propia.} 	
	\label{tab:costos_formulacion_tarifa280}
\end{table}

\begin{table}[H]
	\centering
	\renewcommand{\arraystretch}{1.6}
	\setlength{\tabcolsep}{10pt}
	\Huge
	\begin{adjustbox}{max width=\textwidth}
		\begin{tabular}{|p{9.5cm}|c|r|r|}
			\hline
			\textbf{Tareas (Análisis de proyecto)} & \textbf{Horas} & \textbf{Costo por hora (MXN \$)} & \textbf{Costo total (MXN \$)} \\ \hline
			Definición de actores & 3 & \$260.00 & \$780.00 \\ \hline
			Análisis funcional y no funcional & 8 & \$260.00 & \$2,080.00 \\ \hline
			Creación de documentación de requerimientos & 5 & \$240.00 & \$1,200.00 \\ \hline
			Diagrama de casos de uso & 3 & \$260.00 & \$780.00 \\ \hline
			Diseño de pantallas (mockups) & 10 & \$300.00 & \$3,000.00 \\ \hline
			Análisis de viabilidad y factibilidad & 3 & \$260.00 & \$780.00 \\ \hline
			Análisis financiero & 5 & \$270.00 & \$1,350.00 \\ \hline
			Análisis de riesgos del proyecto & 5 & \$270.00 & \$1,350.00 \\ \hline
			Documentar los requisitos de alto nivel y entregables del proyecto & 5 & \$240.00 & \$1,200.00 \\ \hline
			Priorización de módulos según importancia y complejidad & 1 & \$260.00 & \$260.00 \\ \hline
			\textbf{Total} & \textbf{48} & -- & \textbf{\$12,780.00} \\ \hline
		\end{tabular}
	\end{adjustbox}
	\caption[Costos estimados para la fase de análisis de proyecto (con tarifas ajustadas)]{Costos estimados para la fase de análisis de proyecto (con tarifas ajustadas), elaboración propia.} 	
	\label{tab:costos_analisis_actualizado}
\end{table}

\begin{table}[H]
	\centering
	\renewcommand{\arraystretch}{1.6}
	\setlength{\tabcolsep}{10pt}
	\Huge
	\begin{adjustbox}{max width=\textwidth}
		\begin{tabular}{|p{9.5cm}|c|r|r|}
			\hline
			\textbf{Tareas (Análisis de riesgos)} & \textbf{Horas} & \textbf{Costo por hora (MXN \$)} & \textbf{Costo total (MXN \$)} \\ \hline
			Realizar análisis cualitativo y cuantitativo de riesgos & 4 & \$260.00 & \$1,040.00 \\ \hline
			Planificar respuestas a los riesgos & 2 & \$260.00 & \$520.00 \\ \hline
			Monitoreo de riesgos general & 10 & \$280.00 & \$2,800.00 \\ \hline
			\textbf{Total} & \textbf{16} & -- & \textbf{\$4,360.00} \\ \hline
		\end{tabular}
	\end{adjustbox}
	\caption[Costos estimados para la fase de análisis de riesgos (con tarifas ajustadas)]{Costos estimados para la fase de análisis de riesgos (con tarifas ajustadas), elaboración propia.} 
	\label{tab:costos_riesgos_actualizado}
\end{table}

\begin{table}[H]
	\centering
	\renewcommand{\arraystretch}{1.6}
	\setlength{\tabcolsep}{10pt}
	\Huge
	\begin{adjustbox}{max width=\textwidth}
		\begin{tabular}{|p{9.5cm}|c|r|r|}
			\hline
			\textbf{Tareas (Elaboración de presupuesto)} & \textbf{Horas} & \textbf{Costo por hora (MXN \$)} & \textbf{Costo total (MXN \$)} \\ \hline
			Cotización simbólica de recursos & 5 & \$270.00 & \$1,350.00 \\ \hline
			Estimación de costos por actividades & 10 & \$270.00 & \$2,700.00 \\ \hline
			Estimación de costos por recursos & 10 & \$270.00 & \$2,700.00 \\ \hline
			\textbf{Total} & \textbf{25} & -- & \textbf{\$6,750.00} \\ \hline
		\end{tabular}
	\end{adjustbox}
	\caption[Costos estimados para la fase de elaboración de presupuesto (con tarifas ajustadas)]{Costos estimados para la fase de elaboración de presupuesto (con tarifas ajustadas), elaboración propia.} 
	\label{tab:costos_presupuesto_actualizado}
\end{table}

\begin{table}[H]
	\centering
	\renewcommand{\arraystretch}{1.6}
	\setlength{\tabcolsep}{10pt}
	\Huge
	\begin{adjustbox}{max width=\textwidth}
		\begin{tabular}{|p{9.5cm}|c|r|r|}
			\hline
			\textbf{Tareas (Desarrollo del producto)} & \textbf{Horas} & \textbf{Costo por hora (MXN \$)} & \textbf{Costo total (MXN \$)} \\ \hline
			Definición de la arquitectura básica & 15 & \$320.00 & \$4,800.00 \\ \hline
			Diseño de interfaces de usuario (UI/UX) para cada módulo & 20 & \$300.00 & \$6,000.00 \\ \hline
			Obtención del conjunto de datos & 12 & \$260.00 & \$3,120.00 \\ \hline
			Diagramas de flujo y secuencia para cada funcionalidad & 12 & \$240.00 & \$2,880.00 \\ \hline
			Desarrollo de la funcionalidad de inicio de sesión y validación de credenciales & 20 & \$320.00 & \$6,400.00 \\ \hline
			Implementación del sistema de recuperación de contraseña & 15 & \$320.00 & \$4,800.00 \\ \hline
			Creación de la funcionalidad de registro de nuevos usuarios & 15 & \$320.00 & \$4,800.00 \\ \hline
			Integración de APIs externas & 20 & \$320.00 & \$6,400.00 \\ \hline
			Integración backend y frontend & 25 & \$320.00 & \$8,000.00 \\ \hline
			\textbf{Total} & \textbf{154} & -- & \textbf{\$47,200.00} \\ \hline
		\end{tabular}
	\end{adjustbox}
	\caption[Costos estimados para la fase de desarrollo del producto (con tarifas ajustadas)]{Costos estimados para la fase de desarrollo del producto (con tarifas ajustadas), elaboración propia.} 
	\label{tab:costos_desarrollo_actualizado}
\end{table}


\section{Etapa: despliegue del prototipo}

\begin{table}[H]
	\centering
	\renewcommand{\arraystretch}{1.6}
	\setlength{\tabcolsep}{10pt}
	\Huge
	\begin{adjustbox}{max width=\textwidth}
		\begin{tabular}{|p{9.5cm}|c|r|r|}
			\hline
			\textbf{Tareas (Gestión de calidad)} & \textbf{Horas} & \textbf{Costo por hora (MXN \$)} & \textbf{Costo total (MXN \$)} \\ \hline
			Definir los estándares de calidad aplicables al proyecto & 8 & \$280.00 & \$2,240.00 \\ \hline
			Identificar métricas de calidad & 6 & \$280.00 & \$1,680.00 \\ \hline
			Realizar procedimientos de control de calidad & 10 & \$260.00 & \$2,600.00 \\ \hline
			\textbf{Total} & \textbf{24} & -- & \textbf{\$6,520.00} \\ \hline
		\end{tabular}
	\end{adjustbox}
	\caption[Costos estimados para la fase de gestión de calidad (con tarifas ajustadas)]{Costos estimados para la fase de gestión de calidad (con tarifas ajustadas), elaboración propia.} 
	\label{tab:costos_calidad_actualizado}
\end{table}

\begin{table}[H]
	\centering
	\renewcommand{\arraystretch}{1.6}
	\setlength{\tabcolsep}{10pt}
	\Huge
	\begin{adjustbox}{max width=\textwidth}
		\begin{tabular}{|p{9.5cm}|c|r|r|}
			\hline
			\textbf{Tareas (Gestión de clientes)} & \textbf{Horas} & \textbf{Costo por hora (MXN \$)} & \textbf{Costo total (MXN \$)} \\ \hline
			Identificar y analizar las partes interesadas de la comunidad & 5 & \$260.00 & \$1,300.00 \\ \hline
			Desarrollar y mantener la comunicación con la comunidad & 8 & \$280.00 & \$2,240.00 \\ \hline
			Identificar a todos los interesados & 5 & \$260.00 & \$1,300.00 \\ \hline
			Resolver conflictos con clientes & 10 & \$280.00 & \$2,800.00 \\ \hline
			\textbf{Total} & \textbf{28} & -- & \textbf{\$7,640.00} \\ \hline
		\end{tabular}
	\end{adjustbox}
	\caption[Costos estimados para la fase de gestión de clientes (con tarifas ajustadas)]{Costos estimados para la fase de gestión de clientes (con tarifas ajustadas), elaboración propia.} 
	\label{tab:costos_clientes_actualizado}
\end{table}

\begin{table}[H]
	\centering
	\renewcommand{\arraystretch}{1.6}
	\setlength{\tabcolsep}{10pt}
	\Huge
	\begin{adjustbox}{max width=\textwidth}
		\begin{tabular}{|p{9.5cm}|c|r|r|}
			\hline
			\textbf{Tareas (Gestión de adquisiciones)} & \textbf{Horas} & \textbf{Costo por hora (MXN \$)} & \textbf{Costo total (MXN \$)} \\ \hline
			Planificar futuras compras y adquisiciones & 8 & \$280.00 & \$2,240.00 \\ \hline
			Seleccionar proveedores & 6 & \$280.00 & \$1,680.00 \\ \hline
			Administrar contratos con proveedores & 8 & \$280.00 & \$2,240.00 \\ \hline
			\textbf{Total} & \textbf{22} & -- & \textbf{\$6,160.00} \\ \hline
		\end{tabular}
	\end{adjustbox}
	\caption[Costos estimados para la fase de gestión de adquisiciones (con tarifas ajustadas)]{Costos estimados para la fase de gestión de adquisiciones (con tarifas ajustadas), elaboración propia.} 
	\label{tab:costos_adquisiciones_actualizado}
\end{table}

\begin{table}[H]
	\centering
	\renewcommand{\arraystretch}{1.6}
	\setlength{\tabcolsep}{10pt}
	\Huge
	\begin{adjustbox}{max width=\textwidth}
		\begin{tabular}{|p{9.5cm}|c|r|r|}
			\hline
			\textbf{Tareas (Gestión de regulaciones)} & \textbf{Horas} & \textbf{Costo por hora (MXN \$)} & \textbf{Costo total (MXN \$)} \\ \hline
			Evaluar el impacto ambiental del proyecto & 6 & \$260.00 & \$1,560.00 \\ \hline
			Asegurar el cumplimiento con regulaciones y políticas de privacidad & 10 & \$260.00 & \$2,600.00 \\ \hline
			\textbf{Total} & \textbf{16} & -- & \textbf{\$4,160.00} \\ \hline
		\end{tabular}
	\end{adjustbox}
	\caption[Costos estimados para la fase de gestión de regulaciones (con tarifas ajustadas)]{Costos estimados para la fase de gestión de regulaciones (con tarifas ajustadas), elaboración propia.} 
	\label{tab:costos_regulaciones_actualizado}
\end{table}

\begin{table}[H]
	\centering
	\renewcommand{\arraystretch}{1.6}
	\setlength{\tabcolsep}{10pt}
	\Huge
	\begin{adjustbox}{max width=\textwidth}
		\begin{tabular}{|p{9.5cm}|c|r|r|}
			\hline
			\textbf{Tareas (Gestión de integración)} & \textbf{Horas} & \textbf{Costo por hora (MXN \$)} & \textbf{Costo total (MXN \$)} \\ \hline
			Desarrollar el plan de gestión del proyecto & 15 & \$280.00 & \$4,200.00 \\ \hline
			Dirigir y gestionar el trabajo del proyecto & 20 & \$280.00 & \$5,600.00 \\ \hline
			Monitorear y controlar el trabajo del proyecto & 15 & \$280.00 & \$4,200.00 \\ \hline
			\textbf{Total} & \textbf{50} & -- & \textbf{\$14,000.00} \\ \hline
		\end{tabular}
	\end{adjustbox}
	\caption[Costos estimados para la fase de gestión de integración (con tarifas ajustadas)]{Costos estimados para la fase de gestión de integración (con tarifas ajustadas), elaboración propia.} 
	\label{tab:costos_integracion_actualizado}
\end{table}

\begin{table}[H]
	\centering
	\renewcommand{\arraystretch}{1.6}
	\setlength{\tabcolsep}{10pt}
	\Huge
	\begin{adjustbox}{max width=\textwidth}
		\begin{tabular}{|p{9.5cm}|c|r|r|}
			\hline
			\textbf{Tareas (Pruebas)} & \textbf{Horas} & \textbf{Costo por hora (MXN \$)} & \textbf{Costo total (MXN \$)} \\ \hline
			Costo de las pruebas iniciales solo con desarrolladores & 10 & \$280.00 & \$2,800.00 \\ \hline
			Pruebas unitarias para cada módulo & 20 & \$320.00 & \$6,400.00 \\ \hline
			Pruebas de integración & 10 & \$320.00 & \$3,200.00 \\ \hline
			Pruebas con usuarios para validar la usabilidad & 10 & \$260.00 & \$2,600.00 \\ \hline
			\textbf{Total} & \textbf{50} & -- & \textbf{\$15,000.00} \\ \hline
		\end{tabular}
	\end{adjustbox}
	\caption[Costos estimados para la fase de pruebas (con tarifas ajustadas)]{Costos estimados para la fase de pruebas (con tarifas ajustadas), elaboración propia.} 
	\label{tab:costos_pruebas_actualizado}
\end{table}

\begin{table}[H]
	\centering
	\renewcommand{\arraystretch}{1.6}
	\setlength{\tabcolsep}{10pt}
	\Huge
	\begin{adjustbox}{max width=\textwidth}
		\begin{tabular}{|p{9.5cm}|c|r|r|}
			\hline
			\textbf{Tareas (Lanzamiento)} & \textbf{Horas} & \textbf{Costo por hora (MXN \$)} & \textbf{Costo total (MXN \$)} \\ \hline
			Preparación del entorno de producción & 10 & \$320.00 & \$3,200.00 \\ \hline
			Configuración de servidores y bases de datos & 15 & \$320.00 & \$4,800.00 \\ \hline
			Despliegue del sistema en servidores de producción & 15 & \$320.00 & \$4,800.00 \\ \hline
			Configuración de backups y monitoreo & 10 & \$320.00 & \$3,200.00 \\ \hline
			\textbf{Total} & \textbf{50} & -- & \textbf{\$16,000.00} \\ \hline
		\end{tabular}
	\end{adjustbox}
	\caption[Costos estimados para la fase de lanzamiento (con tarifas ajustadas)]{Costos estimados para la fase de lanzamiento (con tarifas ajustadas), elaboración propia.} 
	\label{tab:costos_lanzamiento_actualizado}
\end{table}

\section{Etapa: costo de venta del prototipo}

\begin{table}[H]
	\centering
	\renewcommand{\arraystretch}{1.6}
	\setlength{\tabcolsep}{10pt}
	\Huge
	\begin{adjustbox}{max width=\textwidth}
		\begin{tabular}{|p{9.5cm}|c|r|r|}
			\hline
			\textbf{Tareas (Manual de usuario)} & \textbf{Horas} & \textbf{Costo por hora (MXN \$)} & \textbf{Costo total (MXN \$)} \\ \hline
			Creación de guías paso a paso para cada módulo & 12 & \$240.00 & \$2,880.00 \\ \hline
			Instrucciones claras y visuales para usuarios no técnicos & 10 & \$300.00 & \$3,000.00 \\ \hline
			\textbf{Total} & \textbf{22} & -- & \textbf{\$5,880.00} \\ \hline
		\end{tabular}
	\end{adjustbox}
	\caption[Costos estimados para la fase de manual de usuario (con tarifas ajustadas)]{Costos estimados para la fase de manual de usuario (con tarifas ajustadas), elaboración propia.} 
	\label{tab:costos_manual_usuario_actualizado}
\end{table}

\begin{table}[H]
	\centering
	\renewcommand{\arraystretch}{1.6}
	\setlength{\tabcolsep}{10pt}
	\Huge
	\begin{adjustbox}{max width=\textwidth}
		\begin{tabular}{|p{9.5cm}|c|r|r|}
			\hline
			\textbf{Tareas (Manual técnico)} & \textbf{Horas} & \textbf{Costo por hora (MXN \$)} & \textbf{Costo total (MXN \$)} \\ \hline
			Documentación de la arquitectura del sistema & 8 & \$240.00 & \$1,920.00 \\ \hline
			Instrucciones sobre la configuración del servidor y despliegue & 12 & \$320.00 & \$3,840.00 \\ \hline
			Descripción del conjunto de datos y APIs & 10 & \$320.00 & \$3,200.00 \\ \hline
			\textbf{Total} & \textbf{30} & -- & \textbf{\$8,960.00} \\ \hline
		\end{tabular}
	\end{adjustbox}
	\caption[Costos estimados para la fase de manual técnico (con tarifas ajustadas)]{Costos estimados para la fase de manual técnico (con tarifas ajustadas), elaboración propia.} 
	\label{tab:costos_manual_tecnico_actualizado}
\end{table}

\begin{table}[H]
	\centering
	\renewcommand{\arraystretch}{1.6}
	\setlength{\tabcolsep}{10pt}
	\Huge
	\begin{adjustbox}{max width=\textwidth}
		\begin{tabular}{|p{9.5cm}|c|r|r|}
			\hline
			\textbf{Tareas (Documentación)} & \textbf{Horas} & \textbf{Costo por hora (MXN \$)} & \textbf{Costo total (MXN \$)} \\ \hline
			Documentación de requerimientos & 15 & \$240.00 & \$3,600.00 \\ \hline
			Documentación de pruebas & 6 & \$240.00 & \$1,440.00 \\ \hline
			Manuales de usuario y técnico & 8 & \$240.00 & \$1,920.00 \\ \hline
			\textbf{Total} & \textbf{29} & -- & \textbf{\$6,960.00} \\ \hline
		\end{tabular}
	\end{adjustbox}
	\caption[Costos estimados para la fase de documentación (con tarifas ajustadas)]{Costos estimados para la fase de documentación (con tarifas ajustadas), elaboración propia.} 
	\label{tab:costos_documentacion_actualizado}
\end{table}


\begin{table}[H]
	\centering
	\renewcommand{\arraystretch}{1.6}
	\setlength{\tabcolsep}{10pt}
	\Huge
	\begin{adjustbox}{max width=\textwidth}
		\begin{tabular}{|p{9.5cm}|c|r|r|}
			\hline
			\textbf{Tareas (Presupuesto de ingresos)} & \textbf{Horas} & \textbf{Costo por hora (MXN \$)} & \textbf{Costo total (MXN \$)} \\ \hline
			Estimación de precio de producto final & 6 & \$270.00 & \$1,620.00 \\ \hline
			\textbf{Total} & \textbf{6} & -- & \textbf{\$1,620.00} \\ \hline
		\end{tabular}
	\end{adjustbox}
	\caption[Costos estimados para la fase de presupuesto de ingresos (con tarifa ajustada)]{Costos estimados para la fase de presupuesto de ingresos (con tarifa ajustada), elaboración propia.} 
	\label{tab:costos_presupuesto_ingresos_s}
\end{table}

\begin{table}[H]
	\centering
	\renewcommand{\arraystretch}{1.6}
	\setlength{\tabcolsep}{10pt}
	\Huge
	\begin{adjustbox}{max width=\textwidth}
		\begin{tabular}{|p{9.5cm}|c|r|r|}
			\hline
			\textbf{Tareas (Estados financieros proforma)} & \textbf{Horas} & \textbf{Costo por hora (MXN \$)} & \textbf{Costo total (MXN \$)} \\ \hline
			Revisión de costos y gastos iniciales & 1 & \$270.00 & \$270.00 \\ \hline
			Proyección de ingresos & 2 & \$270.00 & \$540.00 \\ \hline
			Elaboración de balance proforma & 8 & \$270.00 & \$2,160.00 \\ \hline
			Preparación de estado de resultados & 3 & \$270.00 & \$810.00 \\ \hline
			\textbf{Total} & \textbf{14} & -- & \textbf{\$3,780.00} \\ \hline
		\end{tabular}
	\end{adjustbox}
	\caption[Costos estimados para la fase de estados financieros proforma (con tarifas ajustadas)]{Costos estimados para la fase de estados financieros proforma (con tarifas ajustadas), elaboración propia.} 
	\label{tab:costos_financieros_proforma}
\end{table}

\begin{table}[H]
	\centering
	\renewcommand{\arraystretch}{1.6}
	\setlength{\tabcolsep}{10pt}
	\Huge
	\begin{adjustbox}{max width=\textwidth}
		\begin{tabular}{|p{9.5cm}|c|r|r|}
			\hline
			\textbf{Tareas (Flujos netos de efectivo)} & \textbf{Horas} & \textbf{Costo por hora (MXN \$)} & \textbf{Costo total (MXN \$)} \\ \hline
			Identificación de entradas y salidas de efectivo & 4 & \$270.00 & \$1,080.00 \\ \hline
			Proyección de flujo de efectivo mensual y anual & 2 & \$270.00 & \$540.00 \\ \hline
			Análisis de punto de equilibrio & 6 & \$270.00 & \$1,620.00 \\ \hline
			\textbf{Total} & \textbf{12} & -- & \textbf{\$3,240.00} \\ \hline
		\end{tabular}
	\end{adjustbox}
	\caption[Costos estimados para la fase de flujos netos de efectivo (con tarifas ajustadas)]{Costos estimados para la fase de flujos netos de efectivo (con tarifas ajustadas), elaboración propia.} 
	\label{tab:costos_flujos_efectivo}
\end{table}

\begin{table}[H]
	\centering
	\renewcommand{\arraystretch}{1.6}
	\setlength{\tabcolsep}{10pt}
	\Huge
	\begin{adjustbox}{max width=\textwidth}
		\begin{tabular}{|p{9.5cm}|c|r|r|}
			\hline
			\textbf{Tareas (Evaluación financiera)} & \textbf{Horas} & \textbf{Costo por hora (MXN \$)} & \textbf{Costo total (MXN \$)} \\ \hline
			Análisis de retorno de inversión & 4 & \$270.00 & \$1,080.00 \\ \hline
			Sensibilidad de las proyecciones & 3 & \$270.00 & \$810.00 \\ \hline
			\textbf{Total} & \textbf{7} & -- & \textbf{\$1,890.00} \\ \hline
		\end{tabular}
	\end{adjustbox}
	\caption[Costos estimados para la fase de evaluación financiera (con tarifas ajustadas)]{Costos estimados para la fase de evaluación financiera (con tarifas ajustadas), elaboración propia.} 
	\label{tab:costos_evaluacion_financiera}
\end{table}

\begin{table}[H]
	\centering
	\renewcommand{\arraystretch}{1.6}
	\setlength{\tabcolsep}{10pt}
	\Huge
	\begin{adjustbox}{max width=\textwidth}
		\begin{tabular}{|p{9.5cm}|c|r|r|}
			\hline
			\textbf{Tareas (Mantenimiento)} & \textbf{Horas} & \textbf{Costo por hora (MXN \$)} & \textbf{Costo total (MXN \$)} \\ \hline
			Monitoreo continuo del sistema & 15 & \$320.00 & \$4,800.00 \\ \hline
			Corrección de errores post-despliegue & 10 & \$320.00 & \$3,200.00 \\ \hline
			Costo de mantenimiento de servidores y seguridad & 12 & \$320.00 & \$3,840.00 \\ \hline
			Cotización de salarios del equipo de mantenimiento de la app & 1 & \$270.00 & \$270.00 \\ \hline
			Costos de actualizaciones de la aplicación & 10 & \$320.00 & \$3,200.00 \\ \hline
			\textbf{Total} & \textbf{48} & -- & \textbf{\$15,310.00} \\ \hline
		\end{tabular}
	\end{adjustbox}
	\caption[Costos estimados para la fase de mantenimiento (con tarifas ajustadas)]{Costos estimados para la fase de mantenimiento (con tarifas ajustadas), elaboración propia.} 
	\label{tab:costos_mantenimiento}
\end{table}


\begin{thebibliography}{99}
    \bibitem{ref1}
    MacMillan Education, “Aspectos generales para entender la comunicación. Unidad 1”, MacMillan, 2018. [En línea]. Disponible en: \url{https://www.macmillaneducation.es/wp-content/uploads/2018/10/comunicacion_cliente_libroalumno_unidad1muestra.pdf.} [Recuperado: 26-feb-2025].

    \bibitem{ref2}
    S. L. Hernández Mendoza y D. Aduana Avila, “Barreras de comunicación”, Revista ICEA, vol. 9, no. 18, pp. 47-48, 5 de mayo de 2021. [En línea]. Disponible en: \url{https://repository.uaeh.edu.mx/revistas/index.php/icea/article/view/7125/8008}. [Recuperado: 26-feb-2025].

    \bibitem{ref3}
    Secretaría de Salud, “Con discapacidad auditiva, 2.3 millones de personas: Instituto Nacional de Rehabilitación”, 2021. [En línea]. Disponible en: \url{https://www.gob.mx/salud/prensa/530-con-discapacidad-auditiva-2-3-millones-de-personas-instituto-nacional-de-rehabilitacion?idiom=es}. [Recuperado: 26-feb-2025].

    \bibitem{ref4}
    J. Morales Novas y A. K. Sánchez Zepeda, “Lengua de Señas Mexicana (LSM): Su importancia”, s.f. [En línea]. Disponible en: \url{https://trabajosocial.unam.mx/copred/doc/infografia_2_lengua%20de%20senas_mexicana.pdf}. [Recuperado: 26-feb-2025].

    \bibitem{ref5}
    M. Florencia Melo, “El mapa mundial de Android e iOS”, 2024. [En línea]. Disponible en: \url{https://es.statista.com/grafico/29620/sistema-operativo-movil-con-la-mayor-cuota-de-mercado-por-pais/}. [Recuperado: 26-feb-2025].

    \bibitem{ref6}
    Instituto Federal de Telecomunicaciones (IFT), “ANDROID ES EL SISTEMA OPERATIVO MÁS UTILIZADO EN MÉXICO; POR SU PARTE, GOOGLE/CHROME PREDOMINA PARA REALIZAR BÚSQUEDAS EN INTERNET”, IFT, 2022. [En línea]. Disponible en: \url{https://www.ift.org.mx/sites/default/files/contenidogeneral/usuarios-y-audiencias/encuestassobresistemasoperativosynavegadores2022.pdf}. [Recuperado: 20-abr-2025].

    \bibitem{ref7}
    C. Lugaresi, J. Tang, H. Nash, C. McClanahan, E. Uboweja, M. Hays, F. Zhang, C.-L. Chang, M. G. Yong, J. Lee et al., “Mediapipe: A framework for building perception pipelines”, arXiv preprint, arXiv:1906.08172, 2019.

    \bibitem{ref8}
    J. C. Hernández-Cruz, C. E. Rose-Gómez y S. González-López, “Translation of Spanish Text to Mexican Sign Language Glossed Text Using Rules and Deep Learning”, Resilience and Future of Smart Learning, J. Yang et al., Eds. Springer, 2022. doi: 10.1007/978-981-19-5967-7\_25.

    \bibitem{ref9}
    O. Pichardo-Lagunas y B. Martínez-Seis, “Resource Creation for Automatic Translation System from Texts in Spanish into Mexican Sign Language”, Research in Computing Science, vol. 100, pp. 129-137, 2015.
\end{thebibliography}
\end{document}          
