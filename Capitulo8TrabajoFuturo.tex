\chapter{Trabajo a Futuro}
\section{Trabajo a Futuro}
Como parte del trabajo a futuro, se contempla la ampliación del conjunto de escenarios y situaciones incluidas en el prototipo, incorporando nuevos grupos de frases que abarquen contextos más diversos de la vida cotidiana, con el fin de mejorar la cobertura comunicativa y la utilidad de la aplicación.\\

Asimismo, se considera retomar la integración de modelos de animación 3D, tal como se planteó en la propuesta original, lo cual implicaría la adquisición de un traje de captura de movimiento (motion capture) y la colaboración de un equipo especializado en animación para procesar los movimientos obtenidos en Blender y posteriormente exportarlos a la aplicación móvil. \\

Por último, se proyecta la implementación de nuevas funcionalidades que fortalezcan la interacción del usuario, como un módulo de reconocimiento de voz capaz de transcribir audio a texto, o el desarrollo de una red neuronal convencional que permita identificar letras del alfabeto mediante visión artificial, ampliando así las capacidades del sistema y su aumentar su potencial como herramienta de apoyo inclusiva.\\

Se espera que el presente Trabajo Terminal siente las bases para futuras investigaciones orientadas al desarrollo de sistemas más completos de traducción entre el español y la LSM, con miras a construir en el largo plazo un sistema de traducción general que permita una comunicación bidireccional y natural entre personas oyentes y personas con discapacidad auditiva.