\chapter{Trabajo a Futuro}
\section{Trabajo a Futuro}
Como parte del trabajo a futuro, se plantea la ampliación del conjunto de escenarios y situaciones contemplados en el prototipo, incorporando nuevos grupos de frases que cubran contextos más amplios de la vida cotidiana. Este crecimiento permitiría mejorar la cobertura comunicativa del sistema y fortalecer su utilidad como herramienta de apoyo.\\

Asimismo, se considera retomar la integración de modelos de animación 3D, tal como se propuso en la fase inicial del proyecto. Esto implicaría la adquisición de un traje de captura de movimiento y la colaboración de un equipo especializado en animación para procesar, limpiar y adaptar los movimientos dentro de Blender, para posteriormente exportarlos a la aplicación móvil.\\

En cuanto al módulo de Procesamiento de Lenguaje Natural, se contemplan mejoras tanto operativas como metodológicas. A corto y mediano plazo, se prevé la ampliación del dataset de nombres propios, la incorporación de mecanismos de almacenamiento en caché para disminuir la latencia de búsqueda, así como la instrumentación de herramientas de monitoreo que permitan obtener métricas en tiempo real y generar alertas automatizadas en caso de fallos de rendimiento. De igual manera, se propone habilitar soporte multilingüe, aprovechando que el modelo actual permite trabajar con múltiples idiomas, e implementar un proceso de aprendizaje continuo basado en el registro de consultas no reconocidas, lo que facilitaría la actualización del sistema conforme a las necesidades de los usuarios.\\

A largo plazo, se contempla investigar técnicas de optimización más avanzadas, como el fine-tuning del modelo de embeddings para ajustarlo al dominio específico de la LSM, con el objetivo de mejorar la precisión del sistema. También se prevé la incorporación de mecanismos de contextualización que permitan mantener un historial de interacción, mejorando la interpretación de consultas ambiguas. Finalmente, sería deseable explorar alternativas generativas para que el sistema pueda producir respuestas en lugar de limitarse a recuperar coincidencias mediante similitud semántica, lo que abriría la puerta a un modelo conversacional híbrido orientado a la accesibilidad.\\

En conjunto, estas mejoras permitirán que el presente Trabajo Terminal sirva como base para investigaciones posteriores dirigidas al desarrollo de sistemas más completos de traducción entre español y LSM. En el largo plazo, se busca avanzar hacia soluciones capaces de ofrecer una comunicación bidireccional, natural e inclusiva entre personas oyentes y personas con discapacidad auditiva.