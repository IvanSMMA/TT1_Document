\chapter{Desarrollo}
\section{Análisis de viabilidad y factibilidad}
En esta sección, se presenta el análisis de viabilidad y factibilidad del prototipo de aplicación móvil de apoyo para la traducción de español a Lengua de Señas Mexicana (LSM) empleando técnicas de Procesamiento de Lenguaje Natural (PLN) y modelado 3D, con el objetivo de evaluar si existen las condiciones técnicas, humanas, tecnológicas y financieras necesarias para su desarrollo exitoso del mismo. En primer lugar, se analiza la viabilidad, considerando los conocimientos del equipo, las tecnologías disponibles, las condiciones de ejecución y el potencial de escalabilidad del prototipo. Posteriormente, se estudia la factibilidad, enfocándose en los recursos financieros, humanos y tecnológicos, así como en el tiempo estimado para completar el proyecto.\\

\subsection{Viabilidad}
El análisis de viabilidad considera los conocimientos técnicos del equipo, la madurez de las tecnologías disponibles y las condiciones actuales para el desarrollo. Este análisis permite evaluar si el proyecto puede llevarse a cabo de manera exitosa bajo las condiciones planteadas.

\subsubsection{Conocimientos y experiencia}
El equipo de desarrollo posee conocimientos en Procesamiento de Lenguaje Natural (PLN), así como habilidades básicas en animaciones y recursos visuales. Aunque la experiencia en animación 3D orientada a señas, en programación de aplicaciones móviles y en el manejo de la Lengua de Señas Mexicana (LSM) es limitada, se considera factible adquirir y aplicar los conocimientos necesarios mediante el uso de recursos de investigación, bibliotecas especializadas y la colaboración con expertos en LSM, tomando en cuenta la versión correspondiente al uso y documentación del año 2024. Esta disposición de aprendizaje y fortalecimiento de competencias respalda la viabilidad técnica del proyecto en función de las capacidades del equipo.

De manera complementaria, se cuenta con diversas tecnologías que facilitarán la implementación del prototipo, tal como se describe a continuación.

\subsubsection{Tecnologías disponibles}
Actualmente, existen diversas tecnologías y herramientas que facilitan la creación de sistemas de traducción de texto a señas, tales como conjunto de datos de señas en video, motores de animación 3D y frameworks para el desarrollo de aplicaciones móviles. Asimismo, se dispone de plataformas de código abierto que permiten representar señas mediante modelos animados o videos precargados, optimizando así los recursos disponibles para el desarrollo de prototipos.

Las tecnologías consideradas para el presente proyecto incluyen:
\begin{itemize} 
	\item Librerías y conjunto de datos de señas mexicanas (videos de señas). 
	\item Herramientas de animación 3D como Blender (versión 4.4.3) o Unity (versión 6.1), así como motores ligeros compatibles con aplicaciones móviles. 
	\item Frameworks de desarrollo móvil como Flutter (versión 3.29.3) y React Native (versión 0.79). 
	\item Herramientas de Procesamiento de Lenguaje Natural (PLN) para el análisis y segmentación de frases en español. 
\end{itemize}

\subsubsection{Condiciones para la ejecución}
El proyecto se desarrolla en el marco de un trabajo terminal académico, lo que garantiza el acceso a recursos institucionales, asesoría especializada y bibliografía técnica actualizada. De igual manera, el creciente interés social y académico por fomentar la inclusión de la comunidad sorda en México genera un entorno favorable para la implementación de este tipo de iniciativas, fortaleciendo así las condiciones de ejecución del prototipo.

\subsection{Factibilidad}
La factibilidad del proyecto se analiza considerando los recursos financieros, humanos y tecnológicos disponibles, así como el tiempo estimado para su desarrollo y finalización.

\subsubsection{Recursos financieros}
El presente apartado tiene como objetivo evaluar los costos asociados al desarrollo, despliegue y potencial comercialización del prototipo. 

Para ello, se plantea un análisis financiero estructurado en dos niveles: 

\begin{itemize}
	\item \textbf{Enfoque académico:} incluye estimaciones de costos simbólicos aproximados del mercado laboral en México correspondientes al año 2025 [CITA IEEE], considerando el uso de recursos propios, herramientas gratuitas y asesorías puntuales. Este enfoque representa fielmente la ejecución del proyecto dentro de un marco académico.
	
	\item \textbf{Enfoque exploratorio de comercialización:} presenta una simulación de costos reales basada en tarifas aproximadas de mercado laboral en México correspondientes al año 2025 [CITA IEEE], incluyendo sueldos profesionales, licencias, infraestructura tecnológica y servicios externos. Este análisis permite anticipar los requerimientos financieros de una futura etapa de comercialización.
\end{itemize}

Ambas perspectivas se estructuran mediante dos métodos complementarios de análisis financiero:

\begin{itemize}
	\item \textbf{Estimación por recursos:} identifica y valora los insumos materiales, humanos y tecnológicos necesarios para el desarrollo y operación del prototipo.
	\item \textbf{Estimación por actividades:} desglosa las tareas específicas del proyecto, asignando tiempos estimados y costos unitarios para cada una.
\end{itemize}

Esta estructura metodológica permite identificar con precisión los recursos involucrados, calcular el presupuesto total estimado y analizar la factibilidad financiera del proyecto, tanto en su ejecución académica como en un escenario de comercialización futuro.

\paragraph{\textbf{Informe de costos de creación del prototipo.}} 
Este informe presenta el análisis financiero asociado a la creación del prototipo en cuestión. El objetivo es identificar y estructurar los costos necesarios para garantizar la factibilidad del proyecto en su etapa de creación.

\paragraph{\textbf{Presupuesto mediante recursos.}} 
Dentro del análisis basado en recursos, el proceso de desarrollo se ha estructurado en tres etapas principales, cada una integrada por actividades específicas que permiten avanzar de manera ordenada, asegurando la calidad y funcionalidad del prototipo final. Las etapas contempladas son:

\begin{itemize}
	\item \textbf{Creación del prototipo}. 
	\item \textbf{Despliegue del prototipo}.
	\item \textbf{Evaluación financiera}. 
\end{itemize}

Cada una de estas etapas será detallada en los apartados siguientes.

\paragraph{\textbf{Creación del prototipo.}} 
Esta primera etapa abarca todas las actividades necesarias para el desarrollo inicial del prototipo: definición de objetivos, especificaciones técnicas, requisitos, diseño de animaciones, etc. Aquí se implementan técnicas de PLN para la segmentación de frases y modelado/animación 3D para representar las señas en LSM.

\paragraph{\textbf{Despliegue del prototipo.}} 
Incluye acciones para poner en funcionamiento la app en Android (versión 14), publicación, configuración técnica, documentación técnica y pruebas piloto con usuarios. Esta retroalimentación permitirá ajustar y mejorar la experiencia de uso.

\paragraph{\textbf{Evaluación financiera.}} 
Implica analizar los costos de las etapas anteriores y valorar beneficios esperados como impacto social, accesibilidad y sostenibilidad. Aunque el proyecto tiene fines académicos, se incluye un análisis básico de posibilidad de escalamiento.

Este enfoque estructurado de desarrollo permite garantizar una planeación clara y eficiente, considerando los elementos técnicos, operativos y financieros necesarios para la ejecución del proyecto.

Por último, como medida preventiva ante posibles riesgos o imprevistos, se contempla una reserva de contingencia equivalente al 15\% del costo total estimado.


\paragraph{\textbf{Análisis de recursos.}} 
\paragraph{\textbf{Costos de servicios.}} 
A continuación, se detallan los costos asociados a los servicios necesarios para el desarrollo del prototipo, considerando un periodo estimado de ejecución de cuatro meses. Se incluyen los servicios con ambos enfoques, el académico y el exploratorio de comercialización.


\begin{table}[H]
	\centering
	\renewcommand{\arraystretch}{1.6}
	\setlength{\tabcolsep}{12pt}
	\Huge % tamaño de fuente más grande posible
	\begin{adjustbox}{max width=\textwidth}
		\begin{tabular}{|p{7cm}|r|r|r|}
			\hline
			\textbf{Recurso} & \textbf{Costo mensual (MXN \$)} & \textbf{Costo a 4 meses (MXN \$)} & \textbf{Costo a 1 año (MXN \$)} \\ \hline
			Electricidad e Internet (compartido entre integrantes) & \$500.00 & \$2,000.00 & \$6,000.00 \\ \hline
			Almacenamiento en la nube (Google Drive / GitHub) & \$300.00 & \$1,200.00 & \$3,600.00 \\ \hline
			Pruebas en dispositivos Android (emulador físico o virtual) & \$500.00 & \$2,000.00 & \$6,000.00 \\ \hline
			Reserva para servicios externos o pruebas adicionales & \$1,000.00 & \$4,000.00 & \$12,000.00 \\ \hline
			\textbf{Total estimado de servicios} & \textbf{\$2,300.00} & \textbf{\$9,200.00} & \textbf{\$27,600.00} \\ \hline
		\end{tabular}
	\end{adjustbox}
	\caption[Costos estimados de servicios en el escenario académico durante la creación del prototipo]{Costos estimados de servicios en el escenario académico durante la creación del prototipo, elaboración propia.}
	\label{tab:costos_servicios}
\end{table}



\begin{table}[H]
	\centering
	\renewcommand{\arraystretch}{1.6}
	\setlength{\tabcolsep}{12pt}
	\Huge % Tamaño de fuente más grande disponible en LaTeX estándar
	\begin{adjustbox}{max width=\textwidth}
		\begin{tabular}{|p{7cm}|r|r|r|}
			\hline
			\textbf{Recurso o servicio} & \textbf{Costo mensual (MXN \$)} & \textbf{Costo a 4 meses (MXN \$)} & \textbf{Costo a 1 año (MXN \$)} \\ \hline
			Electricidad e internet (oficina dedicada) & \$2,500.00 & \$10,000.00 & \$30,000.00 \\ \hline
			Renta de espacio de coworking (para 4 personas) & \$9,000.00 & \$36,000.00 & \$108,000.00 \\ \hline
			Suscripción a plataformas de desarrollo (GitHub Copilot, Blender Studio, etc.) & \$800.00 & \$3,200.00 & \$9,600.00 \\ \hline
			API de lenguaje natural (OpenAI GPT, DialogFlow) & \$1,000.00 & \$4,000.00 & \$12,000.00 \\ \hline
			Servicios de almacenamiento en la nube (Google Cloud, Firebase, etc.) & \$750.00 & \$3,000.00 & \$9,000.00 \\ \hline
			Servidor para aplicación móvil (Firebase/AWS) & \$1,200.00 & \$4,800.00 & \$14,400.00 \\ \hline
			Licencia de software para animación (Unity, MediaPipe, etc.) & \$1,500.00 & \$6,000.00 & \$18,000.00 \\ \hline
			Validación profesional de señas LSM (freelancer mensual) & \$5,000.00 & \$20,000.00 & \$60,000.00 \\ \hline
			Servicios de soporte técnico y mantenimiento & \$3,000.00 & \$12,000.00 & \$36,000.00 \\ \hline
			Marketing digital (redes sociales, web, SEO) & \$2,000.00 & \$8,000.00 & \$24,000.00 \\ \hline
			Traducción y adaptación de contenido LSM (consultoría externa) & \$4,000.00 & \$16,000.00 & \$48,000.00 \\ \hline
			\textbf{TOTAL ESTIMADO} & \textbf{\$30,750.00} & \textbf{\$123,000.00} & \textbf{\$369,000.00} \\ \hline
		\end{tabular}
	\end{adjustbox}
	\caption[Costos estimados de servicios en un escenario comercial para la creación del prototipo]{Costos estimados de servicios en un escenario comercial para la creación del prototipo, elaboración propia.}
	\label{tab:costos_comercial}
\end{table}


\paragraph{\textbf{Compras no recurrentes.}} 
En esta sección se presentan los costos estimados de artículos y adquisiciones necesarias para la creación del prototipo, clasificadas como compras no recurrentes. Estas compras representan inversiones únicas que no implican costos periódicos, pero que son fundamentales para el correcto desarrollo y prueba de la aplicación.

\begin{table}[H]
	\centering
	\renewcommand{\arraystretch}{1.6}
	\setlength{\tabcolsep}{10pt}
	\Huge
	\begin{adjustbox}{max width=\textwidth}
		\begin{tabular}{|p{7cm}|c|r|r|}
			\hline
			\textbf{Recurso} & \textbf{Unidades} & \textbf{Costo unitario (MXN \$)} & \textbf{Costo total (MXN \$)} \\ \hline
			Equipo de cómputo personal (propio de los integrantes) & 3 & -- & -- \\ \hline
			Dispositivo Android para pruebas físicas & 1 & -- & -- \\ \hline
			Capacitación online (cursos: Blender, MediaPipe, PLN) & 3 cursos & \$800.00 & \$2,400.00 \\ \hline
			Compra de modelos 3D o recursos gráficos (opcional) & 1 paquete & \$2,500.00 & \$2,500.00 \\ \hline
			\textbf{Total compras no recurrentes} & & & \textbf{\$4,900.00} \\ \hline
		\end{tabular}
	\end{adjustbox}
	\caption[Costos estimados de artículos y compras no recurrentes en el escenario académico]{Costos estimados de artículos y compras no recurrentes en el escenario académico, elaboración propia.}
	\label{tab:compras_no_recurrentes}
\end{table}


\noindent \textbf{Nota aclaratoria:}  
el equipo de cómputo utilizado corresponde a dispositivos personales de los integrantes del proyecto, por lo cual no se ha considerado un costo adicional en esta categoría. La compra de modelos 3D o recursos gráficos es considerada opcional y dependerá de la necesidad de complementar el material gráfico disponible de manera gratuita o de libre acceso.

\begin{table}[H]
	\centering
	\renewcommand{\arraystretch}{1.6}
	\setlength{\tabcolsep}{10pt}
	\Huge
	\begin{adjustbox}{max width=\textwidth}
		\begin{tabular}{|p{7cm}|c|r|r|}
			\hline
			\textbf{Recurso} & \textbf{Unidades} & \textbf{Costo unitario (MXN \$)} & \textbf{Costo total (MXN \$)} \\ \hline
			Equipo de cómputo profesional (para desarrollo y edición 3D) & 4 & \$25,000.00 & \$100,000.00 \\ \hline
			Dispositivos móviles de prueba (Android gama media) & 2 & \$12,000.00 & \$24,000.00 \\ \hline
			Cámara y sensor de movimiento (para captura LSM y pruebas) & 1 & \$12,000.00 & \$12,000.00 \\ \hline
			Paquete profesional de modelos 3D con licencia comercial & 1 & \$10,000.00 & \$10,000.00 \\ \hline
			Cursos y certificaciones profesionales (PLN, UX, IA, Unity) & 4 & \$3,000.00 & \$12,000.00 \\ \hline
			Equipo de audio y grabación (para interfaz voz/signos) & 1 & \$4,500.00 & \$4,500.00 \\ \hline
			\textbf{Total compras no recurrentes} & & & \textbf{\$162,500.00} \\ \hline
		\end{tabular}
	\end{adjustbox}
	\caption[Costos estimados de compras no recurrentes en un escenario de comercialización]{Costos estimados de compras no recurrentes en un escenario de comercialización, elaboración propia.}
	\label{tab:compras_no_recurrentes_comercial}
\end{table}


\paragraph{\textbf{Sueldos y asesorías.}} 
El presente apartado presenta el costo estimado de los sueldos y asesorías considerados para la creación del prototipo en cuestión. Se adopta un enfoque dual que contempla tanto el escenario académico de ejecución como una proyección orientada a una futura etapa de comercialización del producto.

\begin{table}[H]
	\centering
	\renewcommand{\arraystretch}{1.6}
	\setlength{\tabcolsep}{10pt}
	\Huge
	\begin{adjustbox}{max width=\textwidth}
		\begin{tabular}{|p{5.5cm}|c|r|r|r|r|}
			\hline
			\textbf{Equipo} & \textbf{Cantidad} & \multicolumn{2}{c|}{\textbf{Desarrollo}} & \multicolumn{2}{c|}{\textbf{Mantenimiento / Ajustes}} \\ \hline
			\textbf{Tipo} & & \textbf{Mensual (MXN \$)} & \textbf{4 meses (MXN \$)} & \textbf{Mensual (MXN \$)} & \textbf{A un año (MXN \$)} \\ \hline
			Desarrollador Full Stack Junior (estudiante) & 3 & \$72,000.00 (\$24,000.00 c/u) & \$288,000.00 & -- & -- \\ \hline
			Asesoría en animación 3D (freelance) & 1 parcial & \$300.00 (por sesión) & \$1,200.00 (cuatro sesiones) & -- & -- \\ \hline
			Asesoría en LSM (validación de señas) & 1 parcial & \$200.00 (por sesión) & \$800.00 (cuatro sesiones) & -- & -- \\ \hline
			\textbf{Total} & \textbf{5} & \textbf{\$72,500.00} & \textbf{\$290,000.00} & -- & -- \\ \hline
		\end{tabular}
	\end{adjustbox}
	\caption[Costos estimados en el escenario académico de sueldos y asesorías durante la creación del prototipo]{Costos estimados en el escenario académico de sueldos y asesorías durante la creación del prototipo, elaboración propia.}
	\label{tab:sueldos_asesorias}
\end{table}


\noindent \textbf{Nota aclaratoria:}  
el sueldo en este caso, se considera un pago simbólico, sin embargo, el sueldo mensual se consideró como un aproximado para un Desarrollador Full Stack Junior en México con fecha de mayo de 2025 \cite{ref63}.  

\begin{table}[H]
	\centering
	\renewcommand{\arraystretch}{1.6}
	\setlength{\tabcolsep}{10pt}
	\Huge
	\begin{adjustbox}{max width=\textwidth}
		\begin{tabular}{|p{5.8cm}|c|r|r|r|r|}
			\hline
			\textbf{Rol / Servicio} & \textbf{Cantidad} & \multicolumn{2}{c|}{\textbf{Desarrollo}} & \multicolumn{2}{c|}{\textbf{Mantenimiento / Ajustes}} \\ \hline
			\textbf{Tipo} & & \textbf{Mensual (MXN \$)} & \textbf{4 meses (MXN \$)} & \textbf{Mensual (MXN \$)} & \textbf{A un año (MXN \$)} \\ \hline
			Desarrollador backend / frontend & 2 & \$28,000.00 & \$224,000.00 & \$16,000.00 & \$192,000.00 \\ \hline
			Especialista en PLN & 1 & \$30,000.00 & \$120,000.00 & \$15,000.00 & \$180,000.00 \\ \hline
			Diseñador 3D / animador (Unity / Blender) & 1 & \$25,000.00 & \$100,000.00 & \$12,000.00 & \$144,000.00 \\ \hline
			Asesoría profesional en LSM & 1 parcial & \$10,000.00 & \$40,000.00 & \$5,000.00 & \$60,000.00 \\ \hline
			Soporte técnico / DevOps & 1 & \$15,000.00 & \$60,000.00 & \$10,000.00 & \$120,000.00 \\ \hline
			\textbf{Total estimado} & \textbf{6} & \textbf{\$108,000.00} & \textbf{\$544,000.00} & \textbf{\$58,000.00} & \textbf{\$696,000.00} \\ \hline
		\end{tabular}
	\end{adjustbox}
	\caption[Costos estimados de sueldos y asesorías en un entorno comercial durante la creación del prototipo]{Costos estimados de sueldos y asesorías en un entorno comercial durante la creación del prototipo, elaboración propia.}	
	\label{tab:sueldos_comercial}
\end{table}



\noindent \textbf{Nota aclaratoria:}  
La columna de “Mantenimiento / Ajustes” se incluye únicamente como referencia para una posible fase futura de operación continua, en caso de que el prototipo evolucione hacia un producto comercial o requiera soporte extendido. Los sueldos se consideran como un aproximado en México con fecha de mayo de 2025 \cite{ref64, ref65, ref66, ref67, ref68, ref69, ref70}. 


\paragraph{\textbf{Presupuesto mediante actividades.}} 
En el siguiente apartado, se presenta el presupuesto detallado por actividades, donde se especifica el tiempo estimado y el costo asociado a cada tarea del proyecto. Además, los valores por hora se ajustan para una estimación freelance o contratista externo (no asalariado fijo), ideal para la simulación de costos comerciales.  Los sueldos se consideran como un aproximado en México con fecha de mayo de 2025 \cite{ref71, ref72, ref73, ref74, ref75, ref76, ref77}.

Las actividades específicas y sus tiempos estimados se encuentran detallados en los anexos titulados \textbf{\nameref{anexo:actividades_academicas}} y \textbf{\nameref{anexo:actividades_comercial}} .


\paragraph{\textbf{Resumen de costos estimados.}}
El presente apartado presenta el resumen de los costos que incluye ambos escenarios tanto el académico como el de proyección a comercialización en un futuro. El análisis incluye el presupuesto derivado de las actividades de desarrollo, los recursos necesarios para las fases de creación y despliegue, las compras no recurrentes, los sueldos del equipo de trabajo y una reserva contemplada para cubrir posibles riesgos o imprevistos durante la implementación.

\begin{table}[H]
	\centering
	\renewcommand{\arraystretch}{1.5}
	\setlength{\tabcolsep}{12pt}
	\resizebox{\textwidth}{!}{%
		\begin{tabular}{|l|r|}
			\hline
			\textbf{Concepto} & \textbf{Monto estimado (MXN \$)} \\ \hline
			Presupuesto total de actividades & \$71,100.00 \\ \hline
			Presupuesto total de recursos (desarrollo) & \$9,200.00 \\ \hline
			Presupuesto total de recursos (despliegue) & \$27,600.00 \\ \hline
			Compras no recurrentes & \$4,900.00 \\ \hline
			Sueldo del equipo (desarrollo) & \$290,000.00 \\ \hline
			Sueldo del equipo (mantenimiento) & -- \\ \hline
			Reserva para riesgos e imprevistos (15\%) & \$10,665.00 \\ \hline
			\textbf{Presupuesto total estimado del proyecto} & \textbf{\$413,939.00} \\ \hline
		\end{tabular}%
	}
	\caption[Resumen de costos estimados para el desarrollo y despliegue en el escenario académico]{Resumen de costos estimados para el desarrollo y despliegue en el escenario académico, elaboración propia.}	
	\label{tab:costos_academico}
\end{table}

\begin{table}[H]
	\centering
	\renewcommand{\arraystretch}{1.5}
	\setlength{\tabcolsep}{12pt}
	\resizebox{\textwidth}{!}{%
		\begin{tabular}{|l|r|}
			\hline
			\textbf{Concepto} & \textbf{Monto estimado (MXN \$)} \\ \hline
			Presupuesto total de actividades & \$179,280.00 \\ \hline
			Presupuesto total de recursos (desarrollo) & \$123,000.00 \\ \hline
			Presupuesto total de recursos (despliegue) & \$369,000.00 \\ \hline
			Compras no recurrentes & \$162,500.00 \\ \hline
			Sueldo del equipo (desarrollo) & \$544,000.00 \\ \hline
			Sueldo del equipo (mantenimiento) & \$696,000.00 \\ \hline
			Reserva para riesgos e imprevistos (15\%) & \$26,892.00 \\ \hline
			\textbf{Presupuesto total estimado del proyecto} & \textbf{\$2,101,346.00} \\ \hline
		\end{tabular}%
	}
	\caption[Resumen de costos estimados para el desarrollo y despliegue en un entorno comercial del prototipo]{Resumen de costos estimados para el desarrollo y despliegue en un entorno comercial del prototipo, elaboración propia.}	
	\label{tab:costos_comerciall}
\end{table}

\noindent \textbf{Nota aclaratoria:}  
el presupuesto presentado considera un escenario de escalamiento comercial del prototipo, por lo que las cifras reflejan una estimación basada en tarifas de mercado, infraestructura de operación real y recursos humanos contratados de manera formal. Cabe resaltar que, para efectos académicos, los costos efectivos en la fase de prototipo fueron significativamente menores, basados en costos de oportunidad y recursos propios. La reserva para riesgos e imprevistos contempla un porcentaje adicional sobre el subtotal, como medida preventiva ante ajustes, retrasos o necesidades técnicas no previstas.

\paragraph{\textbf{Informe de costo de despliegue del prototipo.}}
El despliegue del prototipo constituye una etapa crucial dentro del proyecto, ya que permite verificar la correcta operación de la aplicación, validar la experiencia de usuario y obtener retroalimentación que contribuya a la mejora continua de la solución. Esta fase busca asegurar que el prototipo sea funcional, accesible y que cumpla con los objetivos de accesibilidad e inclusión establecidos para su prueba en un entorno académico o controlado.

A continuación, se detallan los principales aspectos relacionados con esta etapa:
\paragraph{\textbf{Objetivos del despliegue.}}
Los objetivos principales del despliegue del prototipo son los siguientes:

\begin{itemize}
	\item Verificar la funcionalidad de la aplicación y su compatibilidad en dispositivos con Android 14.
	\item Realizar pruebas piloto para validar la comprensión y fluidez de las animaciones 3D en Lengua de Señas Mexicana (LSM).
	\item Obtener retroalimentación de usuarios potenciales y especialistas para identificar áreas de oportunidad y mejora en el prototipo.
\end{itemize}

\paragraph{\textbf{Cronograma del despliegue.}}
El cronograma de despliegue se fundamenta en la planificación estratégica de las actividades necesarias para la publicación, prueba y evaluación del prototipo. Estas actividades incluyen la ejecución de pruebas internas, la validación con usuarios, la recopilación sistemática de observaciones y la realización de ajustes finales basados en los resultados obtenidos.


Las actividades específicas y sus tiempos estimados se encuentran detallados en el anexo titulado \textbf{\nameref{anexo:actividades_academicas}} y en \textbf{\nameref{anexo:actividades_comercial}}

\paragraph{\textbf{Costos del despliegue.}}
La fase de despliegue representa una etapa crítica del proyecto, ya que permite validar el funcionamiento del prototipo, ajustar detalles técnicos y garantizar que cumpla con los criterios de calidad y usabilidad esperados.

Para ofrecer una visión integral, esta sección presenta dos enfoques diferenciados en la estimación de costos:

\begin{itemize}
	\item \textbf{Enfoque académico:} se consideran costos simbólicos asociados a la ejecución del proyecto en un entorno universitario, utilizando recursos propios y trabajo colaborativo de los estudiantes. Este enfoque refleja la realidad operativa del desarrollo del prototipo en su contexto actual.
	
	\item \textbf{Enfoque exploratorio de comercialización:} se presenta una estimación realista basada en tarifas de mercado, que contempla los costos necesarios para validar, documentar y lanzar el prototipo como un producto funcional en un entorno profesional, incluyendo personal especializado, infraestructura y herramientas de control de calidad.
\end{itemize}

\vspace{1em}
\noindent\textbf{Resumen de costos (despliegue) – Enfoque académico:}
\begin{itemize}
	\item \textbf{Gestión de calidad y pruebas internas:} \$3,600.00 MXN (simbólico).
	\item \textbf{Validación con usuarios (pruebas piloto y retroalimentación):} \$7,500.00 MXN (simbólico).
	\item \textbf{Documentación y ajustes finales del prototipo:} \$1,500.00 MXN (simbólico).
\end{itemize}

\noindent\textbf{Costo total estimado (académico):} \textbf{\$12,600.00 MXN}

\vspace{1em}
\noindent\textbf{Resumen de costos (despliegue) – Enfoque comercial:} (según tarifas profesionales conservadoras)

\begin{itemize}
	\item \textbf{Gestión de calidad (definición de estándares, métricas y control):} \$6,520.00 MXN.
	\item \textbf{Pruebas técnicas e integración (unitarias, con usuarios, corrección de errores):} \$15,000.00 MXN.
	\item \textbf{Preparación de entorno de producción y despliegue técnico:} \$16,000.00 MXN.
\end{itemize}

\noindent\textbf{Costo total estimado (comercial):} \textbf{\$37,520.00 MXN}

\vspace{1em}
Esta doble perspectiva permite comparar la ejecución del despliegue del prototipo en un entorno académico y en un escenario comercial futuro, facilitando la toma de decisiones sobre su escalabilidad y viabilidad económica.

\paragraph{\textbf{Métricas para medir el éxito.}}
Para evaluar la efectividad del despliegue del prototipo, se definirán los siguientes indicadores clave de desempeño (KPIs):

\begin{itemize}
	\item Porcentaje de pruebas funcionales exitosas (fluidez y comprensión de las animaciones 3D): superior al 90\%.
	\item Nivel de satisfacción o retroalimentación positiva de los usuarios evaluadores: superior al 85\%.
\end{itemize}

El cumplimiento de estas métricas permitirá validar que el prototipo alcanza los niveles de funcionalidad, accesibilidad e inclusión planteados como objetivos principales del proyecto.

\paragraph{\textbf{Análisis exploratorio de costo de desarrollo.}}
\paragraph{\textbf{Escenario de posible futura etapa de comercialización.}}
Este apartado presenta una proyección financiera elaborada bajo un enfoque comercial, con el objetivo de estimar los costos y condiciones necesarias para escalar el prototipo de aplicación a un producto funcional en el mercado. Aunque el presente trabajo terminal tiene un carácter académico, esta simulación permite anticipar los requerimientos económicos y técnicos para una eventual implementación comercial.

\paragraph{\textbf{Análisis de costos.}}
A continuación, se presenta el desglose de costos estimados para una fase inicial de comercialización del prototipo, considerando sueldos profesionales, infraestructura tecnológica, servicios especializados, mantenimiento y una reserva para riesgos:

\begin{table}[H]
	\centering
	\renewcommand{\arraystretch}{1.5}
	\setlength{\tabcolsep}{12pt}
	\resizebox{\textwidth}{!}{%
		\begin{tabular}{|l|r|}
			\hline
			\textbf{Concepto} & \textbf{Monto estimado (MXN \$)} \\ \hline
			Presupuesto total de actividades & \$179,280.00 \\ \hline
			Presupuesto total de recursos (desarrollo) & \$123,000.00 \\ \hline
			Presupuesto total de recursos (despliegue) & \$369,000.00 \\ \hline
			Compras no recurrentes & \$162,500.00 \\ \hline
			Sueldo del equipo (desarrollo) & \$544,000.00 \\ \hline
			Sueldo del equipo (mantenimiento) & \$696,000.00 \\ \hline
			Reserva para riesgos e imprevistos (15\%) & \$26,892.00 \\ \hline
			\textbf{Presupuesto total estimado del proyecto} & \textbf{\$2,101,346.00} \\ \hline
		\end{tabular}%
	}
	\caption[Proyección de costos para una posible fase comercial]{Proyección de costos para una posible fase comercial, elaboración propia.}	
	\label{tab:costos_desarrollo}
\end{table}

\noindent \textbf{Nota:} Esta estimación considera valores realistas para una implementación comercial, incluyendo tarifas conservadoras por perfil profesional, recursos tecnológicos y una reserva destinada a cubrir contingencias operativas o técnicas.

\paragraph{\textbf{Determinación del precio de desarrollo.}}

Para calcular un precio base de suscripción mensual, se considera el siguiente escenario:

\begin{itemize}
	\item Número de usuarios esperados: \textbf{200}.
	\item Tiempo promedio de uso por usuario: \textbf{40 horas mensuales}.
	\item Periodo de recuperación de inversión: \textbf{24 meses}.
\end{itemize}

\paragraph{\textbf{Cálculo del precio base mensual.}}

\[
\text{Costo mensual del proyecto} = \frac{\num{2101346.00}}{24} = \num{87556.08}
\]

\[
\text{Costo mensual por usuario} = \frac{\num{87556.08}}{200} = \num{437.78}
\]

\[
\text{Costo por hora de uso} = \frac{\num{437.78}}{40} = \num{10.94}
\]

\paragraph{\textbf{Propuesta de precio final.}}
Considerando un margen de ganancia del \textbf{20\%}, el precio sugerido de suscripción mensual sería:

\[
\text{Precio base mensual} = \num{437.78} \times 1.20 = \num{525.34}
\]

Redondeando al entero más cercano:

\[
\text{Precio base mensual redondeado} = \num{525.00}
\]

\begin{flushleft}
	\textbf{Nota final:} esta simulación financiera tiene carácter exploratorio y busca establecer un marco inicial para la evaluación económica del producto. Las cifras deberán ser validadas y refinadas mediante estudios de mercado y pruebas piloto antes de su implementación real.
\end{flushleft}



\subsubsection{Recursos humanos}
El equipo de desarrollo está conformado por tres estudiantes de la carrera de Ingeniería en Inteligencia Artificial, con experiencia académica en proyectos de Ingeniería de software. Esta preparación garantiza que el equipo cuenta con las competencias necesarias para realizar el análisis, diseño, implementación, prueba y validación del prototipo propuesto.

Adicionalmente, se contempla la posibilidad de realizar capacitaciones específicas en el manejo de conjuntos de datos de señas y técnicas de animación 3D, con el objetivo de reforzar el conocimiento técnico necesario para los módulos especializados del sistema.

La asignación de tres integrantes al proyecto se fundamenta mediante una estimación del esfuerzo utilizando el modelo \textbf{COCOMO II (Constructive Cost Model II)}, específicamente en su modalidad \textit{Early Design Model}, apropiada para proyectos en etapas iniciales [CITA IEEE].

La fórmula del modelo es la siguiente:

\[
\text{Esfuerzo (PM)} = A \cdot (\text{Tamaño})^E \cdot \prod EM_i
\]

Donde:

\begin{itemize}
	\item \textbf{PM} = Persona-meses estimados de esfuerzo.
	\item \textbf{A} = Constante base del modelo. Se usa el valor estándar de $A = 2.94$.
	\item \textbf{Tamaño} = Tamaño estimado del software en miles de líneas de código (KLOC). Se considera un tamaño de $2$ KLOC (2,000 líneas de código).
	\item \textbf{E} = Exponente de escala. Para este proyecto se asume $E = 1.05$, valor apropiado para proyectos con requisitos moderadamente definidos.
	\item $\prod EM_i$ = Producto de los multiplicadores de esfuerzo. Se asume un valor neutral de $1.0$, al tratarse de un entorno académico con condiciones estándar.
\end{itemize}

Sustituyendo los valores en la fórmula:

\[
\text{PM} = 2.94 \cdot (2)^{1.05} \cdot 1.0 \approx 2.94 \cdot 2.07 = \textbf{6.09 persona-meses.}
\]

Considerando un periodo de desarrollo de aproximadamente ocho meses, el esfuerzo mensual requerido se calcula como:

\[
\frac{6.09 \text{ persona-meses}}{8 \text{ meses}} = \num{0.76125} \approx \num{0.76} \text{ personas por mes.}
\]

Aunque el esfuerzo promedio mensual es inferior a una persona de tiempo completo, se justifica la participación de al menos tres integrantes en el proyecto, debido a las siguientes razones:

\begin{itemize}
	\item Permite distribuir responsabilidades de forma especializada entre los tres módulos principales del sistema: procesamiento de lenguaje natural, generación de animaciones 3D y desarrollo móvil.
	\item Favorece el trabajo en paralelo, lo que reduce los tiempos de espera entre etapas y facilita el cumplimiento de plazos intermedios.
	\item Mejora la calidad del prototipo al facilitar revisiones cruzadas, pruebas continuas y retroalimentación iterativa.
	\item Minimiza el riesgo operativo frente a posibles contingencias como cargas académicas variables, imprevistos personales o rotación parcial del equipo.
\end{itemize}


\subsubsection{Recursos tecnológicos}
El equipo de trabajo dispone del equipo de cómputo necesario para el desarrollo, prueba y validación de la aplicación móvil. Además, se cuenta con acceso a las plataformas de desarrollo y a las herramientas de software requeridas, tales como ambientes de programación, motores de animación 3D y bibliotecas especializadas de procesamiento de lenguaje natural (PLN). Esta disponibilidad de recursos tecnológicos garantiza las condiciones adecuadas para la implementación efectiva del prototipo.

\subsubsection{Plazo}
El tiempo estimado para el desarrollo del prototipo es de aproximadamente ocho meses, considerando las fases de análisis, diseño, desarrollo, pruebas y presentación final. Dado que el alcance del proyecto se limita a la traducción de un conjunto predefinido de frases específicas, se considera que el plazo establecido es adecuado para cumplir con los objetivos planteados y la solución propuesta. \\


El análisis realizado en esta sección, permite concluir que el proyecto es tanto viable como factible dentro del contexto académico en el cual se desarrolla. El equipo de trabajo cuenta con los conocimientos fundamentales, el acceso a las tecnologías requeridas y los recursos necesarios para la construcción de un prototipo funcional. Aunque se han identificado áreas que demandarán un proceso de capacitación complementaria, estas no representan un obstáculo significativo para el éxito del proyecto, siempre que se mantenga una adecuada gestión de tiempos y actividades conforme a los alcances establecidos. De este modo, se fortalecen las condiciones para lograr un desarrollo efectivo que cumpla con los objetivos planteados y a futuro siente las bases para una posible expansión comercial.

\section{Análisis y diseño de sistemas}
\subsection{Entregables}

\subsubsection{Diagrama de arquitectura}
\begin{center}
	\includegraphics[width=0.97\textwidth]{Images/Cap 3/Arquitectura_Grande.jpg}
	\captionof{figure}[Diagrama de arquitectura]{Diagrama de arquitectura, elaboración propia.}  % Pie de foto manual
\end{center}

 \begin{flushleft} \href{https://miro.com/app/board/uXjVI24dV0c=/?share_link_id=690276332577}{\textbf{Ver el diagrama de arquitetcura con detalle}} \end{flushleft}

\subsubsection{Reglas de negocio}
\begin{itemize}[leftmargin=1.5cm]
    \item \textbf{RN01:} El sistema debe permitir al usuario solo ingresar texto de forma manual o pegarlo desde otra aplicación.
    \item \textbf{RN02:} El campo de entrada debe tener un límite de caracteres (por ejemplo, 30 caracteres).
    \item \textbf{RN03:} El sistema debe validar el texto ingresado para evitar caracteres no permitidos.
    \item \textbf{RN04:} El botón de traducción debe estar habilitado solo cuando el campo de texto no esté vacío.
    \item \textbf{RN05:} El sistema debe mostrar un mensaje de error si el texto ingresado no es válido.
    \item \textbf{RN06:} El sistema debe permitir al usuario editar el texto antes de enviarlo a traducción.
	\item \textbf{RN07:} El sistema siempre debe mostrar alguna animación, ya sea la correspondiente a la traducción o el deletreo.
\end{itemize}
\subsubsection{Requerimientos funcionales}
\begin{itemize}
    \item \textbf{RF01: Entrada de Texto}  
    La aplicación debe permitir al usuario ingresar y editar texto que será traducido a LSM.
    
    \item \textbf{RF02: Procesamiento de Lenguaje Natural (PLN)}  
    Implementar un módulo que valide, analice y procese el texto ingresado, identificando frases clave y contextos específicos para su traducción.
    
    \item \textbf{RF03: Generación de Avatares 3D}  
    Desarrollar avatares 3D que representen visualmente las señas correspondientes a las frases procesadas. Los avatares deben ser capaces de mostrar expresiones faciales y movimientos corporales que reflejen la gramática y sintaxis de la LSM. 
    
    \item \textbf{RF04: Modo de Deletreo}  
    El sistema debe intentar asignarle una traducción a la frase ingresada. En caso de que a una frase no se le pueda asignar una traducción, la aplicación debe deletrear palabra por palabra utilizando el alfabeto dactilológico de la LSM.
    
    
\end{itemize}


\subsubsection{Requerimientos no funcionales}
\begin{itemize}
    \item \textbf{RnF01: Rendimiento}  
    La aplicación debe cargar y procesar las traducciones en menos de 2 segundos, garantizando una experiencia de usuario rápida y eficiente.
    
    \item \textbf{RnF02: Usabilidad}  
    La interfaz debe ser fácil de navegar, con instrucciones claras y botones de acción bien definidos, siguiendo las heurísticas de usabilidad de Nielsen.

	 \item \textbf{RnF03: Animaciones y Transiciones}  
    Las animaciones entre señas deben ser fluidas y naturales, utilizando IA para optimizar las transiciones y evitar movimientos bruscos o poco realistas.
    
    \item \textbf{RnF04: Interfaz de Usuario (UI)}  
    Diseñar una interfaz intuitiva y accesible que permita a los usuarios ingresar texto personalizado para su traducción.

    \item \textbf{RnF05: Actualización del contenido lingüístico}
	El sistema debe permitir la incorporación y actualización de nuevas versiones del LSM, incluyendo modificaciones en señas existentes, adición de nuevas señas o adaptaciones conforme evolucionen los estándares lingüísticos oficiales.
	\item \textbf{RnF06: Compatibilidad}  
    La aplicación debe ser compatible con la última versión de Android, garantizando su funcionamiento en una amplia gama de dispositivos móviles.
\end{itemize}

\subsubsection{Diagrama de procesos}
\begin{center}
    \makebox[\textwidth]{%
        \includegraphics[width=1.2\textwidth]{Images/Cap 3/Procesos.jpeg}
    }
    \captionof{figure}[Diagrama de procesos del sistema]{Diagrama de procesos del sistema, elaboración propia.}
\end{center}

\subsubsection{Diagrama de actividades} 
\begin{center}
	\makebox[\textwidth]{%
		\includegraphics[width=1\textwidth]{Images/Cap 3/actividades.png}
	}
    \captionof{figure}{Diagrama de actividades del sistema}
\end{center}


\subsubsection{Diagrama de clases}
\begin{center}
	\makebox[\textwidth]{%
		\includegraphics[width=1\textwidth]{Images/Cap 3/tapecito.png}
	}
	\captionof{figure}{Diagrama de clases del sistema}
\end{center}

\subsubsection{Diagrama de secuencia}
\begin{center}
	\makebox[\textwidth]{%
		\includegraphics[width=1\textwidth]{Images/Cap 3/Secuencia.png}
	}
    \captionof{figure}{Diagrama de secuencia del sistema}
\end{center}

\subsubsection{Casos de uso}
\subsubsection{Caso de uso 01: Traducir texto}
\subsubsection{Resumen}
El usuario ingresa texto en la aplicación, que luego es procesado y traducido a LSM para visualizar una animación correspondiente a la traducción.
\subsubsection{Descripción}

\noindent
\begin{tabularx}{\textwidth}{|l|X|}
\hline
\textbf{Caso de uso} & Traducir texto \\ \hline

\textbf{Actor principal} & Usuario (persona que ingresa el texto a traducir). \\ \hline

\textbf{Descripción} & El usuario ingresa texto en el campo indicado de la aplicación. El sistema procesa la frase y determina si existe una traducción directa en LSM. En caso afirmativo, se reproduce la animación correspondiente con un avatar 3D. Si no se encuentra una coincidencia, el sistema recurre al deletreo dactilológico para representar el texto ingresado. \\ \hline

\textbf{Atributos del texto} & Máximo 50 caracteres. Validación para evitar caracteres especiales o símbolos no admitidos. \\ \hline

\textbf{Entrada} & Texto proporcionado por el usuario. \\ \hline

\textbf{Procesamiento} & El módulo de procesamiento de lenguaje natural (PLN) analiza el texto y busca coincidencias en el conjunto de datos. Si no hay coincidencia, se segmenta y transforma en deletreo letra por letra. \\ \hline

\textbf{Salida} & Se muestra la animación en 3D de la traducción en LSM o, en su defecto, del deletreo dactilológico correspondiente. Se incluye un mensaje visual de inicio con la palabra "Vamos". \\ \hline

\textbf{Precondiciones} & El sistema debe estar encendido, funcional y el usuario debe encontrarse en la pantalla principal con acceso al campo de texto. \\ \hline

\textbf{Postcondiciones} & El usuario visualiza la animación correspondiente y tiene la opción de ingresar una nueva frase para traducir. \\ \hline

\textbf{Excepciones} & Si el texto excede el límite o contiene caracteres no válidos, el sistema mostrará un mensaje de error con base en la RN05 e invitará a corregir la entrada. \\ \hline
\end{tabularx}
\captionof{table}[Caso de uso 1]{Caso de uso 1, elaboración propia.}


    

\subsubsection{Trayectoria principal}
\begin{enumerate}[label=\textbf{\arabic*}, leftmargin=1.5cm]
    \item \UCsystem \ El sistema muestra un campo de entrada de texto en la pantalla principal de la aplicación, permitiendo al usuario ingresar texto.  
    Además, la interfaz incluye:  
    \begin{itemize}
        \item Botón para iniciar la traducción.
    \end{itemize}

    \item \UCactor \ Ingresa el texto en el campo de entrada.  
   
    \item \UCsystem \ El sistema procesa el texto ingresado y le asigna una traducción a LSM, mostrandole la animación asignada si la hubo o el deletreo dactilológico si no se le pudo asignar una.

    \item \UCactor \ Puede ingresar texto nuevo y volver a interar la traducción.

\end{enumerate}

\subsubsection{Trayectoria alternativa A}
\begin{enumerate}[label=\textbf{\arabic*}, leftmargin=1.5cm]
    \item \UCactor \ Ingresa el texto con caracteres no permitidos o excede el límite de caracteres.
	\item \UCsystem \ El sistema muestra un mensaje de error indicando que el texto no es válido con base en la RN05.  
	\item \UCactor \ El usuario corrige el texto y vuelve a intentar la traducción.
\end{enumerate}

\subsubsection{Trayectoria alternativa B}
\begin{enumerate}[label=\textbf{\arabic*}, leftmargin=1.5cm]
    \item \UCactor \ Intenta traducir el texto sin haberlo ingresado.
	\item \UCsystem \ El sistema muestra un mensaje de error indicando que el campo de texto está vacío con base en la RN04.
	\item \UCactor \ El usuario ingresa el texto y vuelve a intentar la traducción.
\end{enumerate}

\textit{--- Fin del caso de uso.}


\subsubsection{Diagrama de casos de uso}
\begin{center}
    \includegraphics[width=0.6\textwidth]{Images/Cap 3/casodeuso.png}
    \captionof{figure}{Diagrama de caso de uso 1}
\end{center}


\subsubsection{Identificación de frases}

\begin{table}[H]
\centering
\begin{tabularx}{\textwidth}{|X|X|X|}
\hline
\textbf{Emergencia (3 a 5 palabras)} & \textbf{Saludo (2 a 4 palabras)} & \textbf{Agradecimiento/ Respuestas (2 a 4 palabras)} \\ \hline
Ayuda, por favor. & Hola, ¿cómo estás? & Gracias. \\ \hline
Llama a la policía. & Buenos días. & Muchas gracias. \\ \hline
Necesito un médico. & Buenas tardes. & Te lo agradezco. \\ \hline
Estoy herido/a. & Buenas noches. & Qué amable. \\ \hline
¿Dónde está el hospital? & ¿Qué tal? & Estoy muy agradecido/a. \\ \hline
Es una emergencia. & Mucho gusto. & Gracias por tu ayuda. \\ \hline
¿Puedes ayudarme? & Encantado/a de conocerte. & Cuando quieras. \\ \hline
Necesito asistencia urgente. & ¿Cómo te llamas? & No hay problema. \\ \hline
¿Dónde está la salida? & Me llamo [nombre]. & No pasa nada. \\ \hline
¡Fuego! ¡Fuego! & Hasta luego. & Es un placer. \\ \hline
\end{tabularx}
\caption[Identificación de frases]{Identificación de frases, elaboración propia.}
\end{table}
Las frases seleccionadas son ejemplos representativos de situaciones cotidianas y de emergencia. Se busca que el usuario pueda comunicarse de manera efectiva en contextos importantes, facilitando la comprensión y la interacción con personas que utilizan la Lengua de Señas Mexicana (LSM).

\subsection{Mockups del sistema}
\begin{center}
    \includegraphics[width=0.5\textwidth]{Images/Cap 3/Pantalla1.png}
    \captionof{figure}{Mockup de la pantalla de carga de la aplicación}
\end{center}

La primera pantalla que se muestra es la pantalla de carga de la aplicación, la cual muestra el mensaje " Espere por favor", junto con el ícono de un reloj para indicar que la app está cargando.

\begin{center}
    \includegraphics[width=0.5\textwidth]{Images/Cap 3/Pantalla2.png}
    \captionof{figure}{Mockup de la pantalla de ingreso de texto}
\end{center}
 
La seguda pantalla de la aplicación, la pantalla principal, es la pantalla para ingresar el texto. Se muestra el nombre de la app y  un cuadro para ingresar el texto y un botón para acceder a la traducción.\\

Se debe ingresar texto dentro del cuadro de entrada de texto, para que se realice todo el Procesamiento de Lenguaje Natural y mostrar la animación.

\begin{center}
    \includegraphics[width=0.5\textwidth]{Images/Cap 3/Pantalla3.png}
    \captionof{figure}{Mockup de la pantalla la animación}
\end{center}

La tercera pantalla de la aplicación es la pantalla en donde se muestra la animación 3D asignada a la traducción, para que el usuario sepa en que momento inicia la traducción se le mostrará un mensaje con la palabra "Vamos" . Además se le indica la frase ingresada del usuario y la traducción asignada. Debajo del avatar 3D, se vuelve a mostrar el cuadro de entrada de texto para ingresar un nuevo mensaje.

\begin{center}
    \includegraphics[width=0.5\textwidth]{Images/Cap 3/Pantalla4.png}
    \captionof{figure}{Mockup de la pantalla la animación}
\end{center}

En caso de que no se le pueda asignar una traducción a la frase ingresada, se mostrará la pantalla de deletreo dactilológico, al igual que la Pantalla anterior para que el usuario sepa en que momento inicia la traducción se le mostrará un mensaje con la palabra "Vamos". En esta pantalla se muestra el avatar 3D realizando el deletreo de la frase ingresada por el usuario. Al igual que en la pantalla anterior, se vuelve a mostrar el cuadro de entrada de texto para ingresar un nuevo mensaje.

\begin{center}
	\includegraphics[width=0.5\textwidth]{Images/Cap 3/Paleta.png}
	\captionof{figure}{Paleta de colores de la aplicación}
\end{center}
Se selccionó la siguiente paleta de colores para la aplicación, ya que usa colores representativos de la bandera de LSM, además de que los colores son agradables a la vista y no generan cansancio visual.

\subsection{Identificación y evaluación de riesgos}
\subsubsection{Leyenda de probabilidad y efecto}
\textbf{Probabilidad:}
\begin{itemize}
	\item \textbf{Muy alta ($>$75\%):} Altamente probable que ocurra.
	\item \textbf{Alta (50-75\%):} Probable que ocurra en varias ocasiones durante el proyecto.
	\item \textbf{Moderada (25-50\%):} Existe una posibilidad razonable de que ocurra.
	\item \textbf{Baja (10-25\%):} Poco probable que ocurra, pero no imposible.
\end{itemize}

\textbf{Efecto (Impacto):}
\begin{itemize}
	\item \textbf{Catastrófico:} Afecta gravemente el éxito del proyecto, podría impedir la continuidad del mismo.
	\item \textbf{Serio:} Genera retrasos significativos o pérdida parcial de calidad en el desarrollo o resultados.
	\item \textbf{Tolerable:} Impacto menor, manejable sin afectar los objetivos generales del proyecto.
\end{itemize}


% ===== RIESGOS TÉCNICOS =====
\subsubsection{Riesgos técnicos}
Los riesgos técnicos están relacionados con las limitaciones de la tecnología utilizada, la precisión del modelo de traducción y la adaptación a las particularidades de la Lengua de Señas Mexicana (LSM).

\setlength{\tabcolsep}{4pt}
\renewcommand{\arraystretch}{1.2}

\begin{longtable}{|>{\centering\arraybackslash}p{0.8cm}|>{\raggedright\arraybackslash}p{3.5cm}|>{\raggedright\arraybackslash}p{5.1cm}|>{\raggedright\arraybackslash}p{5.1cm}|}
	\hline
	\textbf{ID} & \textbf{Riesgo específico} & \textbf{Probabilidad} & \textbf{Efecto} \\
	\hline
	T1 & Precisión limitada del modelo de traducción de español a LSM. & Alta (50-75\%) & Serio \\
	\hline
	T2 & Escasa disponibilidad de datasets de calidad para LSM. & Alta (50-75\%) & Serio \\
	\hline
	T3 & Desactualización tecnológica del modelo de IA frente a avances rápidos en el área. & Moderada (25-50\%) & Serio \\
	\hline
	T4 & Dificultad para manejar las variaciones regionales y culturales de LSM. & Media (25-50\%) & Catastrófico \\
	\hline
	T5 & Problemas de compatibilidad y funcionamiento multiplataforma. & Moderado (25-50\%) & Serio \\
	\hline
\caption[Resumen de riesgos técnicos]{Resumen de riesgos técnicos, elaboración propia.} \label{tab:riesgos_tecnicos_resumen} \\

\end{longtable}

% ===== RIESGOS FINANCIEROS Y COMERCIALES =====
\subsubsection{Riesgos financieros y comerciales}
Estos riesgos afectan la viabilidad económica del proyecto y su posible escalamiento hacia una fase comercial.

\setlength{\tabcolsep}{4pt}
\renewcommand{\arraystretch}{1.2}

\begin{longtable}{|>{\centering\arraybackslash}p{0.8cm}|>{\raggedright\arraybackslash}p{3.5cm}|>{\raggedright\arraybackslash}p{5.1cm}|>{\raggedright\arraybackslash}p{5.1cm}|}
	\hline
	\textbf{ID} & \textbf{Riesgo específico} & \textbf{Probabilidad} & \textbf{Efecto} \\
	\hline
	F1 & Subestimación de los costos de producción y operación en una fase comercial. & Alta (50-75\%) & Catastrófico \\
	\hline
	F2 & Falta de modelos de negocio viables para monetizar la solución. & Moderada (25-50\%) & Serio \\
	\hline
	F3 & Dependencia de apoyos gubernamentales o financiamiento social para escalar el proyecto. & Media (25-50\%) & Serio \\
	\hline
	F4 & Competencia con otras soluciones similares con mayor madurez o presencia en el mercado. & Media (25-50\%) & Serio \\
	\hline

\caption[Resumen de riesgos financieros y comerciales]{Resumen de riesgos financieros y comerciales, elaboración propia.} \label{tab:riesgos_financieros_resumen} \\
\end{longtable}

% ===== RIESGOS HUMANOS Y DE GESTIÓN =====
\subsubsection{Riesgos humanos y de gestión}
Estos riesgos se refieren a la disponibilidad, capacitación y coordinación del equipo, así como a la correcta gestión del proyecto.

\setlength{\tabcolsep}{4pt}
\renewcommand{\arraystretch}{1.2}

\begin{longtable}{|>{\centering\arraybackslash}p{0.8cm}|>{\raggedright\arraybackslash}p{3.5cm}|>{\raggedright\arraybackslash}p{5.1cm}|>{\raggedright\arraybackslash}p{5.1cm}|}
	\hline
	\textbf{ID} & \textbf{Riesgo específico} & \textbf{Probabilidad} & \textbf{Efecto} \\
	\hline
	H1 & Falta de experiencia del equipo en procesos de validación lingüística y cultural de LSM. & Alta (50-75\%) & Serio \\
	\hline
	H2 & Dependencia de colaboración externa para la validación de señas. & Media (25-50\%) & Serio \\
	\hline
	H3 & Desmotivación o alta rotación del equipo técnico. & Moderada (25-50\%) & Serio \\
	\hline
	H4 & Falta de claridad o definición adecuada de los requerimientos funcionales. & Alta (50-75\%) & Serio \\
	\hline
\caption[Resumen de riesgos humanos y de gestión]{Resumen de riesgos humanos y de gestión, elaboración propia.} \label{tab:riesgos_humanoss_resumen} \\
\end{longtable}

% ===== RIESGOS ÉTICOS Y REGULATORIOS =====
\subsubsection{Riesgos éticos y regulatorios}
Riesgos asociados a las normativas, certificaciones y cuestiones éticas, especialmente por el tipo de aplicación y su uso potencial en contextos sensibles.

\setlength{\tabcolsep}{4pt}
\renewcommand{\arraystretch}{1.2}

\begin{longtable}{|>{\centering\arraybackslash}p{0.8cm}|>{\raggedright\arraybackslash}p{3.5cm}|>{\raggedright\arraybackslash}p{5.1cm}|>{\raggedright\arraybackslash}p{5.1cm}|}
	\hline
	\textbf{ID} & \textbf{Riesgo específico} & \textbf{Probabilidad} & \textbf{Efecto} \\
	\hline
	E1 & Interpretación incorrecta de mensajes sensibles (contexto médico, legal, etc.). & Baja (10-25\%) & Catastrófico \\
	\hline
	E2 & Ausencia de certificaciones o validaciones oficiales del modelo de traducción. & Moderada (25-50\%) & Serio \\
	\hline
	E3 & Uso no autorizado de bases de datos o glosarios protegidos. & Baja (10-25\%) & Catastrófico \\
	\hline
	E4 & Incumplimiento con normativas de accesibilidad digital o protección de datos. & Moderada (25-50\%) & Catastrófico \\
	\hline
\caption[Resumen de riesgos éticos y regulatorios]{Resumen de riesgos éticos y regulatorios, elaboración propia.} \label{tab:riesgos_eticos_resumen} \\
\end{longtable}

\newpage
\subsection{Planes de prevención y contingencia para riegos}

Esta sección describe las acciones preventivas y los planes de contingencia asociados a los riesgos identificados para el proyecto. La información se organiza por categoría de riesgo, facilitando la identificación y trazabilidad de las medidas frente a cada riesgo.

\subsubsection{Planes de prevención y contingencia para riesgos técnicos}

\setlength{\tabcolsep}{4pt}
\renewcommand{\arraystretch}{1.2}

\begin{longtable}{|>{\centering\arraybackslash}p{0.8cm}|>{\raggedright\arraybackslash}p{3.5cm}|>{\raggedright\arraybackslash}p{5.1cm}|>{\raggedright\arraybackslash}p{5.1cm}|}
	\hline
	\textbf{ID} & \textbf{Riesgo específico} & \textbf{Prevención} & \textbf{Plan de contingencia} \\
	\hline
	T1 & Precisión limitada del modelo de traducción de español a LSM. &
	\begin{itemize}
		\item Validar el modelo con expertos en LSM.
		\item Realizar pruebas piloto con retroalimentación continua.
		\item Incorporar técnicas de mejora continua en el entrenamiento.
	\end{itemize} &
	\begin{itemize}
		\item Ajustar el modelo mediante reentrenamiento con nuevos datos.
		\item Aplicar correcciones manuales temporales mientras se mejora la precisión.
		\item Evaluar modelos alternativos si el desempeño es insatisfactorio.
	\end{itemize} \\
	\hline
	T2 & Escasa disponibilidad de datasets de calidad para LSM. &
	\begin{itemize}
		\item Buscar colaboración con instituciones o expertos en LSM.
		\item Utilizar técnicas de data augmentation para ampliar los datos existentes.
	\end{itemize} &
	\begin{itemize}
		\item Integrar validaciones humanas adicionales para compensar la falta de datos.
		\item Ajustar el alcance del proyecto si los datos son insuficientes.
	\end{itemize} \\
	\hline
	T3 & Desactualización tecnológica del modelo de IA. &
	\begin{itemize}
		\item Mantener vigilancia tecnológica constante.
		\item Participar en foros y comunidades sobre IA y accesibilidad.
	\end{itemize} &
	\begin{itemize}
		\item Migrar a nuevas versiones o tecnologías según los avances detectados.
		\item Ajustar la arquitectura para facilitar futuras actualizaciones.
	\end{itemize} \\
	\hline
	T4 & Dificultad para manejar las variaciones regionales y culturales de LSM. &
	\begin{itemize}
		\item Buscar especialistas del área para el uso de región específica.
	\end{itemize} &
	\begin{itemize}
		\item Limitar el alcance del prototipo a ciertas regiones mientras se extiende la cobertura.
		\item Informar claramente las limitaciones del modelo a los usuarios.
	\end{itemize} \\
	\hline
	T5 & Problemas de compatibilidad y funcionamiento multiplataforma. &
	\begin{itemize}
		\item Desarrollar bajo estándares multiplataforma.
		\item Realizar pruebas en distintos dispositivos y navegadores.
	\end{itemize} &
	\begin{itemize}
		\item Aplicar correcciones específicas por plataforma detectada.
		\item Priorizar las plataformas con mayor demanda de usuarios.
	\end{itemize} \\
	\hline
	\caption[Planes de prevención y contingencia para riesgos técnicos]{Planes de prevención y contingencia para riesgos técnicos, elaboración propia.} \label{tab:riesgos_tecnicos}\\
\end{longtable}


\newpage
% ========================
% Puedes continuar con las siguientes subsecciones:

\subsubsection{Planes de prevención y contingencia para riesgos financieros y comerciales}

\setlength{\tabcolsep}{4pt}
\renewcommand{\arraystretch}{1.2}

\begin{longtable}{|>{\centering\arraybackslash}p{0.8cm}|>{\raggedright\arraybackslash}p{3.5cm}|>{\raggedright\arraybackslash}p{5.1cm}|>{\raggedright\arraybackslash}p{5.1cm}|}
	\hline
	\textbf{ID} & \textbf{Riesgo específico} & \textbf{Prevención} & \textbf{Plan de contingencia} \\
	\hline
	F1 & Subestimación de los costos de producción y operación en una fase comercial. &
	\begin{itemize}
		\item Elaborar un análisis financiero detallado con escenarios conservadores.
		\item Incluir márgenes de contingencia en la planificación de costos.
	\end{itemize} &
	\begin{itemize}
		\item Reevaluar los costos y ajustar el modelo de negocio.
		\item Buscar financiamiento adicional o ajustar el alcance del proyecto.
	\end{itemize} \\
	\hline
	F2 & Falta de modelos de negocio viables para monetizar la solución. &
	\begin{itemize}
		\item Diseñar y evaluar diferentes modelos de negocio desde la fase temprana.
		\item Consultar expertos en comercialización y accesibilidad.
	\end{itemize} &
	\begin{itemize}
		\item Ajustar la estrategia hacia modelos freemium, licencias o apoyos institucionales.
		\item Explorar alianzas con organizaciones del sector social o educativo.
	\end{itemize} \\
	\hline
	F3 & Dependencia de apoyos gubernamentales o financiamiento social para escalar el proyecto. &
	\begin{itemize}
		\item Diversificar las posibles fuentes de financiamiento (fondos privados, crowdfunding).
		\item Preparar la documentación requerida para aplicar a diferentes programas.
	\end{itemize} &
	\begin{itemize}
		\item Redimensionar el proyecto a una escala mínima viable si no se obtienen los fondos esperados.
		\item Buscar inversionistas privados o aliados estratégicos.
	\end{itemize} \\
	\hline
	F4 & Competencia con otras soluciones similares con mayor madurez o presencia en el mercado. &
	\begin{itemize}
		\item Realizar un monitoreo constante del mercado y de las soluciones existentes.
		\item Diferenciar la propuesta de valor en facilidad de uso, precio o calidad.
	\end{itemize} &
	\begin{itemize}
		\item Ajustar el enfoque del producto según las necesidades no cubiertas por la competencia.
		\item Enfocar el desarrollo en nichos específicos o sectores desatendidos.
	\end{itemize} \\
	\hline
	 \caption[Planes de prevención y contingencia para riesgos financieros y comerciales]{Planes de prevención y contingencia para riesgos financieros y comerciales, elaboración propia.} \label{tab:riesgos_financieros}\\
\end{longtable}

\newpage

\subsubsection{Planes de prevención y contingencia para riesgos humanos y de gestión}

\setlength{\tabcolsep}{4pt}
\renewcommand{\arraystretch}{1.2}

\begin{longtable}{|>{\centering\arraybackslash}p{0.8cm}|>{\raggedright\arraybackslash}p{3.5cm}|>{\raggedright\arraybackslash}p{5.1cm}|>{\raggedright\arraybackslash}p{5.1cm}|}
	\hline
	\textbf{ID} & \textbf{Riesgo específico} & \textbf{Prevención} & \textbf{Plan de contingencia} \\
	\hline
	H1 & Falta de experiencia del equipo en validación lingüística y cultural de LSM. &
	\begin{itemize}
		\item Incluir asesoría de expertos en LSM durante el desarrollo.
		\item Capacitar al equipo en aspectos básicos de la cultura y lengua de señas.
	\end{itemize} &
	\begin{itemize}
		\item Contratar o colaborar con intérpretes certificados para cubrir las áreas necesarias.
		\item Ajustar las pruebas de validación incorporando especialistas externos.
	\end{itemize} \\
	\hline
	H2 & Dependencia de colaboración externa para la validación de señas. &
	\begin{itemize}
		\item Establecer acuerdos formales con colaboradores e intérpretes desde el inicio.
		\item Definir tiempos y compromisos claros de participación.
	\end{itemize} &
	\begin{itemize}
		\item Buscar reemplazos o alianzas adicionales si algún colaborador no cumple lo acordado.
		\item Ajustar el cronograma para incluir tiempo adicional si se presentan retrasos.
	\end{itemize} \\
	\hline
	H3 & Desmotivación o alta rotación del equipo técnico. &
	\begin{itemize}
		\item Fomentar un ambiente laboral positivo y flexible.
		\item Ofrecer oportunidades de aprendizaje y desarrollo profesional.
	\end{itemize} &
	\begin{itemize}
		\item Redistribuir funciones y roles de manera temporal.
		\item Contratar personal de refuerzo o apoyo externo si es necesario.
	\end{itemize} \\
	\hline
	H4 & Falta de claridad o definición adecuada de los requerimientos funcionales. &
	\begin{itemize}
		\item Documentar los requisitos de manera detallada y validarlos con los stakeholders.
		\item Realizar sesiones de revisión y ajuste de requisitos de forma periódica.
	\end{itemize} &
	\begin{itemize}
		\item Detener las tareas críticas hasta tener claridad total sobre los requisitos.
		\item Reorganizar el backlog y ajustar las prioridades si es necesario.
	\end{itemize} \\
	\hline
	\caption[Planes de prevención y contingencia para riesgos humanos y de gestión]{Planes de prevención y contingencia para riesgos humanos y de gestión, elaboración propia.} 	\label{tab:riesgos_humanos}\\
\end{longtable}

\subsubsection{Planes de prevención y contingencia para riesgos éticos y regulatorios}

\setlength{\tabcolsep}{4pt}
\renewcommand{\arraystretch}{1.2}

\begin{longtable}{|>{\centering\arraybackslash}p{0.8cm}|>{\raggedright\arraybackslash}p{3.5cm}|>{\raggedright\arraybackslash}p{5.1cm}|>{\raggedright\arraybackslash}p{5.1cm}|}
	\hline
	\textbf{ID} & \textbf{Riesgo específico} & \textbf{Prevención} & \textbf{Plan de contingencia} \\
	\hline
	E1 & Interpretación incorrecta de mensajes sensibles (contexto médico, legal, etc.). &
	\begin{itemize}
		\item Definir claramente los contextos de uso permitidos del sistema.
		\item Incluir advertencias sobre los límites del modelo en el uso de la aplicación.
	\end{itemize} &
	\begin{itemize}
		\item Limitar temporalmente las funcionalidades en contextos de alto riesgo.
		\item Incorporar revisiones humanas en casos sensibles o críticos.
	\end{itemize} \\
	\hline
	E2 & Ausencia de certificaciones o validaciones oficiales del modelo de traducción. &
	\begin{itemize}
		\item Investigar los procesos de certificación existentes y planear la obtención de las acreditaciones.
		\item Involucrar expertos en accesibilidad y traducción en el proceso de validación.
	\end{itemize} &
	\begin{itemize}
		\item Buscar asesoría externa para cumplir los requisitos regulatorios si se identifican brechas.
		\item Ajustar la fase comercial hasta obtener las validaciones requeridas.
	\end{itemize} \\
	\hline
	E3 & Uso no autorizado de conjunto de datos o glosarios protegidos. &
	\begin{itemize}
		\item Verificar licencias de uso de todos los recursos desde la etapa de diseño.
		\item Priorizar el uso de datos open-source o desarrollados internamente.
	\end{itemize} &
	\begin{itemize}
		\item Sustituir inmediatamente los recursos cuestionados por alternativas con licencias válidas.
		\item Consultar asesoría legal para resolver posibles conflictos de derechos de autor.
	\end{itemize} \\
	\hline
	E4 & Incumplimiento con normativas de accesibilidad digital o protección de datos. &
	\begin{itemize}
		\item Asegurar el cumplimiento de la norma ISO 9241-210:2019 y Heurísticas de Jakob Nielsen.
		\item Consultar especialistas en regulación y accesibilidad.
	\end{itemize} &
	\begin{itemize}
		\item Corregir los puntos de incumplimiento detectados antes del lanzamiento comercial.
		\item Documentar y comunicar las acciones correctivas a los usuarios y autoridades si es necesario.
	\end{itemize} \\
	\hline
\caption[Planes de prevención y contingencia para riesgos éticos y regulatorios]{Planes de prevención y contingencia para riesgos éticos y regulatorios, elaboración propia.} 	\label{tab:riesgos_eticos} \\
\end{longtable}

\section{Plan de Pruebas Inicial}
\subsection{Casos de prueba funcionales y de usabilidad}
Para las pruebas iniciales, se contempla que los usuarios objetivo sean personas pertenecientes a la comunidad sorda de México, específicamente personas con discapacidad auditiva que se comunican mediante la Lengua de Señas Mexicana (LSM) e intérpretes de LSM. En particular, las pruebas de funcionalidad y usabilidad se llevarán a cabo con la participación de alumnos e intérpretes de LSM del Centro de Lenguas Extranjeras (CENLEX) del Instituto Politécnico Nacional, así como de personas externas que formen parte de la comunidad sorda.

Las pruebas consideradas son las siguientes:
\begin{itemize}
\item Evaluación de la precisión de los gestos mostrados en las animaciones 3D correspondientes a frases de emergencia.
\item Evaluación de la precisión de los gestos en las animaciones 3D para la traducción dactilológica (alfabeto manual).
\item Valoración de la fluidez en las animaciones 3D.
\item Análisis de la intuitividad de la aplicación y retroalimentación sobre la elección de la paleta de colores.
\item Satisfacción del usuario con la interfaz general (layout, iconografía, navegación).
\item Validación del conjunto de frases (saludos, despedidas y emergencias) incluido.
\end{itemize}


\chapter{Conclusiones}
En la primera etapa desarrollada del Trabajo Terminal, se establecen las bases teóricas y técnicas necesarias para el desarrollo de una solución tecnológica inclusiva orientada a mejorar la comunicación entre personas oyentes y personas con discapacidad auditiva. A partir del análisis de la motivación, la problemática y los objetivos del proyecto, se delimita un enfoque centrado en la accesibilidad lingüística a través de la Lengua de Señas Mexicana (LSM).\\

La revisión del estado del arte y la construcción del marco teórico permiten contextualizar el proyecto dentro del ámbito del procesamiento de lenguaje natural y la representación visual mediante modelado 3D, evidenciando que, si bien existen herramientas similares a nivel internacional, estas no abordan de manera específica las características lingüísticas, culturales y gramaticales de la LSM. Asimismo, el análisis del proceso de comunicación y el estudio profundo de los elementos lingüísticos de esta lengua (como su estructura espacial, dactilología, fonología y gramática) revela los desafíos técnicos de su implementación digital y la necesidad de soluciones adaptadas al contexto mexicano.


% \chapter{Trabajo relacionado}

% \chapter{Desarrollo}

% \chapter{Experimentos}

% \chapter{Conclusiones y trabajo futuro}

% \appendix
\chapter*{Anexos}
\addcontentsline{toc}{chapter}{Anexos}

\chapter{Ley General para la Inclusión de las Personas con Discapacidad}
\label{anexo:ley_inclusion_disc}
\section{Encabezado de la Ley General para la Inclusión de las Personas con Discapacidad}

\begin{center}
	\makebox[\textwidth]{%
		\includegraphics[width=1\textwidth]{Images/Anexos/Encabezado_Ley.png}
	}
    \captionof{figure}[Encabezado de la Ley General para la Inclusión de las Personas con Discapacidad]{Encabezado de la Ley General para la Inclusión de las Personas con Discapacidad, obtenido de \cite{ref34}}
\end{center}

\section{Artículo 2, Fracción XXII}
\begin{center}
	\makebox[\textwidth]{%
		\includegraphics[width=1\textwidth]{Images/Anexos/Art2_FraccXXII.png}
	}
    \captionof{figure}[Artículo 2, Fracción XXII, de la Ley General para la Inclusión de las Personas con Discapacidad]{Artículo 2, Fracción XXII, de la Ley General para la Inclusión de las Personas con Discapacidad, obtenido de \cite{ref34}}
\end{center}

\section{Artículo 20}
\begin{center}
	\makebox[\textwidth]{%
		\includegraphics[width=1\textwidth]{Images/Anexos/Art20.png}
	}
    \captionof{figure}[Artículo 20]{Artículo 20 de la Ley General para la Inclusión de las Personas con Discapacidad, obtenido de \cite{ref34}}
\end{center}

\chapter{Enfoque por actividades (académico)}
\label{anexo:actividades_academicas}  % Etiqueta para hacer referencia
\section{Etapa: creación del prototipo}

\begin{table}[H]
	\centering
	\renewcommand{\arraystretch}{1.6}
	\setlength{\tabcolsep}{10pt}
	\Huge
	\begin{adjustbox}{max width=\textwidth}
		\begin{tabular}{|p{8cm}|c|r|r|}
			\hline
			\textbf{Tareas (Formulación del proyecto)} & \textbf{Horas} & \textbf{Costo por hora (MXN \$)} & \textbf{Costo total (MXN \$)} \\ \hline
			Descripción del proyecto & 1 & \$150.00 & \$150.00 \\ \hline
			Definir el propósito del proyecto & 1 & \$150.00 & \$150.00 \\ \hline
			Planificación del alcance del proyecto & 1 & \$150.00 & \$150.00 \\ \hline
			Definir las actividades necesarias para completar el proyecto & 1 & \$150.00 & \$150.00 \\ \hline
			Definir tareas prioritarias y bloques de trabajo en paralelo & 2 & \$150.00 & \$300.00 \\ \hline
			Estimar recursos y operaciones & 3 & \$150.00 & \$450.00 \\ \hline
			Establecer los objetivos y metas principales & 1 & \$150.00 & \$150.00 \\ \hline
			Identificación de actividades y tareas & 5 & \$150.00 & \$750.00 \\ \hline
			Planificación del cronograma de actividades & 8 & \$150.00 & \$1,200.00 \\ \hline
			\textbf{Total} & \textbf{23} & -- & \textbf{\$3,450.00} \\ \hline
		\end{tabular}
	\end{adjustbox}
	\caption[Costos estimados para la fase de formulación del proyecto]{Costos estimados para la fase de formulación del proyecto, elaboración propia.} 	
	\label{tab:costos_formulacion_nuevo}
\end{table}


\begin{table}[H]
	\centering
	\renewcommand{\arraystretch}{1.6}
	\setlength{\tabcolsep}{10pt}
	\Huge
	\begin{adjustbox}{max width=\textwidth}
		\begin{tabular}{|p{9.5cm}|c|r|r|}
			\hline
			\textbf{Tareas (Análisis del proyecto)} & \textbf{Horas} & \textbf{Costo por hora (MXN \$)} & \textbf{Costo total (MXN \$)} \\ \hline
			Definición de actores & 3 & \$150.00 & \$450.00 \\ \hline
			Análisis funcional y no funcional & 8 & \$150.00 & \$1,200.00 \\ \hline
			Creación de documentación de requerimientos & 5 & \$150.00 & \$750.00 \\ \hline
			Diagrama de casos de uso & 3 & \$150.00 & \$450.00 \\ \hline
			Diseño de pantallas (mockups) & 10 & \$150.00 & \$1,500.00 \\ \hline
			Análisis de viabilidad y factibilidad & 3 & \$150.00 & \$450.00 \\ \hline
			Análisis financiero & 5 & \$150.00 & \$750.00 \\ \hline
			Análisis de riesgos del proyecto & 5 & \$150.00 & \$750.00 \\ \hline
			Documentar los requisitos de alto nivel y entregables del proyecto & 5 & \$150.00 & \$750.00 \\ \hline
			Priorización de módulos según importancia y complejidad & 1 & \$150.00 & \$150.00 \\ \hline
			\textbf{Total} & \textbf{58} & -- & \textbf{\$7,450.00} \\ \hline
		\end{tabular}
	\end{adjustbox}
	\caption[Costos estimados para la fase de análisis del proyecto]{Costos estimados para la fase de análisis del proyecto, elaboración propia.} 	
	\label{tab:costos_analisis_nuevo}
\end{table}

\begin{table}[H]
	\centering
	\renewcommand{\arraystretch}{1.6}
	\setlength{\tabcolsep}{10pt}
	\Huge
	\begin{adjustbox}{max width=\textwidth}
		\begin{tabular}{|p{9.5cm}|c|r|r|}
			\hline
			\textbf{Tareas (Análisis de riesgos)} & \textbf{Horas} & \textbf{Costo por hora (MXN \$)} & \textbf{Costo total (MXN \$)} \\ \hline
			Realizar análisis cualitativo y cuantitativo de riesgos & 4 & \$150.00 & \$600.00 \\ \hline
			Planificar respuestas a los riesgos & 2 & \$150.00 & \$300.00 \\ \hline
			\textbf{Total} & \textbf{6} & -- & \textbf{\$900.00} \\ \hline
		\end{tabular}
	\end{adjustbox}
	\caption[Costos estimados para la fase de análisis de riesgos]{Costos estimados para la fase de análisis de riesgos, elaboración propia.} 	
	\label{tab:costos_riesgos_nuevo}
\end{table}

\begin{table}[H]
	\centering
	\renewcommand{\arraystretch}{1.6}
	\setlength{\tabcolsep}{10pt}
	\Huge
	\begin{adjustbox}{max width=\textwidth}
		\begin{tabular}{|p{9.5cm}|c|r|r|}
			\hline
			\textbf{Tareas (Elaboración de presupuesto)} & \textbf{Horas} & \textbf{Costo por hora (MXN \$)} & \textbf{Costo total (MXN \$)} \\ \hline
			Cotización simbólica de recursos & 5 & \$150.00 & \$750.00 \\ \hline
			Estimación de costos por actividades & 10 & \$150.00 & \$1,500.00 \\ \hline
			Estimación de costos por recursos & 10 & \$150.00 & \$1,500.00 \\ \hline
			\textbf{Total} & \textbf{25} & -- & \textbf{\$3,750.00} \\ \hline
		\end{tabular}
	\end{adjustbox}
	\caption[Costos estimados para la fase de elaboración de presupuesto]{Costos estimados para la fase de elaboración de presupuesto, elaboración propia.} 	
	\label{tab:costos_presupuesto_nuevo}
\end{table}

\begin{table}[H]
	\centering
	\renewcommand{\arraystretch}{1.6}
	\setlength{\tabcolsep}{10pt}
	\Huge
	\begin{adjustbox}{max width=\textwidth}
		\begin{tabular}{|p{9.5cm}|c|r|r|}
			\hline
			\textbf{Tareas (Desarrollo del producto)} & \textbf{Horas} & \textbf{Costo por hora (MXN \$)} & \textbf{Costo total (MXN \$)} \\ \hline
			Definición de la arquitectura básica & 15 & \$150.00 & \$2,250.00 \\ \hline
			Diseño de interfaces de usuario (UI/UX) para cada módulo & 20 & \$150.00 & \$3,000.00 \\ \hline
			Obtención del conjunto de datos & 12 & \$150.00 & \$1,800.00 \\ \hline
			Diagramas de flujo y secuencia & 12 & \$150.00 & \$1,800.00 \\ \hline
			Diagramas correspondientes UML & 20 & \$150.00 & \$3,000.00 \\ \hline
			Integración de APIs externas & 20 & \$150.00 & \$3,000.00 \\ \hline
			Integración backend y frontend & 25 & \$150.00 & \$3,750.00 \\ \hline
			\textbf{Total} & \textbf{124} & -- & \textbf{\$18,600.00} \\ \hline
		\end{tabular}
	\end{adjustbox}
	\caption[Costos estimados para la fase de desarrollo del producto]{Costos estimados para la fase de desarrollo del producto, elaboración propia.} 	
	\label{tab:costos_desarrollo_nuevo}
\end{table}


\section{Etapa: despliegue del prototipo}
\begin{table}[H]
	\centering
	\renewcommand{\arraystretch}{1.6}
	\setlength{\tabcolsep}{10pt}
	\Huge
	\begin{adjustbox}{max width=\textwidth}
		\begin{tabular}{|p{9.5cm}|c|r|r|}
			\hline
			\textbf{Tareas (Gestión de calidad)} & \textbf{Horas} & \textbf{Costo por hora (MXN \$)} & \textbf{Costo total (MXN \$)} \\ \hline
			Definir los estándares de calidad aplicables al proyecto & 8 & \$150.00 & \$1,200.00 \\ \hline
			Identificar métricas de calidad & 6 & \$150.00 & \$900.00 \\ \hline
			Realizar procedimientos de control de calidad & 10 & \$150.00 & \$1,500.00 \\ \hline
			\textbf{Total} & \textbf{24} & -- & \textbf{\$3,600.00} \\ \hline
		\end{tabular}
	\end{adjustbox}
	\caption[Costos estimados para la fase de gestión de calidad]{Costos estimados para la fase de gestión de calidad, elaboración propia.} 	
	\label{tab:costos_calidad_nuevo}
\end{table}

\begin{table}[H]
	\centering
	\renewcommand{\arraystretch}{1.6}
	\setlength{\tabcolsep}{10pt}
	\Huge
	\begin{adjustbox}{max width=\textwidth}
		\begin{tabular}{|p{9.5cm}|c|r|r|}
			\hline
			\textbf{Tareas (Gestión de clientes)} & \textbf{Horas} & \textbf{Costo por hora (MXN \$)} & \textbf{Costo total (MXN \$)} \\ \hline
			Identificar y analizar las partes interesadas de la comunidad & 5 & \$150.00 & \$750.00 \\ \hline
			Desarrollar y mantener la comunicación con la comunidad & 8 & \$150.00 & \$1,200.00 \\ \hline
			Identificar a todos los interesados & 5 & \$150.00 & \$750.00 \\ \hline
			Resolver conflictos con clientes & 10 & \$150.00 & \$1,500.00 \\ \hline
			\textbf{Total} & \textbf{28} & -- & \textbf{\$4,200.00} \\ \hline
		\end{tabular}
	\end{adjustbox}
	\caption[Costos estimados para la fase de gestión de clientes]{Costos estimados para la fase de gestión de clientes, elaboración propia.} 	
	\label{tab:costos_clientes_nuevo}
\end{table}

\begin{table}[H]
	\centering
	\renewcommand{\arraystretch}{1.6}
	\setlength{\tabcolsep}{10pt}
	\Huge
	\begin{adjustbox}{max width=\textwidth}
		\begin{tabular}{|p{9.5cm}|c|r|r|}
			\hline
			\textbf{Tareas (Gestión de adquisiciones)} & \textbf{Horas} & \textbf{Costo por hora (MXN \$)} & \textbf{Costo total (MXN \$)} \\ \hline
			Planificar futuras compras y adquisiciones & 8 & \$150.00 & \$1,200.00 \\ \hline
			\textbf{Total} & \textbf{8} & -- & \textbf{\$1,200.00} \\ \hline
		\end{tabular}
	\end{adjustbox}
	\caption[Costos estimados para la fase de gestión de adquisiciones]{Costos estimados para la fase de gestión de adquisiciones, elaboración propia.} 	
	\label{tab:costos_adquisiciones_nuevo}
\end{table}

\begin{table}[H]
	\centering
	\renewcommand{\arraystretch}{1.6}
	\setlength{\tabcolsep}{10pt}
	\Huge
	\begin{adjustbox}{max width=\textwidth}
		\begin{tabular}{|p{9.5cm}|c|r|r|}
			\hline
			\textbf{Tareas (Gestión de integración)} & \textbf{Horas} & \textbf{Costo por hora (MXN \$)} & \textbf{Costo total (MXN \$)} \\ \hline
			Desarrollar el plan de gestión del proyecto & 15 & \$150.00 & \$2,250.00 \\ \hline
			Dirigir y gestionar el trabajo del proyecto & 20 & \$150.00 & \$3,000.00 \\ \hline
			Monitorear y controlar el trabajo del proyecto & 15 & \$150.00 & \$2,250.00 \\ \hline
			\textbf{Total} & \textbf{50} & -- & \textbf{\$7,500.00} \\ \hline
		\end{tabular}
	\end{adjustbox}
	\caption[Costos estimados para la fase de gestión de integración]{Costos estimados para la fase de gestión de integración, elaboración propia.} 	
	\label{tab:costos_integracion_nuevo}
\end{table}

\begin{table}[H]
	\centering
	\renewcommand{\arraystretch}{1.6}
	\setlength{\tabcolsep}{10pt}
	\Huge
	\begin{adjustbox}{max width=\textwidth}
		\begin{tabular}{|p{9.5cm}|c|r|r|}
			\hline
			\textbf{Tareas (Pruebas)} & \textbf{Horas} & \textbf{Costo por hora (MXN \$)} & \textbf{Costo total (MXN \$)} \\ \hline
			Costo de las pruebas iniciales solo con desarrolladores & 10 & \$150.00 & \$1,500.00 \\ \hline
			Pruebas unitarias para cada módulo & 20 & \$150.00 & \$3,000.00 \\ \hline
			Pruebas de integración & 10 & \$150.00 & \$1,500.00 \\ \hline
			Pruebas con usuarios para validar la usabilidad & 10 & \$150.00 & \$1,500.00 \\ \hline
			\textbf{Total} & \textbf{50} & -- & \textbf{\$7,500.00} \\ \hline
		\end{tabular}
	\end{adjustbox}
	\caption[Costos estimados para la fase de pruebas]{Costos estimados para la fase de pruebas, elaboración propia.} 	
	\label{tab:costos_pruebas_nuevo}
\end{table}

\begin{table}[H]
	\centering
	\renewcommand{\arraystretch}{1.6}
	\setlength{\tabcolsep}{10pt}
	\Huge
	\begin{adjustbox}{max width=\textwidth}
		\begin{tabular}{|p{9.5cm}|c|r|r|}
			\hline
			\textbf{Tareas (Lanzamiento)} & \textbf{Horas} & \textbf{Costo por hora (MXN \$)} & \textbf{Costo total (MXN \$)} \\ \hline
			Preparación del entorno de producción & 10 & \$150.00 & \$1,500.00 \\ \hline
			\textbf{Total} & \textbf{10} & -- & \textbf{\$1,500.00} \\ \hline
		\end{tabular}
	\end{adjustbox}
	\caption[Costos estimados para la fase de lanzamiento]{Costos estimados para la fase de lanzamiento, elaboración propia.} 	
	\label{tab:costos_lanzamiento_nuevo}
\end{table}


\section{Etapa: costo de venta del prototipo}

\begin{table}[H]
	\centering
	\renewcommand{\arraystretch}{1.6}
	\setlength{\tabcolsep}{10pt}
	\Huge
	\begin{adjustbox}{max width=\textwidth}
		\begin{tabular}{|p{9.5cm}|c|r|r|}
			\hline
			\textbf{Tareas (Manual de usuario)} & \textbf{Horas} & \textbf{Costo por hora (MXN \$)} & \textbf{Costo total (MXN \$)} \\ \hline
			Creación de guías paso a paso para cada módulo & 12 & \$150.00 & \$1,800.00 \\ \hline
			Instrucciones claras y visuales para usuarios no técnicos & 10 & \$150.00 & \$1,500.00 \\ \hline
			\textbf{Total} & \textbf{22} & -- & \textbf{\$3,300.00} \\ \hline
		\end{tabular}
	\end{adjustbox}
	\caption[Costos estimados para la fase de elaboración del manual de usuario]{Costos estimados para la fase de elaboración del manual de usuario, elaboración propia.} 	
	\label{tab:costos_manual_nuevo}
\end{table}

\begin{table}[H]
	\centering
	\renewcommand{\arraystretch}{1.6}
	\setlength{\tabcolsep}{10pt}
	\Huge
	\begin{adjustbox}{max width=\textwidth}
		\begin{tabular}{|p{9.5cm}|c|r|r|}
			\hline
			\textbf{Tareas (Manual técnico)} & \textbf{Horas} & \textbf{Costo por hora (MXN \$)} & \textbf{Costo total (MXN \$)} \\ \hline
			Documentación de la arquitectura del sistema & 8 & \$150.00 & \$1,200.00 \\ \hline
			Descripción del conjunto de datos y APIs & 10 & \$150.00 & \$1,500.00 \\ \hline
			\textbf{Total} & \textbf{18} & -- & \textbf{\$2,700.00} \\ \hline
		\end{tabular}
	\end{adjustbox}
	\caption[Costos estimados para la fase de elaboración del manual técnico]{Costos estimados para la fase de elaboración del manual técnico, elaboración propia.} 	
	\label{tab:costos_manual_tecnico_nuevo}
\end{table}

\begin{table}[H]
	\centering
	\renewcommand{\arraystretch}{1.6}
	\setlength{\tabcolsep}{10pt}
	\Huge
	\begin{adjustbox}{max width=\textwidth}
		\begin{tabular}{|p{9.5cm}|c|r|r|}
			\hline
			\textbf{Tareas (Documentación)} & \textbf{Horas} & \textbf{Costo por hora (MXN \$)} & \textbf{Costo total (MXN \$)} \\ \hline
			Documentación de requerimientos & 15 & \$150.00 & \$2,250.00 \\ \hline
			Documentación de pruebas & 6 & \$150.00 & \$900.00 \\ \hline
			Manuales de usuario y técnico & 8 & \$150.00 & \$1,200.00 \\ \hline
			\textbf{Total} & \textbf{29} & -- & \textbf{\$4,350.00} \\ \hline
		\end{tabular}
	\end{adjustbox}
	\caption[Costos estimados para la fase de documentación]{Costos estimados para la fase de documentación, elaboración propia.} 	
	\label{tab:costos_documentacion_nuevo}
\end{table}

\begin{table}[H]
	\centering
	\renewcommand{\arraystretch}{1.6}
	\setlength{\tabcolsep}{10pt}
	\Huge
	\begin{adjustbox}{max width=\textwidth}
		\begin{tabular}{|p{9.5cm}|c|r|r|}
			\hline
			\textbf{Tareas (Presupuesto de ingresos)} & \textbf{Horas} & \textbf{Costo por hora (MXN \$)} & \textbf{Costo total (MXN \$)} \\ \hline
			Estimación de precio de producto final & 6 & \$150.00 & \$900.00 \\ \hline
			\textbf{Total} & \textbf{6} & -- & \textbf{\$900.00} \\ \hline
		\end{tabular}
	\end{adjustbox}
	\caption[Costos estimados para la fase de presupuesto de ingresos]{Costos estimados para la fase de presupuesto de ingresos, elaboración propia.} 	
	\label{tab:costos_presupuesto_ingresos}
\end{table}

\begin{table}[H]
	\centering
	\renewcommand{\arraystretch}{1.6}
	\setlength{\tabcolsep}{10pt}
	\Huge
	\begin{adjustbox}{max width=\textwidth}
		\begin{tabular}{|p{9.5cm}|c|r|r|}
			\hline
			\textbf{Tareas (Estados financieros)} & \textbf{Horas} & \textbf{Costo por hora (MXN \$)} & \textbf{Costo total (MXN \$)} \\ \hline
			Revisión de costos y gastos iniciales & 1 & \$150.00 & \$150.00 \\ \hline
			Proyección de ingresos & 2 & \$150.00 & \$300.00 \\ \hline
			\textbf{Total} & \textbf{3} & -- & \textbf{\$450.00} \\ \hline
		\end{tabular}
	\end{adjustbox}
	\caption[Costos estimados para la fase de estados financieros]{Costos estimados para la fase de estados financieros, elaboración propia.} 	
	\label{tab:costos_estados_financieros}
\end{table}

\chapter{Enfoque por actividades (comercial)}
\label{anexo:actividades_comercial}  % Etiqueta para hacer referencia
\section{Etapa: creación del prototipo}
\begin{table}[H]
	\centering
	\renewcommand{\arraystretch}{1.6}
	\setlength{\tabcolsep}{10pt}
	\Huge
	\begin{adjustbox}{max width=\textwidth}
		\begin{tabular}{|p{9.5cm}|c|r|r|}
			\hline
			\textbf{Tareas (Formulación del proyecto)} & \textbf{Horas} & \textbf{Costo por hora (MXN \$)} & \textbf{Costo total (MXN \$)} \\ \hline
			Descripción del proyecto & 1 & \$280.00 & \$280.00 \\ \hline
			Definir el propósito del proyecto & 1 & \$280.00 & \$280.00 \\ \hline
			Planificación del alcance del proyecto & 1 & \$280.00 & \$280.00 \\ \hline
			Definir las actividades necesarias para completar el proyecto & 1 & \$280.00 & \$280.00 \\ \hline
			Definir tareas prioritarias y bloques de trabajo en paralelo & 2 & \$280.00 & \$560.00 \\ \hline
			Estimar recursos y operaciones & 3 & \$260.00 & \$780.00 \\ \hline
			Establecer los objetivos y metas principales & 1 & \$280.00 & \$280.00 \\ \hline
			Identificación de actividades y tareas & 5 & \$280.00 & \$1,400.00 \\ \hline
			Planificación del cronograma de actividades & 8 & \$280.00 & \$2,240.00 \\ \hline
			\textbf{Total} & \textbf{23} & -- & \textbf{\$6,380.00} \\ \hline
		\end{tabular}
	\end{adjustbox}
	\caption[Costos estimados para la fase de formulación del proyecto (ajustada con nueva tarifa)]{Costos estimados para la fase de formulación del proyecto (ajustada con nueva tarifa), elaboración propia.} 	
	\label{tab:costos_formulacion_tarifa280}
\end{table}

\begin{table}[H]
	\centering
	\renewcommand{\arraystretch}{1.6}
	\setlength{\tabcolsep}{10pt}
	\Huge
	\begin{adjustbox}{max width=\textwidth}
		\begin{tabular}{|p{9.5cm}|c|r|r|}
			\hline
			\textbf{Tareas (Análisis de proyecto)} & \textbf{Horas} & \textbf{Costo por hora (MXN \$)} & \textbf{Costo total (MXN \$)} \\ \hline
			Definición de actores & 3 & \$260.00 & \$780.00 \\ \hline
			Análisis funcional y no funcional & 8 & \$260.00 & \$2,080.00 \\ \hline
			Creación de documentación de requerimientos & 5 & \$240.00 & \$1,200.00 \\ \hline
			Diagrama de casos de uso & 3 & \$260.00 & \$780.00 \\ \hline
			Diseño de pantallas (mockups) & 10 & \$300.00 & \$3,000.00 \\ \hline
			Análisis de viabilidad y factibilidad & 3 & \$260.00 & \$780.00 \\ \hline
			Análisis financiero & 5 & \$270.00 & \$1,350.00 \\ \hline
			Análisis de riesgos del proyecto & 5 & \$270.00 & \$1,350.00 \\ \hline
			Documentar los requisitos de alto nivel y entregables del proyecto & 5 & \$240.00 & \$1,200.00 \\ \hline
			Priorización de módulos según importancia y complejidad & 1 & \$260.00 & \$260.00 \\ \hline
			\textbf{Total} & \textbf{48} & -- & \textbf{\$12,780.00} \\ \hline
		\end{tabular}
	\end{adjustbox}
	\caption[Costos estimados para la fase de análisis de proyecto (con tarifas ajustadas)]{Costos estimados para la fase de análisis de proyecto (con tarifas ajustadas), elaboración propia.} 	
	\label{tab:costos_analisis_actualizado}
\end{table}

\begin{table}[H]
	\centering
	\renewcommand{\arraystretch}{1.6}
	\setlength{\tabcolsep}{10pt}
	\Huge
	\begin{adjustbox}{max width=\textwidth}
		\begin{tabular}{|p{9.5cm}|c|r|r|}
			\hline
			\textbf{Tareas (Análisis de riesgos)} & \textbf{Horas} & \textbf{Costo por hora (MXN \$)} & \textbf{Costo total (MXN \$)} \\ \hline
			Realizar análisis cualitativo y cuantitativo de riesgos & 4 & \$260.00 & \$1,040.00 \\ \hline
			Planificar respuestas a los riesgos & 2 & \$260.00 & \$520.00 \\ \hline
			Monitoreo de riesgos general & 10 & \$280.00 & \$2,800.00 \\ \hline
			\textbf{Total} & \textbf{16} & -- & \textbf{\$4,360.00} \\ \hline
		\end{tabular}
	\end{adjustbox}
	\caption[Costos estimados para la fase de análisis de riesgos (con tarifas ajustadas)]{Costos estimados para la fase de análisis de riesgos (con tarifas ajustadas), elaboración propia.} 
	\label{tab:costos_riesgos_actualizado}
\end{table}

\begin{table}[H]
	\centering
	\renewcommand{\arraystretch}{1.6}
	\setlength{\tabcolsep}{10pt}
	\Huge
	\begin{adjustbox}{max width=\textwidth}
		\begin{tabular}{|p{9.5cm}|c|r|r|}
			\hline
			\textbf{Tareas (Elaboración de presupuesto)} & \textbf{Horas} & \textbf{Costo por hora (MXN \$)} & \textbf{Costo total (MXN \$)} \\ \hline
			Cotización simbólica de recursos & 5 & \$270.00 & \$1,350.00 \\ \hline
			Estimación de costos por actividades & 10 & \$270.00 & \$2,700.00 \\ \hline
			Estimación de costos por recursos & 10 & \$270.00 & \$2,700.00 \\ \hline
			\textbf{Total} & \textbf{25} & -- & \textbf{\$6,750.00} \\ \hline
		\end{tabular}
	\end{adjustbox}
	\caption[Costos estimados para la fase de elaboración de presupuesto (con tarifas ajustadas)]{Costos estimados para la fase de elaboración de presupuesto (con tarifas ajustadas), elaboración propia.} 
	\label{tab:costos_presupuesto_actualizado}
\end{table}

\begin{table}[H]
	\centering
	\renewcommand{\arraystretch}{1.6}
	\setlength{\tabcolsep}{10pt}
	\Huge
	\begin{adjustbox}{max width=\textwidth}
		\begin{tabular}{|p{9.5cm}|c|r|r|}
			\hline
			\textbf{Tareas (Desarrollo del producto)} & \textbf{Horas} & \textbf{Costo por hora (MXN \$)} & \textbf{Costo total (MXN \$)} \\ \hline
			Definición de la arquitectura básica & 15 & \$320.00 & \$4,800.00 \\ \hline
			Diseño de interfaces de usuario (UI/UX) para cada módulo & 20 & \$300.00 & \$6,000.00 \\ \hline
			Obtención del conjunto de datos & 12 & \$260.00 & \$3,120.00 \\ \hline
			Diagramas de flujo y secuencia para cada funcionalidad & 12 & \$240.00 & \$2,880.00 \\ \hline
			Desarrollo de la funcionalidad de inicio de sesión y validación de credenciales & 20 & \$320.00 & \$6,400.00 \\ \hline
			Implementación del sistema de recuperación de contraseña & 15 & \$320.00 & \$4,800.00 \\ \hline
			Creación de la funcionalidad de registro de nuevos usuarios & 15 & \$320.00 & \$4,800.00 \\ \hline
			Integración de APIs externas & 20 & \$320.00 & \$6,400.00 \\ \hline
			Integración backend y frontend & 25 & \$320.00 & \$8,000.00 \\ \hline
			\textbf{Total} & \textbf{154} & -- & \textbf{\$47,200.00} \\ \hline
		\end{tabular}
	\end{adjustbox}
	\caption[Costos estimados para la fase de desarrollo del producto (con tarifas ajustadas)]{Costos estimados para la fase de desarrollo del producto (con tarifas ajustadas), elaboración propia.} 
	\label{tab:costos_desarrollo_actualizado}
\end{table}


\section{Etapa: despliegue del prototipo}

\begin{table}[H]
	\centering
	\renewcommand{\arraystretch}{1.6}
	\setlength{\tabcolsep}{10pt}
	\Huge
	\begin{adjustbox}{max width=\textwidth}
		\begin{tabular}{|p{9.5cm}|c|r|r|}
			\hline
			\textbf{Tareas (Gestión de calidad)} & \textbf{Horas} & \textbf{Costo por hora (MXN \$)} & \textbf{Costo total (MXN \$)} \\ \hline
			Definir los estándares de calidad aplicables al proyecto & 8 & \$280.00 & \$2,240.00 \\ \hline
			Identificar métricas de calidad & 6 & \$280.00 & \$1,680.00 \\ \hline
			Realizar procedimientos de control de calidad & 10 & \$260.00 & \$2,600.00 \\ \hline
			\textbf{Total} & \textbf{24} & -- & \textbf{\$6,520.00} \\ \hline
		\end{tabular}
	\end{adjustbox}
	\caption[Costos estimados para la fase de gestión de calidad (con tarifas ajustadas)]{Costos estimados para la fase de gestión de calidad (con tarifas ajustadas), elaboración propia.} 
	\label{tab:costos_calidad_actualizado}
\end{table}

\begin{table}[H]
	\centering
	\renewcommand{\arraystretch}{1.6}
	\setlength{\tabcolsep}{10pt}
	\Huge
	\begin{adjustbox}{max width=\textwidth}
		\begin{tabular}{|p{9.5cm}|c|r|r|}
			\hline
			\textbf{Tareas (Gestión de clientes)} & \textbf{Horas} & \textbf{Costo por hora (MXN \$)} & \textbf{Costo total (MXN \$)} \\ \hline
			Identificar y analizar las partes interesadas de la comunidad & 5 & \$260.00 & \$1,300.00 \\ \hline
			Desarrollar y mantener la comunicación con la comunidad & 8 & \$280.00 & \$2,240.00 \\ \hline
			Identificar a todos los interesados & 5 & \$260.00 & \$1,300.00 \\ \hline
			Resolver conflictos con clientes & 10 & \$280.00 & \$2,800.00 \\ \hline
			\textbf{Total} & \textbf{28} & -- & \textbf{\$7,640.00} \\ \hline
		\end{tabular}
	\end{adjustbox}
	\caption[Costos estimados para la fase de gestión de clientes (con tarifas ajustadas)]{Costos estimados para la fase de gestión de clientes (con tarifas ajustadas), elaboración propia.} 
	\label{tab:costos_clientes_actualizado}
\end{table}

\begin{table}[H]
	\centering
	\renewcommand{\arraystretch}{1.6}
	\setlength{\tabcolsep}{10pt}
	\Huge
	\begin{adjustbox}{max width=\textwidth}
		\begin{tabular}{|p{9.5cm}|c|r|r|}
			\hline
			\textbf{Tareas (Gestión de adquisiciones)} & \textbf{Horas} & \textbf{Costo por hora (MXN \$)} & \textbf{Costo total (MXN \$)} \\ \hline
			Planificar futuras compras y adquisiciones & 8 & \$280.00 & \$2,240.00 \\ \hline
			Seleccionar proveedores & 6 & \$280.00 & \$1,680.00 \\ \hline
			Administrar contratos con proveedores & 8 & \$280.00 & \$2,240.00 \\ \hline
			\textbf{Total} & \textbf{22} & -- & \textbf{\$6,160.00} \\ \hline
		\end{tabular}
	\end{adjustbox}
	\caption[Costos estimados para la fase de gestión de adquisiciones (con tarifas ajustadas)]{Costos estimados para la fase de gestión de adquisiciones (con tarifas ajustadas), elaboración propia.} 
	\label{tab:costos_adquisiciones_actualizado}
\end{table}

\begin{table}[H]
	\centering
	\renewcommand{\arraystretch}{1.6}
	\setlength{\tabcolsep}{10pt}
	\Huge
	\begin{adjustbox}{max width=\textwidth}
		\begin{tabular}{|p{9.5cm}|c|r|r|}
			\hline
			\textbf{Tareas (Gestión de regulaciones)} & \textbf{Horas} & \textbf{Costo por hora (MXN \$)} & \textbf{Costo total (MXN \$)} \\ \hline
			Evaluar el impacto ambiental del proyecto & 6 & \$260.00 & \$1,560.00 \\ \hline
			Asegurar el cumplimiento con regulaciones y políticas de privacidad & 10 & \$260.00 & \$2,600.00 \\ \hline
			\textbf{Total} & \textbf{16} & -- & \textbf{\$4,160.00} \\ \hline
		\end{tabular}
	\end{adjustbox}
	\caption[Costos estimados para la fase de gestión de regulaciones (con tarifas ajustadas)]{Costos estimados para la fase de gestión de regulaciones (con tarifas ajustadas), elaboración propia.} 
	\label{tab:costos_regulaciones_actualizado}
\end{table}

\begin{table}[H]
	\centering
	\renewcommand{\arraystretch}{1.6}
	\setlength{\tabcolsep}{10pt}
	\Huge
	\begin{adjustbox}{max width=\textwidth}
		\begin{tabular}{|p{9.5cm}|c|r|r|}
			\hline
			\textbf{Tareas (Gestión de integración)} & \textbf{Horas} & \textbf{Costo por hora (MXN \$)} & \textbf{Costo total (MXN \$)} \\ \hline
			Desarrollar el plan de gestión del proyecto & 15 & \$280.00 & \$4,200.00 \\ \hline
			Dirigir y gestionar el trabajo del proyecto & 20 & \$280.00 & \$5,600.00 \\ \hline
			Monitorear y controlar el trabajo del proyecto & 15 & \$280.00 & \$4,200.00 \\ \hline
			\textbf{Total} & \textbf{50} & -- & \textbf{\$14,000.00} \\ \hline
		\end{tabular}
	\end{adjustbox}
	\caption[Costos estimados para la fase de gestión de integración (con tarifas ajustadas)]{Costos estimados para la fase de gestión de integración (con tarifas ajustadas), elaboración propia.} 
	\label{tab:costos_integracion_actualizado}
\end{table}

\begin{table}[H]
	\centering
	\renewcommand{\arraystretch}{1.6}
	\setlength{\tabcolsep}{10pt}
	\Huge
	\begin{adjustbox}{max width=\textwidth}
		\begin{tabular}{|p{9.5cm}|c|r|r|}
			\hline
			\textbf{Tareas (Pruebas)} & \textbf{Horas} & \textbf{Costo por hora (MXN \$)} & \textbf{Costo total (MXN \$)} \\ \hline
			Costo de las pruebas iniciales solo con desarrolladores & 10 & \$280.00 & \$2,800.00 \\ \hline
			Pruebas unitarias para cada módulo & 20 & \$320.00 & \$6,400.00 \\ \hline
			Pruebas de integración & 10 & \$320.00 & \$3,200.00 \\ \hline
			Pruebas con usuarios para validar la usabilidad & 10 & \$260.00 & \$2,600.00 \\ \hline
			\textbf{Total} & \textbf{50} & -- & \textbf{\$15,000.00} \\ \hline
		\end{tabular}
	\end{adjustbox}
	\caption[Costos estimados para la fase de pruebas (con tarifas ajustadas)]{Costos estimados para la fase de pruebas (con tarifas ajustadas), elaboración propia.} 
	\label{tab:costos_pruebas_actualizado}
\end{table}

\begin{table}[H]
	\centering
	\renewcommand{\arraystretch}{1.6}
	\setlength{\tabcolsep}{10pt}
	\Huge
	\begin{adjustbox}{max width=\textwidth}
		\begin{tabular}{|p{9.5cm}|c|r|r|}
			\hline
			\textbf{Tareas (Lanzamiento)} & \textbf{Horas} & \textbf{Costo por hora (MXN \$)} & \textbf{Costo total (MXN \$)} \\ \hline
			Preparación del entorno de producción & 10 & \$320.00 & \$3,200.00 \\ \hline
			Configuración de servidores y bases de datos & 15 & \$320.00 & \$4,800.00 \\ \hline
			Despliegue del sistema en servidores de producción & 15 & \$320.00 & \$4,800.00 \\ \hline
			Configuración de backups y monitoreo & 10 & \$320.00 & \$3,200.00 \\ \hline
			\textbf{Total} & \textbf{50} & -- & \textbf{\$16,000.00} \\ \hline
		\end{tabular}
	\end{adjustbox}
	\caption[Costos estimados para la fase de lanzamiento (con tarifas ajustadas)]{Costos estimados para la fase de lanzamiento (con tarifas ajustadas), elaboración propia.} 
	\label{tab:costos_lanzamiento_actualizado}
\end{table}

\section{Etapa: costo de venta del prototipo}

\begin{table}[H]
	\centering
	\renewcommand{\arraystretch}{1.6}
	\setlength{\tabcolsep}{10pt}
	\Huge
	\begin{adjustbox}{max width=\textwidth}
		\begin{tabular}{|p{9.5cm}|c|r|r|}
			\hline
			\textbf{Tareas (Manual de usuario)} & \textbf{Horas} & \textbf{Costo por hora (MXN \$)} & \textbf{Costo total (MXN \$)} \\ \hline
			Creación de guías paso a paso para cada módulo & 12 & \$240.00 & \$2,880.00 \\ \hline
			Instrucciones claras y visuales para usuarios no técnicos & 10 & \$300.00 & \$3,000.00 \\ \hline
			\textbf{Total} & \textbf{22} & -- & \textbf{\$5,880.00} \\ \hline
		\end{tabular}
	\end{adjustbox}
	\caption[Costos estimados para la fase de manual de usuario (con tarifas ajustadas)]{Costos estimados para la fase de manual de usuario (con tarifas ajustadas), elaboración propia.} 
	\label{tab:costos_manual_usuario_actualizado}
\end{table}

\begin{table}[H]
	\centering
	\renewcommand{\arraystretch}{1.6}
	\setlength{\tabcolsep}{10pt}
	\Huge
	\begin{adjustbox}{max width=\textwidth}
		\begin{tabular}{|p{9.5cm}|c|r|r|}
			\hline
			\textbf{Tareas (Manual técnico)} & \textbf{Horas} & \textbf{Costo por hora (MXN \$)} & \textbf{Costo total (MXN \$)} \\ \hline
			Documentación de la arquitectura del sistema & 8 & \$240.00 & \$1,920.00 \\ \hline
			Instrucciones sobre la configuración del servidor y despliegue & 12 & \$320.00 & \$3,840.00 \\ \hline
			Descripción del conjunto de datos y APIs & 10 & \$320.00 & \$3,200.00 \\ \hline
			\textbf{Total} & \textbf{30} & -- & \textbf{\$8,960.00} \\ \hline
		\end{tabular}
	\end{adjustbox}
	\caption[Costos estimados para la fase de manual técnico (con tarifas ajustadas)]{Costos estimados para la fase de manual técnico (con tarifas ajustadas), elaboración propia.} 
	\label{tab:costos_manual_tecnico_actualizado}
\end{table}

\begin{table}[H]
	\centering
	\renewcommand{\arraystretch}{1.6}
	\setlength{\tabcolsep}{10pt}
	\Huge
	\begin{adjustbox}{max width=\textwidth}
		\begin{tabular}{|p{9.5cm}|c|r|r|}
			\hline
			\textbf{Tareas (Documentación)} & \textbf{Horas} & \textbf{Costo por hora (MXN \$)} & \textbf{Costo total (MXN \$)} \\ \hline
			Documentación de requerimientos & 15 & \$240.00 & \$3,600.00 \\ \hline
			Documentación de pruebas & 6 & \$240.00 & \$1,440.00 \\ \hline
			Manuales de usuario y técnico & 8 & \$240.00 & \$1,920.00 \\ \hline
			\textbf{Total} & \textbf{29} & -- & \textbf{\$6,960.00} \\ \hline
		\end{tabular}
	\end{adjustbox}
	\caption[Costos estimados para la fase de documentación (con tarifas ajustadas)]{Costos estimados para la fase de documentación (con tarifas ajustadas), elaboración propia.} 
	\label{tab:costos_documentacion_actualizado}
\end{table}


\begin{table}[H]
	\centering
	\renewcommand{\arraystretch}{1.6}
	\setlength{\tabcolsep}{10pt}
	\Huge
	\begin{adjustbox}{max width=\textwidth}
		\begin{tabular}{|p{9.5cm}|c|r|r|}
			\hline
			\textbf{Tareas (Presupuesto de ingresos)} & \textbf{Horas} & \textbf{Costo por hora (MXN \$)} & \textbf{Costo total (MXN \$)} \\ \hline
			Estimación de precio de producto final & 6 & \$270.00 & \$1,620.00 \\ \hline
			\textbf{Total} & \textbf{6} & -- & \textbf{\$1,620.00} \\ \hline
		\end{tabular}
	\end{adjustbox}
	\caption[Costos estimados para la fase de presupuesto de ingresos (con tarifa ajustada)]{Costos estimados para la fase de presupuesto de ingresos (con tarifa ajustada), elaboración propia.} 
	\label{tab:costos_presupuesto_ingresos_s}
\end{table}

\begin{table}[H]
	\centering
	\renewcommand{\arraystretch}{1.6}
	\setlength{\tabcolsep}{10pt}
	\Huge
	\begin{adjustbox}{max width=\textwidth}
		\begin{tabular}{|p{9.5cm}|c|r|r|}
			\hline
			\textbf{Tareas (Estados financieros proforma)} & \textbf{Horas} & \textbf{Costo por hora (MXN \$)} & \textbf{Costo total (MXN \$)} \\ \hline
			Revisión de costos y gastos iniciales & 1 & \$270.00 & \$270.00 \\ \hline
			Proyección de ingresos & 2 & \$270.00 & \$540.00 \\ \hline
			Elaboración de balance proforma & 8 & \$270.00 & \$2,160.00 \\ \hline
			Preparación de estado de resultados & 3 & \$270.00 & \$810.00 \\ \hline
			\textbf{Total} & \textbf{14} & -- & \textbf{\$3,780.00} \\ \hline
		\end{tabular}
	\end{adjustbox}
	\caption[Costos estimados para la fase de estados financieros proforma (con tarifas ajustadas)]{Costos estimados para la fase de estados financieros proforma (con tarifas ajustadas), elaboración propia.} 
	\label{tab:costos_financieros_proforma}
\end{table}

\begin{table}[H]
	\centering
	\renewcommand{\arraystretch}{1.6}
	\setlength{\tabcolsep}{10pt}
	\Huge
	\begin{adjustbox}{max width=\textwidth}
		\begin{tabular}{|p{9.5cm}|c|r|r|}
			\hline
			\textbf{Tareas (Flujos netos de efectivo)} & \textbf{Horas} & \textbf{Costo por hora (MXN \$)} & \textbf{Costo total (MXN \$)} \\ \hline
			Identificación de entradas y salidas de efectivo & 4 & \$270.00 & \$1,080.00 \\ \hline
			Proyección de flujo de efectivo mensual y anual & 2 & \$270.00 & \$540.00 \\ \hline
			Análisis de punto de equilibrio & 6 & \$270.00 & \$1,620.00 \\ \hline
			\textbf{Total} & \textbf{12} & -- & \textbf{\$3,240.00} \\ \hline
		\end{tabular}
	\end{adjustbox}
	\caption[Costos estimados para la fase de flujos netos de efectivo (con tarifas ajustadas)]{Costos estimados para la fase de flujos netos de efectivo (con tarifas ajustadas), elaboración propia.} 
	\label{tab:costos_flujos_efectivo}
\end{table}

\begin{table}[H]
	\centering
	\renewcommand{\arraystretch}{1.6}
	\setlength{\tabcolsep}{10pt}
	\Huge
	\begin{adjustbox}{max width=\textwidth}
		\begin{tabular}{|p{9.5cm}|c|r|r|}
			\hline
			\textbf{Tareas (Evaluación financiera)} & \textbf{Horas} & \textbf{Costo por hora (MXN \$)} & \textbf{Costo total (MXN \$)} \\ \hline
			Análisis de retorno de inversión & 4 & \$270.00 & \$1,080.00 \\ \hline
			Sensibilidad de las proyecciones & 3 & \$270.00 & \$810.00 \\ \hline
			\textbf{Total} & \textbf{7} & -- & \textbf{\$1,890.00} \\ \hline
		\end{tabular}
	\end{adjustbox}
	\caption[Costos estimados para la fase de evaluación financiera (con tarifas ajustadas)]{Costos estimados para la fase de evaluación financiera (con tarifas ajustadas), elaboración propia.} 
	\label{tab:costos_evaluacion_financiera}
\end{table}

\begin{table}[H]
	\centering
	\renewcommand{\arraystretch}{1.6}
	\setlength{\tabcolsep}{10pt}
	\Huge
	\begin{adjustbox}{max width=\textwidth}
		\begin{tabular}{|p{9.5cm}|c|r|r|}
			\hline
			\textbf{Tareas (Mantenimiento)} & \textbf{Horas} & \textbf{Costo por hora (MXN \$)} & \textbf{Costo total (MXN \$)} \\ \hline
			Monitoreo continuo del sistema & 15 & \$320.00 & \$4,800.00 \\ \hline
			Corrección de errores post-despliegue & 10 & \$320.00 & \$3,200.00 \\ \hline
			Costo de mantenimiento de servidores y seguridad & 12 & \$320.00 & \$3,840.00 \\ \hline
			Cotización de salarios del equipo de mantenimiento de la app & 1 & \$270.00 & \$270.00 \\ \hline
			Costos de actualizaciones de la aplicación & 10 & \$320.00 & \$3,200.00 \\ \hline
			\textbf{Total} & \textbf{48} & -- & \textbf{\$15,310.00} \\ \hline
		\end{tabular}
	\end{adjustbox}
	\caption[Costos estimados para la fase de mantenimiento (con tarifas ajustadas)]{Costos estimados para la fase de mantenimiento (con tarifas ajustadas), elaboración propia.} 
	\label{tab:costos_mantenimiento}
\end{table}

