\chapter{Conclusiones}
El desarrollo del presente Trabajo Terminal permitió materializar un prototipo funcional de aplicación móvil capaz de traducir texto en español a Lengua de Señas Mexicana (LSM) mediante técnicas de Procesamiento de Lenguaje Natural (PLN). El proyecto, que inició como una propuesta sencilla para vincular texto con señas en video, evolucionó hacia un sistema modular, medible y defendible desde una perspectiva de ingeniería profesional. Aunque el modelado y la animación 3D previstos inicialmente no se implementaron por limitaciones de tiempo, recursos y la necesidad de especialistas en captura de movimiento, la sustitución por videos estilizados con filtros tipo “anime” permitió mantener fluidez visual y congruencia comunicativa sin comprometer la calidad final del prototipo.\\

Durante el desarrollo fue necesario ajustar el conjunto original de frases. Las expresiones de agradecimiento se reemplazaron por expresiones de mínima comunicación, debido a la reducida variabilidad gestual en LSM para dichas frases. Este cambio incrementó la diversidad expresiva y mejoró la utilidad práctica del sistema. El conjunto final se consolidó en 43 frases curadas organizadas en grupos temáticos, lo que permitió aplicar técnicas de búsqueda semántica optimizadas y reducir el número de comparaciones necesarias durante la clasificación.\\

En términos cuantitativos, el sistema superó los objetivos establecidos. La latencia promedio se mantuvo alrededor de 40 ms, por debajo del límite de 100 ms planteado como aceptable; el throughput de más de 25 consultas por segundo fue suficiente para soportar una carga estimada de 100 usuarios concurrentes; y la precisión aproximada del 90\% en la clasificación semántica superó ampliamente el mínimo esperado del 85\%. Además, el sistema logró manejar entradas no previstas, como nombres propios y variaciones de escritura tipo leet speak, gracias a un pipeline de preprocesamiento basado en normalización, detección y posterior deletreo. Estas capacidades reforzaron el comportamiento robusto del sistema ante textos reales y ruidosos.\\

Desde el punto de vista arquitectónico, la aplicación móvil se desarrolló con React Expo, garantizando compatibilidad multiplataforma para Android e iOS. En el backend, la adopción de un monolito modular —en lugar de una arquitectura de microservicios— resultó adecuada para el tamaño y alcance del proyecto. La separación por capas (Presentación, Servicios y Datos), junto con módulos internos bien definidos (API, motor de PLN, normalizador y gestor de grupos), permitió mantener un código ordenado, testeable y con bajo acoplamiento. Esta organización también deja abierta la posibilidad de migrar a microservicios si en un futuro aumenta el tráfico o el equipo de desarrollo crece.\\

El proyecto presentó retos significativos asociados a la interpretación semántica, la escalabilidad y el manejo de entradas atípicas. La transición de una búsqueda lineal O(N) a una búsqueda jerárquica O(K+M) basada en grupos y centroides redujo costos computacionales y mejoró la consistencia de las clasificaciones. Asimismo, el ajuste de thresholds por grupo temático permitió disminuir falsos positivos. Estos resultados mostraron que los hiperparámetros no deben considerarse valores estáticos, sino decisiones que deben calibrarse con datos reales.\\

En el ámbito metodológico, el proyecto transformó la visión sobre el aseguramiento de calidad. Inicialmente se priorizó la cobertura de código, pero posteriormente se incorporó una batería de 168 casos de prueba que incluyó pruebas unitarias, end-to-end, semánticas y de rendimiento. Este enfoque permitió identificar problemas que no se manifestaban en pruebas aisladas y proporcionó una visión integral del comportamiento del pipeline completo.\\

A pesar de los resultados positivos, el proyecto reconoce varias limitaciones. El dataset utilizado, aunque adecuado para validar el funcionamiento del sistema, sigue siendo reducido para una aplicación de uso cotidiano, que requeriría entre 500 y 1000 frases. Tampoco se realizó una validación exhaustiva con la comunidad sorda, por lo que es necesario evaluar la velocidad de los videos, las variaciones dialectales de la LSM y la usabilidad del prototipo en contextos reales. Asimismo, la arquitectura monolítica y el uso de un modelo de lenguaje estático limitan el potencial de adaptación dinámica del sistema.\\

Estas limitaciones establecen rutas claras para el trabajo futuro: ampliar el dataset en colaboración con intérpretes de LSM; evaluar modelos más potentes o especializados; integrar índices vectoriales como FAISS; explorar arquitecturas más escalables; y añadir nuevas funciones, como modos conversacionales, personalización del usuario y herramientas de aprendizaje gamificadas. Estas mejoras son coherentes con la estructura actual y permitirían avanzar hacia un sistema más completo y aplicable en escenarios reales.\\

Finalmente, el proyecto evidenció que construir un sistema funcional implica considerar tanto los aspectos técnicos como las necesidades de los usuarios finales. Más allá de diseñar algoritmos eficientes, fue indispensable tomar decisiones fundamentadas, documentarlas con rigor y validar el sistema con escenarios que representaran situaciones reales. El resultado es un prototipo funcional orientado a la accesibilidad y la inclusión de personas sordas usuarias de LSM. Si esta base técnica contribuye en el futuro al desarrollo de herramientas más amplias o inspira nuevas soluciones para mejorar la comunicación y la inclusión, el propósito de este trabajo terminal habrá sido plenamente alcanzado.\\