\chapter{Conclusiones}
En la primera etapa desarrollada del Trabajo Terminal, se establecen las bases teóricas y técnicas necesarias para el desarrollo de una solución tecnológica inclusiva orientada a mejorar la comunicación entre personas oyentes y personas con discapacidad auditiva. A partir del análisis de la motivación, la problemática y los objetivos del proyecto, se delimita un enfoque centrado en la accesibilidad lingüística a través de la Lengua de Señas Mexicana (LSM).\\

La revisión del estado del arte y la construcción del marco teórico permiten contextualizar el proyecto dentro del ámbito del procesamiento de lenguaje natural y la representación visual mediante modelado 3D, evidenciando que, si bien existen herramientas similares a nivel internacional, estas no abordan de manera específica las características lingüísticas, culturales y gramaticales de la LSM. Asimismo, el análisis del proceso de comunicación y el estudio profundo de los elementos lingüísticos de esta lengua (como su estructura espacial, dactilología, fonología y gramática) revela los desafíos técnicos de su implementación digital y la necesidad de soluciones adaptadas al contexto mexicano.\\

\textbf{Esto de abajo es la versión preliminar de las conclusiones de TT2}\\

El desarrollo del presente Trabajo Terminal permitió cumplir con el objetivo general de crear un prototipo de aplicación móvil que traduce texto en español a Lengua de Señas Mexicana (LSM), integrando técnicas de Procesamiento de Lenguaje Natural (PLN) para el análisis y transformación de oraciones. No obstante, el modelado 3D propuesto en la fase inicial no se implementó debido a limitaciones de tiempo, recursos y la necesidad de un equipo especializado en animación y captura de movimiento. En su lugar, se emplearon videos con una capa de filtros estilo “anime”, los cuales permitieron mantener la fluidez visual en la representación de las señas sin comprometer la calidad de la comunicación.\\

Durante el proceso de desarrollo se modificó el conjunto original de frases. Las expresiones de agradecimiento fueron sustituidas por expresiones de mínima comunicación, ya que muchas de las frases de agradecimiento compartían una misma seña en LSM, lo que reducía la diversidad gestual y la efectividad del prototipo. Este cambio permitió ampliar la variedad de expresiones y mejorar la utilidad práctica del sistema.\\

La aplicación se desarrolló empleando React Expo, lo que ofrece compatibilidad multiplataforma, permitiendo su ejecución tanto en Android como en iOS sin mayores modificaciones. Respecto a los objetivos específicos, se cumplió satisfactoriamente el desarrollo del módulo de PLN, encargado del procesamiento del texto en español y su adaptación a la LSM. En cuanto a los objetivos relacionados con el modelado y la animación 3D, estos se abordaron parcialmente mediante la implementación de videos alojados en la nube, los cuales son cargados dinámicamente por la aplicación, cumpliendo así con la integración de una arquitectura basada en microservicios.

