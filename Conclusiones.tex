\chapter{Conclusiones}
En la primera etapa desarrollada del Trabajo Terminal, se establecen las bases teóricas y técnicas necesarias para el desarrollo de una solución tecnológica inclusiva orientada a mejorar la comunicación entre personas oyentes y personas con discapacidad auditiva. A partir del análisis de la motivación, la problemática y los objetivos del proyecto, se delimita un enfoque centrado en la accesibilidad lingüística a través de la Lengua de Señas Mexicana (LSM).\\

La revisión del estado del arte y la construcción del marco teórico permiten contextualizar el proyecto dentro del ámbito del procesamiento de lenguaje natural y la representación visual mediante modelado 3D, evidenciando que, si bien existen herramientas similares a nivel internacional, estas no abordan de manera específica las características lingüísticas, culturales y gramaticales de la LSM. Asimismo, el análisis del proceso de comunicación y el estudio profundo de los elementos lingüísticos de esta lengua (como su estructura espacial, dactilología, fonología y gramática) revela los desafíos técnicos de su implementación digital y la necesidad de soluciones adaptadas al contexto mexicano.