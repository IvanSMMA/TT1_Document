\chapter{Conclusiones}
El desarrollo del presente Trabajo Terminal permitió construir un prototipo funcional capaz de traducir texto en español a Lengua de Señas Mexicana (LSM) mediante técnicas de Procesamiento de Lenguaje Natural (PLN). Lo que inició como una propuesta sencilla evolucionó hacia un sistema modular y medible, diseñado con criterios de ingeniería y orientado a la accesibilidad. Aunque el modelado y la animación 3D planteados inicialmente no se implementaron por restricciones de tiempo y recursos, la incorporación de videos estilizados permitió mantener fluidez visual y consistencia comunicativa sin comprometer la calidad del prototipo.\\

Durante el proceso fue necesario refinar el conjunto de frases para mejorar la diversidad expresiva, consolidándolo en 32 expresiones organizadas en grupos temáticos. Esto permitió optimizar la búsqueda semántica y reducir el costo computacional de la clasificación. En cuanto al desempeño, el sistema superó las metas establecidas: una latencia promedio cercana a 40 ms, un \textit{throughput} superior a 25 consultas por segundo y una precisión aproximada del 90\% en la interpretación semántica. El \textit{pipeline} también logró manejar entradas no previstas, como nombres propios y textos con variaciones de escritura, gracias a un esquema de normalización, detección y deletreo.\\

La elección de React Expo y un modelo de arquitectura C4 fueron adecuados para el tamaño del proyecto, ya que esto permitió el mantenimiento y la extensibilidad del sistema. En materia de calidad, la integración de pruebas unitarias, semánticas, end-to-end y de desempeño permitió validar el comportamiento del pipeline completo y resolver fallas que no eran visibles en pruebas aisladas.\\

Una parte importante del aprendizaje surgió de los desafíos técnicos en la reproducción secuencial de videos dentro del entorno móvil Expo Go. Resolver problemas de avance, pausas y sincronización exigió un dominio profundo de los ciclos de vida de React (\texttt{useEffect}, \texttt{useCallback}) y la adopción de una arquitectura defensiva basada en estados de control y temporizadores. Estas soluciones reforzaron la estabilidad del sistema y permitieron comprender la complejidad de los entornos móviles modernos.\\

A pesar de los avances, es importante reconocer limitaciones que surgieron a lo largo del desarrollo del Trabajo Terminal: el dataset sigue siendo reducido para un uso cotidiano y las dificultades para implementar avatares 3D. No obstante, persisten oportunidades de mejora, como ampliar el corpus en colaboración con intérpretes, explorar modelos de lenguaje más robustos, incorporar índices vectoriales especializados y añadir funciones orientadas al aprendizaje y a la personalización.\\

El desarrollo de SignAI representó un proceso formativo que fortaleció la capacidad técnica y el trabajo en equipo, y permitió enfrentar problemas reales con soluciones fundamentadas. El prototipo resultante constituye una base sólida para futuras extensiones y demuestra el potencial de la tecnología para mejorar la comunicación y promover la inclusión de personas sordas usuarias de LSM.\\