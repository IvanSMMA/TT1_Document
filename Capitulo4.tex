\chapter{Estudio de viabilidad y factibilidad}
\section{Viabilidad}
El análisis de viabilidad considera los conocimientos técnicos del equipo, la madurez de las tecnologías disponibles y las condiciones actuales para el desarrollo del prototipo de una aplicación para la traducción de frases del español a Lengua de Señas Mexicana (LSM). Este estudio permite evaluar si el proyecto puede llevarse a cabo de manera exitosa bajo las condiciones planteadas.

\subsection{Conocimientos y experiencia}
El equipo de desarrollo cuenta con conocimientos en procesamiento de lenguaje natural (PLN), programación de aplicaciones móviles y web, así como en el manejo básico de animaciones y recursos visuales. Aunque la experiencia en animación 3D para señas es limitada, se considera viable el aprendizaje y la implementación de modelos básicos, apoyándose en recursos de investigación, bibliotecas existentes y posibles colaboraciones con expertos en LSM. Esto respalda la viabilidad técnica del proyecto en cuanto a conocimientos y capacidades del equipo.

\subsection{Tecnologías disponibles}
Existen diversas tecnologías y herramientas que permiten la creación de un sistema de traducción de texto a señas, tales como bases de datos de señas en video, motores de animación 3D, y frameworks para el desarrollo de aplicaciones móviles. Además, se dispone de plataformas de código abierto que pueden facilitar la representación de señas a través de modelos animados o videos precargados.

Entre las tecnologías consideradas para el proyecto se encuentran:
\begin{itemize}
	\item Librerías y bases de datos de señas mexicanas (videos de señas).
	\item Herramientas de animación 3D como Blender o motores ligeros compatibles con aplicaciones móviles.
	\item Frameworks de desarrollo móvil como Flutter o React Native.
	\item Herramientas de PLN para el análisis y segmentación de frases en español.
\end{itemize}

\subsection{Condiciones para la ejecución}
El proyecto se desarrolla como parte de un trabajo terminal académico, lo que facilita el acceso a recursos institucionales, asesoría docente y bibliografía especializada. Además, el creciente interés social y académico por promover la inclusión de la comunidad sorda en México crea un contexto favorable para la ejecución de este tipo de proyectos.

\subsection{Escalabilidad}
Si bien el proyecto se plantea inicialmente como un prototipo académico, la arquitectura modular permitirá una posible escalabilidad hacia una versión comercial en el futuro. Esto incluiría la incorporación de más frases, optimización de los motores de animación y adaptación a otros dispositivos o plataformas. Sin embargo, se reconoce que la escalabilidad implicaría mayores recursos financieros y humanos, los cuales deberán evaluarse en etapas posteriores.

\section{Factibilidad}
La factibilidad del proyecto se analiza en función de los recursos financieros, humanos y tecnológicos disponibles, así como del tiempo estimado para su desarrollo.

\subsection{Recursos financieros}
Dado que el proyecto tiene carácter académico, los costos asociados son mínimos, centrados principalmente en:
\begin{itemize}
	\item Uso de plataformas de desarrollo (gratuitas o con licencias de estudiante).
	\item Adquisición o generación de materiales visuales (videos o animaciones de señas).
	\item Posibles servicios de almacenamiento o hosting para la aplicación, en caso de requerirse.
\end{itemize}

El siguiente cuadro muestra una estimación preliminar de costos:

\begin{table}[h!]
	\centering
	\begin{tabular}{|c|c|}
		\hline
		\textbf{Concepto}                    & \textbf{Costo estimado (MXN)} \\ \hline
		Licencias de software o herramientas (si aplica) & 0 -- 2,000                   \\ \hline
		Hosting / almacenamiento en la nube             & 0 -- 1,500                   \\ \hline
		Producción o adquisición de videos de señas     & 0 -- 5,000                   \\ \hline
		\textbf{Total estimado}                         & \textbf{0 -- 8,500}          \\ \hline
	\end{tabular}
	\caption{Estimación de costos para el desarrollo del prototipo}
\end{table}

\subsection{Recursos humanos}
El equipo está conformado por estudiantes con formación en ingeniería en inteligencia artificial y desarrollo de software, lo que asegura las competencias necesarias para llevar a cabo el proyecto. Se considera factible, además, realizar capacitaciones específicas en el manejo de bases de datos de señas y en técnicas básicas de animación si el proyecto lo requiere.

\subsection{Recursos tecnológicos}
Se cuenta con el equipo de cómputo necesario para el desarrollo de la aplicación, así como con acceso a las plataformas y herramientas de software requeridas (ambientes de desarrollo, motores de animación, bibliotecas de PLN, etc.). Esto asegura la disponibilidad de los recursos tecnológicos para la implementación del prototipo.

\subsection{Plazo}
El tiempo estimado para el desarrollo del prototipo es de aproximadamente 4 meses, dividido en fases de análisis, diseño, desarrollo, pruebas y presentación. Dado que el proyecto se enfoca en un alcance controlado (traducción de un conjunto limitado de frases predefinidas), el plazo es considerado suficiente para cumplir con los objetivos planteados.

\section{Conclusión}
El estudio realizado permite concluir que el proyecto es viable y factible dentro del contexto académico. El equipo cuenta con los conocimientos básicos necesarios, acceso a las tecnologías requeridas y recursos suficientes para la construcción de un prototipo funcional. Si bien se identifican áreas que pueden requerir aprendizaje adicional, estas no representan un impedimento significativo para el desarrollo exitoso del sistema, siempre y cuando se respeten los alcances definidos y se mantenga una adecuada gestión del proyecto.
