\chapter{Estudio de viabilidad y factibilidad}

Este capítulo presenta el análisis de viabilidad y factibilidad del proyecto, con el objetivo de evaluar si existen las condiciones técnicas, humanas, tecnológicas y financieras necesarias para su desarrollo exitoso. En primer lugar, se analiza la viabilidad, considerando los conocimientos del equipo, las tecnologías disponibles, las condiciones de ejecución y el potencial de escalabilidad del prototipo. Posteriormente, se estudia la factibilidad, enfocándose en los recursos financieros, humanos y tecnológicos, así como en el tiempo estimado para completar el proyecto. Por último, se concluye sobre la posibilidad real de implementar el prototipo propuesto dentro del contexto académico establecido.\\

\section{Viabilidad}
El análisis de viabilidad considera los conocimientos técnicos del equipo, la madurez de las tecnologías disponibles y las condiciones actuales para el desarrollo del prototipo de un prototipo de aplicación móvil de apoyo para la traducción de frases del español a Lengua de Señas Mexicana (LSM). Este estudio permite evaluar si el proyecto puede llevarse a cabo de manera exitosa bajo las condiciones planteadas.

\subsection{Conocimientos y experiencia}
El equipo de desarrollo posee conocimientos en procesamiento de lenguaje natural (PLN), así como habilidades básicas en animaciones y recursos visuales. Aunque la experiencia en animación 3D orientada a señas, en programación de aplicaciones móviles y en el manejo de la Lengua de Señas Mexicana (LSM) es limitada, se considera factible adquirir y aplicar los conocimientos necesarios mediante el uso de recursos de investigación, bibliotecas especializadas y la colaboración con expertos en LSM. Esta disposición de aprendizaje y fortalecimiento de competencias respalda la viabilidad técnica del proyecto en función de las capacidades del equipo.

De manera complementaria, se cuenta con diversas tecnologías que facilitarán la implementación del prototipo, tal como se describe a continuación.

\subsection{Tecnologías disponibles}
Actualmente, existen diversas tecnologías y herramientas que facilitan la creación de sistemas de traducción de texto a señas, tales como conjunto de datos de señas en video, motores de animación 3D y frameworks para el desarrollo de aplicaciones móviles. Asimismo, se dispone de plataformas de código abierto que permiten representar señas mediante modelos animados o videos precargados, optimizando así los recursos disponibles para el desarrollo de prototipos.

Las tecnologías consideradas para el presente proyecto incluyen:
\begin{itemize} 
	\item Librerías y conjunto de datos de señas mexicanas (videos de señas). 
	\item Herramientas de animación 3D como Blender (versión 4.4.3) o Unity (versión 6.1), así como motores ligeros compatibles con aplicaciones móviles. 
	\item Frameworks de desarrollo móvil como Flutter (versión 3.29.3) y React Native (versión 0.79). 
	\item Herramientas de procesamiento de lenguaje natural (PLN) para el análisis y segmentación de frases en español. 
\end{itemize}

\subsection{Condiciones para la ejecución}
El proyecto se desarrolla en el marco de un trabajo terminal académico, lo que garantiza el acceso a recursos institucionales, asesoría especializada y bibliografía técnica actualizada. Asimismo, el creciente interés social y académico por fomentar la inclusión de la comunidad sorda en México genera un entorno favorable para la implementación de este tipo de iniciativas, fortaleciendo así las condiciones de ejecución del prototipo.

Además de considerar las condiciones actuales para la ejecución del proyecto, se ha previsto su potencial de crecimiento a futuro, como se expone a continuación.

\subsection{Escalabilidad}
Aunque el proyecto está concebido inicialmente como un prototipo académico, su arquitectura modular permitirá una potencial expansión hacia una versión comercial en etapas futuras. Esta expansión contemplaría la incorporación de un mayor número de frases, la optimización de los motores de animación y la adaptación a múltiples dispositivos o plataformas. No obstante, se reconoce que dicho crecimiento requerirá una inversión adicional en recursos financieros y humanos, factores que deberán evaluarse conforme avance el desarrollo del proyecto.

\section{Factibilidad}
La factibilidad del proyecto se analiza considerando los recursos financieros, humanos y tecnológicos disponibles, así como el tiempo estimado para su desarrollo y finalización.

\subsection{Recursos financieros}
Dado que el proyecto tiene carácter académico, los costos asociados son mínimos, centrados principalmente en:
\begin{itemize}
	\item Uso de plataformas de desarrollo gratuitas o con licencias de estudiante.
	\item Adquisición o generación de materiales visuales, como videos o animaciones de señas.
	\item Contratación de servicios de almacenamiento en la nube, en caso de ser necesarios para el despliegue de la aplicación.
\end{itemize}

En el siguiente apartado, se presenta una estimación preliminar de los costos requeridos para el desarrollo del prototipo:

\begin{table}[H]
	\centering
	\begin{tabular}{|p{8cm}|c|}
		\hline
		\textbf{Concepto} & \textbf{Costo estimado (MXN \$)} \\ \hline
		Licencias de software o herramientas (en caso de requerirse) & 0 -- 2,000 \\ \hline
		Hosting y almacenamiento en la nube & 0 -- 1,500 \\ \hline
		Producción o adquisición de videos de señas & 0 -- 5,000 \\ \hline
		\textbf{Total estimado} & \textbf{0 -- 8,500} \\ \hline
	\end{tabular}
	\caption{Estimación preliminar de costos asociados al desarrollo del prototipo académico}
	\label{tab:costos-prototipo}
\end{table}


Cabe señalar que esta estimación contempla un rango de costos, ya que algunas plataformas y herramientas pueden ofrecer versiones gratuitas o descuentos académicos, reduciendo así el gasto total del proyecto.

\subsection{Recursos humanos}
El equipo de desarrollo está integrado por estudiantes con formación en Ingeniería en Inteligencia Artificial, quienes cuentan con experiencia académica en proyectos de Ingeniería de software. Esta preparación asegura que el equipo posee las competencias necesarias para llevar a cabo el diseño, implementación y validación del prototipo propuesto. También, se contempla la posibilidad de realizar capacitaciones específicas en el manejo de bases de datos de señas y en técnicas básicas de animación 3D, a fin de fortalecer los conocimientos requeridos para el éxito del proyecto.

\subsection{Recursos tecnológicos}
El equipo de trabajo dispone del equipo de cómputo necesario para el desarrollo, prueba y validación de la aplicación móvil. Además, se cuenta con acceso a las plataformas de desarrollo y a las herramientas de software requeridas, tales como ambientes de programación, motores de animación 3D y bibliotecas especializadas de procesamiento de lenguaje natural (PLN). Esta disponibilidad de recursos tecnológicos garantiza las condiciones adecuadas para la implementación efectiva del prototipo.

\subsection{Plazo}
El tiempo estimado para el desarrollo del prototipo es de aproximadamente cuatro meses, considerando las fases de análisis, diseño, desarrollo, pruebas y presentación final. Dado que el alcance del proyecto se limita a la traducción de un conjunto predefinido de frases específicas, se considera que el plazo establecido es adecuado para cumplir con los objetivos planteados y la solución propuesta. \\


El análisis realizado permite concluir que el proyecto es tanto viable como factible dentro del contexto académico en el cual se desarrolla. El equipo de trabajo cuenta con los conocimientos fundamentales, el acceso a las tecnologías requeridas y los recursos necesarios para la construcción de un prototipo funcional. Aunque se han identificado áreas que demandarán un proceso de capacitación complementaria, estas no representan un obstáculo significativo para el éxito del proyecto, siempre que se mantenga una adecuada gestión de tiempos y actividades conforme a los alcances establecidos. De este modo, se fortalecen las condiciones para lograr un desarrollo efectivo que cumpla con los objetivos planteados y a futuro siente las bases para una posible expansión comercial.
