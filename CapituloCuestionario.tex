\chapter{Cuestionario de Usabilidad}
\section{Prueba de Usabilidad mediante Cuestionario Estructurado}

Con el propósito de evaluar la experiencia de uso, la percepción de calidad del módulo de traducción y la claridad de las animaciones en Lengua de Señas Mexicana (LSM), se aplicó una prueba de usabilidad basada en un cuestionario estructurado.\\

Este instrumento fue diseñado considerando las recomendaciones de la norma ISO 9241-210 \cite{ref62}, así como principios derivados de las heurísticas de Nielsen \cite{ref61}, con el fin de evaluar aspectos como:

\begin{itemize}
    \item Facilidad de uso.
    \item Pertinencia de la traducción.
    \item Calidad perceptual del movimiento de la animación.
    \item Utilidad del vocabulario incluido.
    \item satisfacción general del usuario.
\end{itemize}

\noindent \textbf{Procedimiento}\\
La prueba se llevó a cabo siguiendo el siguiente protocolo:
\begin{itemize}
    \item A cada participante se le mostró un video demostrativo del funcionamiento de la aplicación.
    \item Posteriormente, se le invitó a responder un cuestionario en línea compuesto por preguntas cerradas de opción múltiple, escalas Likert de 1 a 5 \cite{refpru1}, y preguntas abiertas para permitir observaciones cualitativas.
    \item La participación fue anónima y sin recopilar datos personales sensibles.
    \item No se requirió experiencia previa en LSM para participar.
\end{itemize}

\noindent \textbf{Estructura del cuestionario}\\
El cuestionario se compone de cinco bloques temáticos:
\begin{itemize}
    \item Datos demográficos.
    \item Conocimientos previos en LSM.
    \item Percepción de velocidad y pertinencia de la traducción.
    \item Calidad del movimiento del avatar.
    \item Cubrimiento del vocabulario y opinión general.
\end{itemize}

En la \textbf{Tabla \ref{tab:instrumento-usabilidad}} se resume la estructura completa del instrumento, indicando el tipo de pregunta y su propósito.

\begin{table}[H]
\centering
\renewcommand{\arraystretch}{1.6}
\begin{tabular}{|p{5cm}|p{4cm}|p{4cm}|}
\hline
\textbf{Categoría} & \textbf{Pregunta evaluada} & \textbf{Tipo de respuesta} \\ \hline

Datos demográficos & Edad del participante & Opción múltiple \\ \hline

Conocimientos previos en LSM & Nivel de dominio de LSM & Opción múltiple \\ \hline

Experiencia previa & Uso previo de herramientas de traducción & Sí / No \\ \hline

Comparativa con herramientas previas & Diferencias observadas respecto a herramientas de traducción previamente utilizadas & Respuesta abierta \\ \hline

Velocidad y pertinencia de traducción & Evaluación de qué tan rápido y pertinente fue el emparejamiento de la frase (escala 1–5) & Escala Likert \\ \hline

Utilidad del deletreo & Utilidad del modo de deletreo dentro de una conversación & Escala Likert \\ \hline

Calidad del movimiento del avatar & Percepción de fluidez y claridad del movimiento en las animaciones & Opción múltiple \\ \hline

Cobertura del vocabulario & Adecuación del conjunto de frases y categorías incluidas en la aplicación & Opción múltiple \\ \hline

Fortalezas percibidas & Aspectos positivos o elementos destacados por el usuario & Respuesta abierta \\ \hline

Áreas de oportunidad & Problemas, fallas o aspectos a mejorar identificados por el usuario & Respuesta abierta \\ \hline

\end{tabular}
\caption[Instrumento de evaluación de usabilidad]{Resumen del cuestionario empleado para la evaluación de usabilidad de la aplicación, elaboración propia.}
\label{tab:instrumento-usabilidad}
\end{table}

\newpage
Una copia íntegra del cuestionario, con el texto exacto de cada reactivo, se incluye en el \textbf{\nameref{anexo:cuestionario}} para referencia completa.

\section{Resultados de la Evaluación de Usabilidad}
A continuación se presentan los resultados obtenidos del cuestionario aplicado a los participantes. Las figuras muestran los resultados de las preguntas cerradas, mientras que las respuestas abiertas fueron analizadas siguiendo una codificación temática simple.\\

\textbf{Perfil de los Participantes}\\
La Figura \ref{fig:edad-participantes} muestra la distribución por edad de los participantes. El rango de 18–24 años abarca un 43.8\%, seguido del grupo de 25–50 años que abarca el mismo porcentaje (43.8\%), mientras que un 12.5\% pertenece al grupo de 50 años o más. Esta distribución es representativa del público potencial de la aplicación, principalmente jóvenes adultos.\\

\begin{center}
    \includegraphics[width=0.75\textwidth]{Images/Cap6/1_Edad.jpeg}
    \captionof{figure}[Distribución por edad]{Distribución por edad de los participantes, elaboración propia.} 
    \label{fig:edad-participantes}
\end{center}

En cuanto al nivel de dominio de LSM, los resultados en la Figura \ref{fig:nivel_dominio_lsm} indican una composición equilibrada:
\begin{itemize}
    \item 25\% sin experiencia previa.
    \item 31.3\% con nivel básico.
    \item 25\% con nivel intermedio.
    \item 18.8\% con nivel avanzado o fluido.
\end{itemize}

\begin{center}
    \includegraphics[width=0.95\textwidth]{Images/Cap6/2_Conocimientos_LSM.jpeg}
    \captionof{figure}[Nivel de Dominio LSM]{Nivel de dominio de LSM, elaboración propia.} 
    \label{fig:nivel_dominio_lsm}
\end{center}

\textbf{Experiencia previa con herramientas similares}\\
De acuerdo con la Figura \ref{fig:uso_herramientas_existentes}, el 62.5\% de los participantes indicó no haber utilizado herramientas de traducción similares, mientras que el 37.5\% sí tenía experiencia previa.\\

\begin{center}
    \includegraphics[width=0.75\textwidth]{Images/Cap6/3_Uso_Herramientas.jpeg}
    \captionof{figure}[Experiencias previas con herramientas]{Experiencia previa con aplicaciones y herramientas similares por parte de los usuarios, elaboración propia.} 
    \label{fig:uso_herramientas_existentes}
\end{center}

Solo 9 participantes reportaron haber utilizado previamente aplicaciones similares. Entre ellos, se identificaron tres diferencias principales:
\begin{enumerate}
    \item \textbf{Mayor naturalidad en la animación}\\
    Varios participantes destacaron que la fluidez del avatar supera la de herramientas previas:
    \begin{itemize}
        \item “La fluidez de la animación”.
        \item “Mejor calidad y velocidad percibida en los videos / animaciones”.
    \end{itemize}

    \item \textbf{Posibilidad de concatenar frases}\\
    Una funcionalidad ampliamente valorada fue la capacidad de realizar traducciones por frase y no solo deletreo:
    \begin{itemize}
        \item “Se pueden concatenar las frases, eso es algo muy ventajoso”.
        \item “Posibilidad de identificar frases y no solo deletreo”.
    \end{itemize}

    \item \textbf{Interfaz más clara o intuitiva}\\
    Algunos comentarios reconocieron mejoras visuales:
    \begin{itemize}
        \item “Mejor distribución de colores”.
    \end{itemize}
\end{enumerate}

En general, los usuarios con experiencia previa percibieron la herramienta como más intuitiva, más clara y con animaciones superiores. Esta proporción revela que la mayoría evaluó la aplicación sin sesgos comparativos y que una minoría pudo aportar información contextual respecto a soluciones existentes.\\

\textbf{Percepción sobre la velocidad y pertinencia de la traducción}\\
Los resultados de la Figura \ref{fig:fluidez} mostraron una percepción predominantemente positiva. En la escala del 1 al 5:
\begin{itemize}
    \item 31.3\% calificó la traducción con 4.
    \item 31.3\% la calificó con 5.
    \item 25\% asignó un 3.
    \item Solo 12.5\% la calificó con 2.
    \item Ningún usuario seleccionó 1.
\end{itemize}

\begin{center}
    \includegraphics[width=0.85\textwidth]{Images/Cap6/5_Fluidez.jpg}
    \captionof{figure}[Velocidad y pertinencia de la traducción]{Velocidad y pertinencia de la traducción, elaboración propia.} 
    \label{fig:fluidez}
\end{center}

Estos resultados indican que más del 60\% considera que la traducción es rápida, pertinente y adecuada, mientras que únicamente una minoría percibió lentitud o falta de pertinencia. Esto valida el diseño del módulo de emparejamiento implementado.\\

\textbf{Utilidad del modo de deletreo}\\
La valoración del modo de deletreo fue notablemente positiva:
\begin{itemize}
    \item 31.3\% lo considera “extremadamente útil”.
    \item 25\% lo considera “muy útil”.
    \item 25\% “útil”.
    \item 12.5\% “poco útil”.
    \item Solo 6.3\% lo percibe como “nada útil”.
\end{itemize}

\begin{center}
    \includegraphics[width=0.85\textwidth]{Images/Cap6/6_Utilidad_Deletreo.jpeg}
    \captionof{figure}[Utilidad del modo de deletreo]{Utilidad del modo de deletreo, elaboración propia.} 
    \label{fig:utilidad_deletreo}
\end{center}

En conjunto, 81.3\% lo considera “útil” o “muy útil”, lo que confirma que este mecanismo funciona como un soporte importante en situaciones donde el vocabulario no está disponible o cuando se requiere precisión (como nombres propios).\\

\textbf{Percepción del movimiento del avatar}\\
La Figura \ref{fig:percepcion_movimiento} indica que el 75\% de los usuarios describió el movimiento del avatar como claro y natural, lo que indica que la animación es comprensible y suficientemente fluida para usuarios no expertos.\\

Sin embargo:
\begin{itemize}
    \item 18.8\% señaló movimientos lentos o trabados.
    \item 6.2\% describió la animación como confusa.
\end{itemize}

\begin{center}
    \includegraphics[width=0.85\textwidth]{Images/Cap6/7_Calidad_Movimiento.jpeg}
    \captionof{figure}[Percepción del movimiento]{Percepción del movimiento del avatar, elaboración propia.} 
    \label{fig:percepcion_movimiento}
\end{center}

Aunque la mayoría percibe una buena calidad de animación, estos resultados revelan oportunidades de mejora en la suavidad, velocidad y naturalidad de ciertos gestos.\\

\textbf{Evaluación del vocabulario disponible}\\
Respecto a la cobertura del vocabulario, la Figura \ref{fig:evaluacion_vocabulario} establece que:
\begin{itemize}
    \item 56.3\% considera que las frases y categorías incluidas cubren las necesidades básicas de comunicación.
    \item 43.8\% opina que faltan frases o categorías importantes.
    \item Ningún participante consideró que existieran frases innecesarias.
\end{itemize}

\begin{center}
    \includegraphics[width=0.85\textwidth]{Images/Cap6/8_Seleccion_Frases.jpeg}
    \captionof{figure}[Evaluación del vocabulario]{Evaluación del vocabulario seleccionado, elaboración propia.} 
    \label{fig:evaluacion_vocabulario}
\end{center}

Este resultado sugiere que, si bien la base actual de frases es funcional, existe una expectativa clara de ampliar el repertorio para cubrir más contextos comunicativos cotidianos.\\

\newpage
\textbf{Fortalezas percibidas}\\
Entre las 15 respuestas registradas, surgieron cinco temas recurrentes:
\begin{enumerate}
    \item \textbf{Concatenación y pertinencia de frases}\\
    La capacidad de traducir frases completas fue mencionada como la principal fortaleza:
    \begin{itemize}
        \item “Su plus es que añadieron lo de concatenar las frases”.
        \item “Permite identificar frases y no solo deletreo”.
    \end{itemize}

    \item \textbf{Fluidez y calidad de las animaciones}\\
    Los usuarios destacaron la naturalidad del movimiento:
    \begin{itemize}
        \item “Las animaciones se ven bien, no es muy complicado usar la app”.
        \item “Las animaciones no se traban”.
    \end{itemize}

    \item \textbf{Interfaz simple, limpia y no invasiva}\\
    La aplicación fue percibida como fácil de usar:
    \begin{itemize}
        \item “La interfaz es simple y minimalista”.
        \item “Excelente distribución de botones”.
        \item “Buena elección de colores”.
    \end{itemize}

    \item \textbf{Utilidad para el aprendizaje de LSM}\\
    Varios participantes valoraron la función educativa:
    \begin{itemize}
        \item “Aprender LSM por cómo en verdad se expresan las señas es lo más valioso”.
        \item “Sirve para situaciones donde se necesita comunicar algo”.
    \end{itemize}

    \item \textbf{Respuesta inmediata y sistema fluido}\\
    Comentarios recurrentes señalaron buen rendimiento:
    \begin{itemize}
        \item “La velocidad de los videos es buena”.
        \item “El tener siempre una respuesta del sistema hace que sea una aplicación completa”.
        \item “Está bien que no dejen al usuario sin respuesta”.
    \end{itemize}
\end{enumerate}

En conjunto, los usuarios describen la aplicación como fluida, intuitiva y útil, especialmente para principiantes.\\

\newpage
\textbf{Áreas de oportunidad}\\
Entre las 14 respuestas abiertas, se identificaron cuatro líneas principales de mejora:
\begin{enumerate}
    \item \textbf{Ampliación del vocabulario}\\
    Fue el tema más mencionado:
    \begin{itemize}
        \item “Agregar más frases”.
        \item “Faltan algunas categorías de frases o funciones extra”.
        \item “A veces la seña va muy rápido y no se distingue”.
    \end{itemize}

    La percepción general indica que el vocabulario actual es funcional, pero insuficiente para cubrir situaciones más amplias.\\

    \item \textbf{Mejora de la fluidez del avatar}\\
    Aunque las animaciones son valoradas positivamente, algunos comentaron que podrían ser más naturales:
    \begin{itemize}
        \item “La seña va muy rápido y no se distingue”.
        \item “Hacen falta gestos faciales o expresividad”.
    \end{itemize}

    \item \textbf{Mayor personalización}\\
    Algunos usuarios desean opciones para adaptar la experiencia:
    \begin{itemize}
        \item “Quisiera poder tener mi propio personaje y guardar frases”.
        \item “Poder crear tu propio video con tu animación”.
    \end{itemize}

    \item \textbf{Funcionalidades adicionales mediante IA}
    Algunos participantes sugieren extender el sistema más allá de traducción a LSM:
    \begin{itemize}
        \item “Agregar reconocimiento de voz y texto en imágenes”.
        \item “Un traductor que permita comunicación bidireccional”.
    \end{itemize}
\end{enumerate}

En conjunto, las áreas de oportunidad se enfocan en ampliar categorías, mejorar expresividad del avatar y considerar nuevas funciones de interacción y reconocimiento.

