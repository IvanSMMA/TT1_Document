\chapter{Marco Teórico: conceptos teóricos}
\section{Comunicación}
La comunicación es un proceso dinámico, en el que participa una fuente o emisor que envía un mensaje a través de un canal o medio a un potencial receptor que, a su vez, puede convertirse también en emisor \cite{ref20}. Cuando se transmite el mensaje de una forma clara y efectiva para el receptor sin generar dudas ni confusiones, se logra un comunicación efectiva \cite{ref21}.\\

Comunicar es el acto que permite establecer relaciones efectivas, compartir experiencias, experimentar emociones y sentimientos, así como hacer que los demás lo experimenten \cite{ref22}. A continuación, se describen los tipos de comunicación que existen.

\subsection{Tipos de Comunicación}
Uno de los tipos de comunicación está basado en si se usan palabras o no, es decir, comunicación verbal o no verbal \cite{ref23}:

\begin{itemize}
    \item \textbf{Comunicación verbal}: se emplean palabras y se lleva a cabo a través del habla o de manera escrita.
    \item \textbf{Comunicación no verbal}: se emplea el lenguaje corporal, gestos, signos no lingüísticos y sonidos que no forman palabras.
\end{itemize}

Otro de los tipos de comunicación son la formal y la informal, las cuales se describen a continuación \cite{ref23}:

\begin{itemize}
    \item \textbf{Formal}: se utiliza un lenguaje especializado y estandarizado, sin errores ni coloquialismos, además de que se toman en cuenta las jerarquías sociales.
    \item \textbf{Informal}: no se emplea lenguaje estandarizado, no se siguen protocolos jerárquicos y se emplean coloquialismos.
\end{itemize}

Un tercer tipo de clasificación es aquella que está basada en el tipo de acto comunicativo, la cual contiene los siguientes elementos \cite{ref23}:
\begin{itemize}
    \item \textbf{Comunicación intrapersonal}: conversaciones que un ser humano entabla consigo mismo.
    \item \textbf{Comunicación interpersonal}: intercambio de ideas y pensamientos entre dos personas, la cuál debe ser directa e interactiva.
    \item \textbf{Comunicación grupal}: intercambio de ideas y pensamientos entre un grupo de más de dos personas, las cuales se comunican con un propósito.
    \item \textbf{Comunicación masiva}: dirigida a un gran número de personas, mediante un medio masivo de comunicación como lo puede ser las redes sociales, radio, televisión, entre otros.
\end{itemize}

\subsection{Elementos de la comunicación}
Dentro del proceso de comunicación hay una serie de elementos que hacen posible la transmisión de un mensaje. A continuación, se enlistan cada uno de ellos: 
\begin{itemize}
    
\item \textbf{Emisor}: es el individuo que inicia el intercambio de información al transmitir el mensaje \cite{ref22}. Dicho mensaje debe ser codificado en un sistema de símbolos que deberá ser entendible para el receptor. 

\item \textbf{Receptor}: individuo que recibe el mensaje enviado, el cual es interpretado con base en las experiencias, opiniones, contexto y situación del receptor \cite{ref20}. El receptor también puede ser el emisor.

\item \textbf{Código}: es el sistema de signos que es empleado tanto por el emisor como por el receptor para llevar a cabo el proceso de comunicación. Ese sistema debe ser conocido por ambos para facilitar la codificación y descodificación \cite{ref23}.

\item \textbf{Mensaje}: es la información que el emisor transmite al receptor por medio del código \cite{ref24}.

\item \textbf{Canal}: medio en el que los mensajes del emisor se transmiten hacia el receptor \cite{ref20}.

\item \textbf{Contexto}: se refiere a la situación en la que se lleva a cabo el proceso de comunicación, la cual tiene influencia directa en el entendimiento e interpretación del mensaje \cite{ref24}.

\item \textbf{Retroalimentación}: es la respuesta que el receptor emite tras haber recibido e interpretado un mensaje, convirtiéndose momentáneamente en emisor. Este elemento permite cerrar el ciclo comunicativo al brindar al emisor una señal clara sobre si su mensaje fue comprendido, aceptado o necesita ser aclarado o reformulado \cite{ref23}.

\item \textbf{Ruido o interferencia}: dentro del proceso de comunicación puede haber factores externos que dificultan o impiden el entendimiento de los mensajes \cite{ref23}.
\end{itemize}

% \begin{center}
%     \includegraphics[width=0.9\textwidth]{Images/Cap 2/ProcesoComunicación.png}
%     \captionof{figure}[Proceso de comunicación]{Proceso de comunicación, elaboración propia.} 
% \end{center}

% TOMAR EN CUENTA
% \subsubsection{Diagrama de actividades} 
% \begin{center}
% 	\makebox[\textwidth]{%
% 		\includegraphics[width=1\textwidth]{Images/Cap 3/Actividades.png}
% 	}
%     \captionof{figure}{Diagrama de actividades del sistema}
% \end{center}

\begin{center}
	\makebox[\textwidth]{%
		\includegraphics[width=1\textwidth]{Images/Cap 2/ProcesoComunicación.png}
	}
    \captionof{figure}[Proceso de comunicación]{Proceso de comunicación, elaboración propia.}
\end{center}

La comunicación es un proceso indispensable para la interacción humana ya que por medio de ella las personas pueden intercambiar ideas, pensamientos y emociones. No obstante, como se menciona en el concepto de ruido, en ocasiones hay elementos que impiden que la comunicación se lleve a cabo, como lo pueden ser las barreras de la comunicación.

\newpage
\subsection{Barreras de la comunicación}
Las barreras de la comunicación son elementos que limitan o dificultan que las personas puedan comunicarse, a la par que se dificulta su proceso de comunicación \cite{ref2}. Son todas las perturbaciones que sufre un mensaje, en cualquiera de los elementos que forman parte del proceso de comunicación.\\

Los principales tipos de barreras son:
\begin{enumerate}
    \item \textbf{Barreras físicas}: son interferencias causadas por elementos del entorno o en el medio donde se lleva a cabo la comunicación \cite{ref25}.
    \item \textbf{Barreras psicológicas}: son aquellas que surgen por emociones, prejuicios o estados mentales que afectan la interpretación del mensaje \cite{ref25}.
    \item \textbf{Barreras semánticas}: surgen cuando hay confusión en el significado de las palabras, debido a una interpretación incorrecta del lenguaje. Generalmente ocurren cuando se habla en un idioma que el emisor o el receptor no entienden, o se emplean conceptos técnicos desconocidos \cite{ref25}.
    \item \textbf{Barreras administrativas}: generalmente se presentan en entornos laborales y son causadas por falta de planeación, malentendidos, falta de claridad en los procesos de comunicación y distorsiones semánticas \cite{ref25}.
    \item \textbf{Barreras culturales}: este tipo de barreras se presentan cuando hay diferencias en costumbres, valores, normas o expresiones entre culturas, que imposibilitan la comunicación \cite{ref25}.
    \item \textbf{Barreras interpersonales}: hace referencia a las barreras en las que hay suposiciones incorrectas y diferentes percepciones \cite{ref25}.
    \item \textbf{Barreras tecnológicas}: fallas y limitaciones que se presentan en medios tecnológicos empleados para la comunicación \cite{ref25}.
    \item \textbf{Barreras fisiológicas}: impedimentos físicos o biológicos causados por deficiencias en los sentidos, enfermedades o condiciones médicas que afectan cualquiera de los sentidos de manera parcial o total, afectando la transmisión de información \cite{ref25}. Por ejemplo, voz débil, pronunciación defectuosa, sordera, problemas del habla, problemas visuales, etc.
\end{enumerate}
Para efectos de este Trabajo Terminal se analizarán las barreras fisiológicas, concretamente las que son causadas por problemas de sordera. En el siguiente apartado se describen los términos correctos para referirse a las personas con capacidad de escucha y a las personas con discapacidad auditiva.\\

\section{Personas con discapacidad auditiva}
\subsection{Personas Oyentes}
Un oyente se define como aquella persona con la capacidad de escuchar sonidos que le permiten interpretar mensajes. El término procede del verbo oír, que refiere a la capacidad que posee un individuo para poder percibir sonidos \cite{ref26}.

\subsection{Personas con discapacidad auditiva (sordas)}
Por otro lado, una persona que padece de discapacidad auditiva es aquella que ha sufrido la pérdida de la función del sistema auditivo, teniendo como consecuencia una discapacidad para poder oír, lo que dificulta el acceso al lenguaje oral \cite{ref27}.\\ 

Los términos adecuados para referirse a las personas que padecen de esta condición son personas sordas, personas con discapacidad auditiva o personas de la comunidad sorda \cite{refsordos}. \\

De acuerdo con la Federación Mundial de Sordos, existen aproximadamente 70 millones de personas sordas en todo el mundo, las cuales emplean más de 300 diferentes lenguas de señas \cite{ref28}. Las lenguas de señas varían entre países, presentando cambios principalmente en la estructura gramatical, sintaxis, vocabulario, signos, alfabeto y expresiones corporales \cite{ref26}.\\

Por otro lado, la Secretaría de Salud menciona que en México hay aproximadamente 2.3 millones de personas con discapacidad auditiva, de las cuales más del 50\% son mayores de 60 años, 34\% tienen entre 30 y 59 años, y el 2\% son niñas y niños \cite{ref3}.\\

Las principales causas de problemas de audición son antecedentes familiares de sordera heredados, edad avanzada, enfermedades infecciosas, exposición continua a sonidos intensos, entre otras \cite{ref3}.\\

Las personas sordas enfrentan consecuencias en ámbitos académicos, laborales, sociales y emocionales, debido a que las situaciones de aislamiento, deficiencia en la comunicación y dificultades del día a día repercuten negativamente para integrarse en grupos y para socializar \cite{ref29}. \\

\newpage
\subsection{Tipos de Discapacidad Auditiva}
La discapacidad auditiva se clasifica en tres tipos según distintos criterios: según la parte del oído afectada, según el grado de pérdida auditiva y según el momento en que se adquiere \cite{ref30}:\\
\newline\textbf{Según la parte del oído afectada}
\begin{itemize}
    \item \textbf{Hipoacusia conductiva}: es producida por un impedimento en el trayecto de las ondas sonoras del oído externo y medio al oído interno, causado por tumores, perforación del tímpano, traumatismos o disfunciones del oído.  
    \item \textbf{Hipoacusia neurosensorial}: se produce cuando el nervio auditivo o las células ciliadas son dañadas, ya sea por herencia, anormalidades al momento del nacimiento, exposición a ruidos fuertes, traumatismos, entre otras causas.  
    \item \textbf{Hipoacusia mixta}: combinación de hipoacusia conductiva e hipoacusia neurosensorial, causadas por anormalidades al nacer, infecciones, tumores y lesiones en la cabeza.  
\end{itemize}

\begin{center}
    \includegraphics[width=0.9\textwidth]{Images/Cap 2/PartesOido.jpg}
    \captionof{figure}[Partes del oído humano]{Partes del oído humano, obtenido de \cite{ref31}.} 
\end{center}

\newpage
\textbf{Según el grado de pérdida}\\
El rango normal de audición oscila entre 0 y 20 decíbeles (dB). Tomando en consideración ese rango, se establece la siguiente clasificación de acuerdo con los dB que se hayan perdido:

\begin{itemize}
    \item \textbf{Leve:} 20-40 dB.  
    \item \textbf{Moderada:} 40-70 dB.  
    \item \textbf{Severa:} 70-90 dB.  
    \item \textbf{Profunda:} más de 90 dB.  
\end{itemize}

\textbf{Según el momento de la adquisición}\\
En esta clasificación, la discapacidad auditiva puede ser:

\begin{itemize}
    \item \textbf{Hereditaria}: la discapacidad está contenida en algunos de los genes de uno o ambos progenitores.  
    \item \textbf{Adquirida}: la discapacidad puede ser prenatal (antes del nacimiento) o postnatal (después del nacimiento), y en este último caso se deben tomar en cuenta otros criterios:
        \begin{itemize}
        \item \textbf{Prelocutiva:} antes del desarrollo del lenguaje.  
        \item \textbf{Postlocutiva:} después del desarrollo del lenguaje.  
        \end{itemize}
    \end{itemize}

Las personas sordas enfrentan consecuencias en ámbitos académicos, laborales, sociales y emocionales, debido a que las situaciones de aislamiento, deficiencia en la comunicación y dificultades del día a día repercuten negativamente para integrarse en grupos y para socializar \cite{ref29}. En la siguiente sección, se abordan las brechas entre las personas oyentes y las personas con discapacidad.

\subsection{Brechas entre personas oyentes y personas con discapacidad auditiva}
En el plano sociocultural el lenguaje es esencial en las formas de comunicación en una comunidad, pero cuando no todos los individuos pueden responder a esa lógica comunicativa se crean brechas en los discursos que giran en torno a las formas de relacionarse con los demás, puesto que aquellos que tienen códigos y configuraciones diferentes pasan a estar en un plano de invisibilidad \cite{ref32}.\\

La comunidad sorda, a pesar de ser un grupo portador de un lenguaje cultural particular, debe responder a la lengua “natural” de las personas oyentes, y de no poder hacerlo ocasiona que sean excluidos en diferentes escenarios de la vida cotidiana. Esta comunidad ha sido estereotipada como personas incapaces o con limitaciones para insertarse en la sociedad, por lo que, si no pueden entrar en la “lógica natural” para comunicarse con las personas, se ven forzados a interactuar solamente con las personas que comparten su misma condición \cite{ref32}.\\

A lo largo de los últimos años, se han realizado múltiples esfuerzos a nivel gubernamental y se han puesto en marcha discursos que giran alrededor del reconocimiento e inclusión de todas las personas por igual, como lo es la Ley General para la Inclusión de las Personas con Discapacidad \cite{ref37}, para garantizar una mayor participación de las personas con discapacidad auditiva en escenarios sociales. No obstante, lo expresado en la legalidad dista mucho de las realidades particulares de las personas sordas en el marco sociocultural. La comunidad sorda ha sido reconocida como minoría lingüística y, por sus mismas condiciones, ha sido ubicada socialmente en el plano de la exclusión y la invisibilidad \cite{ref32}. \\ 

La presencia de barreras de comunicación generan aislamiento e impiden el desarrollo de una existencia satisfactoria, lo que puede generar graves problemas psicológicos como la depresión, ansiedad, insomnio, estrés, ideas paranoides y sensibilidad interpersonal \cite{ref27}.\\

Además, la comunidad sorda presenta dificultad para acceder a la información proveniente de la televisión, radio, llamadas telefónicas, megafonías en estaciones de metro y salidas de aeropuertos, etc., debido a que esta es principalmente transmitida hacia la población oyente.\\

A pesar de que las personas sordas presentan muchas dificultades en su vida diaria, hoy en día disponen de numerosas herramientas de apoyo (ver \textbf{\autoref{sec:edoArte}}) para impulsar su inclusión en entornos sociales y favorecer su crecimiento personal, como lo son las prótesis auditivas, señales acústicas y su propia Lengua de Señas. \\

En este Trabajo Terminal, únicamente se centrará el estudio en las Lenguas de Señas, concretamente en la Lengua de Señas Mexicana (LSM), revisando toda la documentación existente hasta el 2024, año de la elaboración de este trabajo.\\

\newpage
\section{Lengua de Señas Mexicana}
\subsection{Definición de Lengua de Señas}
La Lengua de Señas es definida como la lengua natural de expresión y configuración gesto-espacial y percepción visual gracias a la cual los sordos pueden comunicarse con su entorno social, la cual está basada en movimientos y expresiones a través de manos, ojos, rostro, boca y cuerpo \cite{ref33}.\\

En el mundo existen cerca de 300 lenguas de señas distintas, siendo así que cada país posee su propia lengua de señas. Por ejemplo, la Lengua de Señas Mexicana (LSM) es diferente a la Lengua de Señas Española (LSE), que a pesar de estar articulados en el mismo idioma (español), no comparten muchas señas en común debido a que ambas lenguas presentan señas que pueden ser regionalismos de cada país \cite{ref33}.\\

Por su parte, la Lengua de Señas Mexicana (LSM) es la lengua de señas que se emplea en México, que cuenta con su propio vocabulario y gramática. A la LSM se le considera como una lengua, debido a que es completamente capaz de expresar una amplia gama de pensamientos y emociones como cualquier otra lengua \cite{ref33}.

\subsection{Lengua de Señas Mexicana (LSM)}
La Ley General para la Inclusión de las Personas con Discapacidad \cite{ref34} define a la LSM, en el Artículo 2, como la lengua de una comunidad de sordos que consiste en una serie de signos gestuales articulados con las manos y acompañados de expresiones faciales, mirada intencional y movimiento corporal, dotados de función lingüística, que forma parte del patrimonio lingüistico de dicha comunidad y es tan rica y compleja en gramática y vocabulario como cualquier lengua oral \cite{ref34}.\\

Por su parte, el Artículo 20 de dicha ley establece que los medios de comunicación deben implementar la tecnología, más concretamente, de intérpretes de LSM que permitan a la comunidad de sordos las facilidades de comunicación \cite{ref34}.\\

En México hay entre 87,000 y 100,000 personas hablantes de LSM que la dominan y la emplean como vía de comunicación, siendo incluso una población mucho más grande que algunas comunidades hablantes de lenguas indígenas del país \cite{ref35}.\\

\newpage
\subsection{Abecedario de la LSM}
La siguiente tabla explica detalladamente cómo se conforma cada una de las letras del abecedario de LSM:

\begin{longtable}{|m{2cm}|m{5cm}|m{5cm}|}
    \hline
    \textbf{Letra} & \textbf{Descripción} & \textbf{Seña} \\
    \hline
    \endfirsthead
    
    \hline
    \textbf{Letra} & \textbf{Descripción} & \textbf{Seña} \\
    \hline
    \endhead
    
    \hline
    \endfoot
    
    \endlastfoot
    
    A & Con la mano cerrada, se muestran las uñas y se estira el dedo pulgar hacia un lado. La palma mira al frente.
    & \makecell{\colorbox{white}{\includegraphics[width=4cm]{Images/Cap 2/Alfabeto LSM/A.png}}} \\
    \hline
    
    B & Los dedos índice, medio, anular y meñique se estiran unidos y el pulgar se dobla hacia la palma, la cual mira al frente.
    & \makecell{\colorbox{white}{\includegraphics[width=4cm]{Images/Cap 2/Alfabeto LSM/B.png}}} \\
    \hline
    
    C & Los dedos índice, medio, anular y meñique se mantienen unidos y en posición cóncava; el pulgar también se coloca en esa posición. La palma mira a un lado.
    & \makecell{\colorbox{white}{\includegraphics[width=4cm]{Images/Cap 2/Alfabeto LSM/C.png}}} \\
    \hline

    D & Los dedos medio, anular, meñique y pulgar se unen por las puntas y el dedo índice se estira. La palma mira al frente.
    & \makecell{\colorbox{white}{\includegraphics[width=4cm]{Images/Cap 2/Alfabeto LSM/D.png}}} \\
    \hline

    E & Se doblan los dedos completamente y se muestran las uñas. La palma mira al frente.
    & \makecell{\colorbox{white}{\includegraphics[width=4cm]{Images/Cap 2/Alfabeto LSM/E.png}}} \\
    \hline

    F & Con la mano abierta y los dedos unidos, se dobla el índice hasta que su parte lateral toque la yema del pulgar. La palma mira a un lado.
    & \makecell{\colorbox{white}{\includegraphics[width=4cm]{Images/Cap 2/Alfabeto LSM/F.png}}} \\
    \hline

    G & Se cierra la mano y los dedos índice y pulgar se estiran. La palma mira hacia la persona que se comunica.
    & \makecell{\colorbox{white}{\includegraphics[width=4cm]{Images/Cap 2/Alfabeto LSM/G.png}}} \\
    \hline

    H & Se cierra la mano y los dedos índice y medio se unen y se estiran, se extiende el dedo pulgar señalando hacia arriba. La palma mira hacia la persona que se comunica.
    & \makecell{\colorbox{white}{\includegraphics[width=4cm]{Images/Cap 2/Alfabeto LSM/H.png}}} \\
    \hline

    I & Con la mano cerrada, el dedo meñique se estira señalando hacia arriba. La palma se coloca de lado.
    & \makecell{\colorbox{white}{\includegraphics[width=4cm]{Images/Cap 2/Alfabeto LSM/I.png}}} \\
    \hline

    J & Con la mano cerrada, el dedo meñique estirado señala hacia arriba y la palma señala a un lado. La mano dibuja una “j” en el aire.
    & \makecell{\colorbox{white}{\includegraphics[width=4cm]{Images/Cap 2/Alfabeto LSM/J.png}}} \\
    \hline

    K & Se cierra la mano con los dedos índice, medio y pulgar estirados. La yema del pulgar se coloca entre el índice y el medio, moviendo la muñeca hacia arriba.
    & \makecell{\colorbox{white}{\includegraphics[width=4cm]{Images/Cap 2/Alfabeto LSM/K.png}}} \\
    \hline

    L & Con la mano cerrada y los dedos índice y pulgar estiados, se forma una “L”. La palma mira al frente.
    & \makecell{\colorbox{white}{\includegraphics[width=4cm]{Images/Cap 2/Alfabeto LSM/L.png}}} \\
    \hline

    M & Con la mano cerrada, se ponen los dedos índice, medio y anular sobre el pulgar.
    & \makecell{\colorbox{white}{\includegraphics[width=4cm]{Images/Cap 2/Alfabeto LSM/M.png}}} \\
    \hline
 
    N & Con la mano cerrada, se ponen los dedos índice y medio sobre el pulgar. 
    & \makecell{\colorbox{white}{\includegraphics[width=4cm]{Images/Cap 2/Alfabeto LSM/N.png}}} \\
    \hline

    Ñ & Con la mano cerrada, se ponen los dedos índice y medio sobre el pulgar. Se mueve la muñeca a los lados. 
    & \makecell{\colorbox{white}{\includegraphics[width=4cm]{Images/Cap 2/Alfabeto LSM/Ñ.png}}} \\
    \hline

    O & Con la mano se forma una letra “o”. Todos los dedos se tocan por las puntas. 
    & \makecell{\colorbox{white}{\includegraphics[width=4cm]{Images/Cap 2/Alfabeto LSM/O.png}}} \\
    \hline

    P & Con la mano cerrada y los dedos índice, medio y pulgar estirados, se coloca la yema del pulgar entre el índice y el medio.
    & \makecell{\colorbox{white}{\includegraphics[width=4cm]{Images/Cap 2/Alfabeto LSM/P.png}}} \\
    \hline
    
    Q & Con la mano cerrada, se colocan los dedos índice y pulgar en posición de garra. La palma mira hacia abajo, y se mueve hacia los lados.
    & \makecell{\colorbox{white}{\includegraphics[width=4cm]{Images/Cap 2/Alfabeto LSM/Q.png}}} \\
    \hline
    
    R & Con la mano cerrada, se estiran y entrelazan los dedos índice y medio. La palma mira al frente.
    & \makecell{\colorbox{white}{\includegraphics[width=4cm]{Images/Cap 2/Alfabeto LSM/R.png}}} \\
    \hline

    S & Con la mano cerrada, se pone el pulgar sobre los otros dedos. La palma mira al frente.
    & \makecell{\colorbox{white}{\includegraphics[width=4cm]{Images/Cap 2/Alfabeto LSM/S.png}}} \\
    \hline

    T & Con la mano cerrada, el pulgar se pone entre el índice y el medio. La palma mira al frente. 
    & \makecell{\colorbox{white}{\includegraphics[width=4cm]{Images/Cap 2/Alfabeto LSM/T.png}}} \\
    \hline

    U & Con la mano cerrada, se estiran los dedos índice y medio unidos. La palma mira al frente. 
    & \makecell{\colorbox{white}{\includegraphics[width=4cm]{Images/Cap 2/Alfabeto LSM/U.png}}} \\
    \hline

    V & Con la mano cerrada, se estiran los dedos índice y medio separados. La palma mira al frente. 
    & \makecell{\colorbox{white}{\includegraphics[width=4cm]{Images/Cap 2/Alfabeto LSM/V.png}}} \\
    \hline

    W & Con la mano cerrada, se estiran los dedos índice, medio y anular separados. La palma mira al frente. 
    & \makecell{\colorbox{white}{\includegraphics[width=4cm]{Images/Cap 2/Alfabeto LSM/W.png}}} \\
    \hline

    X & Con la mano cerrada, el índice y el pulgar en posición de garra y la palma dirigida a un lado, se realiza un movimiento al frente y de regreso. 
    & \makecell{\colorbox{white}{\includegraphics[width=4cm]{Images/Cap 2/Alfabeto LSM/X.png}}} \\
    \hline
    
    Y & Con la mano cerrada, se estira el meñique y el pulgar. La palma mira hacia la persona que se comunica. 
    & \makecell{\colorbox{white}{\includegraphics[width=4cm]{Images/Cap 2/Alfabeto LSM/Y.png}}} \\
    \hline

    Z & Con la mano cerrada, el dedo índice estirado y la palma al frente, se dibuja una letra z en el aire. 
    & \makecell{\colorbox{white}{\includegraphics[width=4cm]{Images/Cap 2/Alfabeto LSM/Z.png}}} \\
    \hline
    
    \caption[Abecedario de la LSM]{Abecedario de la LSM, obtenido de \cite{ref36}.} \label{tabla:LSM}
\end{longtable}

\clearpage

\subsection{Grámatica de la Lengua de Señas Mexicana}
La gramática estudia cómo se conectan los elementos de una lengua para crear oraciones con sentido. En la Lengua de Señas Mexicana (LSM), esa estructura no depende de sonidos ni palabras habladas, sino del uso visual del cuerpo, el espacio y los movimientos \cite{ref37}.\\

La LSM se desarrolla en un espacio frente al cuerpo, dividido en tres zonas principales \cite{ref37}:

\begin{itemize}
    \item Una línea vertical que va desde la cintura hasta la parte superior de la cabeza.
    \item Un límite horizontal, que se extiende hasta los codos con los brazos en ángulo.
    \item Un tercer límite que marca qué tan lejos están las manos del cuerpo.
\end{itemize}

Si la seña se realiza fuera de estos límites, suele entenderse como un énfasis o exageración del mensaje.\\

A diferencia del español, la LSM no se basa en sonidos, sino en aspectos visuales y espaciales. Las señas suelen representar ideas complejas, como si fueran palabras individuales con sentido propio. Estas señas se consideran morfemas libres, ya que no necesitan agregarse a otras ni modificarse con terminaciones \cite{ref38}.\\

En esta lengua, no se usan frecuentemente sufijos y prefijos para cambiar el significado. En su lugar, expresiones faciales, movimientos de cabeza o del cuerpo ayudan a matizar lo que se dice. Estos gestos pueden aportar información como si la acción se repite, si está terminada, si es deseada, obligatoria, o si es posible \cite{ref38}.\\

Las señas no tienen una categoría gramatical fija. Una misma seña puede funcionar como verbo, adjetivo o sustantivo, dependiendo del contexto en que se use. Esta flexibilidad se debe a un fenómeno llamado prototipicidad, que permite a ciertas formas adaptarse a distintas funciones según la necesidad \cite{ref38}.\\

Existen señas que siguen patrones más estables, como los verbos direccionales, que cambian su movimiento para indicar quién realiza una acción y hacia quién va dirigida. Sin embargo, estas señas también pueden usarse en otros contextos sin perder su significado \cite{ref38}.\\

Las oraciones en LSM se forman con señas que expresan acciones, participantes, tiempo, condiciones o características. En oraciones simples, las señas que representan sujetos u objetos se comportan como nombres, aunque no siempre sean sustantivos. Las acciones o ideas principales se representan con señas que funcionan como predicados \cite{ref38}.\\

También es común ver señas que expresan pronombres, ubicaciones o tiempos. A veces se usa el deletreo dactilológico o nombres propios, los cuales pueden repetirse al final de la oración como una forma de remarcar la información, lo que algunos estudios llaman “etiquetado” o “tags” \cite{ref38}.\\

\begin{center}
    \includegraphics[width=0.9\textwidth]{Images/Cap 2/Estructura_gramatica_LSM.png}
    \captionof{figure}[Estructura gramatical de una oración de LSM]{Estructura gramatical de una oración de LSM, obtenido de \cite{ref38}.}  % Pie de foto manual
\end{center}

\textbf{Fonología}\\
En las lenguas de señas, los fonemas, las unidades mínimas con significado, pueden descomponerse en siete componentes esenciales \cite{ref39}:

\begin{enumerate}
    \item \textbf{Configuración manual}: es la forma específica que adopta la mano al ejecutar un signo determinado.
    \item \textbf{Orientación de la mano}: hace referencia a la dirección en la que se posiciona la palma, ya sea orientada hacia arriba, abajo o frente al emisor.
    \item \textbf{Zona de ejecución}: indica la parte del cuerpo en la que se realiza el signo, como por ejemplo la frente, la boca, el pecho o los hombros.
    \item \textbf{Desplazamiento}: describe el tipo de movimiento que se lleva a cabo con las manos al hacer un signo; este puede ser giratorio, lineal, en vaivén o segmentado.
    \item \textbf{Área de contacto}: se refiere a la parte de la mano dominante (la derecha para personas diestras o la izquierda para personas zurdas) que entra en contacto con el cuerpo. Puede involucrar la palma, las yemas o el dorso de los dedos.
    \item \textbf{Plano de producción}: se trata de la distancia entre el cuerpo y el lugar donde se articula el signo. El Plano 1 está en contacto directo con el cuerpo, mientras que el Plano 4 se encuentra más alejado, con los brazos completamente extendidos.
    \item \textbf{Elementos corporales no manuales}: son señales complementarias que refuerzan el mensaje, como expresiones faciales, movimientos del torso, gesticulaciones orales o el uso del cuello y los hombros. Por ejemplo, para comunicar una acción futura se inclina el cuerpo hacia delante, y para indicar el pasado, hacia atrás.\\
\end{enumerate}

\textbf{La Configuración Manual (CM) en la LSM}\\
\label{sec:config_manual}
En las lenguas de señas, las manos son las principales herramientas para comunicar, aunque no son las únicas. Además de considerar la dirección del movimiento y el lugar en el espacio donde se hace una seña, también es esencial observar la forma que toman las manos, conocida como configuración manual (CM). Esta configuración puede variar tanto en la mano dominante como en la no dominante \cite{ref37}.\\

La configuración manual, entonces, representa la forma específica que adoptan las manos al momento de hacer una seña. Esto incluye aspectos como \cite{ref37}:

\begin{itemize}
    \item \textbf{La posición de los dedos}: si están juntos o separados, doblados o rectos.
    \item \textbf{La forma general de la mano}: abierta, en puño, en forma de garra, etc.
    \item \textbf{La ubicación del pulgar y del índice}: suelen tener movimientos propios.
\end{itemize}

Desde un punto de vista técnico, la configuración manual forma parte de lo que se llama la matriz articulatoria. Dentro de ella, se distinguen dos grupos importantes \cite{ref37}:

\begin{itemize}
    \item Los dedos (índice, medio, anular y meñique), que suelen moverse como bloque.
    \item El pulgar, que, por su movilidad más independiente, se analiza aparte.
\end{itemize}

Debido a todas las combinaciones posibles entre estos elementos, las configuraciones manuales no pueden reducirse a formas simples, sino que son estructuras complejas que generan significado cuando se combinan con otros componentes de la seña.

\newpage

\textbf{Orientación de la Palma de la Mano}\\
Se refiere a la dirección en la que se encuentra la palma de la mano en relación con el cuerpo de la persona que está haciendo la seña, justo en el momento en que adopta la configuración manual \cite{ref37}.\\

En la Lengua de Señas Mexicana (LSM), se han identificado nueve posibles formas de orientar la palma durante la articulación de una seña \cite{ref37}. Estas son:

\begin{enumerate}
    \item Palma hacia arriba, con los dedos apuntando a la izquierda.
    \item Palma hacia arriba, con los dedos apuntando hacia el frente.
    \item Palma hacia abajo, con los dedos dirigidos a la izquierda.
    \item Palma hacia abajo, con los dedos apuntando hacia adelante.
    \item Palma hacia la izquierda, con los dedos hacia arriba.
    \item Palma hacia la izquierda, con los dedos hacia el frente.
    \item Palma hacia el frente, con los dedos señalando hacia arriba.
    \item Palma frente al cuerpo, dedos hacia arriba.
    \item Palma frente al cuerpo, dedos hacia la izquierda.
\end{enumerate}

\begin{center}
    \includegraphics[width=0.7\textwidth]{Images/Cap 2/Orientacion_Palma_Mano.png}
    \captionof{figure}[Orientaciones de la palma de la mano en LSM]{Orientaciones de la palma de la mano en LSM, obtenido de \cite{ref37}.}  % Pie de foto manual
\end{center}

Cada una de estas orientaciones forma parte de la estructura visual y espacial de una seña, y su correcta ejecución es clave para transmitir el significado deseado.\\

\textbf{Ubicación en la Lengua de Señas Mexicana (LSM)}\\
Hace referencia al lugar específico en el espacio donde se realiza una seña. Este espacio, conocido como espacio señante, puede estar frente al cuerpo o sobre él y es clave para transmitir el significado correcto \cite{ref37}.\\

Para describir las señas en este diccionario, se dividió el espacio señante principalmente en niveles de altura y direcciones laterales. En algunos casos, también se toma en cuenta una tercera dimensión, que implica mayor cercanía o profundidad \cite{ref37}.\\

Cuando las señas se hacen sobre el cuerpo, se habla de “alturas” específicas, como por ejemplo:
\begin{itemize}
    \item A la altura del cuello.
    \item A la altura del hombro.
    \item A la altura del pecho.
    \item A la altura del plexo.
    \item A la altura de la cintura.
    \item A la altura de la cadera.
\end{itemize}

Si la seña se mueve entre dos puntos, se describe como un desplazamiento, por ejemplo:

\begin{itemize}
    \item Del pecho a la cintura.
    \item Del hombro a la cadera.
    \item Del cuello a la cadera.
\end{itemize}

Cuando el movimiento es horizontal o de un lado a otro, también se aclara, por ejemplo:
\begin{itemize}
    \item A la altura del pecho, de izquierda a derecha.
\end{itemize}

En el caso de las señas realizadas en la cara, se puede ser más preciso indicando zonas como:
\begin{itemize}
    \item A la altura de los ojos.
    \item A la altura de las cejas.    
\end{itemize}

Finalmente, si la seña se hace sobre el tronco del cuerpo, se especifica si es:
\begin{itemize}
    \item Del lado izquierdo.
    \item Del lado derecho.
    \item Al centro.
\end{itemize}

Aunque algunas ubicaciones pueden detallarse aún más, se busca usar una descripción unificada para facilitar la comprensión \cite{ref37}.\\

\textbf{Dirección del Movimiento de la mano en la LSM}\\
Es la trayectoria que la mano sigue al realizar una seña \cite{ref37}. En la \autoref{direcciones_Mano} se muestran las nueve direcciones que las configuraciones manuales pueden seguir durante la articulación de las señas respecto al cuerpo.

\begin{center}
    \includegraphics[width=0.8\textwidth]{Images/Cap 2/Direccion_Manos_LSM.png}
    \captionof{figure}[Direcciones posibles que sigue la mano en la LSM]{Direcciones posibles que sigue la mano en la LSM, obtenido de \cite{ref37}.} 
    \label{direcciones_Mano}
\end{center}

\newpage
\textbf{Explicación de los movimientos y sus símbolos}\\
La \autoref{tabla:Movimiento_LSM} muestra todos los movimientos que las manos pueden realizar en la LSM. En la primera columna se menciona el nombre del movimiento; en la segunda, la descripción del mismo y en la tercera aparecen flechas o imágenes de la mano para indicar la dirección o la manera en que las configuraciones manuales se mueven.\\

\begin{longtable}{|m{5cm}|m{5cm}|m{5cm}|}
    \hline
    \textbf{Movimiento} & \textbf{Descripción del movimiento} & \textbf{Imagen} \\
    \hline
    \endfirsthead
    
    \hline
    \textbf{Movimiento} & \textbf{Descripción del movimiento} & \textbf{Imagen} \\
    \hline
    \endhead
    
    \hline
    \endfoot
    
    \endlastfoot
    
     & Se emplea una numeración progresiva para señalar cómo cambian las formas de las manos y los desplazamientos en las señas que combinan varios movimientos.
    & \makecell{\colorbox{white}{\includegraphics[width=4cm]{Images/Cap 2/Movimientos LSM/1.png}}} \\
    \hline

    Lineal (lin) & Movimiento rectilíneo.
    & \makecell{\colorbox{white}{\includegraphics[width=4cm]{Images/Cap 2/Movimientos LSM/2.png}}} \\
    \hline

    Arco (ar) & El desplazamiento del brazo, la muñeca o la mano dibuja una curva en forma de arco.
    & \makecell{\colorbox{white}{\includegraphics[width=4cm]{Images/Cap 2/Movimientos LSM/3.png}}} \\
    \hline

    Extensión de dedos (E) & Los dedos se extienden.
    & \makecell{\colorbox{white}{\includegraphics[width=4cm]{Images/Cap 2/Movimientos LSM/4.png}}} \\
    \hline

    Vaivén (va) & Se realiza un movimiento intercalado entre ambas manos o brazos.
    & \makecell{\colorbox{white}{\includegraphics[width=4cm]{Images/Cap 2/Movimientos LSM/5.png}}} \\
    \hline
    
    Circular (circ) & La trayectoria de la mano, muñeca o brazo describe movimientos circulares o semicirculares.
    & \makecell{\colorbox{white}{\includegraphics[width=4cm]{Images/Cap 2/Movimientos LSM/6.png}}} \\
    \hline

    Espiral (es) & La mano o el brazo giran siguiendo un patrón redondeado.
    & \makecell{\colorbox{white}{\includegraphics[width=4cm]{Images/Cap 2/Movimientos LSM/7.png}}} \\
    \hline

    Flexión de dedos (f) & Los dedos se retraen.
    & \makecell{\colorbox{white}{\includegraphics[width=4cm]{Images/Cap 2/Movimientos LSM/8.png}}} \\
    \hline

    Ondular (ond) & El movimiento de la mano o el brazo imita una forma de onda.
    & \makecell{\colorbox{white}{\includegraphics[width=4cm]{Images/Cap 2/Movimientos LSM/9.png}}} \\
    \hline
    
    Salto & La mano o los dedos simulan uno o varios saltos.
    & \makecell{\colorbox{white}{\includegraphics[width=4cm]{Images/Cap 2/Movimientos LSM/10.png}}} \\
    \hline

    Movimiento vibratorio local (vib) & La mano tiembla.
    & \makecell{\colorbox{white}{\includegraphics[width=4cm]{Images/Cap 2/Movimientos LSM/11.png}}} \\
    \hline
    
    Cabeceo de muñeca (cab) & El movimiento parte desde la parte posterior y avanza al frente, usando solo el giro de la muñeca.
    & \makecell{\colorbox{white}{\includegraphics[width=4cm]{Images/Cap 2/Movimientos LSM/12.png}}} \\
    \hline
    
    Aplanado (apl) & Se realiza un contacto breve entre el índice y medio, o el índice y el pulgar, seguido de una separación.
    & \makecell{\colorbox{white}{\includegraphics[width=4cm]{Images/Cap 2/Movimientos LSM/13.png}}} \\
    \hline

    Apulgarado (p) & El índice o medio se libera con impulso del pulgar, pasando de estar doblado a completamente recto.
    & \makecell{\colorbox{white}{\includegraphics[width=4cm]{Images/Cap 2/Movimientos LSM/14.png}}} \\
    \hline

    Cambios progresivos en los dedos (prog) & Los dedos se mueven uno a uno de forma intercalada.
    & \makecell{\colorbox{white}{\includegraphics[width=4cm]{Images/Cap 2/Movimientos LSM/15.png}}} \\
    \hline

    Deslizamiento (desl) & Se realiza un movimiento deslizante de los dedos sobre la superficie del pulgar.
    & \makecell{\colorbox{white}{\includegraphics[width=4cm]{Images/Cap 2/Movimientos LSM/16.png}}} \\
    \hline

    Zig-zag (zig) & El índice traza en el aire la forma de la letra Z.
    & \makecell{\colorbox{white}{\includegraphics[width=4cm]{Images/Cap 2/Movimientos LSM/17.png}}} \\
    \hline
    
    Siete (7) & Se realiza un movimiento que dibuja el número siete en el espacio.
    & \makecell{\colorbox{white}{\includegraphics[width=4cm]{Images/Cap 2/Movimientos LSM/18.png}}} \\
    \hline

    Rotación de muñeca (rot) & La rotación del antebrazo o la muñeca provoca que la mano cambie su dirección.
    & \makecell{\colorbox{white}{\includegraphics[width=4cm]{Images/Cap 2/Movimientos LSM/19.png}}} \\
    \hline
    
    Choque (ch) & Las manos se encuentran y se tocan.
    & \makecell{\colorbox{white}{\includegraphics[width=4cm]{Images/Cap 2/Movimientos LSM/20.png}}} \\
    \hline
    
    Doblar (dob) & Mientras el pulgar permanece quieto, los demás dedos se doblan hacia el centro de la mano.
    & \makecell{\colorbox{white}{\includegraphics[width=4cm]{Images/Cap 2/Movimientos LSM/21.png}}} \\
    \hline
    
    Cruzado (crz) & Los brazos se mueven hacia el centro cruzándose, y las manos se aproximan en un punto medio.
    & \makecell{\colorbox{white}{\includegraphics[width=4cm]{Images/Cap 2/Movimientos LSM/22.png}}} \\
    \hline
    
    Simétrico (sim) & Desde una posición inicial común, las manos se separan hacia diferentes direcciones: superior, inferior o lateral.
    & \makecell{\colorbox{white}{\includegraphics[width=4cm]{Images/Cap 2/Movimientos LSM/23.png}}} \\
    \hline

    Prensar & El índice y el pulgar realizan un gesto de pinza al agarrar otra mano o una zona del cuerpo.
    & \makecell{\colorbox{white}{\includegraphics[width=4cm]{Images/Cap 2/Movimientos LSM/24.png}}} \\
    \hline

    \caption[Movimientos de las manos]{Movimientos de las manos, obtenido de \cite{ref37}.} \label{tabla:Movimiento_LSM} \\
\end{longtable}

\textbf{Rasgos no manuales: expresión facial y gestos}\\
Los rasgos no manuales (RNM) en la Lengua de Señas Mexicana incluyen la expresión facial, los gestos y los movimientos del cuerpo \cite{ref37}. Estos elementos se realizan al mismo tiempo que las señas y tienen una función gramatical clave, ya que aportan significado al mensaje, similar a cómo el tono de voz o la velocidad lo hacen en el español hablado \cite{ref37}.\\

También existen gestos universales que no dependen del idioma o la cultura, como los que expresan felicidad, tristeza, dolor o alegría y que se entienden en cualquier parte del mundo \cite{ref37}.\\

\textbf{Tipos de Señas en la LSM}\\
En la Lengua de Señas Mexicana (LSM), las señas se pueden clasificar de diferentes maneras, principalmente según cuántas manos se usan y cómo se mueven \cite{ref37}:

\begin{itemize}
    \item \textbf{Seña manual (SM)}: se realiza con una sola mano.
    \item \textbf{Seña bimanual (SB)}: usa ambas manos, pero no necesariamente hacen lo mismo; puede haber movimientos distintos entre una y otra.
    \item \textbf{Seña simétrica (SS)}: ambas manos se mueven al mismo tiempo con movimientos similares o en espejo (como si se reflejaran una a la otra).
    \item \textbf{Seña compuesta (SC)}: se forma a partir de dos o más señas simples o al menos tres formas diferentes de las manos.
\end{itemize}

\newpage
Además, según la relación entre la seña y su significado, también se pueden clasificar así \cite{ref37}:
\begin{itemize}
    \item \textbf{Icónicas}: estas señas imitan la forma o alguna característica del objeto al que se refieren. Por ejemplo, la seña de “árbol” muestra cómo es un árbol.
    \item \textbf{Referenciales en el cuerpo}: Algunas señas se hacen en partes del cuerpo relacionadas con el objeto, como la seña de “manzana”, que se realiza en la mejilla.
    \item \textbf{Arbitrarias}: no guardan ninguna relación visual con el objeto o concepto. Ejemplos de estas son las señas de “gracias” u “oportunidad”.
    \item \textbf{Inicializadas o alfabéticas}: se utilizan las letras del alfabeto manual, generalmente la inicial de la palabra en español, como en “mamá” o “alumno”.
    \item \textbf{Indéxicas}: estas señas señalan un lugar, persona u objeto en el espacio, como los pronombres: yo, tú, él, ella, allá, aquí, etc.
    \item \textbf{Numéricas}: son aquellas donde la forma de la mano representa un número, y se usan para nombrar cosas como países o acciones. Ejemplos: “Dinamarca” (con el número 8), “mujer” (número 1), “abanico” (número 4) y “atención” (número 6).\\
\end{itemize}

\textbf{Clasificadores en la LSM}\\
En la LSM, los clasificadores son señas especiales que permiten describir mejor las características de un objeto, las cuales combinan dos elementos: uno que indica a qué tipo de objeto se hace referencia (como persona, animal, cosa) y otro que muestra sus propiedades, como forma, tamaño o cómo se mueve \cite{ref37}. Son muy útiles cuando no existe una seña exacta para algo y ayudan a transmitir el mensaje de manera visual y clara\\

Están basados en la configuración manual (CM), es decir, en la forma en la que se colocan y usan las manos. A través de estas configuraciones, se puede representar si un objeto es redondo, plano, grande, pequeño, blando, rígido, entre otros rasgos y, de igual manera, pueden indicar su ubicación, cantidad u orden \cite{ref37}.\\

Además, existen clasificadores de predicado, que no solo describen el objeto, sino que también aportan información sobre su movimiento, posición o estado. Estos se combinan con “raíces de movimiento” para señalar si algo se mueve, está quieto o entra en contacto con algo \cite{ref37}. Las raíces de movimiento pueden ser:
\begin{itemize}
    \item Movimiento o proceso.
    \item Descripción estática.
    \item Contacto con otro objeto.
\end{itemize}

Los clasificadores también se organizan en diferentes tipos, dependiendo de lo que representan \cite{ref37}:
\begin{itemize}
    \item Clasificadores de entidad (personas, animales, cosas).
    \item De superficie (planos, mesas, pisos).
    \item De profundidad y anchura (algo profundo o ancho).
    \item De extensión o límites (como bordes o extremos).
    \item De perímetro (formas cerradas).
    \item De instrumento (objetos usados para realizar acciones).
\end{itemize}

\textbf{Afijos}\\
De acuerdo con la definición de la Real Academia Española \cite{refrae}, los afijos son morfemas que se agregan a la raíz de una palabra para alterar su sentido o función gramatical. No obstante, en la Lengua de Señas, el uso de afijos no es común como lo es en las lenguas orales \cite{ref37}.\\

En las lenguas habladas, los afijos como prefijos y sufijos son parte esencial de la estructura de muchas palabras. En cambio, en la Lengua de Señas, su presencia es limitada y aparece sobre todo en situaciones en las que interviene el español escrito \cite{ref37}. Generalmente, son personas oyentes que conocen la LSM quienes tienden a emplear estos elementos con mayor frecuencia, adaptando la estructura del español al sistema visual y gestual de la lengua de señas.\\

\textbf{Prefijos}\\
En la Lengua de Señas Mexicana (LSM), los prefijos son elementos que se colocan al inicio de una seña para agregar información como el tiempo, el género o el número \cite{ref37}. Aunque en las lenguas orales son comunes, en LSM su uso es más limitado y, en muchos casos, surge por influencia del español. Actualmente, se conservan pocos prefijos, como por ejemplo “in-” o “im-”, que se representan con una “I” hecha por la mano dominante al tocar la mano base \cite{ref37}.\\

Para expresar el género femenino, se utiliza una seña específica que aparece después de indicar el masculino. El número (singular o plural) suele marcarse antes del sustantivo, y si se quiere enfatizar, puede colocarse también después \cite{ref37}.\\

El tiempo del mensaje se indica al inicio, ya sea mediante una seña específica o con movimientos del cuerpo: inclinarse hacia adelante para el futuro, o hacia atrás para el pasado \cite{ref37}.\\

En cuanto a la negación, existen señas conocidas como dobletes, que tienen una forma afirmativa y otra negativa completamente distinta, sin necesidad de añadir gestos como mover la cabeza \cite{ref37}. Algunos ejemplos son:
\begin{itemize}
    \item Gustar / No gustar.
    \item Poder / No poder.
    \item Saber / No saber.
    \item Todavía / Todavía no.
    \item Querer / No querer.
    \item Haber / No haber.
    \item Sirve / No sirve.
\end{itemize}

\textbf{Sufijos}\\
Son pequeñas unidades de significado que se colocan después de una seña para dar más precisión o detalle al mensaje. Su uso en la Lengua de Señas Mexicana (LSM) se ha visto influenciado principalmente por el idioma español, ya que en las lenguas orales es común agregar estas terminaciones a las palabras \cite{ref37}.\\

En el pasado, los sufijos más frecuentes en LSM eran “-ción” y “-mente”, ya que ayudan a formar sustantivos abstractos o adverbios \cite{ref37}. Sin embargo, con el tiempo han dejado de usarse tanto y actualmente los sufijos que se siguen utilizando con mayor frecuencia son:

\begin{itemize}
    \item -ito / -ita, que indican diminutivo o cercanía con afecto.
    \item -al, usado para formar adjetivos relacionados a un lugar o cosa.
    \item -or / -ora, que hacen referencia a profesiones o a quien realiza una acción.
    \item -dad, que se usa para formar conceptos abstractos, como en “amistad” o “bondad”.
\end{itemize}

\newpage
\subsection{Dactilología}
La dactilología es un sistema de comunicación que transmite información mediante el deletreo manual, y en ocasiones es usado en conjunto con la lengua de señas. Se emplea la mano de diferente manera para pronunciar cada una de las letras \cite{ref30}.\\

Otra definición de la dactilología es que es la representación manual de cada una de las letras que componen el alfabeto para poder transmitir a las personas sordas cualquier palabra que se desee comunicar. Todas las lenguas de señas poseen mecanismos internos que les permiten generar mensajes \cite{ref40}.\\

Para comunicarse por medio de dactilología se emplea la mano dominante a la altura de la barbilla, en conjunto con la articulación oral, siendo necesario que la cara y la boca sean visibles \cite{ref40}. Principalmente se usa para sustantivos, nombres propios, direcciones y palabras para los cuales no existe un signo creado.\\

Si bien la discapacidad auditiva representa una barrera de la comunicación, las personas sordas en los últimos años han buscado superar esa barrera con ayuda de dispositivos tecnológicos que puedan fungir como intérpretes. El desarrollo de la Inteligencia Artificial (IA), más concretamente las técnicas de Procesamiento de Lenguaje Natural (PLN), Visión Artificial, \textit{Deep Learning} y modelado de animaciones 3D (ver \textbf{\autoref{sec:edoArte}}), han ayudado a crear nuevos sistemas que faciliten la interacción entre personas oyentes y personas de la comunidad sorda, derribando las barreras de la comunicación.\\

\subsection{Diferencia entre Traducción e Interpretación}
La traducción es el proceso de convertir textos escritos de un idioma a otro, conservando el sentido, la intención y el tono del mensaje original. La misma se realiza sobre materiales tangibles: documentos, contratos, libros, artículos, guiones, entre otros. El objetivo es transcribir por escrito un mensaje de una lengua origen a una lengua meta, conservando su significado, estilo y contexto \cite{reftradint}. \\

La traducción implica un proceso cognitivo complejo que requiere no sólo la sustitución lingüística, sino también la comprensión del contexto cultural y semántico en el que se enmarca el mensaje original. En el ámbito computacional, este proceso es llevado a cabo por sistemas de traducción automática que analizan la estructura lingüística, la semántica y las relaciones gramaticales del texto fuente, empleando modelos estadísticos o de aprendizaje profundo para generar un texto equivalente en el idioma destino \cite{reftradint2}.\\

Por otro lado, la interpretación es la conversión del mensaje oral de un orador de una lengua hablada a otra, la cual trabaja con el discurso oral, y el proceso se realiza en tiempo real. Por lo tanto, el intérprete debe reaccionar rápidamente, sin margen de corrección ni consulta \cite{reftradint}.\\

La interpretación requiere habilidades cognitivas diferentes a las de la traducción, ya que el intérprete debe procesar simultáneamente la comprensión y la reformulación del mensaje, implicando una alta capacidad de concentración y dominio de ambas lenguas en contextos culturales diversos. A diferencia de la traducción donde existe la posibilidad de revisar el resultado en materiales de consulta, la interpretación se caracteriza por su inmediatez y espontaneidad, lo que la hace esencial en entornos como conferencias, juicios o emergencias médicas \cite{reftradint2}.\\ 

En el caso de la Lengua de Señas Mexicana (LSM), los intérpretes desempeñan un papel fundamental al facilitar la comunicación en espacios presenciales, mientras que un sistema automatizado como el propuesto en este trabajo realiza un proceso de traducción escrita a representación visual animada.\\

La LSM, al ser una lengua visogestual, no puede traducirse de manera literal palabra por palabra, sino que requiere un proceso de adaptación sintáctica y semántica. Por ello, los sistemas automáticos deben incorporar etapas de análisis lingüístico y modelado computacional que permitan representar la estructura y el contexto de la oración en español dentro del marco gramatical propio de la LSM.\\

El desarrollo de este tipo de sistemas contribuye al campo de la accesibilidad tecnológica al proporcionar alternativas comunicativas que promueven la inclusión social. El uso de técnicas de Procesamiento de Lenguaje Natural (PLN) y animaciones permite automatizar la traducción escrita a visual, garantizando resultados comprensibles y gramaticalmente coherentes con las reglas lingüísticas de la LSM. Esta distinción resulta fundamental para entender el alcance del proyecto, centrado en la traducción automatizada y no en la interpretación en tiempo real mediante medios humanos.\\

Este prototipo no pretende reemplazar el trabajo de los intérpretes humanos, sino complementar sus funciones mediante herramientas digitales que permitan traducir texto escrito en español a representaciones visuales animadas en LSM. El proyecto se enmarca dentro del campo de la traducción asistida por computadora, en la cual el procesamiento previo del texto constituye una fase esencial para transformar el lenguaje natural en representaciones visuales comprensibles. Además, si en el futuro el sistema logra procesar voz o señas en tiempo real, podría evolucionar hacia un sistema de interpretación automática, ampliando su alcance y abriendo una nueva línea de investigación en accesibilidad e inteligencia artificial aplicada a las lenguas de señas.\\

En los siguientes apartados se analizarán un par de herramientas que serán necesarias para el desarrollo del prototipo planteado en el capítulo 1 (ver \textbf{\autoref{sec:Intro}}), como lo es la Inteligencia Artificial (IA) y el Procesamiento de Lenguaje Natural (PLN).\\

\section{Inteligencia Artificial}
La Inteligencia Artificial (IA) es la capacidad que poseen las máquinas para usar algoritmos y aprender de los datos para tomar decisiones tal como lo haría un ser humano \cite{ref41}. A diferencia del ser humano, la IA no necesita descansar y es capaz de analizar grandes cantidades de información, reduciendo el margen de error.\\

La IA se basa en el uso de algoritmos y tecnologías de aprendizaje automático para dar a las máquinas la capacidad de aplicar ciertas habilidades cognitivas y realizar tareas por sí mismas de manera autónoma o semiautónoma. A medida que la IA mejora, muchos procesos son más eficientes y algunas tareas que parecían complicadas se realizan con mayor rapidez y precisión \cite{ref42}.\\

\subsection{Clasificación de la Inteligencia Artificial}
La IA puede ser clasificada de varias maneras, ya sea a partir de su grado de capacidad cognitiva o a partir de su grado de autonomía \cite{ref42}.\\

\textbf{Clasificación a partir de su grado de capacidad cognitiva:}
\begin{itemize}
    \item \textbf{Inteligencia Artificial débil o limitada}: está diseñada para realizar tareas específicas de manera eficiente, pero no tiene la capacidad de razonar ni aprender de nuevas situaciones \cite{ref42}.\\
    \item \textbf{Inteligencia Artificial general o fuerte}: este tipo de IA tiene la capacidad de realizar varias tareas cognitivas como el razonamiento, el aprendizaje y la resolución de problemas. A diferencia de la IA débil, la IA fuerte es capaz de adaptarse a nuevas situaciones y entornos \cite{ref42}.\\
    \item \textbf{Super Inteligencia Artificial}: tiene la capacidad de realizar cualquier tarea compleja que requiere Inteligencia Humana, ya que es muy poderosa, y puede superar a los seres humanos en términos de capacidad cognitiva y de aprendizaje \cite{ref42}.\\
\end{itemize}

\textbf{Clasificación de acuerdo con su grado de autonomía:}

\begin{itemize}
    \item \textbf{Inteligencia Artificial Reactiva}: este tipo de IA realiza tareas específicas de manera autónoma, pero no tiene la capacidad de recordar eventos pasados ni de anticipar situaciones futuras. Es útil en situaciones en las que se requieren respuestas rápidas y precisas a situaciones específicas \cite{ref42}.\\
    \item \textbf{Inteligencia Artificial Deliberativa}: tiene la capacidad de planificar y tomar decisiones basándose en información del entorno y en objetivos predeterminados. Es decir, puede analizar situaciones y elegir opciones que le permitan cumplir con objetivos, o adaptarse a entornos empleando información del pasado y del futuro \cite{ref42}.\\
    \item \textbf{Inteligencia Artificial Cognitiva}: se caracteriza por su capacidad de imitar las funciones cognitivas humanas como lo son el razonamiento, la percepción y el aprendizaje, y tienen la capacidad de adaptarse a nuevas situaciones y entornos \cite{ref42}.\\
    \item \textbf{Inteligencia Artificial Autónoma}: es capaz de interactuar de manera autónoma con su entorno, tomar decisiones y aprender de nuevas situaciones, y cambiar sus objetivos y estrategias en función de las estrategias sin la necesidad de la intervención humana \cite{ref42}.
\end{itemize}

De igual manera, la IA emplea diferentes técnicas, las cuales se enlistan a continuación \cite{ref43}:

\begin{itemize}
    \item \textbf{Búsqueda de soluciones}: esta técnica tiene por objetivo encontrar mecanismos de deducción y búsqueda de soluciones para la resolución de problemas cuando no se cuenta con un método directo \cite{ref43}.\\
    \item \textbf{Representación del conocimiento}: elaboración de métodos y técnicas eficientes que sean capaces de organizar conocimientos en un sistema, para posteriormente ser usados en la búsqueda de soluciones para diferentes problemáticas \cite{ref43}.\\
    \item \textbf{Reconocimiento de patrones}: son técnicas de clasificación para identificar subgrupos midiendo el parecido o similitud entre formas, con el objetivo de obtener conclusiones \cite{ref43}.\\
    \item \textbf{Robótica}: esta técnica tiene por objetivo la construcción de robots inteligentes capaces de funcionar con autonomía, que cuenten con la habilidad de realizar procesos mecánicos y manuales con el fin de obtener mayor productividad, suplir mano de obra y proporcionalidad flexibilidad en procesos industriales \cite{ref43}.\\
    \item \textbf{Redes Neuronales}: son sistemas compuestos por estructuras de red, con un gran número de conexiones entre diferentes capas de procesadores, que a su vez tienen asignadas diferentes funciones. Las redes neuronales efectúan una labor de aprendizaje por la reproducción de las salidas de un conjunto de entrenamiento \cite{ref43}.\\
    \item \textbf{Algoritmos genéticos}: son los tipos de algoritmos que tratan de emular el proceso de selección natural a un problema dado, en el que se aplican operadores genéticos para evaluar cada una de las soluciones propuestas. Se emplean procedimientos de búsqueda y optimización para mejorar las soluciones existentes y generar nuevas \cite{ref43}.\\
    \item \textbf{Sistemas expertos}: sistemas que almacenan conocimientos de expertos sobre un área o campo especializado, para obtener una solución mediante una deducción lógica \cite{ref43}. \\
    \item \textbf{Procesamiento de Lenguaje Natural (PLN)}: se centra en el diseño de métodos y algoritmos que toman como entrada o producen como salida datos en la forma del lenguaje humano, ya sea en forma de texto, audio o animación \cite{ref44}.
\end{itemize}

En este Trabajo Terminal nos centraremos en la técnica de Procesamiento de Lenguaje Natural (PLN). En el siguiente apartado se profundizará más en el concepto, características y usos del PLN.\\

\section{Procesamiento de Lenguaje Natural (PLN)}
El Procesamiento de Lenguaje Natural (PLN, o NLP por sus siglas en inglés) es el campo de estudio que busca entender cómo funciona el lenguaje, su construcción, la generación de nuevo lenguaje, así como todas las tareas que tienen relación con el tratamiento del lenguaje como lo es la generación de texto, traductores, generadores de resúmenes, chatbots, entre otros \cite{ref45}.\\

El PLN emplea el lenguaje natural para establecer comunicación entre un ser humano y una computadora. Esta última deberá entender las oraciones que le sean proporcionadas mediante modelos que le ayuden a entender los mecanismos humanos relacionados con el lenguaje \cite{ref46}. 

\subsection{Arquitectura de un sistema de PLN}

La arquitectura de un sistema de PLN está dividida en los siguientes niveles \cite{ref47}:\

\begin{enumerate}[label=\alph*.]
    \item \textbf{Nivel Fonético}: en este nivel se interpretan los sonidos dentro de las palabras.
    \item \textbf{Nivel Fonémico}: se trabajan con los fonemas, los cuales son unidades teóricas básicas para estudiar el nivel fonológico de la lengua humana, ya que analizan la varianza en la pronunciación cuando las palabras están conectadas.
    \item \textbf{Nivel Morfológico}: indica cómo es que las palabras se construyen a partir de unidades de significado más pequeñas, llamadas morfemas.
    \item \textbf{Nivel Léxico}: se encarga del significado individual de cada palabra, analizando cada una de las palabras para conocer su significado y función dentro de una oración, tomando en cuenta el contexto en el que se encuentre.
    \item \textbf{Nivel Sintáctico}: se analiza cómo es que las palabras se unen para formar oraciones, entendiendo la función estructural que cada palabra posee.
    \item \textbf{Nivel Semántico}: se refiere al significado de las palabras, y cómo los mismos se unen para darle sentido a una oración, considerando también el contexto de la oración.
    \item \textbf{Nivel de Discurso}: se encarga de trabajar con unidades de texto grandes, haciendo conexiones entre las oraciones. Se identifica la función que cumple cada oración en el texto, sumando información al significado del texto completo.
    \item \textbf{Nivel Pragmático}: trata de cómo las oraciones son empleadas en diferentes situaciones y cómo es que el uso afecta el significado de las mismas.
\end{enumerate}

\begin{center}
    \includegraphics[width=0.8\textwidth]{Images/Cap 2/Niveles_Arquitectura_PLN.png}
    \captionof{figure}[Niveles de la arquitectura de un Sistema de Procesamiento de Lenguaje Natural]{Niveles de la arquitectura de un Sistema de Procesamiento de Lenguaje Natural, obtenido de \cite{ref46}.}  % Pie de foto manual
\end{center}

\textbf{Los pasos que sigue la arquitectura del sistema de PLN son los siguientes \cite{ref46}:}
\begin{enumerate}
    \item El usuario le expresa a la computadora lo que desea hacer.\\
    \item La computadora analiza las oraciones que el usuario le proporciona, en el sentido morfológico y sintáctico. En otras palabras, se verifican los componentes léxicos definidos y se verifica si se cumple un orden gramatical entre los elementos identificados.\\
    \item Se realiza un análisis sintáctico de las oraciones, para saber cuál es el significado de cada oración.\\
    \item Después de realizar el paso anterior, se lleva a cabo un análisis pragmático de todas las oraciones juntas. Al final de este paso, la computadora obtiene la expresión final.\\
    \item Una vez obtenida la expresión final, la misma es ejecutada para obtener un resultado que será proporcionado al usuario.
\end{enumerate}

\subsection{Técnicas de PLN}
El PLN se apoya de un conjunto de técnicas mediante las cuales se extrae información determinada de un texto. A continuación, se describen algunas de las técnicas más comunes utilizadas \cite{ref47}:

\begin{enumerate}
    \item \textbf{Detección de oraciones}: esta técnica se encarga de recortar una secuencia de caracteres entre dos signos de puntuación; el signo debe estar acompañado por un espacio en blanco y se excluye el caso de la primer frase y en posibles ocasiones la última frase. Corresponde el nivel de procesamiento sintáctico dentro de la arquitectura de PLN.\\
    
La detección de oraciones puede presentar algunas dificultades a la hora de procesar títulos, abreviaturas, o algunos elementos que no siguen algún patrón de texto plano. En esos casos se emplean bancos de palabras, que incluyen aquellos símbolos o abreviaturas necesarias para detectar las sentencias, y posteriormente son cargadas en el modelo \cite{ref47}.
\begin{center}
    \includegraphics[width=0.8\textwidth]{Images/Cap 2/Deteccion_Oraciones.png}
    \captionof{figure}[Ejemplo de la delimitación de oraciones dentro de un párrafo]{Ejemplo de la delimitación de oraciones dentro de un párrafo, obtenido de \cite{ref47}.}
    \label{tabla_delimitacion_oraciones}
\end{center}
La \autoref{tabla_delimitacion_oraciones} muestra que, en el párrafo, el modelo en español determina que “Sr.” es una abreviatura de la palabra “Señor” y por consiguiente ignora el signo de puntuación como final de la oración.

\item \textbf{Segmentación por palabras}: después de que se identifican cada una de las oraciones que componen un texto, se procede a la segmentación por palabras, más conocida como analizador léxico o “\textit{Tokenizer}”.\\

Esta técnica, perteneciente al nivel léxico, consiste en la identificación de \textit{tokens}, los cuales son unidades lingüísticas como palabras, puntuación, números, caracteres alfanuméricos, etc. Para identificar \textit{tokens} en idiomas modernos, se delimitan espacios en blanco con límites de palabra, entre comillas, paréntesis y puntuación.\\

El trato con las abreviaciones es similar a la detección de oraciones, ya que se emplea una lista de palabras recortadas reconocidas \cite{ref47}.\begin{center}
    \includegraphics[width=0.8\textwidth]{Images/Cap 2/Separacion_Palabras.png}
    \captionof{figure}[Ejemplo de separación de palabras en un párrafo]{Ejemplo de separación de palabras en un párrafo, obtenido de \cite{ref47}.}
    \label{separacion_palabras}
\end{center}
En la \autoref{separacion_palabras} se obtiene la lista de palabras del texto, con la separación por palabras indicada por los espacios en blanco y los signos de puntuación. 

\item \textbf{Etiquetado gramatical o \textit{Part-of-Speech (POS) - tagging}}: el proceso de etiquetado gramatical consiste en asignar la categoría gramatical a cada una de las palabras de un texto, de acuerdo con la definición de esta o el contexto en el que aparece, como lo pueden ser los sustantivos, adjetivos, adverbios, etc. \\

Para lograr lo anterior, es primordial establecer las relaciones de una palabra con sus adyacentes dentro de una frase o de un párrafo. Un mismo \textit{token} puede tener múltiples etiquetas POS, pero solo una es válida dependiendo del contexto.
\begin{center}
	\makebox[\textwidth]{%
		\includegraphics[width=1\textwidth]{Images/Cap 2/POS-tagging.png}
	}
    \captionof{figure}[POS Tagger]{\textit{POS Tagger}, obtenido de \cite{ref47}.}  % Pie de foto manual
\end{center}

\item \textbf{Segmentación morfológica}: en esta etapa, se realiza la identificación de morfemas, que son un fragmento mínimo capaz de expresar el significado de una palabra, es decir, es la unidad significativa más pequeña de un idioma.\\

La identificación de morfemas permite el análisis en profundidad de una palabra en un texto, ya que de esta forma se obtiene información específica como el género, modo, tiempo, etc., y es posible ubicar de manera precisa cada palabra de una oración.\\

Los morfemas se clasifican en 2 categorías. Los morfemas independientes admiten cierta libertad fonológica del lexema: 

\begin{itemize}
    \item \textbf{Pronombres}: cuíde-se, di-le, él, ella.
    \item \textbf{Preposiciones}: desde, a, con, de.
    \item \textbf{Conjunciones}: y, e, o, pero, aunque.
    \item \textbf{Determinantes}: él, ella, ese, un, una.
\end{itemize}

Por otro lado, los morfemas dependientes van unidos a otra unidad mínima dotada de significado, conocidos como monemas, para completar su significado.  Los tipos de morfemas dependientes son: 

\begin{enumerate}
    \item \textbf{Derivativos}: estos morfemas son facultativos, es decir, añaden matices al significado de los lexemas.
    \begin{itemize}
        \item Prefijos.
        \item Sufijos.
        \item Interfijos.
    \end{itemize}
    \item \textbf{Flexivos}: estos morfemas son constitutivos, es decir, señalan relaciones gramaticales y sus accidentes entre los diferentes agentes de una acción verbal o una expresión nominal.
    \begin{itemize}
        \item Género
\item Número.
\item Persona.
\item Modo y tiempo.
    \end{itemize}
\end{enumerate}

\item \textbf{Eliminación de \textit{Stop words}}. Mediante esta técnica, se excluyen palabras comunes que tienen poco valor para la recuperación de información, con el fin de reducir el tamaño de un texto y seleccionar las palabras clave. La cantidad de ocurrencias de una palabra en un texto determina si es o no una “\textit{stop word}”, siendo que cuanto más ocurrencias existan menos relevancia tiene en el texto; en su mayoría, los artículos, los pronombres, las preposiciones y las conjunciones.\\

A partir de un listado de palabras \textit{Stop words}, se hace una busqueda de aquellas palabras con mayor ocurrencia dentro de un texto, para su posible eliminación. En ocasiones, al listado de palabras de uso común se le agrega un conjunto de palabras propias del documento que se analiza, empleando la técnica TF-IDF (\textit{Term Frequency - Inverse Document Frequency}), que permite determinar qué palabras son importantes para un documento de acuerdo con la frecuencia de aparición dentro de un texto.
\begin{center}
	\makebox[\textwidth]{%
		\includegraphics[width=1\textwidth]{Images/Cap 2/Deteccion_Stopwords.png}
	}
    \captionof{figure}[Ejemplo de detección de Stop words]{Ejemplo de Detección de \textit{Stop words}, obtenido de \cite{ref47}.}  % Pie de foto manual
\end{center}

\item \textbf{Reconocimiento de Entidades Nombradas (NER)}: se realiza una busqueda y clasificación de elementos de texto que pertenecen a categorías predefinidas, como lo son nombres de personas, nombres de entidades, organizaciones, lugares, expresiones temporales, cantidades, porcentajes. etc.
Para poder hacer el reconocimiento de las diferentes entidades, se utilizan una serie de aproximaciones, siendo necesario ademas, tener una noción del contexto en el cual se encuentra cada una de las entidades para determinar su significado. Finalmente, dentro de las posibles entidades se realiza una asociación con los conceptos del contexto dentro de una base de datos de conocimiento.
\begin{center}
	\makebox[\textwidth]{%
		\includegraphics[width=1\textwidth]{Images/Cap 2/NER.png}
	}
    \captionof{figure}[Reconocimiento de Entidades Nombradas (NER)]{Reconocimiento de Entidades Nombradas (NER), obtenido de \cite{ref47}}  % Pie de foto manual
\end{center}


\item \textbf{\textit{Stemming}}: esta técnica busca un concepto de una palabra mediante la eliminación de prefijos y sufijos para obtener la raíz. De esta manera, se reduce la palabra a su mínimo elemento con significado.
\begin{center}
    \includegraphics[width=0.8\textwidth]{Images/Cap 2/Stemming.png}
    \captionof{figure}[Ejemplo de los términos derivados de la raíz “catalog”]{Ejemplo de los términos derivados de la raíz “catalog”, obtenido de \cite{ref47}.}  % Pie de foto manual
\end{center}
No obstante, es importante mencionar que esta técnica no siempre funciona correctamente debido a que hay palabras que poseen raíces compartidas por más de un significado. 
\begin{table}[H]
    \centering
    \begin{tabular}{|p{3cm}|p{2.5cm}|p{6cm}|}
        \hline
        \textbf{Término con prefijo} & \textbf{Raíz/Stem} & \textbf{Término con el que causaría confusión} \\
        \hline
        Prevalencia & valenc & Valencia, valencia, valenciano, ambivalencia, polivalencia \\
        \hline
        Precatalogar & catalog & Descatalogar, catálogo \\
        \hline
    \end{tabular}
    \caption[Ejemplos de términos con raíces compartidas]{Ejemplos de términos con raíces compartidas, obtenido de \cite{ref47}.}
    \label{tabla:confusion}
\end{table}

\end{enumerate}

\subsection{Embeddings Semánticos}
Los embeddings semánticos constituyen una forma de representar palabras, frases o incluso documentos mediante valores numéricos. En esencia, consisten en transformar el lenguaje natural, difícil de procesar directamente por una computadora, en vectores matemáticos que sí pueden ser manipulados mediante operaciones algebraicas. Esto permite que los sistemas automáticos capten relaciones de significado más allá de la simple coincidencia de palabras, logrando una comprensión más profunda del contenido lingüístico \cite{refebd1}.\\

Para obtener estas representaciones, se emplean algoritmos de aprendizaje automático entrenados con grandes volúmenes de texto, como libros, artículos científicos o recopilaciones masivas de contenido web. Durante el entrenamiento, el modelo identifica patrones, asociaciones y regularidades en el uso del lenguaje. De esta manera, aprende a ubicar cada palabra dentro de un espacio vectorial de muchas dimensiones, donde la proximidad entre vectores refleja su similitud semántica. Así, términos relacionados aparecen cercanos entre sí, mientras que aquellos con significados distintos se separan en el espacio \cite{refebd1}.\\

\noindent \textbf{Vectores semánticos o representaciones vectoriales}

Los vectores semánticos, también denominados representaciones vectoriales o simplemente embeddings, son la base del procesamiento moderno del significado en modelos computacionales. Un vector puede entenderse como una secuencia ordenada de números que define un punto dentro de un espacio con múltiples dimensiones; cada una de estas dimensiones expresa alguna característica aprendida por el modelo durante su entrenamiento \cite{refebd1}.\\

Una forma de visualizar este concepto es compararlo con un mapa geográfico. En un mapa tradicional, la posición de un punto se determina mediante dos coordenadas, como latitud y longitud. De manera análoga, en un espacio semántico cada palabra se asocia a un vector que indica su ubicación dentro de cientos o miles de dimensiones. Cuando dos vectores se encuentran cercanos, significa que sus palabras representan ideas o significados similares; si están lejanos, la relación semántica entre ellas es escasa o inexistente \cite{refebd1}.\\

En la práctica, los embeddings suelen ser representaciones densas de alta dimensionalidad, generalmente entre 128 y 1024 dimensiones, lo que permite capturar matices complejos del significado lingüístico con un grado notable de precisión \cite{refebd2}.\\

\begin{center}
    \includegraphics[width=0.8\textwidth]{Images/Cap 2/Embeddings/1_RepresentaciónVectorial.png}
    \captionof{figure}[Representación Vectorial]{Representación vectorial, obtenido de \cite{refebd1}.}  % Pie de foto manual
\end{center}

\noindent \textbf{Dimensiones de un embedding como una analogía con mapas}

Las dimensiones de un vector semántico pueden entenderse del mismo modo que las coordenadas que ubican un punto en un mapa. Sin embargo, en lugar de señalar una posición geográfica, cada dimensión sitúa una palabra o frase dentro de un espacio semántico abstracto. Cada una de estas dimensiones refleja alguna característica latente del significado, como matices emocionales, atributos gramaticales o cualquier otro rasgo aprendido por el modelo durante el entrenamiento \cite{refebd2}.\\

Cuando todas las dimensiones se combinan, forman una especie de “mapa conceptual” en el que las palabras se distribuyen según sus semejanzas. Por esta razón, dos términos con significados próximos aparecerán cercanos en este espacio, mientras que aquellos que no comparten relación se encontrarán a mayor distancia \cite{refebd2}.\\

Por ejemplo, si una de las dimensiones capturara aspectos relacionados con la polaridad emocional, podría esperarse que palabras como “feliz” o “alegre” obtuvieran valores similares, mientras que palabras como “triste” o “deprimido” se localizarían en posiciones opuestas dentro de esa misma dimensión.\\

\noindent \textbf{Cómo los vectores representan el significado en los textos}

Los embeddings permiten que un sistema automático comprenda el significado de un texto al reflejar patrones de uso y coocurrencia entre las palabras. Aquellos términos que suelen aparecer en contextos similares tienden a adquirir vectores parecidos. De esta manera, palabras como rey y reina se ubicarán cerca entre sí porque suelen relacionarse con conceptos comunes, como la monarquía o el poder \cite{refebd3}.\\

Además, al tratarse de valores numéricos, estos vectores pueden ser manipulados matemáticamente \cite{refebd3}. Por ejemplo, si un modelo ha aprendido atributos vinculados al género gramatical, podría aproximarse la relación:

\begin{center}
    \includegraphics[width=0.9\textwidth]{Images/Cap 2/Embeddings/2_RelacionesSemánticas.png}
    \captionof{figure}[Relaciones Semánticas]{Relaciones semánticas entre palabras, obtenido de \cite{refebd1}.}  % Pie de foto manual
\end{center}

Si bien los modelos reales manejan estructuras mucho más complejas, este tipo de ejemplos ilustra cómo los embeddings capturan no solo definiciones individuales, sino también las relaciones semánticas que existen entre los conceptos.\\

\noindent \textbf{Embeddings: representaciones de alta dimensionalidad}

Los embeddings actuales, como los generados por modelos avanzados de Google, OpenAI y otros sistemas basados en redes neuronales, suelen tener cientos o miles de dimensiones. Estas dimensiones no están definidas manualmente; surgen de manera automática durante el entrenamiento y cada una representa un aspecto abstracto del significado aprendido por el modelo \cite{refebd1}.\\

El uso de una gran cantidad de dimensiones permite representar los textos con un nivel de detalle muy elevado, similar a disponer de un mapa extremadamente preciso con numerosas coordenadas. Gracias a esta alta dimensionalidad, los modelos pueden capturar relaciones semánticas complejas, matices contextuales, analogías y diferentes sentidos de una misma palabra \cite{refebd1}.\\

Por ejemplo, el siguiente vector podría representar el embedding de la frase “La materia no se crea ni se destruye, solo se transforma”:\\

\begin{center}
    \includegraphics[width=0.5\textwidth]{Images/Cap 2/Embeddings/3_RepresentaciónEmbedding.png}
    \captionof{figure}[Representación Vectorial de un embedding]{Representación vectorial de un embedding, obtenido de \cite{refebd1}.}  % Pie de foto manual
\end{center}

En dicho vector, cada componente corresponde a una dimensión del espacio semántico. La combinación de todos estos valores numéricos produce una representación única del significado global de la frase. Aunque estos números no son interpretables por una persona en términos directos, sí permiten que un sistema automático manipule esa representación de forma matemática \cite{refebd1}.\\

De este modo, cuando se genera el embedding de otra frase con un significado parecido. Por ejemplo: “La energía no desaparece, únicamente cambia de forma”, ambos vectores tienden a aparecer muy próximos dentro del espacio semántico, ya que comparten patrones conceptuales y contextuales. Esta cercanía numérica refleja la similitud entre las ideas expresadas en ambas oraciones.\\

\noindent \textbf{Entrenamiento de los modelos de embedding (usualmente con 768 dimensiones)}

Los modelos encargados de producir embeddings se entrenan utilizando cantidades masivas de texto, que pueden incluir libros, artículos científicos, publicaciones web, código y diversos tipos de documentos. En los últimos años, este proceso ha escalado de forma significativa, con modelos que se entrenan sobre volúmenes de datos cada vez mayores \cite{refebd4}.\\

Durante este entrenamiento, el sistema aprende a identificar relaciones entre palabras y a predecir el contexto en el que podrían aparecer. Este aprendizaje es lo que permite que, posteriormente, pueda transformar cualquier frase en un vector numérico dentro del espacio semántico.\\

\begin{center}
    \includegraphics[width=0.9\textwidth]{Images/Cap 2/Embeddings/4_DimensionesEmbedding.png}
    \captionof{figure}[Dimensiones de un embedding]{Dimensiones de un embedding, obtenido de \cite{refebd1}.}  % Pie de foto manual
\end{center}

El número de dimensiones que tendrá cada embedding se establece como un parámetro del modelo. Un valor muy común en modelos modernos es 768 dimensiones, aunque pueden existir variantes con más o menos dimensiones según los objetivos del sistema y la arquitectura utilizada. Este proceso de entrenamiento exige una enorme capacidad computacional y puede requerir varios días o incluso semanas para completarse, especialmente cuando se trabaja con grandes volúmenes de datos o modelos de última generación \cite{refebd4}.\\

Además de comprender el concepto general de los embeddings, es importante reconocer que existen diversas técnicas para generarlos. Estas técnicas han evolucionado con el tiempo y pueden clasificarse en dos grandes categorías: los métodos clásicos, que producen representaciones estáticas, y los métodos modernos, que generan representaciones contextuales capaces de capturar matices más complejos del lenguaje. A continuación, se describen brevemente los modelos más representativos de cada categoría.\\

\noindent \textbf{Técnicas clásicas de generación de embeddings}

Los métodos clásicos fueron pioneros en la representación vectorial del lenguaje y permitieron reemplazar las palabras por vectores que conservaban relaciones semánticas básicas. Entre los modelos más influyentes se encuentran:

\begin{itemize}
    \item \textbf{Word2Vec}, introducido por Mikolov y colaboradores \cite{refebd5}.
    \item \textbf{GloVe}, basado en la factorización de matrices de co-ocurrencias de palabras \cite{refebd6}.
    \item \textbf{FastText}, que mejora los modelos anteriores al incorporar información de subpalabras y prefijos, permitiendo representar adecuadamente palabras nuevas o poco frecuentes \cite{refebd7}.
\end{itemize}

Estos modelos producen embeddings estáticos: cada palabra posee un único vector sin importar el contexto en el que aparece.\\

\noindent \textbf{Técnicas modernas (embeddings contextuales)}

Con el desarrollo de arquitecturas más avanzadas, como redes bidireccionales y modelos transformers, surgieron métodos capaces de generar embeddings que dependen del contexto de cada oración. Esto permite capturar significados más finos y relaciones más complejas entre las palabras. Entre los modelos destacados se encuentran:

\begin{itemize}
    \item \textbf{ELMo}, basado en redes LSTM bidireccionales que generan representaciones sensibles al contexto \cite{refebd8}.
    \item \textbf{BERT}, modelo basado en transformers y entrenado mediante Masked Language Modeling, que produce representaciones contextuales profundas \cite{refebd9}.
    \item \textbf{Sentence-BERT}, una adaptación de BERT diseñada específicamente para generar embeddings de oraciones altamente útiles en tareas de similitud semántica \cite{refebd10}.
\end{itemize}

Sentence-BERT será el modelo utilizado en este Trabajo Terminal debido a su capacidad para producir representaciones compactas y comparables de oraciones completas.\\

\noindent \textbf{Ventajas de Sentence-BERT}

Este modelo presenta características que lo convierten en una opción especialmente adecuada para aplicaciones que requieren medir similitud semántica entre textos \cite{refebd10}:

\begin{itemize}
    \item Captura el significado global de oraciones completas, no solo de palabras individuales.
    \item Ha sido preentrenado utilizando millones de pares de frases comparadas semánticamente.
    \item Está optimizado para tareas de similitud, búsqueda semántica y recuperación de información.
    \item Permite inferencia rápida sin necesidad de entrenamiento adicional (\textit{fine-tuning}) en la mayoría de los casos.
\end{itemize}

\subsection{Similitud coseno}
La similitud del coseno es una medida ampliamente utilizada para evaluar el grado de semejanza entre dos vectores dentro de un espacio multidimensional. A diferencia de otras métricas basadas en distancia, este método no considera la magnitud de los vectores, sino la dirección hacia la cual apuntan. Esta característica la vuelve especialmente útil en espacios de alta dimensionalidad, donde las distancias euclidiana o Manhattan pueden volverse menos representativas.\\

El cálculo consiste en obtener el coseno del ángulo formado por dos vectores distintos de cero. El resultado es un valor numérico comprendido entre –1 y 1, donde:
\begin{itemize}
    \item \textbf{1} indica que ambos vectores apuntan exactamente en la misma dirección, es decir, tienen una similitud completa.
    \item \textbf{0} señala que los vectores son ortogonales, lo que implica que no existe relación direccional entre ellos.
    \item \textbf{-1} representa direcciones totalmente opuestas, lo cual se interpreta como máxima disimilitud.
\end{itemize}

Esta medida puede visualizarse como la comparación entre dos flechas: si ambas apuntan hacia el mismo lugar, la similitud es alta; si forman un ángulo recto, no guardan relación; y si se orientan en direcciones contrarias, son totalmente diferentes.\\

En el ámbito del Machine Learning, Procesamiento de Lenguaje Natural (PLN) e Inteligencia Artificial, la similitud del coseno resulta fundamental debido a que estos sistemas operan con datos representados como vectores (por ejemplo, embeddings de texto). Al convertir palabras, frases o documentos en vectores numéricos, se pueden comparar semánticamente mediante esta métrica.\\

Por ejemplo, un chatbot puede transformar una consulta en un embedding, compararlo con los vectores almacenados en una base de datos y emplear la similitud del coseno para identificar cuál respuesta es la más cercana en significado. Este proceso permite medir cuánta relación existe entre dos representaciones vectoriales y seleccionar la opción más adecuada.\\

En resumen, la similitud del coseno es la métrica estándar para comparar embeddings, ya que ofrece una forma robusta de evaluar el parecido semántico entre textos mediante la orientación de sus vectores en el espacio.\\

\noindent \textbf{Definición matemática:}

La similitud del coseno se define a partir de la relación angular entre dos vectores. Matemáticamente, se expresa como:

\[
\text{cosine\_similarity}(A, B)
= \frac{A \cdot B}{\|A\| \, \|B\|}
\]

donde:
\begin{itemize}
    \item $A \cdot B$ = producto punto (suma de productos elemento a elemento).
    \item $\|A\|$ = norma euclidiana del vector $A$.
    \item Rango: $[-1, 1]$, típicamente $[0, 1]$ cuando se usan embeddings normalizados.
\end{itemize}

Esta formulación permite medir qué tan alineados están dos vectores dentro de un espacio multidimensional, lo cual es fundamental para comparar representaciones semánticas en tareas de PLN.\\

\noindent Propiedades:

La similitud coseno presenta características que la hacen especialmente útil en modelos basados en embeddings:

\begin{itemize}
    \item Invariante a la magnitud de los vectores
    \item Cómputo eficiente: $O(d)$ donde $d$ es la dimensionalidad
    \item Interpretabilidad: 1.0 = idéntico, 0.0 = ortogonal
\end{itemize}

En el proceso de selección de una métrica adecuada para comparar embeddings, se evaluaron diversas alternativas empleadas comúnmente en tareas de análisis vectorial. Sin embargo, varias de ellas presentan limitaciones importantes cuando se trabajan datos de alta dimensionalidad o representaciones semánticas continuas, razón por la cual fueron descartadas:

\noindent \textbf{Alternativas consideradas y rechazadas:}
\begin{itemize}
    \item \textbf{Distancia Euclidiana}: Su desempeño se ve afectado por la magnitud de los vectores, lo que genera resultados inconsistentes cuando los embeddings tienen distintas escalas o longitudes.
    \item \textbf{Distancia de Manhattan}: Aunque útil en ciertos contextos, no captura la direccionalidad entre los vectores, un aspecto clave para evaluar similitud semántica.
    \item \textbf{Índice de Jaccard}: Diseñado para conjuntos discretos; no es adecuado para vectores densos y continuos, como los generados por modelos modernos de PLN.
\end{itemize}

La similitud del coseno se seleccionó como la métrica óptima debido a sus propiedades, que la han consolidado como un estándar en sistemas basados en embeddings. Entre sus principales ventajas destacan:
\begin{itemize}
    \item \textbf{Robustez en espacios de alta dimensionalidad}: A diferencia de las distancias tradicionales, la similitud coseno mantiene un comportamiento estable incluso cuando los vectores poseen cientos o miles de dimensiones.
    \item \textbf{Independencia de la magnitud}: Al centrarse únicamente en el ángulo entre vectores, evita que diferencias en longitud distorsionen la percepción de similitud.
    \item \textbf{Eficiencia computacional}: Su cálculo es ligero y puede implementarse con gran rapidez utilizando herramientas ampliamente adoptadas como NumPy o SciPy.
    \item \textbf{Versatilidad en múltiples dominios}: Es aplicable en tareas como minería de textos, recuperación de información, sistemas de recomendación y búsquedas semánticas en tiempo real, lo que la convierte en una métrica generalista y ampliamente probada.
\end{itemize}

En conjunto, estas características la hacen especialmente adecuada para comparar representaciones vectoriales en el contexto del Procesamiento de Lenguaje Natural empleado en este proyecto.

\subsection{Transformers y Modelos Pre-Entrenados}
Desde su presentación en 2017, la arquitectura Transformer supuso un cambio definitivo en el desarrollo de modelos de aprendizaje profundo, especialmente en el ámbito del Procesamiento de Lenguaje Natural (PLN). Antes de su aparición, el PLN dependía en gran medida de arquitecturas secuenciales como las redes neuronales recurrentes (RNN) y sus variantes (LSTM y GRU). Aunque efectivas para ciertas tareas, estas redes tenían dificultades para capturar dependencias de largo alcance y requerían procesar la información paso a paso, lo que limitaba su escalabilidad.\\

El trabajo \textit{“Attention Is All You Need”} introducido por Vaswani et al. en 2017 [e6] propuso un enfoque completamente distinto: eliminar la recurrencia y reemplazarla por un mecanismo de atención capaz de modelar relaciones globales entre tokens en una sola operación. Esta innovación permitió acelerar drásticamente el entrenamiento y mejorar la calidad contextual de los modelos, convirtiéndose en la base de sistemas modernos como BERT, GPT, T5 y otras variantes avanzadas.\\

A continuación, se describen los elementos esenciales que conforman la arquitectura Transformer, los cuales permiten comprender por qué se ha convertido en el estándar en modelos de PLN:
\begin{enumerate}
    \item \textbf{Embeddings}. Cada token del texto se representa como un vector denso de alta dimensionalidad. Esto permite trabajar con significado numérico en lugar de cadenas de texto.
    \item \textbf{Codificación posicional}: Como el modelo no es secuencial, se introduce información sobre el orden de los tokens usando vectores posicionales, normalmente basados en funciones senoidales.
    \item \textbf{Mecanismo de atención (Self-Attention)}: Este mecanismo permite que cada token “preste atención” a otros tokens de la secuencia para capturar relaciones contextuales. Se utilizan tres vectores: query, key y value.
    \item \textbf{Multi-Head Attention}: Permite al modelo enfocarse en diferentes partes del texto desde distintas perspectivas. Cada “cabeza” aprende un aspecto contextual distinto.
    \item \textbf{Capas Feed-Forward}: Redes neuronales tradicionales que procesan los datos post-atención token por token.
    \item \textbf{Normalización y conexiones residuales}: Estas técnicas estabilizan el entrenamiento y ayudan a preservar el flujo de información.
\end{enumerate}

\noindent \textbf{Arquitectura general: encoder y decoder}

El Transformer original está compuesto por dos bloques principales:
\begin{itemize}
    \item \textbf{Encoder}: procesa la entrada analizando relaciones entre los tokens.
    \item \textbf{Decoder}: genera la salida, utilizando atención tanto a la entrada codificada como al contexto generado anteriormente.
\end{itemize}

Dependiendo del tipo de modelo, puede utilizarse solo una de estas partes:
\begin{itemize}
    \item \textbf{Modelos encoder-only}: Por ejemplo, BERT, empleados para clasificación, extracción de características o embeddings.
    \item \textbf{Modelos decoder-only}: Como GPT, orientados a generación de texto.
    \item \textbf{Modelos encoder–decoder}: Como T5, útiles para tareas de traducción o reformulación.
\end{itemize}

\noindent \textbf{Procesamiento general de un Transformer}
\begin{enumerate}
    \item Tokenización del texto de entrada.
    \item Conversión de tokens a embeddings.
    \item Suma de las codificaciones posicionales.
    \item Procesamiento mediante múltiples capas de self-attention y feed-forward.
    \item (En modelos de generación) aplicación de atención enmascarada para controlar el orden autoregresivo.
    \item Obtención del siguiente token mediante una capa final softmax.
\end{enumerate}

En términos generales, la arquitectura Transformer integra los componentes que se describen a continuación:
\begin{itemize}
    \item \textbf{Encoder}: procesa y contextualiza el texto de entrada.
    \item \textbf{Decoder}: produce la salida token por token (no presente en BERT).
    \item \textbf{Self-Attention}: cada palabra puede relacionarse con todas las demás en la secuencia.
    \item \textbf{Multi-Head Attention}: permite analizar el texto desde diversas perspectivas simultáneamente.
    \item \textbf{Feed-Forward Networks}: capas densas que refinan la representación.
    \item \textbf{Positional Encoding}: incorpora información relativa al orden.
\end{itemize}

A continuación, se enlistan algunos modelos pre-entrenados basados en Transformers.\\

\noindent \textbf{BERT (Bidirectional Encoder Representations from Transformers)}

\begin{itemize}
    \item Entrenado con grandes corpus como Wikipedia y BookCorpus (aprox. 3.3B palabras).
    \item Utiliza dos tareas principales durante el pre-entrenamiento:
    \begin{itemize}
        \item \textbf{Masked Language Model}: predecir palabras ocultas dentro de una oración.
        \item \textbf{Next Sentence Prediction}: aprender relaciones entre oraciones consecutivas.
    \end{itemize}
    \item Su naturaleza bidireccional lo hace ideal para tareas de comprensión del lenguaje.
\end{itemize}

\noindent \textbf{SENTENCE-TRANSFORMERS (BERT)}

Extensión de BERT adaptada para generar embeddings semánticos de oraciones con alta calidad. Se caracteriza por:
\begin{itemize}
    \item Uso de redes siamesas o triplet loss para aprender similitud textual.
    \item Optimización específica para tareas de búsqueda semántica, clustering y clasificación basada en similitud.
    \item Disponibilidad de modelos especializados para distintos idiomas y aplicaciones.
\end{itemize}

El modelo utilizado en este Trabajo Terminal es paraphrase-multilingual-MiniLM-L12-v2. Sus características principales son:
\begin{itemize}
    \item \textbf{Familia}: MiniLM (una variante compacta y eficiente de BERT).
    \item \textbf{Número de capas}: 12 (menor profundidad que BERT-large).
    \item \textbf{Dimensionalidad de los embeddings}: 384.
    \item \textbf{Compatibilidad multilingüe}: soporta más de 50 idiomas, incluido el español.
    \item \textbf{Optimización}: especializado en detección de paráfrasis y similitud semántica.
    \item \textbf{Tamaño reducido (aprox. 120 MB)}: facilita su uso en proyectos con recursos limitados.
    \item \textbf{Velocidad}: capaz de procesar consultas en aproximadamente 40 ms sobre CPU.
\end{itemize}

Este modelo fue elegido debido a su equilibrio entre rendimiento, precisión semántica y eficiencia computacional, lo que lo hace adecuado para el sistema de reconocimiento y correspondencia lingüística requerido en el proyecto.

\subsection{Aplicaciones del PLN}
A continuación, se enlistan las principales aplicaciones del PLN:
\begin{itemize}
    \item \textbf{Recuperación y extracción de información}: la recuperación de información es el proceso de encontrar datos en un repositorio grande, para satisfacer una necesidad de información \cite{ref48}. Por su parte, la extracción de información consiste en la obtención de ciertos elementos dentro de un texto que son de interés, para posteriormente ser pasadas a un formato de base de datos \cite{ref48}.    
    \item \textbf{Minería de datos}: permite descubrir patrones ocultos y relaciones en datos estructurados, empleando técnicas de reconocimiento de información, extracción de información y corpus procesados con técnicas de lingüística computacional \cite{ref48}.
    \item \textbf{Sistemas de búsqueda de respuestas}: son sistemas diseñados para interpretar una pregunta en lenguaje natural y proporcionar una respuesta, para evitar que los usuarios naveguen y lean varias páginas de resultados de búsqueda. Estos sistemas son alimentados con contenido fuente para entender las preguntas del usuario y encontrar las respuestas \cite{ref48}.
    \item \textbf{Generación de resúmenes automáticos}: consiste en emplear herramientas de PLN, para tomar una colección de términos, frases o párrafos significativos que definen el significado del texto original para generar un resumen. También se pueden emplear técnicas de PLN para parafrasear un texto y producir una síntesis \cite{ref48}.
    \item \textbf{Análisis de sentimientos}: identificación y extracción de información subjetiva, empleando herramientas de PLN y software de análisis de textos para automatizar el proceso. El análisis de sentimientos emplea una clasificación polarizada de sentimientos que consiste en un rango de -10 a 10 que se basa en el aprendizaje para evaluar emociones tanto negativas como positivas en corpus etiquetados de entrenamiento \cite{ref48}.
    \item \textbf{Traducción automática}: consiste en tomar el texto escrito en un lenguaje y traducirlo a otro, manteniendo el mismo significado. El proceso de traducción automática sigue tres pasos: en primer lugar, el texto en el lenguaje original se transforma a una representación intermedia, luego se realizan modificaciones a esta representación intermedia basándose en la morfología del lenguaje, y finalmente se transforma al lenguaje destino \cite{ref48}.
\end{itemize}

\section{Sistema Operativo Android}
Un sistema operativo móvil es un conjunto de programas que habilitan características específicas de un teléfono móvil y brindan servicios a las aplicaciones móviles que se ejecutan en él \cite{ref57}.\\

El sistema operativo Android es un sistema operativo móvil desarrollado por la empresa estadounidense Google y que está basado en el sistema operativo Linux. Es un sistema operativo abierto, gratuito, versátil, seguro y altamente personalizable que está desarrollado para dispositivos móviles como smartphones y tablets \cite{ref57}.

\begin{center}
    \includegraphics[width=0.6\textwidth]{Images/Cap 2/Android_Logo.png}
    \captionof{figure}[Logo de Android]{Logo de Android, obtenido de \cite{ref58}.} 
\end{center}

Las principales características de Android son las siguientes \cite{ref57}:

\begin{itemize}

    \item \textbf{Interfaz de Usuario (UI) personalizable}: los usuarios son capaces de cambiar el aspecto de sus dispositivos para adaptarlos a sus necesidades.
    \item \textbf{Compatibilidad con múltiples fabricantes}: este sistema operativo es ejecutado en una gran cantidad de dispositivos de múltiples fabricantes.
    \item \textbf{Google Play Store}: cuenta con una tienda de aplicaciones que permite descargar diferentes aplicaciones de diversa índole, basadas en las necesidades de cada usuario.
    \item \textbf{Asistente Virtual}: los usuarios tienen acceso a un asistente virtual llamado Google Assistant, que ayuda en la realización de tareas.
    \item \textbf{Integración con servicios de Google}: Android está integrado con servicios de Google como Gmail, Google Drive, Google Photos, Maps, entre otros.
    \item \textbf{Compatibilidad con tecnologías emergentes}: es compatible con tecnologías como la Realidad Virtual (VR), la Realidad Aumentada (AR) y los asistentes virtuales. 
\end{itemize}

La versión actual del Sistema Operativo Android, al momento, es la versión Android 14, la cual fue anunciada el 04 de Octubre de 2023.\\

\section{React Native}
React Native es un framework de aplicaciones móviles desarrollado por Meta (Facebook) que permite crear aplicaciones para iOS y Android empleando JavaScript y React \cite{ref59}.\\

Permite crear aplicaciones con un rendimiento similar a las nativas, empleando las mismas API nativas que emplean otras aplicaciones. Sus principales características son \cite{ref59}:

\begin{itemize}
    \item \textbf{Desarrollo multiplataforma}: puede ser desplegado en Android, iOS, Web, etc.
    \item \textbf{Uso de JavaScript}: la base de React Native es JavaScript.
    \item \textbf{Componentes reutilizables}: se emplean componentes de React, que a su vez son bloques de código reutilizables.
    \item \textbf{Acceso a las API nativas}: se puede acceder a características como la cámara, mícrofono, GPS, etc.
    \item \textbf{Rendimiento nativo}: las apps de React Native tienen un rendimiento similar a las apps nativas.
\end{itemize}

La última versión de React Native, al momento de la elaboración de este proyecto, es la 0.79 \cite{ref60}.\\

\begin{center}
    \includegraphics[width=0.6\textwidth]{Images/Cap 2/ReactNative-Logo.png}
    \captionof{figure}[Logo de React Native]{Logo de React Native, obtenido de \cite{ref60}.} 
\end{center}

\subsection{Expo}
Expo es un framework diseñado para facilitar el desarrollo de aplicaciones en React, el cual integra un conjunto de herramientas y servicios que funcionan sobre React Native y las principales plataformas móviles. Su propósito es simplificar las tareas de creación, compilación y despliegue de aplicaciones en iOS, Android e incluso en la Web, utilizando siempre una misma base de código escrita en JavaScript o TypeScript con React \cite{refexpo1}.\\

\begin{center}
    \includegraphics[width=0.8\textwidth]{Images/Cap 2/expo.jpg}
    \captionof{figure}[Logo de Expo]{React Expo, obtenido de \cite{refexpologo}.} 
\end{center}

Además, Expo ofrece acceso a una gran variedad de funciones nativas desde JavaScript, permitiendo trabajar con elementos como la cámara, notificaciones, síntesis de voz, sensores del dispositivo, detección facial y manejo de video, entre otros \cite{refexpo1}.\\

Las características principales de Expo son las siguientes \cite{refexpo2}:
\begin{itemize}
    \item \textbf{Configuración automatizada}: la mayor parte de la preparación del entorno es gestionada directamente por Expo.
    \item \textbf{Código multiplataforma}: con un solo desarrollo se puede ejecutar la aplicación tanto en Android como en iOS.
    \item \textbf{Sin necesidad de configuración nativa}: es posible probar la aplicación mediante Expo Go sin instalar herramientas como Android Studio o Xcode.
    \item \textbf{Bibliotecas integradas}: incluye múltiples API preconstruidas que permiten acceder fácilmente a funcionalidades nativas del dispositivo.
\end{itemize}

Las herramientas destacadas del ecosistema de Expo incluyen \cite{refexpo3}:
\begin{itemize}
    \item \textbf{Expo Go App}: aplicación móvil que permite visualizar y probar el proyecto en tiempo real, sin necesidad de compilarlo por completo.
    \item \textbf{Expo SDK}: conjunto de API listas para usar, que proporcionan acceso rápido a servicios como la cámara, notificaciones push o geolocalización.
    \item \textbf{Expo CLI}: herramienta de línea de comandos que gestiona el ciclo de vida completo del proyecto, desde su creación hasta su publicación.
\end{itemize}

A continuación, se enlistan las ventajas de utilizar Expo para el desarrollo de aplicaciones móviles \cite{refexpo4}:
\begin{itemize}
    \item \textbf{Desarrollo ágil}: crear una app con React Native y Expo resulta más rápido que mantener proyectos nativos separados. Además, permite integrar fácilmente numerosas bibliotecas externas, reduciendo tiempos de implementación.
    \item \textbf{Pruebas simplificadas}: las aplicaciones pueden evaluarse en distintos dispositivos de forma sencilla, lo que mejora la calidad y la transparencia del proceso de desarrollo.
    \item \textbf{Actualizaciones rápidas}: gracias a EAS Update, los cambios o parches pueden implementarse de manera inmediata, garantizando un rendimiento estable y eficiente en todas las plataformas.
\end{itemize}

Expo para React Native constituye una solución potente y optimizada para acelerar el desarrollo de aplicaciones móviles sin perder flexibilidad. Ya sea para la creación de prototipos o proyectos multiplataforma, Expo proporciona un ecosistema integral de herramientas y servicios que simplifican el flujo de trabajo y mejoran la eficiencia en todo el proceso \cite{refexpo3}.\\

\section{API}
Las API (del inglés Application Programming Interfaces) son mecanismos que permiten la comunicación entre dos componentes de software mediante definiciones, protocolos y reglas de intercambio. Actúan como un contrato de servicio que especifica cómo deben comunicarse las aplicaciones a través de solicitudes y respuestas estructuradas \cite{refapi1}. En este modelo, el cliente envía la solicitud y el servidor la procesa y devuelve una respuesta, permitiendo la integración entre distintos sistemas \cite{refapi1}.\\
\newline \textbf{Tipos de API según su funcionamiento}

\begin{center}
    \includegraphics[width=0.8\textwidth]{Images/Cap 2/API.png}
    \captionof{figure}[APIS]{Funcionamiento de las API, obtenido de \cite{refapi2}.} 
\end{center}

Las API pueden operar de distintas formas \cite{refapi1}:
\begin{itemize}
    \item \textbf{SOAP}: usa XML para intercambiar información entre cliente y servidor; actualmente se considera menos flexible.
    \item \textbf{RPC}: permite ejecutar funciones remotas en el servidor y recibir los resultados.
    \item \textbf{WebSocket}: admite comunicación bidireccional en tiempo real mediante objetos JSON.
    \item \textbf{REST}: es el tipo más utilizado, basado en HTTP y en el intercambio de datos entre cliente y servidor.
\end{itemize}

\noindent\textbf{Clasificación por ámbito y uso}

Según su alcance y aplicación, las API se agrupan en \cite{refapi1}\cite{refapi2}:
\begin{itemize}
    \item \textbf{Privadas o internas}: se usan dentro de una empresa para conectar sistemas propios.
    \item \textbf{Públicas o abiertas}: accesibles a cualquier usuario o desarrollador.
    \item \textbf{De socios o aliados}: restringidas a colaboradores autorizados.
    \item \textbf{Compuestas}: combinan varias API para atender procesos más complejos.
\end{itemize}

\vspace{1em} % ← agrega espacio aquí

También pueden clasificarse por su propósito \cite{refapi2}:
\begin{itemize}
    \item \textbf{De datos}: gestionan operaciones CRUD sobre bases de datos.
    \item \textbf{De sistemas operativos}: permiten a las aplicaciones interactuar con los recursos del sistema.
    \item \textbf{Remotas}: conectan sistemas distribuidos a través de una red.
    \item \textbf{Web}: posibilitan la comunicación entre servicios en la web.
\end{itemize}

Las API exponen un conjunto limitado de acciones y puntos de acceso (endpoints) para que otros sistemas interactúen con ellas. Las solicitudes especifican la acción y los datos requeridos, y las respuestas devuelven la información procesada \cite{refapi3}. También pueden activarse por eventos de negocio, como pagos, registros o actualizaciones de inventario, facilitando procesos automatizados entre múltiples sistemas \cite{refapi3}.\\

Entre las principales ventajas de las API destacan \cite{refapi4}:
\begin{itemize}
    \item \textbf{Ahorro de tiempo y costos}: simplifican el desarrollo al reutilizar funcionalidades existentes.
    \item \textbf{Colaboración e integración}: permiten ampliar arquitecturas con nuevos servicios.
    \item \textbf{Seguridad y control}: limitan el acceso mediante pasarelas y autenticación.
    \item \textbf{Innovación}: fomentan la creación de nuevas aplicaciones y oportunidades de negocio.
\end{itemize}

\subsection{API REST}
La Transferencia de Estado Representacional (REST, por sus siglas en inglés) constituye un estilo arquitectónico que define un conjunto de principios para el diseño de interfaces entre sistemas. Su propósito es establecer reglas que permitan crear APIs capaces de comunicarse de forma eficiente, escalable y fiable, incluso en entornos distribuidos como Internet. Una arquitectura basada en REST se caracteriza por ser flexible, fácil de mantener y por ofrecer portabilidad entre distintas plataformas, lo que la vuelve una opción común para el desarrollo de servicios web modernos.\\

Cuando una API se implementa siguiendo estos principios, se le denomina API REST o API RESTful. Del mismo modo, los servicios web construidos bajo esta arquitectura reciben el nombre de servicios web RESTful. En la práctica, los términos REST API y RESTful API suelen emplearse como equivalentes.\\

A continuación, se describen los principales lineamientos que definen el estilo REST:
\begin{enumerate}
    \item \textbf{Interfaz Uniforme}: La uniformidad de la interfaz es el eje central de REST, ya que garantiza que la comunicación entre cliente y servidor siga un formato consistente. Los datos que se transmiten se denominan representaciones y pueden diferir de la forma en la que el servidor almacena internamente sus recursos. Por ejemplo, la información podría gestionarse como texto plano y enviarse al cliente en formato JSON, XML o HTML.\\
    
    Este principio incluye cuatro restricciones importantes:

    \begin{itemize}
        \item Cada solicitud debe identificar el recurso al que desea acceder, normalmente mediante una URL.
        \item La representación enviada por el servidor proporciona suficiente información para que el cliente pueda modificar o eliminar el recurso si es necesario.
        \item Los mensajes deben ser autodescriptivos, es decir, contener los metadatos necesarios para que el cliente sepa cómo procesarlos.
        \item Las representaciones deben incluir enlaces a otros recursos relevantes, de modo que el cliente pueda descubrirlos sin requerir conocimiento previo del sistema.
    \end{itemize}

    \item \textbf{Tecnología sin estado}: REST establece que cada solicitud enviada al servidor debe contener toda la información necesaria para ser atendida, sin depender de solicitudes anteriores. Esto implica que el servidor no mantiene sesiones ni estados entre interacciones. Gracias a esta independencia, el sistema se vuelve más escalable y sencillo de distribuir, ya que cualquier servidor puede atender cualquier petición sin requerir contexto adicional.
    \item \textbf{Sistema por capas}: En REST, la comunicación entre cliente y servidor puede pasar por uno o más intermediarios, como balanceadores de carga, servidores de seguridad o cachés, sin que el cliente sea consciente de ello. Las capas pueden encargarse de tareas como autenticación, lógica de negocio o almacenamiento, y funcionan de manera transparente para quien consume la API. Esta estructura permite mejorar la organización del sistema y distribuir responsabilidades sin afectar la interacción con el cliente.
    \item \textbf{Almacenamiento en caché}: Los servicios RESTful permiten aprovechar mecanismos de almacenamiento en caché para mejorar el rendimiento. Si ciertos datos no cambian con frecuencia, pueden ser guardados temporalmente por el cliente o por intermediarios para evitar solicitudes repetitivas al servidor. La API debe indicar, mediante cabeceras específicas, si una respuesta puede almacenarse y por cuánto tiempo, permitiendo así reducir la carga del servidor y disminuir los tiempos de respuesta.
    \item \textbf{Código bajo demanda}: Opcionalmente, REST permite que el servidor envíe código ejecutable al cliente para ampliar sus capacidades de forma temporal. Un ejemplo habitual ocurre en páginas web que envían fragmentos de JavaScript al navegador para validar formularios o añadir interactividad. Aunque no es obligatorio, este principio proporciona flexibilidad adicional al aumentar las funciones disponibles del lado del cliente sin requerir instalaciones externas.
\end{enumerate}

\subsection{API RESTful}
Una API RESTful es una interfaz que permite que dos sistemas informáticos intercambien información de forma segura a través de Internet. En entornos empresariales, es habitual que diferentes aplicaciones, internas o de terceros, necesiten comunicarse entre sí para completar diversas tareas. Por ejemplo, un sistema de nóminas puede requerir información de un servicio bancario externo para gestionar pagos, o datos de una aplicación interna para validar horas trabajadas.\\

Las API diseñadas bajo el estilo REST facilitan este tipo de integración, ya que emplean estándares abiertos y bien definidos que permiten una comunicación eficiente, confiable y sencilla de mantener.\\

\noindent \textbf{Principios fundamentales de REST}

Los servicios RESTful se basan en una serie de principios arquitectónicos que garantizan interoperabilidad, escalabilidad y simplicidad:

\begin{enumerate}
    \item \textbf{Cliente–Servidor}: Se mantiene una separación clara entre el cliente (quien consume la información) y el servidor (quien la provee), lo que facilita el mantenimiento y la evolución independiente de cada parte.
    \item \textbf{Sin estado (Stateless)}: Cada petición enviada al servidor debe contener toda la información necesaria para procesarla. El servidor no almacena el estado de las interacciones previas, lo que mejora la escalabilidad.
    \item \textbf{Cacheable}: Las respuestas pueden indicar si es posible almacenarlas en caché, reduciendo la carga sobre el servidor y acelerando la comunicación.
    \item \textbf{Interfaz uniforme}: Todos los recursos deben identificarse mediante URIs y utilizar representaciones estandarizadas, proporcionando una interacción consistente entre cliente y servidor.
    \item \textbf{Sistema en capas}: La arquitectura puede incluir múltiples capas como seguridad, balanceo de carga o intermediarios sin que el cliente necesite conocer su existencia.
\end{enumerate}

\noindent \textbf{Métodos HTTP utilizados en REST}

Los servicios RESTful emplean los métodos del protocolo HTTP para definir la operación que se desea realizar sobre un recurso:

\begin{itemize}
    \item \textbf{GET}: Recupera un recurso o una colección de recursos. Es un método seguro e idempotente.
    \item \textbf{POST}: Crea un nuevo recurso o ejecuta una acción específica en el servidor.
    \item \textbf{PUT}: Reemplaza por completo un recurso existente con una nueva representación.
    \item \textbf{PATCH}: Realiza una actualización parcial del recurso.
    \item \textbf{DELETE}: Elimina el recurso identificado por la URI.
\end{itemize}

\noindent \textbf{Códigos de estado HTTP}

Las respuestas de una API RESTful utilizan códigos de estado estándar para indicar el resultado de la operación:

\begin{itemize}
    \item \textbf{2xx – Éxito}: Ejemplos comunes incluyen 200 OK para peticiones correctas y 201 Created para la creación de un nuevo recurso.
    \item \textbf{4xx – Errores del cliente}: Incluyen situaciones como 400 Bad Request cuando la petición está mal formada o 404 Not Found si el recurso solicitado no existe.
    \item \textbf{5xx – Errores del servidor}: Indican fallos internos, como 500 Internal Server Error.
\end{itemize}

El estilo REST ofrece múltiples beneficios que explican su amplia adopción en la industria del software:
\begin{itemize}
    \item \textbf{Alta escalabilidad}: La ausencia de estado facilita el uso de balanceadores de carga y distribuciones horizontales.
    \item \textbf{Desacoplamiento}: Cliente y servidor pueden evolucionar de manera independiente, siempre que mantengan los contratos de la API.
    \item \textbf{Amplio soporte y estandarización}: REST es uno de los enfoques más utilizados, por lo que existe abundante documentación, librerías y compatibilidad multiplataforma.
    \item \textbf{Herramientas maduras}: Ecosistemas como Postman, Swagger/OpenAPI, entre otros, permiten documentar, probar y mantener APIs de manera eficiente.
\end{itemize}

\section{Arquitectura de Microservicios}
La arquitectura de microservicios, abreviada comúnmente como microservicios, es un estilo de diseño de software orientado al desarrollo de aplicaciones como un conjunto de servicios pequeños, independientes y desplegables de forma separada. Cada microservicio posee un dominio de responsabilidad propio y puede implementarse utilizando distintos lenguajes o tecnologías, siempre que mantenga una comunicación coherente mediante contratos de API bien definidos \cite{refmic1,refmic2,refmic3}.\\

A diferencia de las arquitecturas monolíticas, en las que todos los componentes de una aplicación se integran en un único bloque de código, los microservicios promueven un enfoque distribuido y débilmente acoplado, lo que implica que un fallo en un componente no afecta el funcionamiento del resto del sistema \cite{refmic2}\cite{refmic3}. Esta independencia facilita el mantenimiento, la escalabilidad y la actualización continua sin interrumpir el servicio general \cite{refmic3}\cite{refmic4}.\\

Gracias a su flexibilidad, la arquitectura de microservicios permite desarrollar e implementar funcionalidades individuales de manera autónoma, lo que acelera el ciclo de desarrollo y mejora la capacidad de respuesta ante las necesidades del usuario o del negocio \cite{refmic3}\cite{refmic4}. Además, los equipos de desarrollo pueden trabajar de forma paralela en distintos servicios, reduciendo el tiempo de entrega de nuevas características y evitando dependencias entre módulos \cite{refmic2}\cite{refmic4}.\\

Entre los ejemplos más comunes del uso de microservicios se encuentran \cite{refmic1}:
\begin{itemize}
    \item \textbf{Migración de sitios web}: permite dividir una plataforma monolítica en varios microservicios desplegados en la nube, favoreciendo la escalabilidad y el mantenimiento.
    \item \textbf{Gestión de contenido multimedia}: posibilita el almacenamiento y distribución de videos o imágenes mediante sistemas de almacenamiento de objetos escalables.
    \item \textbf{Procesamiento de transacciones y facturación}: separa los módulos de pago y facturación, garantizando que el sistema continúe funcionando incluso si un servicio falla.
    \item \textbf{Procesamiento de datos}: amplía la compatibilidad con la nube de los servicios de análisis y gestión de datos.
\end{itemize}

Las aplicaciones basadas en microservicios presentan varios atributos distintivos \cite{refmic2}\cite{refmic3}, como lo son:
\begin{itemize}
    \item \textbf{Independencia de desarrollo y despliegue}: cada servicio puede diseñarse, probarse y desplegarse sin afectar a los demás.
    \item \textbf{Comunicación basada en API}: los microservicios se comunican entre sí mediante interfaces estandarizadas, lo que permite la integración sin modificar los datos de origen.
    \item \textbf{Escalabilidad selectiva}: los servicios pueden ampliarse individualmente según la demanda, optimizando los recursos y reduciendo costos.
    \item \textbf{Diversidad tecnológica}: cada servicio puede desarrollarse con el lenguaje o framework más adecuado a su función.
\end{itemize}

La adopción de microservicios ofrece múltiples beneficios frente a las arquitecturas tradicionales \cite{refmic3}\cite{refmic4}. A continuación se enlistan algunos:
\begin{itemize}
    \item \textbf{Resiliencia y tolerancia a fallos}: al ser componentes independientes, los errores en un servicio no afectan al resto de la aplicación.
    \item \textbf{Agilidad y velocidad de desarrollo}: la división en módulos pequeños acelera la creación, prueba e implementación del software.
    \item \textbf{Escalabilidad y mantenimiento mejorado}: los equipos pueden escalar servicios específicos y depurar errores sin interrumpir toda la aplicación.
    \item \textbf{Eficiencia operativa}: algunos equipos pueden gestionar sus propios microservicios, reduciendo la necesidad de departamentos de operaciones separados.
    \item \textbf{Dinamismo tecnológico}: se facilita la adopción de nuevas herramientas o tecnologías sin reescribir toda la aplicación.
\end{itemize}

Los microservicios representan un modelo arquitectónico moderno y modular, ideal para entornos nativos de la nube y para organizaciones que buscan agilidad, escalabilidad y resiliencia en sus aplicaciones \cite{refmic1,refmic2,refmic3,refmic4}.\\

\begin{center}
    \includegraphics[width=0.95\textwidth]{Images/Cap 2/microservicios.png}
    \captionof{figure}[Arquitectura de Microservicios]{Arquitectura de Microservicios, obtenido de \cite{refmic5}.} 
\end{center}


\section{Servicios en la Nube}
Los servicios en la nube son soluciones tecnológicas que permiten a usuarios y organizaciones acceder bajo demanda a recursos informáticos, aplicaciones y datos a través de internet, sin necesidad de poseer infraestructura o hardware especializado. Estos servicios son administrados por un proveedor de nube, lo que permite a los usuarios obtener acceso flexible y económico a las herramientas que necesitan, sin asumir los costos ni la complejidad de su gestión \cite{refnub1}.\\

El funcionamiento de la nube se basa en la virtualización de recursos que residen en servidores físicos ubicados en centros de datos gestionados por los proveedores. Los usuarios acceden a estos recursos mediante una conexión a internet, almacenando y procesando la información en servidores remotos en lugar de en equipos locales. Generalmente, el uso de la nube se cobra por suscripción o consumo, lo que representa una alternativa más económica frente al mantenimiento de infraestructura propia \cite{refnub1}.\\

\begin{center}
    \includegraphics[width=0.8\textwidth]{Images/Cap 2/Nube.png}
    \captionof{figure}[Servicios en la Nube]{Funcionamiento de los Servicios en la Nube, obtenido de \cite{refnub2}.} 
\end{center}

Entre las principales ventajas de los servicios en la nube se destacan \cite{refnub1}:
\begin{itemize}
    \item Reducción de hardware físico y costos operativos.
    \item Flexibilidad de pago según el consumo real.
    \item Menor carga del área de TI, ya que el mantenimiento recae en el proveedor.
    \item Acceso remoto desde cualquier dispositivo y ubicación.
    \item Colaboración en tiempo real entre equipos mediante plataformas compartidas.
\end{itemize}

\noindent \textbf{Modelos de servicios en la nube}

Los servicios de nube se clasifican comúnmente en tres modelos principales, con una categoría adicional emergente \cite{refnub1}:
\begin{itemize}
    \item \textbf{Infraestructura como servicio (IaaS)}: ofrece recursos virtualizados como servidores, almacenamiento y redes. Ejemplos: Amazon Web Services (AWS) y Microsoft Azure.
    \item \textbf{Plataforma como servicio (PaaS)}: proporciona entornos completos para el desarrollo, prueba y despliegue de aplicaciones, sin gestionar la infraestructura subyacente.
    \item \textbf{Software como servicio (SaaS)}: ofrece aplicaciones completas alojadas en la nube, accesibles desde internet, como Google Workspace o Microsoft 365.
    \item \textbf{Todo como servicio (XaaS)}: extiende el modelo a cualquier recurso tecnológico disponible como servicio bajo demanda.
\end{itemize}

\vspace{1em} % ← agrega espacio aquí

Contar con un proveedor especializado en la nube aporta beneficios adicionales \cite{refnub3}, como lo son:
\begin{itemize}
    \item Agilidad empresarial y mantenimiento simplificado.
    \item Costos reducidos mediante esquemas de pago por uso.
    \item Escalabilidad inmediata según la demanda.  
    \item Alta fiabilidad e infraestructura redundante.  
    \item Acceso centralizado y movilidad global.
    \item Recuperación ante fallos gracias a planes de continuidad del negocio.  
    \item Componentes de la infraestructura de nube  
\end{itemize}

\vspace{1em} % ← agrega espacio aquí

La infraestructura de nube se compone de varios elementos esenciales \cite{refnub4}:
\begin{itemize}
    \item \textbf{Servidores}: equipos de alto rendimiento ubicados en centros de datos distribuidos.
    \item \textbf{Redes}: conectividad entre aplicaciones, servicios y datos mediante equipos de conmutación y balanceadores de carga.
    \item \textbf{Almacenamiento}: espacios escalables para guardar datos de manera segura y accesible.
    \item \textbf{Software}: herramientas de administración, máquinas virtuales y entornos de análisis que facilitan el uso de los recursos virtualizados.
\end{itemize}

Los servicios en la nube ofrecen una amplia gama de beneficios, que van desde bajos costos iniciales hasta una mayor libertad de acceso al software. Su naturaleza adaptable y escalable permite que se ajusten a diferentes proyectos y empresas de cualquier tamaño o sector. La diversidad de opciones disponibles brinda la posibilidad de seleccionar el tipo de servicio que mejor se alinee con las necesidades de cada proyecto, lo que convierte a la nube en una solución versátil y aplicable en prácticamente cualquier entorno \cite{refnub5}.

\subsection{AWS}
Amazon Web Services (AWS) es uno de los principales proveedores de servicios en la nube a nivel mundial. Ofrece una amplia gama de soluciones tecnológicas que incluyen almacenamiento, recursos de computación, bases de datos, desarrollo de aplicaciones móviles, inteligencia artificial y herramientas empresariales, todo bajo el modelo de cloud computing \cite{refaws1}.\\

AWS tiene presencia en más de 190 países y cuenta con una extensa red de centros de datos distribuidos en América, Europa, Asia y Oceanía. Su infraestructura global garantiza alta disponibilidad, seguridad y escalabilidad. Entre sus ventajas más destacadas se encuentran \cite{refaws1}:
\begin{itemize}
    \item \textbf{Seguridad}: cuenta con certificaciones internacionales como PCI DSS, ISO 27001, HIPAA y auditorías SOC 1 y SOC 2.
    \item \textbf{Amplia oferta de bases de datos}: soporta sistemas como MySQL, Oracle, PostgreSQL, SQL Server, MongoDB y Amazon Aurora.
    \item \textbf{Bajo costo}: elimina la necesidad de inversión en infraestructura local mediante un modelo de pago por uso.
    \item \textbf{Accesibilidad y flexibilidad}: los servicios pueden adaptarse rápidamente a las necesidades del mercado, con incorporación constante de nuevas herramientas.
    \item \textbf{Gobernanza y visibilidad}: permite auditar y controlar los datos a pesar de que la infraestructura sea administrada por AWS.
    \item \textbf{Resiliencia}: sus centros de datos están diseñados para operar incluso en situaciones críticas o de contingencia.
\end{itemize}

Las herramientas y servicios de AWS abarcan múltiples áreas de la computación en la nube \cite{refaws2}:
\begin{itemize}
    \item \textbf{Cloud computing}: incluye modelos de servicio como IaaS, PaaS y SaaS, permitiendo escalar aplicaciones según la demanda.
    \item \textbf{Redes privadas virtuales}: con Amazon VPC es posible crear entornos seguros y personalizados dentro de la nube.
    \item \textbf{Gestión de bases de datos}: Amazon RDS facilita la creación y administración de bases de datos accesibles desde cualquier dispositivo o ubicación.
    \item \textbf{Aplicaciones empresariales y móviles}: AWS ofrece herramientas para el desarrollo de apps, respaldo de datos (backup), recuperación ante desastres (disaster recovery), autenticación doble, IoT e inteligencia de negocios.
\end{itemize}

Entre los servicios más utilizados y representativos de la plataforma se encuentran \cite{refaws3}:
\begin{itemize}
    \item \textbf{Servicios informáticos}: Amazon EC2 (Elastic Compute Cloud) proporciona potencia de cómputo escalable, y AWS Lambda permite ejecutar código sin necesidad de administrar servidores.
    \item \textbf{Almacenamiento}: Amazon S3 ofrece almacenamiento de objetos escalable, Amazon Glacier permite archivado de bajo costo, y Amazon EBS proporciona almacenamiento en bloques de alto rendimiento.
    \item \textbf{Bases de datos}: Amazon RDS gestiona bases relacionales, mientras que DynamoDB ofrece bases NoSQL rápidas y flexibles.
    \item \textbf{Red y distribución de contenido}: AWS Direct Connect ofrece conexiones dedicadas y Amazon CloudFront acelera la entrega global de contenido mediante una red CDN.
    \item \textbf{Seguridad y cumplimiento}: AWS IAM gestiona el acceso a los recursos y AWS Shield protege contra ataques DDoS.
    \item \textbf{Servicios avanzados}: la plataforma incluye soluciones para aprendizaje automático, análisis de datos, IoT y herramientas de desarrollo, facilitando la integración de funciones inteligentes y automatización.
\end{itemize}

AWS constituye una plataforma integral y versátil que ha transformado la forma en que las organizaciones gestionan sus operaciones digitales. Sus ventajas en seguridad, rendimiento y escalabilidad la convierten en una de las opciones más sólidas del mercado, y su amplia oferta de servicios, alcance global y constante innovación la posicionan como un referente líder en computación en la nube \cite{refaws3}.\\

\begin{center}
    \includegraphics[width=0.7\textwidth]{Images/Cap 2/aws.png}
    \captionof{figure}[Logo de AWS]{Amazon Web Services (AWS), obtenido de \cite{refaws4}.} 
\end{center}

\section{Las 10 Reglas Heurísticas de Usabilidad de Nielsen}
En 1994, Jakob Nielsen estableció un conjunto de reglas que todos los sistemas deben cumplir, para la detección de fallos sin realizar pruebas de usuario. Dichas reglas son útiles para realizar una evaluación de usabilidad de un sitio web, aplicación o producto digital \cite{ref61}.\\

A continuación, se enlistan las 10 reglas heurísticas de usabilidad de Nielsen \cite{ref61}:
\begin{enumerate}
    \item \textbf{Visibilidad y estado del sistema}: el diseño de cualquier interfaz debe mantener informado a los usuarios sobre lo que sucede, para evitar confusiones.
    \item \textbf{Coincidencia entre el mundo real y el sistema}: se deben emplear palabras y conceptos que sean familiares para el usuario, para asegurar la comprensión de la información.
    \item \textbf{Control y libertad al usuario}: los usuarios deben tener la posibilidad de poder deshacer y rehacer acciones, para poder tener el control y evitar que se queden atascados al momento de realizar acciones erróneas.
    \item \textbf{Estándares y consistencia}: los elementos visuales y comportamientos del sistema deben ser uniformes, permitiendo que el usuario pueda navegar de forma intuitiva.
    \item \textbf{Prevención de errores}:se debe priorizar la eliminación de aquellas condiciones que sean propensas al error, por medio de mensajes o validaciones.
    \item \textbf{Reconocimiento para evitar el recuerdo}: la reducción de la información que un usuario tiene que recordar facilita el uso de un sistema o aplicación. Los elementos necesarios deben ser visibles.
    \item \textbf{Flexibilidad y eficiencia de uso}: se le debe permitir a los usuarios a los usuarios adaptar las acciones frecuentes, para acelerar la interacción de un usuario.
    \item \textbf{Diseño estético y minimalista}: el contenido y el diseño deben estar centrados en los elementos esenciales, evitando que las interfaces estén sobrecargadas de información irrelevante o innecesaria.
    \item \textbf{Ayudar a los usuarios para reconocer, diagnosticar y recuperarse de los errores}: los mensajes de error deben ser expresados en un lenguaje sencillo, para su facil detección y tratamiento.
    \item \textbf{Ayuda y documentación}: es fundamental contar con documentación en la que se enlisten los pasos concretos que debe seguir el usuario para evitar errores y poder completar tareas.
\end{enumerate}

\section{Norma ISO 9241-210}
La norma ISO 9241-210, o también conocida como norma ISO 9241-210:2019, es una norma internacional que se enmarca en la categoría más amplia de la norma ISO 9241, un conjunto de normas relacionadas con la ergonomía de la interacción persona-sistema. Esta norma proporciona directrices y principios para el diseño de sistemas interactivos que priorizan las necesidades, capacidades y preferencias de los usuarios, con el objetivo de mejorar su satisfacción y usabilidad \cite{ref62}.\\

Los puntos y objetivos clave descritos en la norma ISO 9241-210:2019 \cite{ref62}:

\begin{itemize}
    \item \textbf{Diseño Centrado en el Ser Humano (HCD)}: este punto resalta la importancia de involucrar a los usuarios finales durante todo el proceso de diseño y desarrollo, para que los sistemas desarrollados sean más intuitivos, eficientes y eficaces.
    \item \textbf{Proceso iterativo}: promueve un proceso de diseño iterativo, donde los diseñadores recopilan continuamente la opinión de los usuarios, perfeccionan sus diseños y los vuelven a probar, con el objetivo de identificar y abordar problemas de usabilidad en las primeras etapas del proceso de diseño.
    \item \textbf{Enfoque centrado en el usuario}: la norma subraya la necesidad de comprender a fondo las características, los objetivos y las tareas de los usuarios. El diseño debe ajustarse a las necesidades de los usuarios.
    \item \textbf{Aplicabilidad}: es aplicable a varios tipos de sistemas interactivos, incluidas aplicaciones de software, sitios web, aplicaciones móviles e interfaces de hardware.
    \item \textbf{Evaluación de usabilidad}: el estándar fomenta el uso de métodos de evaluación de usabilidad, como pruebas de usuarios y evaluaciones de expertos, para evaluar la eficacia del diseño y realizar mejoras.
    \item \textbf{Accesibilidad}: los diseñadores deben garantizar que sus sistemas interactivos sean accesibles para personas con discapacidad y personas de la tercera edad.
    \item \textbf{Documentación}: es primordial documentar todo el proceso de diseño, incluidos los resultados de la investigación de usuarios, las decisiones de diseño y los resultados de las pruebas de usabilidad.
\end{itemize}